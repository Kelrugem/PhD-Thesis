\chapter{Future works}\label{ConclusionTheEnd}
%
%In this thesis we have seen a lot. We have restated a covariantized theory of gauge theory, allowing non-flat vector bundle connections $\nabla$ on the Lie algebroid replacing the Lie algebra, and we got an additional 2-form $\zeta$ in the field strength; this theory was originally introduced by Alexei Kotov and Thomas Strobl. Additionally, there is a field redefinition which keeps the Lagrangian invariant, but which might lead to a standard formulation of gauge theory. With this redefinition it is also possible to motivate $\zeta$, then it is not just an auxiliary map allowing non-flat connections as in the original motivation of $\zeta$ provided by Alexei Kotov and Thomas Strobl.
%
%In the case of Lie algebra bundles (LABs) and using the compatibility conditions, we were able to see that the studied connection $\nabla$ is a Lie derivation law covering a Lie algebroid morphism $\Xi: \mathrm{T}N \to \mathrm{Out}(\mathcal{D}_{\mathrm{Der}}(K))$, the latter then given by $\Xi = \sharp \circ \nabla$.
%
%We have proven that the field redefinitions and the question about, whether or not we can transform to a pre-classical gauge theory where $\nabla$ is flat, has a strong relationship to Mackenzie's study about extending tangent bundles with LABs over the same base in sense of Lie algebroids. Using Mackenzie's results, we were able to use an obstruction class $\mathrm{Obs}(\Xi)$ and to argue that locally we can always transform to a pre-classical gauge theory, while globally it is a different question: Having a non-trivial obstruction class leads quickly to a non-pre-classical gauge theory, while even a trivial obstruction class can still imply a non-pre-classical theory.
%
%The obstruction class is also strongly related to $\mathrm{d}^\nabla \zeta$, an invariant of the field redefinition. Studying this object leads to the quick result that $\mathrm{d}^\nabla \zeta \neq 0$ already implies that there is no field redefinition leading to a classical theory because then $\zeta$ cannot vanish after any field redefinition. This condition implies the failure of the (covariantized) Bianchi identity of the field strength. Moreover, there is a canonical construction of such a theory with $\mathrm{d}^\nabla \zeta \neq 0$ when starting with a classical theory.
%
%In the case of tangent bundles we have quickly seen that they are locally always pre-classical, while globally we have found an example in $\mathds{S}^7$. The CYMH GT of $\mathds{S}^7$ cannot be flattened by the field redefinition since otherwise it would carry a Lie group structure.
%
%We were able to generalize certain statements to direct products of tangent bundles and LABs; resulting into that those are locally pre-classical.
%
%As a side effect of the reformulation of CYMH GTs, we got a coordinate-free version of CYMH GTs and a new formulation of infinitesimal gauge transformations. We were able to rewrite infinitesimal gauge transformations as derivations induced by a generalized pullback of Lie algebroid connections. After solving issues related to the formulation of the commutator of two infinitesimal gauge transformations due to that the parameters $\varepsilon$ for such transformations are functionals themselves, \textit{i.e.}~elements of $\mathcal{F}^0_E(M; {}^*E)$ (recall the discussion about the pre-bracket $\Delta$ in Def.~\ref{def:PrebracketonPullbackLiealgebroid}), we studied the "curvature"/commutator of the infinitesimal gauge transformation acting on vector bundle valued functionals. Its result was that infinitesimal gauge transformations may actually be formulated by using the basic connection and not using $\nabla_\rho$; the latter might look more natural since it restricts to the canonical flat connection in the classical situation, but the basic connection is in the context of CYMH GT always flat, while $\nabla_\rho$ is not. Flatness of the Lie algebroid connection behind the definition of infinitesimal gauge transformations was essential in order to know that the commutator of two transformations is still a transformation. This lead to explicit changes in the formulas of the infinitesimal gauge transformations for the physical quantities like the field strength, while the infinitesimal gauge invariance of the Lagrangian was unaffected. Using the basic connection is also in better alignment of the symmetries of a gauge theory, as also emphasized by the compatibility conditions.
%
One may take these results as a motivation to always assume that a CYMH GT is pre-classical. 
There is hope to generalize the construction of the obstruction class to every Lie algebroid by assuming that the isotropy of the Lie algebroid is stable under the chosen connection. As we have seen, this stability condition is invariant under the field redefinition, and it may allow to reduce the study "roughly" to a study of Lie algebra bundles because the isotropy is a Lie algebra bundle around \textbf{regular points} in our case, also recall Thm.~\ref{thm:BLALAB}. Of course, a Lie algebroid consists of more than an isotropy. To take care of the remaining structure one could "decouple" the Lie algebroid along the foliation and along a transversal submanifold using the splitting theorem. However, we also have seen that there are certain difficulties in that approach.

%Sadly, nothing of this is published yet, but I am going to write two or even three preprints soon. This task was way more difficult than expected such that I couldn't publish anything earlier yet. I attached a section of my thesis which will be roughly the same as my first preprint which I want to upload in the next weeks.

Future plans for research could be studying a possible generalized definition of the obstruction class, using the previously-mentioned idea or another ansatz; in general, there are still a lot of open questions regarding general Lie algebroids which need to be answered. The question about the (physical) significance of the tensor $\zeta$ is interesting, too. For this it is also necessary to quantize this theory.

One could also think about integrating this theory, probably using Lie groupoids instead of Lie groups. Often it is of advantage if underlying curvatures are flat when it is about integrability, which may mean that $\nabla$ needs to be flat for a suitable integration and that may be a further argument for assuming that the theory is already pre-classical. However, since we used the basic connection to define infinitesimal gauge transformations, which is always flat in our context, we may or may not have solved a certain problem in integrating CYMH GTs.

Another possible plan is to go back to the example of unit octonions. $\mathbb{S}^7$ is a Moufang loop and its corresponding tangent space at its neutral element is an algebra known as Malcev algebra. Hence, this example may show that a suitable new formulation of gauge theory may be in replacing Lie groups and Lie algebras with Moufang loops and Malcev algebras, respectively. In a private talk to Alessandra Frabetti I learned that one seemingly only needs the structure of Moufang loops for renormalizations such that it might be fruitful to develop a gauge theory using that notion.

\textbf{\emph{Thanks}} for reading and your support! Do not hesitate to ask me further questions. I wish you a nice and pleasant time.

\textbf{Acknowledgements about finances:} This work was produced within the scope of the NCCR SwissMAP which was funded by the Swiss National Science Foundation. I would like to thank the Swiss National Science Foundation for their financial support.

This thesis was also supported by the LABEX MILYON (ANR-10-LABX-0070) of Universit\'{e} de Lyon, within the program "Investissements d'Avenir" (ANR-11-IDEX- 0007) operated by the French National Research Agency (ANR).