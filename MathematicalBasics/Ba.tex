\chapter{General theory of Lie algebroids}\label{MathematicalBasics}

\section{Lie algebroids}\label{LieAoids}

In the following we follow \cite[\S VII]{DaSilva}.

\begin{definitions}{Lie algebroid, \cite[reduced definition of \S 16.1, page 113]{DaSilva}}{test}
%\leavevmode\newline
Let $\gls{E} \to N$ be a real vector bundle of finite rank. Then $E$ is a smooth Lie algebroid if there is a bundle map $\gls{1rho}: E \to \mathrm{T}N$, called the \textbf{anchor}, and a Lie algebra structure on $\Gamma(E)$ with Lie bracket $\gls{0[]E}$ satisfying
\ba
  \mleft[\mu, f \nu\mright]_E = f \mleft[\mu, \nu\mright]_E + \mathcal{L}_{\rho(\mu)}(f) ~ \nu
\label{eq:E-Leibniz}
\ea
for all $f \in C^\infty(N)$ and $\mu, \nu \in \Gamma(E)$, where $\mathcal{L}_{\rho(\mu)}(f)$ is the action of the vector field $\rho(\mu)$ on the function $f$ by derivation. We will sometimes denote a Lie algebroid by $\mleft( E, \rho, \mleft[ \cdot, \cdot \mright]_E \mright)$.
\end{definitions}

\begin{remarks}{Transitive Lie algebroids, \cite[very beginning of \S 17; page 123]{DaSilva}}{TransitiveLieALgeoids}
If the anchor $\rho$ is surjective, then we say that \textbf{$E$ is transitive}.
\end{remarks}

\begin{remark}
\leavevmode\newline
We often will just write "Let $E$ be a Lie algebroid.", with that we canonically also denote the anchor by $\rho$ or $\rho_E$ and the Lie bracket by $\mleft[ \cdot, \cdot \mright]_E$ without further clarifying these notations. Furthermore, \cite[\S 16.1, page 113]{DaSilva} imposes that $\rho$ is a homomorphism of Lie brackets as a part of the definition of Lie algebroids, but we will see in the following that this is not needed, it will be already a consequence of this reduced definition as explained in \textit{e.g.}~\cite[page 68]{Homomrho}.
\end{remark}

\begin{examples}{\cite[\S 16.2, page 114]{DaSilva}}{LAoidsGeneralizeLAAndTN}
The two basic examples of Lie algebroids are the following.
\begin{enumerate}
	\item Each finite dimensional real Lie algebra is a Lie algebroid over a point set $\{*\}$ with zero anchor.
	\item The tangent bundle $\mathrm{T}N$ of any manifold $N$ where the anchor is the identity map and where the Lie bracket is the usual one of vector fields.
\end{enumerate}
\end{examples}

As shown by the basic examples above, the idea behind Lie algebroids is that they are a simultaneous generalization of tangent bundles and Lie algebras, this allows a generalization of specific terms of their calculus to Lie algebroids. We will also always view tangent bundles as Lie algebroids given by the structure presented in Ex.~\ref{ex:LAoidsGeneralizeLAAndTN}.

\begin{definitions}{Basic calculus on Lie algebroids $E$}{BasicCalculusOf LieAlgoide}
Let $E \to N$ be a Lie algebroid and $V\to N$ a vector bundle, then we define the following:
\begin{itemize}
	\item \textbf{Structure functions}, \cite[\S 16.5, page 119]{DaSilva}
	\newline Let $\mleft( e_a \mright)_a$ be some local frame over some open subset $U \subset N$. Then the \textbf{structure functions} $\gls{Cbca} \in C^\infty(U)$ are defined by 
	\ba
	[e_b, e_c]_E = C^a_{bc} e_a.
	\ea
	\item \textbf{$E$-Lie derivatives}, \cite[\S 16.1; page 113]{DaSilva}
	\newline One can define \textbf{$E$-Lie derivatives}, similar as in the situation of tangent bundles, by
	\ba
	\gls{Lie}_\mu(\nu) &\coloneqq [\mu, \nu]_E, \\
	\gls{Lie}_\mu(f) &\coloneqq \mathcal{L}_{\rho(\mu)}(f)
	\ea
	for all $f \in C^\infty(N)$ and $\mu, \nu \in \Gamma(E)$. The Leibniz rule \eqref{eq:E-Leibniz} then reads
	\ba
	\mathcal{L}_\mu (f\nu) &= f \mathcal{L}_\mu(\nu) + \mathcal{L}_\mu (f) ~ \nu
	\ea
	for all $f \in C^\infty(N)$ and $\mu, \nu \in \Gamma(E)$. We will use both notations, $\mathcal{L}_\mu$ and $\mathcal{L}_{\rho(\mu)}$; it is clear by context which is meant.
	\item \textbf{$E$-forms}, \cite[\S 18.1; page 131]{DaSilva}
	\newline The antisymmetric parts of $(0,s)$-$E$-tensors define the \textbf{$E$-forms}, \textit{i.e.}~$\gls{1ZOmegas(E)} \coloneqq \Gamma\mleft(\bigwedge^s E^*\mright)$ ($s \in \mathbb{N}_0$). The previously defined Lie derivative can be extended to those forms (and general $E$-tensors) with the typical definitions by imposing the Leibniz rule. As for typical forms, one can define \textbf{$E$-forms with values in $V$} by $\gls{1ZOmegap(EV)} \coloneqq \Gamma\mleft(\bigwedge^s E^* \otimes V \mright)$.
	\item \textbf{$E$-differential}, \cite[\S 18.1, page 131]{DaSilva}
	\newline The \textbf{$E$-differential} is defined as $\gls{dE}: \Omega^\bullet(E) \to \Omega^{\bullet+1}(E)$ by
	\ba
	\mleft(\mathrm{d}_E \omega\mright) \mleft( \nu_0, \dots, \nu_s \mright)
	&\coloneqq
	\sum_i (-1)^i ~ \mathcal{L}_{\nu_i} \mleft( \omega\mleft( \nu_0, \dots, \widehat{\nu}_i, \dots, \nu_s \mright) \mright) \nonumber \\
	&\hspace{1cm}
		+ \sum_{i < j} (-1)^{i+j} ~ \omega\mleft( \mleft[ \nu_i, \nu_j \mright]_E, \nu_0, \dots, \widehat{\nu}_i, \dots, \widehat{\nu}_j, \dots, \nu_s \mright)
	\ea
	for all $\omega \in \Omega^s(E)$ and $\nu_0, \dots, \nu_s \in \Gamma(E)$.
\end{itemize}
\end{definitions}

\begin{remark}
\leavevmode\newline
\indent $\bullet$ $\Gamma(E)$ is an infinite-dimensional Lie algebra w.r.t.~$[\cdot, \cdot]_E$ but it should be seen as a generalization of finite dimensional Lie algebras whose "finite dimension" is the finite rank of $E$: Choose a local frame $\mleft(e_a\mright)_a$ of $E$ over an open subset $U \subset N$. As introduced, one gets in general now \textit{structure functions} $C^a_{bc} \in C^\infty(U)$ instead of \textit{structure constants} and a base of the Lie algebra is replaced by such a (local) frame on the vector bundle; recall the last section about classical gauge theory where we viewed the basis of the Lie algebra as a global constant frame.

$\bullet$ In the following we will argue that the anchor of a Lie algebroid is a homomorphism of Lie brackets (if viewed as a tensor acting on sections). With that one can then show $\mathrm{d}_E^2=0$ by precisely the same calculation as one does with respect to the de-Rham differential. As argued in \cite[\S 18.1, page 131f.]{DaSilva}, there is a one-to-one correspondence between Lie algebroid structures and such differential operators squaring to zero and satisfying the graded Leibniz rule with respect to the wedge product. Moreover, there is also a correspondence to vector bundles admitting a cohomological vector field; but we won't use these relationships which is why we are not going to state or explain these relationships explicitly.
	%
	%Due to the Leibniz rule the structure functions do not transform as a tensor in general, \textit{i.e.} let $f_a \coloneqq M_a^b e_b$ be another local frame on some open subset $V \subset U$ for some invertible matrix function $M$, then with $[f_b, f_c]_E = \tilde{C}_{bc}^a f_a$ one can calculate in a straightforward manner that
	%\ba
	%\tilde{C}_{bc}^a
	%&= M_b^d M_c^g C_{dg}^h \left( M^{-1} \right)^a_h + M_b^d \left( M^{-1} \right)^a_g \mathcal{L}_{e_d}( M^g_c) - M_c^d \left( M^{-1} \right)^a_g \mathcal{L}_{e_d}( M^g_b).
	%\ea
	%
	%It is clear that the structure functions are exactly the structure constants in the case of a Lie algebra. In the situation of tangent bundles one could take \textit{e.g.} some coordinate vector fields $\mleft(\partial_i\mright)_i$ as a local frame, and their structure functions are zero.
	%
	%With other straightforward calculations one also shows
	%\ba
	%C^a_{bc} &= - C^a_{cb}, \\
	%0
	%&=
	%C^d_{ae} C^e_{bc} + C^d_{be} C^e_{ca} + C^d_{ce} C^e_{ab}
		%+ \mathcal{L}_{e_a}\mleft( C^d_{bc} \mright)
		%+ \mathcal{L}_{e_b}\mleft( C^d_{ca} \mright)
		%+ \mathcal{L}_{e_c}\mleft( C^d_{ab} \mright),
	%\ea
	%where the first equation comes from the antisymmetry and the second one from the Jacobi identity of $\mleft[ \cdot, \cdot \mright]_E$.
\end{remark}

In older works about Lie algebroids (also in \cite{DaSilva}) one often sees that the definition also contains the condition about that the induced map $\Gamma(\rho): \Gamma(E) \to \mathfrak{X}(N)$ (which we will still denote as $\rho$) is a homomorphism of Lie algebras w.r.t. $\mleft[ \cdot, \cdot \mright]_E$ and $\mleft[ \cdot, \cdot \mright]$, the Lie bracket of vector fields $\mathfrak{X}(N)$. But that is not needed, see \textit{e.g.}~\cite[page 68]{Homomrho}. To show this we want to introduce some measures for the homomorphism property and the Jacobi identity. Let us start with the former.

\begin{definitions}{Curvature of morphisms, \newline \cite[variant of Definition 5.2.9; page 187]{mackenzieGeneralTheory}}{GeneralDefOfCurvMorphisms}
Let $E_1, E_2$ be two Lie algebroids over the same base manifold $N$. Then the \textbf{curvature of a vector bundle morphism $\xi: E_1 \to E_2$} is a map $\gls{Rxi}: \Gamma(E_1) \times \Gamma(E_1) \to \Gamma(E_2)$ defined by
\ba
R_\xi(\mu, \nu)
&\coloneqq
\mleft[ \xi(\mu), \xi(\nu) \mright]_{E_2}
	- \xi \mleft(
		\mleft[ \mu, \nu \mright]_{E_1}
	\mright)
\ea
for all $\mu, \nu \in \Gamma(E_1)$.
\end{definitions}

\begin{remark}
\leavevmode\newline
$R_\xi$ is clearly anti-symmetric.

For an anchor $\rho$ of a Lie algebroid we therefore expect $R_\rho = 0$ in case it is a homomorphism of Lie brackets.
\end{remark}

Later, in the sections about connections, we will see that it makes sense to call $R_\xi$ curvature, though one may already see why by its definition. What we want to show is that $R_\rho = 0$ for an anchor $\rho$ of a Lie algebroid. Hence, let us first show that those curvature are tensors if $\xi$ is an anchor preserving vector bundle morphism, which basically describes a morphism related to the structure given by the anchor:\footnote{In fact, one can also define vector bundles known as anchored vector bundles which are just vector bundles with a bundle map like the anchor; see \textit{e.g.}~\cite[\S 3, first part of Definition 3.1]{meinrenkensplitting}. Then the following definition is the definition of morphisms of anchored vector bundles.}

\begin{definitions}{Anchor-preserving vector bundle morphism, \newline \cite[\S 4.3, Equation (22); page 157]{mackenzieGeneralTheory}}{DefOfAnchorPreservingStuff}
Let $E_i\stackrel{\pi_i}{\to} N_i$ ($i \in \{1,2\}$) be two Lie algebroids over smooth manifolds $N_i$. Then we say that a vector bundle morphism $\xi: E_1 \to E_2$ over a smooth map $f: N_1 \to N_2$\footnote{That means $\pi_2 \circ \xi = f \circ \pi_1$.} is \textbf{anchor-preserving} if it satisfies
\ba\label{EqFuerAnchorBundleMorphisms}
\mathrm{D}f \circ \rho_{E_1}
&=
\rho_{E_2} \circ \xi.
\ea
\end{definitions}

\begin{remarks}{Notations and base-preserving morphisms}{SomeExtraNotationForAnchorBundleMorphs}
$\bullet$ As it is well-known, $\xi$ does not necessarily induce a map $\Gamma(E_1) \to \Gamma(E_2)$ on sections, that depends on how $f$ is structured. However, we have 
\bas
\pi_2 \bigl( \xi(\nu) \bigr)
&=
f\bigl( \underbrace{\pi_1 (\nu)}_{= \mathds{1}_{N_1}} \bigr)
=
f
\eas
for all $\nu \in \Gamma(E_1)$, such that $\xi$ induces a tensor on $\Gamma(E_1) \to \Gamma(f^*E_2)$ (the $C^\infty(N_1)$-linearity follows trivially); see \textit{e.g.}~\cite[paragraph after Propositon 7.10]{meinrenkenlie}, too. Recall, that we introduced that already for maps like $\mathrm{D}f$ at the end of the introduction, that is, $\mathrm{D}f \in \Omega^1(N_1; f^*\mathrm{T}N_2)$, which is also trivially an anchor-preserving vector bundle morphism over $f$. This is why we write equations like Eq.~\eqref{EqFuerAnchorBundleMorphisms} often as
\ba
\mathrm{D}f \circ \rho_{E_1}
&=
(f^*\rho_{E_2}) \circ \xi
\ea
when we view that condition as an equation for sections, in order to emphasize the relationship with the pullback;
recall that $f^*\rho_{E_2}: \Gamma(f^*E_2) \to \Gamma(f^*\mathrm{T}N_2)$. However, sometimes we also omit the notation of that pullback in that case.
\newline
\newline
$\bullet$ If $E_1, E_2$ are two Lie algebroids over the same base manifold $N$, then a vector bundle morphism $\xi: E_1 \to E_2$ is anchor-preserving if it satisfies
\ba
\rho_{E_1}
&=
\rho_{E_2} \circ \xi.
\ea
For this recall, that in this case we always mean base-preserving morphisms if not mentioning otherwise, that is, $f = \mathds{1}_N$. The anchor is therefore a trivial example for an anchor-preserving morphism.
\end{remarks}

\begin{remark}
\leavevmode\newline
As in \cite[Definition 5.2.5; page 186]{mackenzieGeneralTheory} one may also call such anchor-preserving morphisms ($E_1$-) connections; also here it will be clearer later why, but to avoid confusion with typical connections carrying a Leibniz rule (also called Koszul connection in \cite{mackenzieGeneralTheory}), we will not denote those as such.
\end{remark}

\begin{lemmata}{Curvatures are tensorial in case of anchor-preservation, \newline \cite[variant of Lemma 5.2.8; page 187]{mackenzieGeneralTheory}}{KruemmungenSindTensorenMitAnkerErhaltung}
Let $E_1, E_2$ be two Lie algebroids over the same base manifold $N$, and $\xi:E_1\to E_2$ an anchor-preserving vector bundle morphism. Then $R_\xi$ is an anti-symmetric tensor, \textit{i.e.}~it is $C^\infty(N)$-bilinear.
\end{lemmata}

\begin{remark}
\leavevmode\newline
This also shows that one could test the homomorphism property of anchors in just one frame around each point locally, because anchors are trivially anchor-preserving morphisms.
\end{remark}

\begin{proof}[Proof of Lemma \ref{lem:KruemmungenSindTensorenMitAnkerErhaltung}]
\leavevmode\newline
$R_\xi$ is clearly antisymmetric and, thus, we only need to show the $C^\infty(N)$-linearity with respect to one argument. That is, applying the Leibniz rule on both summands,
\bas
R_\rho(\mu, f \nu)
&=
\mleft[ \xi(\mu),f \xi(\nu) \mright]_{E_2}
	- \xi \mleft(
		\mleft[ \mu, f \nu \mright]_{E_1}
	\mright)
\\
&=
f R_\xi(\mu, \nu)
	+  \underbrace{\mathcal{L}_{(\rho_{E_2} \circ \xi)(\mu)}(f)}
	_{\mathcal{L}_{\rho_{E_1}(\mu)}(f)}
	~ \xi(\nu)
	- \xi\mleft( \mathcal{L}_{\rho_{E_1}(\mu)}(f) ~ \nu \mright)
\\
&=
f R_\xi(\mu, \nu)
\eas
for all $\mu, \nu \in \Gamma(E_1)$ and $f \in C^\infty(N)$.
\end{proof}

%\begin{remark}
%\leavevmode\newline
%By using what we discussed in Remark \ref{rem:SomeExtraNotationForAnchorBundleMorphs}, one can define a curvature also for vector bundle morphisms of Lie algebroids over different bases, and that notion should still be a tensor in case of anchor-preserving morphisms, too.
%\end{remark}
There is a certain relationship between the curvature of an anchor $\rho$ using the Jacobiator which will help us to show that anchors are also Lie bracket homomorphisms.

\begin{definitions}{Jacobiator, \cite[Remark 6.12; page 35]{meinrenkenlie}}{JacobiatorOfLieAlgebras}
Let $W$ be a vector space, not necessarily finite-dimensional, equipped with an antisymmetric bilinear bracket $\mleft[ \cdot, \cdot \mright]_W: W\times W \to W, (v, w) \mapsto \mleft[ v, w \mright]_W$. Then we define the \textbf{Jacobiator $\gls{J}: W \times W \times W \to W$} by 
\ba
J(\mu, \nu, \eta)
&\coloneqq 
\mleft[\mu, \mleft[\nu, \eta\mright]_W\mright]_W
	+ \mleft[\nu, \mleft[\eta, \mu\mright]_W\mright]_W 
	+ \mleft[\eta, \mleft[\mu, \nu\mright]_W\mright]_W
\ea
for all $\mu, \nu \in W$.
\end{definitions}

\begin{remark}
\leavevmode\newline
It is clear that $J = 0$ if $W = \Gamma(E)$ as Lie algebra, for $E$ a Lie algebroid. It is also trivial to see that $J$ is $\mathbb{R}$-trilinear and antisymmetric.
\end{remark}

%\begin{definitions}{Jacobiator and curvature of a bundle map}{DefMeasureofJacobiandHomom}
%Let $E \to N$ be a real vector bundle of finite rank, equipped with a bundle map $\rho: E \to \mathrm{T}N$ and an antisymmetric bilinear bracket $\mleft[ \cdot, \cdot \mright]_E$ on the space of sections $\Gamma(E)$ satisfying the Leibniz rule \eqref{eq:E-Leibniz}. Then we define the following objects.
%\begin{itemize}
	%\item \textbf{Curvature of $\rho$}, \cite[variant of Definition 5.2.9, page 187]{mackenzieGeneralTheory}
	%\newline We define the curvature of $\rho$ by $R_\rho: \Gamma(E) \times \Gamma(E) \to \mathfrak{X}(N)$, \textbf{COMMENT: Change sign here}
	%\ba
	%R_\rho(\mu, \nu)
	%&\coloneqq
	%\rho \mleft( \mleft[ \mu, \nu \mright]_E \mright)
		%- \mleft[ \rho(\mu), \rho(\nu) \mright]
	%\ea
	%for all $\mu, \nu \in \Gamma(E)$.
	%\item \textbf{Jacobiator}, \cite[Remark 6.12, page 35]{meinrenkenlie}
	%\newline We define the Jacobiator $\gls{J}: \Gamma(E) \times \Gamma(E) \times \Gamma(E) \to \Gamma(E)$,
	%\ba
	%J(\mu, \nu, \eta)
	%&\coloneqq [\mu, [\nu, \eta]_E]_E + [\nu, [\eta, \mu]_E]_E + [\eta, [\mu, \nu]_E]_E
	%\ea
	%for all $\mu, \nu, \eta \in \Gamma(E)$.
%\end{itemize}
%\end{definitions}
%

\begin{propositions}{Relation of Jacobiator and anchor, \cite[page 68]{Homomrho}}{MeasureofJacobiandHomom}
Let $E \to N$ be a real vector bundle of finite rank, equipped with a bundle map $\rho: E \to \mathrm{T}N$ and an antisymmetric bi-linear bracket $\mleft[ \cdot, \cdot \mright]_E$ on the space of sections $\Gamma(E)$ satisfying the Leibniz rule \eqref{eq:E-Leibniz} with respect to $\rho$. Then the following are equivalent:
\begin{itemize}
	\item $J$ is a tensor, where $J$ is the Jacobiator related to $\Gamma(E)$ with bracket $\mleft[ \cdot, \cdot\mright]_E$.
	\item $R_\rho = 0$.
\end{itemize}
\end{propositions}

\begin{remarks}{Anchor is a Homomorphism}{AnchorAHomom}
This implies that the anchor of a Lie algebroid is a homomorphism of Lie algebras because the definition of Lie algebroids assumes the Jacobi identity on $\mleft[ \cdot, \cdot \mright]_E$, so, $J = 0$, the zero-tensor. Vice versa, when we know that $R_\rho = 0$, then we only need to check the Jacobi identity in one frame around each point because $J$ behaves like a tensor.
\end{remarks}

\begin{proof}[Proof of Prop.~\ref{prop:MeasureofJacobiandHomom}]
\leavevmode\newline
We have
\bas
J(\mu, \nu, f \eta)
&= 
\mleft[\mu, \mleft[\nu, f \eta\mright]_E\mright]_E 
	+ \mleft[\nu, \mleft[f \eta, \mu\mright]_E\mright]_E 
	+ \mleft[f \eta, \mleft[\mu, \nu\mright]_E\mright]_E 
\\
&= \left[\mu, f \left[\nu, \eta\right]_E + \mathcal{L}_{\rho(\nu)}(f) ~ \eta\right]_E + \left[\nu, f \left[\eta, \mu\right]_E - \mathcal{L}_{\rho(\mu)}(f) ~ \eta\right]_E 
\\
&\hspace{1cm} 
+ f [\eta, [\mu, \nu]_E]_E - \mathcal{L}_{\rho([\mu, \nu]_E)}(f) ~ \eta 
\\ 
&= 
f \underbrace{\left( [\mu, [\nu, \eta]_E]_E + [\nu, [\eta, \mu]_E]_E + [\eta, [\mu, \nu]_E]_E \right)}_{= J(\mu, \nu, \eta)}
\\
&\hspace{1cm}
				+ \mathcal{L}_{\rho(\mu)}(f) ~ [\nu, \eta]_E + \mathcal{L}_{\rho(\nu)}(f) ~ [\eta, \mu]_E  - \mathcal{L}_{\rho(\mu)}(f) ~ [\nu, \eta]_E + \mathcal{L}_{\rho(\nu)}(f) ~ [\mu, \eta]_E 
\\
&\hspace{1cm}
		 + \mathcal{L}_{\rho(\mu)}\left( \mathcal{L}_{\rho(\nu)}(f) \right) \eta - \mathcal{L}_{\rho(\nu)}\left( \mathcal{L}_{\rho(\mu)}(f) \right) \eta
		- \mathcal{L}_{\rho([\mu, \nu]_E)}(f) ~ \eta 
\\
&=
f J(\mu,\nu,\eta) + \left[ \mathcal{L}_{\rho(\mu)}, \mathcal{L}_{\rho(\nu)} \right](f) ~ \eta - \mathcal{L}_{\rho([\mu, \nu]_E)}(f) ~ \eta 
\\
&= 
f J(\mu,\nu,\eta) + \mathcal{L}_{[\rho(\mu), \rho(\nu)]}(f) ~ \eta - \mathcal{L}_{\rho([\mu, \nu]_E)}(f) ~ \eta 
\\
&= 
f J(\mu,\nu,\eta) - \mathcal{L}_{R_\rho(\mu, \nu)}(f) ~ \eta
\eas
for all $\mu, \nu, \eta \in \Gamma(E)$ and $f \in C^\infty(N)$. Thus, we have
\bas
J(\mu, \nu, f \eta)
&= 
f J(\mu, \nu, f \eta)
\eas
if and only if
\bas
R_\rho(\mu,\nu)
&=
0,
\eas
	%\bas
	%J \text{ is a tensor}
	%&\Leftrightarrow
	%\mathcal{L}_{R_\rho(\mu, \nu)}(f) = 0
	%\Leftrightarrow
	%R_\rho(\mu, \nu) = 0,
	%\eas
where we use that a vector field of $N$ is zero when it always acts as zero derivation. The same argument holds for all arguments due to the antisymmetry of $J$. Hence, we get the desired equivalence of statements.
\end{proof}

In the following we introduce other important examples of Lie algebroids which we need later, see \cite[\S 16.2]{DaSilva}.

\begin{examples}{Bundle of Lie algebras, \newline \cite[\S 16.2, Example 2; page 114]{DaSilva} and \cite[\S 16.3; page 116f.]{DaSilva}}{BLA}
A \textbf{bundle of Lie algebras}, or \textbf{\gls{BLA}}, is a bundle whose fibers consist of Lie algebras, necessarily of the same dimension, giving rise to structure functions on the base manifold which should be smooth.
	
Such a bundle is a Lie algebroid with the anchor $\rho \equiv 0$.
	
The converse is also true, every Lie algebroid with zero anchor is a bundle of Lie algebras because then $\mleft[ \cdot, \cdot  \mright]_E$ behaves as a tensor due to the lack of a real Leibniz rule and is thence a field of Lie algebra brackets. This is why BLAs may be just defined as Lie algebras with zero anchor.
\end{examples}

As argued in \cite[Theorem 6.4.5; page 238f.]{mackenzieGeneralTheory}, when the Lie algebras of each fibre of a bundle of Lie algebras are isomorphic to each as Lie algebras, then we denote that as \textbf{Lie algebra bundle} (in short \textbf{LAB}).

\begin{definitions}{Lie algebra bundle (LAB), \cite[Definition 3.3.8; page 104]{mackenzieGeneralTheory}}{LAB}
Let $\mathfrak{g}$ be a Lie algebra. A \textbf{Lie algebra bundle}, or \textbf{\gls{LAB}}, is a vector bundle $K \to N$ equipped with a field of Lie algebra brackets $\mleft[ \cdot, \cdot \mright]_{\mathfrak{g}}: \Gamma(K) \times \Gamma(K) \to \Gamma(K)$, \textit{i.e.}\ $\mleft[ \cdot, \cdot \mright]_{\mathfrak{g}} \in \Gamma\mleft(\bigwedge^2 K^* \otimes K \mright)$ such that it restricts to a Lie algebra bracket on each fibre, and such that $K$ admits an \textbf{LAB atlas $\{ \psi_i: K|_{U_i} \to U_i \times \mathfrak{g} \}$ of LAB charts} subordinate to some open covering $\mleft( U_i \mright)_i$ of $N$, that is, an atlas such that each induced map $\psi_{i, p}: K_p \to \mathfrak{g}$ is a Lie algebra isomorphism, where $p \in U_i$, $K_p$ the fiber at $p$, $\psi_{i, p} \coloneqq \mathrm{pr}_2 \circ \mleft.\psi_i\mright|_{K_p}$ and $\mathrm{pr}_2$ is the projection onto the second factor.
\end{definitions}

We are going to discuss those later in more detail. For gauge theory the following example is of special importance, and this example emphasizes why we are interested into Lie algebroids.

\begin{definitions}{Action Lie algebroids, \cite[\S 16.2, Example 5; page 114]{DaSilva}}{ActionLieAlgebroids}
Let $\mleft(\mathfrak{g}, \mleft[\cdot, \cdot \mright]_{\mathfrak{g}}\mright)$ be a Lie algebra equipped with a Lie algebra action $\gamma: \mathfrak{g} \to \mathfrak{X}(N)$ on a smooth manifold $N$. A \textbf{transformation Lie algebroid} or \textbf{action Lie algebroid} is defined as the bundle $E \coloneqq N \times \mathfrak{g}$ over $N$ with anchor
\ba
\rho(p, v) &\coloneqq \gamma(v)|_p
\ea
for $(p, v) \in E$, and Lie bracket
\ba\label{LieBracketActionLieAlg}
	\mleft.\mleft[\mu, \nu\mright]_E\mright|_p
	&\coloneqq 
	\mleft[\mu_p, \nu_p\mright]_{\mathfrak{g}}
		+ \mleft.\mleft(\mathcal{L}_{\gamma(\mu(p))}(\nu^a) - \mathcal{L}_{\gamma(\nu(p))}(\mu^a) \mright)\mright|_p ~ e_a
\ea
	for all $p \in N$ and $\mu, \nu \in \Gamma(E)$, where one views a section $\mu \in \Gamma(E)$ as a map $\mu: N \to \mathfrak{g}$ and $\mleft( e_a \mright)_a$ is some arbitrary frame of constant sections.
\end{definitions}

\begin{remark}
\leavevmode\newline
$\mleft[ \cdot,\cdot, \mright]_E$ is here clearly well-defined since one just allows global constant frames. That is, another global and constant frame is just given by $f_b = M_b^a e_a$, where $M_b^a$ are constants (and invertible as matrix). Due to this constancy, $\mleft.\mleft(\mathcal{L}_{\gamma(\mu(p))}(\nu^a) - \mathcal{L}_{\gamma(\nu(p))}(\mu^a) \mright)\mright|_p ~ e_a$ is clearly independent of the chosen global constant frame. 

Observe also that we have
\bas
\rho(\nu)
&=
\gamma(\nu),
\\
\mleft[\mu, \nu\mright]_E
&=
\mleft[\mu, \nu\mright]_{\mathfrak{g}}
\eas
for all constant sections $\mu, \nu \in \Gamma(E)$. We can trivially view constant sections of $E$ as elements of $\mathfrak{g}$ as we did in Chapter \ref{ClassicGaugeTheory}; doing so implies that action Lie algebroids encode the Lie algebra and its action.
\end{remark}

\begin{propositions}{Action Lie algebroids are Lie algebroids, \newline \cite[\S 16.2, Example 5; page 114]{DaSilva}}{ActionLieoidsAreOids}
Let $\mleft(\mathfrak{g}, \mleft[\cdot, \cdot \mright]_{\mathfrak{g}}\mright)$ be some Lie algebra equipped with a Lie algebra action $\gamma: \mathfrak{g} \to \mathfrak{X}(N)$ on a smooth manifold $N$. Then the action Lie algebroid as defined in Def.~\ref{def:ActionLieAlgebroids} is a Lie algebroid structure on $E = N \times \mathfrak{g}$. Moreover, it is the unique Lie algebroid structure on $E$ with
\ba
\rho(\nu)
&=
\gamma(\nu),
\\
\mleft[\mu, \nu\mright]_E
&=
\mleft[\mu, \nu\mright]_{\mathfrak{g}}
\ea
for all constant sections $\mu, \nu \in \Gamma(E)$.
\end{propositions}

\begin{remark}
\leavevmode\newline
The statement about uniqueness is equivalent to say that the action Lie algebroid is the unique Lie algebroid structure on $E= N \times \mathfrak{g}$ such that the map $h$, defined by
\bas
\mathfrak{g} &\to \Gamma(E),
\\
X &\mapsto h(X) = X,
\eas
is a Lie algebra homomorphism with $\rho \circ h = \gamma$,\footnote{Observe the similarity to the definition of anchor-preserving morphisms.} where we mean with $h(X) = X$ that $h(X)$ is $X$ as constant section in $E$. That emphasizes why we are interested into Lie algebroids when we want to generalize gauge theory. Together with the uniqueness this also implies that action Lie algebroids are the unique Lie algebroid structure related to classical gauge theory; which is why we want to use those later to recover the classical theory.
\end{remark}

\begin{proof}[Proof of Prop.~\ref{prop:ActionLieoidsAreOids}]
\leavevmode\newline
First, let us show that we have a Lie algebroid structure.
By construction it is clear that $\rho$ is a bundle map, $\mleft[ \cdot, \cdot \mright]_E$ is antisymmetric and satisfies the Leibniz rule w.r.t.~$\rho$. Using a global frame of constant sections $\mleft( e_a \mright)_a$, the curvature $R_\rho$ of $\rho$ (see Def.~\ref{def:GeneralDefOfCurvMorphisms}) is zero, in fact, for any $p \in N$ we have 
\bas
R_\rho(e_a, e_b)|_p
&=
[ \rho(e_a), \underbrace{\rho(e_b)}_{\mathclap{\stackrel{\text{const.}}{=} ~\gamma(e_b|_p) = \gamma(e_b) }} ]|_p 
	- \rho_p \mleft( \mleft.\mleft[ e_a, e_b \mright]_E\mright|_p \mright)
\\
&\stackrel{\mathclap{\text{const.}}}{=} ~~
\mleft.\mleft[ \gamma(e_a), \gamma(e_b) \mright]\mright|_p 
	- \mleft.\gamma\mleft( \mleft[ e_a, e_b \mright]_{\mathfrak{g}} \mright)\mright|_p
\\
&= 
0,
\eas
where we used that $\gamma$ is a homomorphism for the last equality. Thence, $\rho$ is a homomorphism.
	
Then by using Prop.~\ref{prop:MeasureofJacobiandHomom} one can finally show that the Jacobi identity is satisfied. By using again a global constant frame $\mleft( e_a \mright)_a$ and $\mleft[ e_a, e_b \mright]_E = \mleft[ e_a, e_b \mright]_{\mathfrak{g}}$, we get
\bas
J(e_a, e_b, e_c)
&=
\mleft[e_a, \mleft[e_b, e_c\mright]_E\mright]_E 
	+ \mleft[e_b, \mleft[e_c, e_a\mright]_E\mright]_E
	+ \mleft[e_c, \mleft[e_a, e_b\mright]_E\mright]_E(p) 
\\
&\stackrel{\mathclap{\text{const.}}}{=} ~~
\mleft[e_a, \mleft[e_b, e_c\mright]_E\mright]_{\mathfrak{g}} 
	+ \mleft[e_b, \mleft[e_c, e_a\mright]_E\mright]_{\mathfrak{g}} 
	+ \mleft[e_c, \mleft[e_a, e_b\mright]_E\mright]_{\mathfrak{g}} 
\\
&\stackrel{\mathclap{\text{const.}}}{=} ~~
\mleft[e_a, \mleft[e_b, e_c\mright]_{\mathfrak{g}}\mright]_{\mathfrak{g}} 
	+ \mleft[e_b, \mleft[e_c, e_a\mright]_{\mathfrak{g}}\mright]_{\mathfrak{g}}
	+ \mleft[e_c, \mleft[e_a, e_b\mright]_{\mathfrak{g}}\mright]_{\mathfrak{g}} 
\\
&= 0.
\eas
Therefore we can conclude that this defines a Lie algebroid. Uniqueness comes by construction because constant sections describe a global frame and since we require that the anchor is a bundle morphism, and that the Lie bracket on $\Gamma(E)$ needs to satisfy the Leibniz rule; in other words the definition of the action Lie algebroid comes precisely from the motivation to impose those conditions. That is, assume that we have another bundle map $\rho^\prime: E \to \mathrm{T}N$ with
\bas
\rho^\prime(\nu)
&=
\gamma(\nu)
=
\rho(\nu)
\eas
for all constant sections $\nu \in \Gamma(E)$. Then for all sections $\eta= \eta^a e_a \in \Gamma(E)$ we have
\bas
\rho^\prime(\eta)
&=
\eta^a \rho^\prime(e_a)
=
\eta^a \rho(e_a)
=
\rho(\eta),
\eas
hence, $\rho^\prime = \rho$ follows trivially, and, so, we can assume the same anchor for any other Lie algebroid structure. For the Lie bracket assume that there is another Lie bracket $\mleft[ \cdot, \cdot \mright]_E^\prime$ on $\Gamma(E)$, satisfying the Leibniz rule with respect to $\rho^\prime=\rho$, with
\bas
\mleft[ \mu, \nu \mright]_E^\prime
&=
\mleft[\mu, \nu\mright]_{\mathfrak{g}}
=
\mleft[\mu, \nu\mright]_E
\eas
for all constant sections $\mu, \nu$. Therefore we can show for all sections $\eta = \eta^a e_a, \xi = \xi^b e_b \in \Gamma(E)$ that
\bas
\mleft[ \eta, \xi \mright]_E^\prime
&=
\eta^a \xi^b ~ \underbrace{\mleft[e_a, e_b\mright]_E^\prime}_{=\mleft[e_a, e_b\mright]_E}
	+ \mleft(\mathcal{L}_{\rho(\eta)}(\xi^a)
	- \mathcal{L}_{\rho(\xi)}(\eta^a) \mright) ~ e_a
\\
&=
\mleft[ \eta, \xi \mright]_E
\eas
for all $p \in N$, using the Leibniz rule of both brackets with respect to $\rho^\prime = \rho$. This proves the uniqueness.
\end{proof}

Recall Prop.~\ref{prop:LieRepAndLieAct}, with that we can use previous examples of Lie algebra actions to construct action Lie algebroids.

\begin{examples}{$\mathrm{su}(2)$-action Lie algebroid, recall Ex. \ref{ex:sutwoliealgactionasLiealg} and its references}{sutwoliealgactionasLiealgoid}
Let $E \coloneqq \mathbb{R}^3 \times \mathbb{R}^3 \to \mathbb{R}^3$; $e_x, e_y, e_z$ are the standard unit vectors (which we will also denote by $e_1, e_2, e_3$, corresponding to $x^1=x, x^2=y, x^3=z$), the anchor is given by $\rho(e_j) = - \epsilon_{jkl} x^k ~ \partial/\partial x^l$, where $\epsilon_{jkl}$ is the Levi-Civita tensor. The Lie bracket is given by the cross product w.r.t. $\mleft(e_i\mright)_i$, \textit{i.e.} $\mleft[ e_i, e_j \mright]_E \coloneqq e_i \times e_j$.
	
That this is an action Lie algebroid simply follows by that its Lie algebra action is induced by the Lie algebra representation introduced in Ex.~\ref{ex:sutwoliealgactionasLiealg}.
\end{examples}

\begin{examples}{Electroweak interaction coupled to a Higgs field, \newline recall Ex. \ref{ex:electroweakinteractionasLiealg} and its references}{electroweakinteractionasLiealgoid}
The action Lie algebroid corresponding to the \textbf{electroweak interaction coupled to a Higgs field} is defined as action Lie algebroid for $\mathfrak{g} \coloneqq \mathrm{su}(2) \times \mathrm{u}(1)$ over $N \coloneqq \mathbb{C}^2 (\cong \mathbb{R}^4)$. Let $\mathrm{i}$ be the imaginary number, $g_w$ and $g^\prime$ be positive real numbers (the \textbf{coupling constants}), $n_\gamma$ be a non-zero natural number (a normalization constant) and
\bas
\beta_l
&\coloneqq
g_w \frac{\mathrm{i} \sigma_l}{2}
\in \mathrm{su}(2), \quad l \in \{1,2,3\}, \\
\beta_4
&\coloneqq
g^\prime \frac{\mathrm{i}}{2 n_\gamma}
\in \mathrm{u}(1),
\eas
where the $\sigma_l$ are the Pauli matrices
\bas
\sigma_1
&\coloneqq
\begin{pmatrix}
0 & 1 \\
1 & 0
\end{pmatrix}, &
\sigma_2
&\coloneqq
\begin{pmatrix}
0 & -\mathrm{i} \\
\mathrm{i} & 0
\end{pmatrix}, &
\sigma_3
&\coloneqq
\begin{pmatrix}
1 & 0 \\
0 & -1
\end{pmatrix}.
\eas
Writing $\mathbb{C}^2 \ni \omega \coloneqq \begin{pmatrix} \omega^1 \\ \omega^2 \end{pmatrix} = \begin{pmatrix} x^1 + \mathrm{i} x^2 \\ x^3 + \mathrm{i} x^4 \end{pmatrix} \cong \begin{pmatrix} x^1 \\ x^2 \\ x^3 \\ x^4 \end{pmatrix}$ and denoting the coordinate vector fields for the $\mleft(x^i\mright)_i$ by $\partial_i$, the Lie algebra action $\gamma$ is then defined by 
\bas
\gamma\mleft(\beta_1\mright)_\omega
&\coloneqq
\frac{g_w}{2} ~ \mleft.\mleft( x^4 \partial_1 - x^3 \partial_2 + x^2 \partial_3 - x^1 \partial_4 \mright)\mright|_\omega, 
\\
\gamma\mleft(\beta_2\mright)_\omega
&\coloneqq
\frac{g_w}{2} ~ \mleft.\mleft( -x^3 \partial_1 - x^4 \partial_2 + x^1 \partial_3 + x^2 \partial_4 \mright)\mright|_\omega, 
\\
\gamma\mleft(\beta_3\mright)_\omega
&\coloneqq
\frac{g_w}{2} ~ \mleft.\mleft( x^2 \partial_1 - x^1 \partial_2 - x^4 \partial_3 + x^3 \partial_4 \mright)\mright|_\omega, 
\\
\gamma\mleft(\beta_4\mright)_\omega
&\coloneqq
\frac{g^\prime}{2} ~ \mleft.\mleft( x^2 \partial_1 - x^1 \partial_2 + x^4 \partial_3 - x^3 \partial_4 \mright)\mright|_\omega,
\eas
which is induced by the Lie algebra representation introduced in Ex.~\ref{ex:electroweakinteractionasLiealg}, hence, it defines an action Lie algebroid.
\end{examples}

Let us conclude this section by revisiting the isotropy introduced in Section \ref{IsotropyClassical}. In order to do so it is useful to start with action Lie algebroids $E = N \times \mathfrak{g} \to N$ related to a Lie algebra $\mathfrak{g}$ action $\gamma$ on a smooth manifold $N$. By Def.~\ref{def:IsotropySubalgebra} the isotropy at $p \in N$ is given by the kernel of $\gamma$ with point evaluation at $p$. However, as we have seen, this is precisely the kernel of the anchor then at point $p$. Hence, we can immediately generalize the definition of isotropies.

\begin{definitions}{Isotropies of Lie algebroids, \newline \cite[\S 16.1, comment after the remark on page 113]{DaSilva}}{IsotropyForLieAlgeoids}
Let $E \to N$ be a Lie algebroid over a smooth manifold $N$. Then the \textbf{isotropy of $E$} is defined as the kernel of the anchor $\rho$, $\mathrm{Ker}(\rho)$.
\end{definitions}

Recall the discussion after Cor.~\ref{cor:IsotropyVonLieAlgMitAdjoint}, the isotropy at a point is in general not an ideal of $\mathfrak{g}$, however, the isotropy as a kernel of the anchor is an ideal of $E$ in the sense of
\bas
\rho\bigl(\mathrm{ad}(\nu)\bigr)
&=
\rho\mleft(
	\mleft[ \nu, \cdot \mright]_E
\mright)
=
0
\eas
for all $\nu \in \Gamma(E)$ with $\rho(\nu) = 0$, using that $\rho$ is a homomorphism of Lie brackets; one can generalize this of course to open subsets of $N$. The Leibniz rule in $\mleft[ \cdot, \cdot \mright]_E$ is basically canceling the failure of being an ideal as it happened in the discussion after Cor.~\ref{cor:IsotropyVonLieAlgMitAdjoint}. Also observe that
\bas
\mleft.\mleft[ \nu, f\mu \mright]_E\mright|_p
&=
f(p) ~ \mleft.\mleft[ \nu, \mu \mright]_E\mright|_p
	+ \underbrace{\mathcal{L}_{\rho(\nu)_p}(f)}_{=0} ~ \mu_p
\eas
for all $f \in C^\infty(N)$ and $\nu, \mu \in \Gamma(E)$ such that $\rho(\nu)_p = 0$ at a fixed point $p \in N$. Hence, the Lie bracket becomes tensorial if restricted onto sections with values in the isotropy (at a point), therefore it is then a typical Lie bracket and it restricts onto each fibre such that $\mathrm{Ker}(\rho_p)$ is a Lie algebra at each point $p \in N$, as also argued in \cite[\S 16.1, comment after the remark on page 113]{DaSilva}. However, the dimension of the isotropy is in general not constant which is why the isotropy is in general not a bundle of Lie algebras; simply take an action Lie algebroid as in Ex.~\ref{ex:electroweakinteractionasLiealgoid}, especially the action is induced by a Lie algebra representation on a vector space $N = W$. The isotropy at $0 \in W$ is then always the full Lie algebra while aside that this is in general of course not the case; we called this symmetry breaking, recall the discussion after Def.~\ref{def:ClassicYMHLagrangian}. 

If the anchor is always zero, then the rank of the isotropy is constant and equals the ranks of $E$. Hence, a Lie algebroid with zero anchor is a bundle of Lie algebras, as also argued in \cite[second example in \S 16.2; page 114]{DaSilva}.

In general, the anchor gives rise to a singular foliation on $N$ due to that it is a homomorphism of Lie brackets; we will discuss this later. Let us first turn very shortly to morphisms and then to Lie algebroid connections.

\section{Morphism of Lie algebroids}\label{MorphsOfLieOids}

It is of course a natural question what a morphism of Lie algebroids is; we will only need the easier definition of morphisms for Lie algebroids over the same base, which is straightforward to formulate.

\begin{definitions}{Base-preserving morphism of Lie algebroids, \newline \cite[\S 3.3, second part of Definition 3.3.1; page 100]{mackenzieGeneralTheory}}{BasePreservingMorphismOfLieAlgebroids}
Let $\mleft( E_1, \rho_{E_1}, \mleft[ \cdot, \cdot \mright]_{E_1} \mright)$ and $\mleft( E_2, \rho_{E_2}, \mleft[ \cdot, \cdot \mright]_{E_2} \mright)$ be two Lie algebroids over the same base manifold $N$. Then a \textbf{morphism of Lie algebroids $\phi: E_1 \to E_2$ over $N$}, or a \textbf{base-preserving morphism of Lie algebroids}, is a vector bundle morphism with
\bas
\rho_{E_2} \circ \phi &= \rho_{E_1}, \\
\phi\mleft( \mleft[ \mu, \nu \mright]_{E_1} \mright) &= \mleft[ \phi(\mu), \phi(\nu) \mright]_{E_2}
\eas
for all $\mu, \nu \in \Gamma(E_1)$.

When $\phi$ is additionally an isomorphism of vector bundles then we call it an \textbf{isomorphism of Lie algebroids over $N$}, or a \textbf{base-preserving isomorphism of Lie algebroids}.
\end{definitions}

\begin{remark}
\leavevmode\newline
For a Lie algebroid $E \to N$ over a smooth manifold $N$ its anchor $\rho$ is therefore also a Lie algebroid morphism $E \to \mathrm{T}N$; recall Remark \ref{rem:AnchorAHomom}.

The first condition is actually the same as for anchor-preservation for morphisms over the same base; recall the second point in Remark \ref{rem:SomeExtraNotationForAnchorBundleMorphs}.
\end{remark}

There is also a definition of morphisms for Lie algebroids over different bases, but we will not need it which is why we are going to omit its definition; see \textit{e.g.}~\cite[\S 7]{meinrenkenlie}.

We want to introduce connections as anchor-preserving morphisms; flatness is then equivalent to say that connections are morphisms of Lie algebroids. In order to define connections like that we need to introduce the derivations on vector bundles.
 %For two Lie algebroids $E_i \to N_i$ ($i \in \{1,2\}$) over smooth manifolds $N_i$, the idea is to use that a vector bundle morphism $\phi$ over a smooth map $f$ induces a map $\Gamma(E_1) \to \Gamma(f^*E_2)$ as we have discussed before. $f^*$Then replace the previously-used 

\section{\texorpdfstring{Derivations on vector bundles $V$}{Derivations on a vector bundle}}\label{DerivationsOnvector}

In Chapter \ref{ClassicGaugeTheory} we defined Lie algebra connections to define infinitesimal gauge transformations. Let us now start to reintroduce that concept for Lie algebroids, going towards Lie algebroid connections, generalizing typical vector bundle connections.

Moreover, we want to view connections slightly different, as a certain morphism of Lie algebroids. Before we can do this we need to introduce the Lie algebroid of derivations now, which have a relationship to certain vector fields known as \textbf{linear vector fields} on a vector bundle. The following constructions are motivated by \cite[Example 3.3.4; page 102f.; and \S 3.4; page 110ff.]{mackenzieGeneralTheory}.

\begin{definitions}{Derivations on a vector bundle at a fixed point, \newline \cite[variation of Example 3.3.4, page 102f.]{mackenzieGeneralTheory}}{DifferentialOperatorsOfLieAlgebroids}
Let $V \to N$ be a vector bundle over a smooth manifold $N$ and $p \in N$; the fibre of $V$ at $p$ we denote with $V_p$. Then a \textbf{derivation on $V$ at $p$} is an $\mathbb{R}$-linear map $L: \Gamma(V) \to V_p$ for which there exists a tangent vector $a_p(L) \in \mathrm{T}_pN$ such that
\ba
L(fv)
&=
f(p) ~ L(v) + \mathcal{L}_{a_p(L)}(f) ~ v_p
\ea
for all $f \in C^\infty(N)$ and $v \in \Gamma(V)$. We say that $L$ lifts $a_p(L)$.

We define the space of all derivations on $V$ at $p$ by
\ba
\gls{DApV}
&\coloneqq
\left\{ L: \Gamma(V) \to V_p ~ \middle| ~ L \text{ a derivation on $V$ at $p$} \right\}.
\ea
%
%Similarly, a \textbf{derivation on $V$} is defined as an $\mathbb{R}$-linear map $D: \Gamma(V) \to \Gamma(V)$ for which there exists a vector field $a(D) \in \mathfrak{X}(N)$ such that
%\bas
%D(fv)
%&=
%f D(v) + \mathcal{L}_{a(D)}(f) ~ v
%\eas
%for all $f \in C^\infty(N)$ and $v \in \Gamma(V)$.
\end{definitions}

\begin{remark} \label{RemDerivationsatapointarelocally}
\leavevmode\newline
It is clear that $\mathcal{D}_p(V)$ is a vector space, where the zero element is just the zero map with $a_p(0) = 0$, and all $L \in \mathcal{D}_p(V)$ can be restricted to open subsets $U$ around $p$ with the typical arguments.
%
%Take two sections $v, v^\prime \in \Gamma(V)$ with $v|_U = v^\prime|_U$, then take another open neighbourhood $U^\prime$ around $p$ such that the closure $\overline{U^\prime}$ in $N$ is a subset of $U$. Use any bump function $f \in C^\infty(N)$ with $f|_{U^\prime} \equiv 1$ and whose support lies in $U$, \textit{i.e.} $f(q) = 0$ for all $q \notin U$.\footnote{The existence of $U^\prime$ and $f$ will be explained later.} Then clearly $f ~ \mleft(v-v^\prime\mright) = 0$ and
%\bas
%0
%&=
%L\mleft(f ~ \mleft(v-v^\prime\mright)\mright)
%=
%\underbrace{f(p)}_{=1} ~ L\mleft(v-v^\prime\mright) + \mathcal{L}_{a_p(L)}(f) ~ \underbrace{\mleft(v_p - v^\prime_p\mright)}_{=0}
%=
%L(v) - L(v^\prime).
%\eas
%Hence, $L(v) = L(v^\prime)$, and so $L$ can be restricted to an operator just acting on $\Gamma(V|_U)$ by
%\bas
%L|_U(w) &\coloneqq L(f w)
%\eas
%for all $w \in \Gamma(V|_U)$, where we view $fw$ as a section on $V$ extended by zero outside of $U$. This is independent of the choice of bump function by the calculation above: For any other open subset $\widetilde{U}$ around $p$ with $\overline{\widetilde{U}} \subset U$ (the closure taken in $N$) and a bump function $g$ with $g|_{\widetilde{U}} \equiv 1$ and support in $U$ we would get 
%\bas
%f|_{U^\prime \cap \widetilde{U}} &\equiv 1 \equiv g|_{U^\prime \cap \widetilde{U}}
%\eas
%and, thence, $\mleft(f w\mright)|_{U^\prime \cap \widetilde{U}} = \mleft(g w\mright)|_{U^\prime \cap \widetilde{U}}$ such that
%\bas
%L(f w) &= L(g w).
%\eas
%Especially, this also holds for $g = (f)^2$. Thence, this also clearly defines a derivation on $V|_U$ at $p$, for this observe
%\bas
%L|_U(h w)
%&=
%L(fhw)
%=
%L(fh ~ fw) \\
%&=
%\underbrace{f(p)}_{=1} h(p) ~ L(fw) + \underbrace{\mathcal{L}_{a_p(L)}(fh)}_{\mathclap{= f(p) ~ \mathcal{L}_{a_p(L)}(h) + \mathcal{L}_{a_p(L)}(f) ~ h(p)}} ~ f(p) ~ w_p \\
%&=
%h(p) ~ L|_U(w)
	%+ \mathcal{L}_{a_p(L)}(h) ~ w_p
%\eas
%for all $h \in C^\infty(U)$, where $fh$ is viewed as a smooth function on $N$, extended by zero. Here we used that the action of vector fields is local around $p$, therefore also $\mathcal{L}_{a_p(L)}(f) = 0$ due to the fact that $f|_{U^\prime} \equiv 1$.
%
%\textit{\textbf{About the existence of $U^\prime$:}} Take \textit{e.g.} any open subset $V$ around $p$ on which some coordinate function $\psi$ is defined. Then $\psi\mleft( V \cap U \mright)$ is open in $\mathbb{R}^{\mathrm{dim}(N)}$, and for $\mathbb{R}^{\mathrm{dim}(N)}$ we know that there is an open subset $W$ around $\psi(p)$ with $\overline{W} \subset \psi\mleft( V \cap U \mright)$. Then define $U^\prime \coloneqq \psi^{-1} \mleft( W \mright)$ which is open by continuity of $\psi$. We then have $\overline{\psi^{-1} \mleft( W \mright)} \subset \psi^{-1} \mleft( \overline{W} \mright)$ because $\psi^{-1} \mleft( \overline{W} \mright)$ is closed by continuity of $\psi$ and closedness of $\overline{W}$. Thence, $\overline{U^\prime} \subset V \cap U \subset U$.
%
%\textit{\textbf{About the existence of $f$:}} Simply take a partition of unity subordinate to the open covering of $U$ and $N \setminus \overline{U^\prime}$, \textit{i.e.} we have two smooth functions $\psi_U$ and $\psi_{N \setminus \overline{U^\prime}}$, whose supports lie in $U$ and $N \setminus \overline{U^\prime}$, respectively, such that $1 \equiv \psi_U + \psi_{N \setminus \overline{U^\prime}}$. Especially,
%\bas
%1 = \psi_U(q)
%\eas
%for all $q \in U^\prime$. Therefore define $f \coloneqq \psi_U$.
\end{remark}

Our aim is to show that the disjoint union $\mathcal{D}(V)$ of $\mathcal{D}_p(V)$ admits a vector bundle structure and even forms a Lie algebroid. Its sections have then the following form, formally already denoted by $\Gamma(\mathcal{D}(V))$.

\begin{definitions}{Derivations on a vector bundle $V$, \newline \cite[Example 3.3.4; page 102f.]{mackenzieGeneralTheory}}{DerivationsOnV}
Let $V \to N$ be a vector bundle over a smooth manifold $N$. Then a \textbf{derivation on $V$} is an $\mathbb{R}$-linear map $\mathcal{T}: \Gamma(V) \to \Gamma(V)$ such that there is a smooth vector field $a\mleft( \mathcal{T} \mright) \in \mathfrak{X}(N)$ with
\ba\label{eqDerivationsLiftASuperDuperVectorField}
\mathcal{T}(fv)
&=
f ~ \mathcal{T}(v)
	+ \mathcal{L}_{a\mleft( \mathcal{T} \mright)}(f) ~ v
\ea
for all $f \in C^\infty(N)$ and $v \in \Gamma(V)$. We say that $\mathcal{T}$ lifts $a(\mathcal{T})$.

We define the space of all derivations on $V$ by
\ba
\Gamma(\mathcal{D}(V))
&\coloneqq
\left\{
\mathcal{T}: \Gamma(V) \to \Gamma(V) ~ \middle| ~
\mathcal{T} \text{ a derivation on } V
\right\}.
\ea
\end{definitions}

\begin{remark}
\leavevmode\newline
%This definition of course also applies to $V|_U$ where $U$ is some open subset of $N$. One could also write $\Gamma(\mathcal{D}(U)) \coloneqq \Gamma(\mathcal{D}(V|_U))$.
%
It is clear that $\sEnd(V) \subset \Gamma(\mathcal{D}(V))$ with $a(A) \equiv 0$ for all $A \in \sEnd(V)$, and that $\Gamma(\mathcal{D}(V))$ is a $C^\infty(N)$-module.
\end{remark}

The following result can be seen as a generalization of the section around Remark \ref{RemTVGleichV}.
 %recall also the definition of $\overline{\mathrm{End}}$ mentioned there. 

\begin{propositions}{Isomorphisms of the space of derivations of $V$ at $p$, \newline \cite[Example 3.10]{basicconn}}{IsomorphismofDerivationonVectorbundleatabasepoint}
Let $V \to N$ be a real vector bundle with non-zero finite rank and $p \in N$ whose fiber we denote with $V_p$. Then each vector bundle connection $\nabla$ on $V$ induces a vector space isomorphism
\ba
\mathcal{D}_p(V)
&\cong
\mathrm{T}_pN \oplus \mathrm{End}(V_p)
\ea
%and
%\ba
%\mathcal{D}_p(V)
%&\cong
%\mathrm{T}_pN \oplus \overline{\mathrm{End}(V_p)}
%\subset
%\mathrm{T}_pN \oplus \mathfrak{X}(V_p).
%\ea
Under such isomorphisms $a_p: \mathcal{D}_p(V) \to \mathrm{T}_pN$, $L \mapsto a_p(L)$ is the projection onto the first factor.
\end{propositions}

\begin{remark}
\leavevmode\newline
The last statement shows why we say that $a_p(L)$ is lifted by $L \in \mathcal{D}_p(V)$.
\end{remark}

\begin{proof}
\leavevmode\newline
Define $T: \mathrm{T}_pN \oplus \mathrm{End}(V_p) \to \mathcal{D}_p(V)$ by
\ba 
(X, A) &\mapsto T(X, A), 
\nonumber\\ \label{EqFibrewisesupderduperisomorphismofderivations}
\mleft( T(X, A) \mright) (v) 
&\coloneqq T(X, A)(v) \coloneqq 
\nabla_X v|_p + A(v_p)
\ea
for all $v \in \Gamma(V)$. $T$ is clearly bilinear, and $T(X,A)$ clearly defines a derivation at $p$. For injectivity, observe
\bas
\nabla_X
&=
-A,
\eas
for all $(X, A)$ in the kernel of $T$, which is clearly a contradiction to the Leibniz rule in $\nabla_X$ when $X \neq 0$ due to the fact that $V$ has a non-zero rank. Thus, for such $(X,A)$, $X=0$ and then clearly also $A=0$; so, injectivity is given.

For surjectivity observe for all $L \in \mathcal{D}_p(V)$, 
%using a local frame $\mleft( e_a \mright)_a$ of $V$,
\bas
L(v)
&=
\mleft.\nabla_{a_p(L)} v\mright|_p
	+ L(v) - \mleft.\nabla_{a_p(L)} v\mright|_p
\eas
hence, use $X \coloneqq a_p(L) \in \mathrm{T}_pN$ and define $A \coloneqq L - \nabla_{a_p(L)}$, which is clearly an element of $\mathrm{End}(V_p)$. Hence, $T$ is surjective, too.

That $a_p$ is under such an isomorphism the projection onto the first factor is clear by construction.
%
%The last isomorphism comes by using Cor. \ref{CorEndVGleichBarEndV}, but just using the underlying vector space isomorphism because we have not defined any Lie bracket on $\mathcal{D}_p(V)$. That $a_p$ is the projection onto the first factor is clear by construction, as argued, $X = a_p(L)$, and that is independent of the choice of frame in the definition of $T$.
\end{proof}

%\begin{remark}
%\leavevmode\newline
%Recall the isomorphism mentioned in Cor. \ref{CorEndVGleichBarEndV} such that we can now combine it with the isomorphism of the proof above to explicitly construct $\mathcal{D}_p(V) \cong \mathrm{T}_pN \oplus \overline{\mathrm{End}(V_p)}$, so, we are now going to use exactly the same notation as in the previous proof and as in Cor. \ref{CorEndVGleichBarEndV}. Take for all $x \in V_p$ for a fixed $p \in N$ a (local) section $v \in \Gamma(V)$ with $v_p = x$, and then observe that
%\bas
%T(v)
%&\stackrel{\text{Prop. \ref{PropIsomorphismofDerivationonVectorbundleatabasepoint}}}{\mapsto}
%\mleft(a_p(T), \overline{A}_{v_p} \mright)
%\in
%\mathrm{T}_pN \oplus \mathrm{T}_{x}V_p
%\eas
%for all $T \in \mathcal{D}_p(V)$
%\end{remark}

Trivially extending that isomorphism to all $p \in N$, leads to a canonical vector bundle structure inherited by the Whitney sum $\mathrm{T}N \oplus \mathrm{End}(V)$.
%
%\begin{lemma}[Derivations as a sheaf of modules over $C^\infty$] \cite[variation of Example 3.3.4, page 102f.]{mackenzieGeneralTheory} \label{LemmaDerivationsASheafOfModules}
%\leavevmode\newline
%Let $V$ be a real vector bundle with non-zero rank. Then the space of derivations describe a locally free sheaf of modules of rank $\mathrm{dim}(N)+\mleft( \mathrm{rank}(V) \mright)^2$ over the sheaf of rings $C^\infty$, and every element $\mathcal{T}\in\Gamma(\mathcal{D}(V))$ restricts onto an element of $\mathcal{D}_p(V)$ by point evaluation at $p \in N$, lifting $a(\mathcal{T})|_p$; similarly for $\Gamma(\mathcal{D}(V|_U))$ for all open subsets $U$ with $p \in U$.
%
%Moreover, around each $p$ we can find an open neighbourhood $U$ with
%\ba
%\mathfrak{X}(U) \oplus \sEnd\mleft(\mleft.V\mright|_U\mright) &\cong \Gamma(\mathcal{D}(\mleft.V\mright|_U))
%\ea
%as modules over $C^\infty(U)$, under this isomorphism $a: \Gamma(\mathcal{D}(V)) \to \mathfrak{X}(N)$ of Def. \ref{def:DerivationsOnV} acts as projection onto the first factor.
%\end{lemma}
%
%\begin{remark}
%\leavevmode\newline
%As for $\mathcal{D}_p(V)$, the last statement motivates why a derivation $\mathcal{T}$ on $V$ lifts $a(\mathcal{T})$, see also \cite[beginning of \S 2; $\Gamma(\mathcal{D}(V))$ is there denoted as $\saut(E)$]{meinrenkensplitting}.
%
%The discussion for zero-rank vector bundles is trivial since then $\Gamma(\mathcal{D}(V))=\{0\}$, and then it is clear that $\Gamma(\mathcal{D}(V))$ is just the space of sections of the unique zero rank vector bundle with fibers $\mathcal{D}_p(V) = \{0\}$ for $p \in N$.
%\end{remark}
%
%\begin{proof}
%\leavevmode\newline
%\indent$\bullet$ $\Gamma(\mathcal{D}(V))$ is clearly a module over the ring of smooth functions $C^\infty(N)$. Let $p \in N$ and we denote with $\mathrm{ev}_p: \Gamma(V) \to V_p, v \mapsto v_p$, the point evaluation of sections, then $\mathrm{ev}_p \circ \mathcal{T}: \Gamma(V) \to V_p$ for all $\mathcal{T} \in \Gamma(\mathcal{D}(V))$, and
%\bas
%\mleft(\mathrm{ev}_p \circ \mathcal{T}\mright)(fv)
%&=
%f(p) ~ \mleft(\mathrm{ev}_p \circ \mathcal{T}\mright)(v)
	%+ \mathcal{L}_{\mleft. a(\mathcal{F}) \mright|_p}(f) ~ v_p
%\eas
%for all $f \in C^\infty(N)$ and $v \in \Gamma(V)$. Thence, $\mathrm{ev}_p \circ \mathcal{T}$ defines a derivation on $V$ at $p$ with $a_p\mleft(\mathrm{ev}_p \circ \mathcal{T}\mright) = \mleft. a(\mathcal{T}) \mright|_p$. That every element of $\mathcal{D}_p(V)$ is reached will be seen at the end of this proof.
%
%$\bullet$ That also implies that the argument of Rem. \ref{RemDerivationsatapointarelocally} can be applied here, too, \textit{i.e.}, there is a restriction $\mathcal{T}|_U: \Gamma(V|_U) \to \Gamma(V|_U)$ as a derivation on $V|_U$ with $a(\mathcal{T}|_U) = a(\mathcal{T})|_U \in \mathfrak{X}(U)$ over some open subset $U \subset N$, defining $\mathcal{T}|_U(v)$ point-wise using bump functions as in Rem. \ref{RemDerivationsatapointarelocally} for all $v \in \Gamma(V|_U)$. That shows that the space of derivations describe a sheaf of modules over $C^\infty$.
%
%$\bullet$ The rank is given by Prop. \ref{PropIsomorphismofDerivationonVectorbundleatabasepoint} by extending the isomorphism given there: Take an open subset $U$ of $N$ such that we have a local frame $\mleft( e_a \mright)_a$ of $V$ and coordinates $\partial_i$ on $N$. Then the isomorphism of Prop. \ref{PropIsomorphismofDerivationonVectorbundleatabasepoint} extends to an isomorphism $T$
%\ba
%\mathfrak{X}(U) \oplus \sEnd\mleft(\mleft.V\mright|_U\mright) &\to \Gamma(\mathcal{D}(\mleft.V\mright|_U)), \nonumber\\
%(X,A) &\mapsto T(X, A), T(X, A)(v) \coloneqq \mathcal{L}_X(v^a) ~ e_a + A(v) \label{DefSuperCoolerIsomorphismus}
%\ea
%for all $v = v^a e_a \in \Gamma(V|_U)$. That this is a vector space isomorphism is exactly the same proof as in Prop. \ref{PropIsomorphismofDerivationonVectorbundleatabasepoint}. Moreover, this is also clearly an isomorphism of modules over $C^\infty(U)$ since this function is also $C^\infty$-linear. Similar to the argument of the proof of Prop. \ref{PropIsomorphismofDerivationonVectorbundleatabasepoint} $a$ will be the projection onto the first factor.
%
%$\bullet$ The isomorphism \eqref{DefSuperCoolerIsomorphismus} restricts to the isomorphism \eqref{EqFibrewisesupderduperisomorphismofderivations} by point evaluation, and that shows that $\Gamma(\mathcal{D}(V|_U))$ projects onto $\mathcal{D}_p(V)$ by point evaluation for all $p \in N$. Now fix $p$ and take an open subset $U^\prime$ such that its closure $\overline{U^\prime}$ in $N$ is a subset of $U$ and $p \in U^\prime$, and take a bump function $f \in C^\infty(N)$ with support on $U$ and $f\equiv 1$ on $U^\prime$; see Remark \ref{RemDerivationsatapointarelocally} for their existence. Then take any $\mathcal{T} \in \Gamma(\mathcal{D}(V|_U))$ projecting onto some $T \in \mathcal{D}_p(V)$ by point evaluation. Then $f \mathcal{T} \in \Gamma(\mathcal{D}(V))$ (extended by zero outside of $U$) with $\mleft( f \mathcal{T} \mright)|_{U^\prime} = \mathcal{T}|_{U^\prime}$. Thus, we have an element of $\Gamma(\mathcal{D}(V))$ projecting onto $T$ by point evaluation.
%%Then we can extend every $L \in \mathcal{D}_p(V)$ for all $p \in U^\prime$ by first writing it as
%%\bas
%%\mathcal{L}_{X_p}(v^a) ~ \mleft.e_a\mright|_p + A_p(v_p)
%%\eas
%%for all $v \in \Gamma(V)$ and for some $(X_p, A_p) \in \mathrm{T}_pN \oplus \mathrm{End}(V_p)$ (see \eqref{EqFibrewisesupderduperisomorphismofderivations}), and then extend $(X_p, A_p)$ constantly to $(X, A) \in \mathfrak{X}(U) \times \sEnd(V|_U)$; identify $(X,A)$
%\end{proof}
%
%With that the vector bundle structure is given.

\begin{lemmata}{Vector bundle of derivations, \newline \cite[variation of the introduction in Example 3.3.4, page 102f.]{mackenzieGeneralTheory} and \cite[Example 3.10]{basicconn}}{LemmaVectorbundlestructureofDV}
Let $V \to N$ be a real vector bundle with non-zero rank. Then there is a unique vector bundle structure on $\mathcal{D}(V) \coloneqq \coprod_{p \in N} \mathcal{D}_p(V)$ such that $\Gamma(\mathcal{D}(V))$ of Def.~\ref{def:DerivationsOnV} is its space of smooth sections, where $\coprod$ is the disjoint union of sets.

Moreover, each connection $\nabla$ on $V$ defines a vector bundle isomorphism
\ba
\mathcal{D}(V) &\cong \mathrm{T}N \oplus \mathrm{End}(V), 
\ea
where $\mathrm{T}N \oplus \mathrm{End}(V)$ is the Whitney sum of vector bundles.
\end{lemmata}

\begin{proof}
\leavevmode\newline
This follows by Prop.~\ref{prop:IsomorphismofDerivationonVectorbundleatabasepoint}: Given a connection $\nabla$, we can define an isomorphism $T: \mathfrak{X}(N) \oplus \sEnd(V) \to \Gamma(\mathcal{D}(V))$ of $C^\infty(N)$-modules 
\ba
T(X, A)
&\coloneqq
\nabla_X + A
\ea
for all $(X, A) \in \mathfrak{X}(N) \oplus \sEnd(V)$. This shows that $\Gamma(\mathcal{D}(V))$ is a locally free sheaf of modules of constant rank, and it restricts to $\mathcal{D}_p(V)$ at $p \in N$ because $T$ restricts to the isomorphism of Prop.~\ref{prop:IsomorphismofDerivationonVectorbundleatabasepoint}. Then we make use of the 1:1 correspondence of vector bundles and locally free sheaf of modules of constant rank (over a sheaf of rings coming from a ringed space), which implies a unique vector bundle structure on $\mathcal{D}(V) \coloneqq \coprod_{p \in N} \mathcal{D}_p(V)$ such that $\Gamma(\mathcal{D}(V))$ is its space of smooth sections. Since $T$ is clearly $C^\infty(N)$-linear, we also have an isomorphism of vector bundles $\mathcal{D}(V) \cong \mathrm{T}N \oplus \mathrm{End}(V)$ by $T$.
\end{proof}

This leads to the following definitions.

\begin{definitions}{The bundle of derivations, \newline\cite[variation of Example 3.3.4, page 102f.]{mackenzieGeneralTheory}}{LieAlgebroidOfDerivations}
Let $V \to N$ be a real vector bundle with finite rank. Then we define the \textbf{bundle of derivations on $V$} as the vector bundle $\gls{DAV}$ equipped with the vector bundle structure of Lemma \ref{lem:LemmaVectorbundlestructureofDV}, assuming that the rank of $V$ is non-zero; if the rank is zero, then we define $\mathcal{D}(V) \coloneqq N \times \{0\}$.
\end{definitions}

%\begin{remark}
%\leavevmode\newline
%Sections of $\mathcal{D}(V)$
%\end{remark}

\begin{propositions}{Lie algebroid structure on $\mathcal{D}(V)$,\newline \cite[Example 3.3.4, page 102f.]{mackenzieGeneralTheory}}{LieAlgebroidOfDerivationOnV}
%\leavevmode\newline
Let $V \to N$ be a real vector bundle. $\mathcal{D}(V)$ together with $\gls{a1}$ defined by
\ba
\mathcal{D}(V) &\to \mathrm{T}N, \\
\mathcal{D}_p(V) \ni D &\mapsto a(D) \coloneqq a_p(D),
\ea
and $\mleft[ \cdot, \cdot \mright]_{\mathcal{D}(V)}$, defined by
\ba
\Gamma(\mathcal{D}(V)) \times \Gamma(\mathcal{D}(V)) &\to \Gamma(\mathcal{D}(V)), \\
(\mathcal{T}_1, \mathcal{T}_2) &\mapsto \mleft[ \mathcal{T}_1, \mathcal{T}_2 \mright]_{\mathcal{D}(V)}
\coloneqq
\mathcal{T}_1 \circ \mathcal{T}_2 - \mathcal{T}_2 \circ \mathcal{T}_1,
\ea
is a Lie algebroid with anchor $a$ and Lie bracket $\mleft[ \cdot, \cdot \mright]_{\mathcal{D}(V)}$. The anchor extended on sections is exactly the same $a$ as in Def. \ref{def:DerivationsOnV}.
\end{propositions}

\begin{remark}
\leavevmode\newline
By Prop.~\ref{prop:IsomorphismofDerivationonVectorbundleatabasepoint}, $\mathcal{D}(V)$ is also transitive.
\end{remark}

\begin{proof}
\leavevmode\newline
For $p \in N$ and for all $f \in C^\infty(N)$, $v \in \Gamma(V)$, $\alpha, \beta \in \mathbb{R}$ and $D_1, D_2 \in \mathcal{D}_p(V)$ we have
\bas
\mleft( \alpha D_1 + \beta D_2 \mright)(fv)
&=
f(p) ~ \mleft( \alpha D_1 + \beta D_2 \mright)(v) + \mathcal{L}_{a_p(\alpha D_1 + \beta D_2)}(f) ~ v_p
\eas
and
\bas
\mleft( \alpha D_1 + \beta D_2 \mright)(fv)
&=
\alpha D_1 (fv)
	+ \beta D_2 (fv) \\
&=
f(p) ~ \mleft( \alpha D_1 + \beta D_2 \mright)(v)
	+ \mleft( \alpha \mathcal{L}_{a_p(D_1)}(f) + \beta \mathcal{L}_{a_p(D_2)}(f) \mright) ~ v_p \\
&=
f(p) ~ \mleft( \alpha D_1 + \beta D_2 \mright)(v)
	+ \mathcal{L}_{\alpha a_p(D_1) + \beta a_p(D_2)}(f) ~ v_p
\eas
and, hence,
\bas
\mathcal{L}_{\alpha a_p(D_1) + \beta a_p(D_2)}(f) ~ v_p 
&=
\mathcal{L}_{a_p(\alpha D_1 + \beta D_2)}(f) ~ v_p.
\eas
For a non-zero rank we can therefore conclude
\bas
\alpha a_p(D_1) + \beta a_p(D_2)
&=
a_p(\alpha D_1 + \beta D_2).
\eas
That means that $a$ extends on sections, which gives the $a$ given in Def. \ref{def:DerivationsOnV} on sections by Lemma \ref{lem:LemmaVectorbundlestructureofDV} ($\Rightarrow$ Def. \ref{def:DerivationsOnV} gives the sections of $\mathcal{D}(V)$ $\Rightarrow$ point evaluation at $p$ of $\mathcal{T} \in \Gamma(\mathcal{D}(V))$ gives a derivation of $V$ at $p$ lifting the tangent vector $a(\mathcal{T})|_p$ which we therefore identify as $a_p(\mathcal{T}_p)$). While all of that is trivial for zero rank since then $a \equiv 0$.
%
%Now take two sections $\mathcal{T}_1, \mathcal{T}_2 \in \Gamma(\mathcal{D}(V))$, at each point $p \in N$ they describe a derivation on $V$ at $p$ which are projected onto some tangent vector at $p$, $a_p\mleft(\mleft.\mathcal{T}_1\mright|_p\mright)$ and $a_p\mleft(\mleft.\mathcal{T}_2\mright|_p\mright)$. Since $a$ is a smooth vector bundle morphism we can conclude that $\mathfrak{X}(N) \ni a\mleft(\mathcal{T}_i\mright), p \mapsto a_p\mleft( \mathcal{T}_i \mright)$, and
%\ba
%\mathcal{T}_i(fv)
%&=
%f ~ \mathcal{T}_i(v)
	%+ \mathcal{L}_{a\mleft(\mathcal{T}_i\mright)}(f) ~ v\label{eqDerivationsLiftASuperDuperVectorField}
%\ea
%for all $f \in C^\infty(N)$, $v \in \Gamma(V)$ and $i \in \{1,2\}$.

That $\mleft[ \cdot, \cdot \mright]_{\mathcal{D}(V)}$ is a Lie bracket is clear since it is just the typical commutator of linear operators on a (infinite-dimensional) vector space. Thence, the only thing left is to show the Leibniz rule, which simply follows by
\bas
\mleft[ \mathcal{T}_1, f \mathcal{T}_2 \mright]_{\mathcal{D}(V)}(v)
&=
\mathcal{T}_1\mleft( f \mathcal{T}_2(v) \mright)
	- f ~ \mathcal{T}_2\mleft( \mathcal{T}_1(v) \mright)
\stackrel{\text{Eq. \eqref{eqDerivationsLiftASuperDuperVectorField}}}{=}
f ~ \mleft[ \mathcal{T}_1, \mathcal{T}_2 \mright]_{\mathcal{D}(V)}(v)
	+ \mathcal{L}_{a\mleft( \mathcal{T}_1 \mright)}(f) ~ \mathcal{T}_2(v)
\eas
for all $f \in C^\infty(N)$, $v \in \Gamma(V)$.
\end{proof}

As usual for differential operators, we will identify those derivations as certain vector fields, following \cite[beginning of \S 2; $\Gamma(\mathcal{D}(V))$ is there denoted as $\saut(E)$]{meinrenkensplitting} and \cite[\S 3.4 \textit{et seq.}; page 110ff.]{mackenzieGeneralTheory}. For the following recall that for each vector bundle $V \stackrel{\pi}{\to} N$ there is also a vector bundle structure for $\mathrm{T}V \stackrel{\mathrm{D}\pi}{\to} \mathrm{T}N$, and the following diagram describes a double vector bundle
\begin{center}
	\begin{tikzcd}
		 \mathrm{T}V \arrow{r}{\mathrm{D}\pi} \arrow{d}{\pi_{\mathrm{T}V}} & \mathrm{T}N \arrow{d}{\pi_{\mathrm{T}N}} \\
		V \arrow[r, "\pi"]& N
	\end{tikzcd}
\end{center}
that is, each horizontal and vertical line is a vector bundle, and the horizontal and vertical scalar multiplications on $\mathrm{T}V$ commute, see \textit{e.g.}~\cite[\S 3ff.]{Highervectorbundles}. Let us shortly recap the vector bundle structure of $\mathrm{T}V \stackrel{\mathrm{D}\pi}{\to} \mathrm{T}N$, following \cite[discussion at the beginning of \S 3.4; page 110ff.]{mackenzieGeneralTheory}: The linear structure at $v \in \mathrm{T}_p N$ ($p \in N$) is basically given by the vertical structure of $V$ prolonged along the fibre $V_p$, but as an affine space whose offset is given by $v$. That is, let $\xi, \eta \in \mathrm{T}V$ with 
\bas
\mathrm{D}_{\pi_{\mathrm{T}V}(\xi)}\pi(\xi)
&=
\mathrm{D}_{\pi_{\mathrm{T}V}(\eta)}\pi(\eta)
\eqqcolon
v,
\eas
and, hence, due to $\pi_{\mathrm{T}N}(v) = p$,
\bas
p
&= 
(\pi \circ \pi_{\mathrm{T}V})(\xi)
=
(\pi \circ \pi_{\mathrm{T}V})(\eta).
\eas
Thus, one can take curves $f,h: I \to V$ ($I \in \mathbb{R}$ an open interval around 0) with
\bas
f(0)
&=
\pi_{\mathrm{T}V}(\xi),
&
\mleft.\frac{\mathrm{d}}{\mathrm{d}t}\mright|_{t=0} f
&=
\xi,
\\
h(0)
&=
\pi_{\mathrm{T}V}(\eta),
&
\mleft.\frac{\mathrm{d}}{\mathrm{d}t}\mright|_{t=0} h
&=
\eta,
\eas
such that
\bas
\pi \circ f = \pi \circ h,
\eas
because the condition on $\xi$ and $\eta$ imply on the base paths $\pi \circ f, \pi \circ h: I \to N$ that
\bas
(\pi\circ f)(0)
&=
p
=
(\pi \circ h)(0),
\\
\mleft.\frac{\mathrm{d}}{\mathrm{d}t}\mright|_{t=0} \bigl( \pi \circ f \bigr)
&=
\mathrm{D}_{\pi_{\mathrm{T}V}(\xi)}(\xi)
=
\mathrm{D}_{\pi_{\mathrm{T}V}(\eta)}(\eta)
=
\mleft.\frac{\mathrm{d}}{\mathrm{d}t}\mright|_{t=0} \bigl( \pi \circ h \bigr).
\eas
%So, one takes one base path satisfying those, and then $f$ and $h$ are the lifts this base path, which is possible because $\xi$ and $\eta$ just differ along the vertical structures (base points in same fibre, and sharing the same projection under $\mathrm{D}\pi$).
Then the addition and scalar multiplication with $\lambda \in \mathbb{R}$ for $\mathrm{T}V \stackrel{\mathrm{D}\pi}{\to} \mathrm{T}N$ is defined by
\bas
\xi \RPlus \eta
&\coloneqq
\mleft.\frac{\mathrm{d}}{\mathrm{d}t}\mright|_{t=0} (f + h),
\\
\lambda \boldsymbol{\cdot} \xi
&\coloneqq
\mleft.\frac{\mathrm{d}}{\mathrm{d}t}\mright|_{t=0} (\lambda h),
\eas
where the addition of curves is well-defined because of $\pi \circ f = \pi \circ h$ which implies $\pi(f+h)= \pi(f) = \pi(h)$; so, one can take the sum of the curves and
\bas
\mathrm{D}\pi(\xi \RPlus \eta)
&=
\mleft.\frac{\mathrm{d}}{\mathrm{d}t}\mright|_{t=0} \bigl( \underbrace{\pi (f+h)}_{= \pi(f)} \bigr)
=
\mathrm{D}\pi(\xi)
=
v.
\eas
In other words, those operations come from interpreting tangent vectors as equivalence classes of curves, assuming there are representatives of the classes sharing the same base path ($\pi \circ f = \pi \circ h$) with which one can do those operations.
It is trivial to show that we have a double vector bundle. The operations of the linear structure in $\mathrm{T}V \stackrel{\pi_{\mathrm{T}V}}{\to} V$ is still denoted in the same manner as usual, and by definition one also gets
\bas
\pi_{\mathrm{T}V}(\xi \RPlus \eta)
&=
\pi_{\mathrm{T}V}(\xi)
	+ \pi_{\mathrm{T}V}(\eta),
\\
\pi_{\mathrm{T}V}(\lambda \boldsymbol{\cdot} \xi)
&=
\lambda ~ \pi_{\mathrm{T}V}(\xi).
\eas

\begin{definitions}{Linear vector fields, \cite[Definition 3.4.1; page 113]{mackenzieGeneralTheory}}{LinearVectorFieldsOnVectorBundles}
Let $V \stackrel{\pi}{\to} N$ be a vector bundle over a smooth manifold $N$. Then a \textbf{linear vector field on $V$} is a vector field $\xi \in \mathfrak{X}(V)$ which is also a vector bundle morphism $V \to \mathrm{T}V$ over a vector field $X \in \mathfrak{X}(N)$, \textit{i.e.}~on one hand the following diagram commutes
\begin{center}
	\begin{tikzcd}
		 V \arrow{r}{\xi} \arrow{d}{\pi} & \mathrm{T}V \arrow{d}{\mathrm{D}\pi} \\
		N \arrow[r, "X"]& \mathrm{T}N
	\end{tikzcd}
\end{center}
that is
\ba\label{LiftingVectorFieldsByLinearOnes}
\mathrm{D}\pi \circ \xi &= X \circ \pi = \pi^*X,
\ea
and on the other hand we have additionally
\ba\label{LinearityOfLinearVectorFields}
\xi_{\alpha x + \beta y}
&=
\alpha \boldsymbol{\cdot} \xi_x
	\RPlus \beta \boldsymbol{\cdot} \xi_y
\ea
for all $x, y \in V$ with $\pi(x) = \pi(y)$ and $\alpha, \beta \in \mathbb{R}$.

We say that \textbf{$\xi$ lifts $X$}.
\end{definitions}

\begin{remarks}{Coordinates on $\mathrm{T}V$}{CoordinateOnTangentStuffFOrLinearVectorFields}
As usual, vector fields are locally determined by their action on coordinate functions, that is, denote with $x^i$ coordinates on $N$, then coordinates on $V$ are given by $\pi^*x^i$ and $y^j$, where the latter are the fibre coordinates, given by a local trivialization, especially $y^j$ are (local) smooth and fibre-linear functions on $V$, elements of $\Gamma(V^*)$, whose set we denote by $C^\infty_{\mathrm{lin}}(V) \coloneqq \Gamma(V^*)$ as in \cite{mackenzieGeneralTheory}. That means that (linear) vector fields on $V$ are uniquely given by their action on $\pi^*C^\infty(N)$ and $C^\infty_{\mathrm{lin}}(V) \coloneqq \Gamma(V^*)$, we will emphasize this in the following proposition.
%
%Using a trivialisation, it is trivial to see that the coordinate vector fields of the defined coordinates are linear vector fields:
%\ba
%\frac{\partial}{\partial \mleft(\pi^*x^i\mright)}
%&\text{ linear and lifting }
%\frac{\partial}{\partial x^i},
%\\
%\frac{\partial}{\partial y^j}
%&\text{ linear and lifting }
%0.
%\ea
%For the former use curves parallel to the base manifold, and for the latter curves perpendicular to the base, using the trivialization induced by those coordinates.
\end{remarks}

The following proposition shows the idea behind the linear vector fields.

\begin{propositions}{Action of linear vector fields, \newline \cite[first two statements of Proposition 3.4.2; page 113f.]{mackenzieGeneralTheory}}{ActionOfLinearVecFields}
Let $V \stackrel{\pi}{\to} N$ be a vector bundle over a smooth manifold $N$, and $\xi \in \mathfrak{X}(V)$. Then $\xi$ is a linear vector field on $V$ if and only if $\xi\mleft( \pi^*C^\infty(N) \mright) \subset \pi^*C^\infty(N)$ and $\xi \mleft( C^\infty_{\mathrm{lin}}(V) \mright) \subset C^\infty_{\mathrm{lin}}(V)$.
\end{propositions}

\begin{proof}
\leavevmode\newline
\indent $\bullet$ We prove that by first showing that Eq.~\eqref{LiftingVectorFieldsByLinearOnes} is equivalent to $\xi\mleft( \pi^*C^\infty(N) \mright) \subset \pi^*C^\infty(N)$ for $\xi \in \mathfrak{X}(V)$. Let $f \in C^\infty(N)$, then 
\bas
\xi\mleft(\pi^*f\mright)
&=
\mathrm{d}\mleft(\pi^*f\mright)(\xi)
=
\mleft(\pi^*\mathrm{d}f \mright)\bigl( \mathrm{D}\pi(\xi) \bigr).
\eas
If $\mathrm{D}\pi(\xi) = \pi^*X$ for an $X \in \mathfrak{X}(N)$, then clearly
\bas
\xi\mleft(\pi^*f\mright)
&=
\pi^*\mleft( \mathrm{d}f(X) \mright)
\in \pi^*(C^\infty(N)).
\eas
Therefore let us now show the other direction.
 %by assuming the contrary, \textit{i.e.}~assume now $\xi\mleft(\pi^*f\mright) \notin \pi^*(C^\infty(N))$.
We know that $\mathrm{D}\pi(\xi) \in \Gamma(\pi^*\mathrm{T}N)$. Let $\mleft( \partial_i = \partial/\partial x^i \mright)_i$ local coordinate vector fields on $N$, then we can write
\bas
\mathrm{D}\pi(\xi)
&=
\mathrm{d}\pi^i(\xi) ~ \pi^*\partial_i,
\eas
and, so, we get the well-known formula $\mleft(\text{for } f = x^j\mright)$
\bas
\xi\mleft(\pi^*x^j\mright)
&=
\mathrm{d}\pi^i(\xi) ~ \pi^*\mleft(\partial_i x^j\mright)
=
\mathrm{d}\pi^j(\xi).
\eas
Hence, when there is for all $f$ an $h_f \in C^\infty(N)$ with $\xi(\pi^*f) = \pi^* h_f$,\footnote{That restricts trivially to local subsets, that is, it will work for $f=x^j$, too.} then
\bas
\mathrm{D}\pi(\xi)
&=
\underbrace{\mathrm{d}\pi^i(\xi)}_{= \xi\mleft(\pi^*x^i\mright)} ~ \pi^*\partial_i
=
\pi^*\mleft(
	\sum_i h_{x^i} ~ \partial_i
\mright).
\eas
Since the coordinates $x^j$ were arbitrary, we can conclude that there is a vector field $X \in \mathfrak{X}(N)$ such that $\mathrm{D}\pi(\xi) =\pi^*X$; that is, define $X \coloneqq \sum_i h_{x^i} ~ \partial_i$, and then show it is independent of coordinates, that is, take another coordinate system $\mleft( \partial_\alpha^\prime= \partial/\partial z^\alpha \mright)_\alpha$ of $N$. Then denote with $M$ the (local) invertible Jacobian with $\partial^\prime_\alpha = M_\alpha^i \partial_i$. Since terms like $\xi\mleft(\pi^*x^i\mright)$ describe the components of $\xi$ along the coordinates $\pi^*x^i$, we can immediately conclude
\bas
\pi^*h_{z^\alpha}
&=
\xi\mleft(\pi^*z^\alpha\mright)
=
\pi^*\mleft(\mleft(M^{-1}\mright)^\alpha_i\mright) ~ \xi\mleft(\pi^*x^i\mright)
=
\pi^*\mleft(\mleft(M^{-1}\mright)^\alpha_i ~ h_{x^i}\mright).
\eas
Therefore
\bas
\sum_\alpha h_{z^\alpha} ~ \partial^\prime_\alpha
&=
\sum_i h_{x^i} ~ \partial_i,
\eas
thence, $X$ is well-defined. Thus, Eq.~\eqref{LiftingVectorFieldsByLinearOnes} is equivalent to $\xi\mleft( \pi^*C^\infty(N) \mright) \subset \pi^*C^\infty(N)$.

$\bullet$ Now let $\xi \in \mathfrak{X}(V)$ satisfying Eq.~\eqref{LiftingVectorFieldsByLinearOnes} and lifting a vector field $X \in \mathfrak{X}(N)$, $x, y \in V$ with $\pi(x)=\pi(y)$ (such that $\mathrm{D}_x\pi(\xi_x) = \mathrm{D}_y\pi(\xi_y)$ by Eq.~\eqref{LiftingVectorFieldsByLinearOnes}), and let $f_x, f_y:I \to V,$ ($I \subset \mathbb{R}$ an open interval around 0) be curves with $f_x(0)=x, f_y(0)=y$, $\pi(f_x) = \pi(f_y)$ and
\bas
\mleft. \frac{\mathrm{d}}{\mathrm{d}t} \mright|_{t=0} f_x
&=
\xi_x,
&
\mleft. \frac{\mathrm{d}}{\mathrm{d}t} \mright|_{t=0} f_y
&=
\xi_y,
\eas
then observe for all $\lambda \in C^\infty_{\mathrm{lin}}(V)$ that
\bas
\mleft(
	\alpha \boldsymbol{\cdot} \xi_x
	\RPlus \beta \boldsymbol{\cdot} \xi_y
\mright)(\lambda)
&=
\mleft(
	\mleft. \frac{\mathrm{d}}{\mathrm{d}t} \mright|_{t=0}\mleft(
		\alpha f_x + \beta f_y
	\mright)
\mright)
(\lambda)
\\
&=
\mleft. \frac{\mathrm{d}}{\mathrm{d}t} \mright|_{t=0}\bigl(
	\underbrace{\lambda \circ \mleft(
		\alpha f_x + \beta f_y
	\mright)}_
	{\mathclap{ \stackrel{\lambda \text{ linear}}{=} \alpha (\lambda \circ f_x) + \beta (\lambda \circ f_y) }}
\bigr)
\\
&=
\alpha ~ \xi_x(\lambda)
	+ \beta ~ \xi_y(\lambda)
\eas
for all $\alpha, \beta \in \mathbb{R}$. 

If $\xi$ satisfies Eq.~\eqref{LinearityOfLinearVectorFields}, then by those results
\bas
\xi_{\alpha x +\beta y}(\lambda)
&=
\alpha ~ \xi_x(\lambda)
	+ \beta ~ \xi_y(\lambda),
\eas
therefore $\xi(\lambda) \in C^\infty_{\mathrm{lin}}(V)$ and the proof is finished (due to the previous bullet point).

If, on the other hand, $\xi(\lambda) \in C^\infty_{\mathrm{lin}}(V)$, then also
\bas
\xi_{\alpha x +\beta y}(\lambda)
&=
\alpha ~ \xi_x(\lambda)
	+ \beta ~ \xi_y(\lambda)
=
\mleft(
	\alpha \boldsymbol{\cdot} \xi_x
	\RPlus \beta \boldsymbol{\cdot} \xi_y
\mright)(\lambda).
\eas
For an $h \in C^\infty(N)$ observe
\bas
\mleft(
	\alpha \boldsymbol{\cdot} \xi_x
	\RPlus \beta \boldsymbol{\cdot} \xi_y
\mright)(\pi^*h)
&=
\mleft(
	\mleft. \frac{\mathrm{d}}{\mathrm{d}t} \mright|_{t=0}\mleft(
		\alpha f_x + \beta f_y
	\mright)
\mright)
(\pi^*h)
\\
&=
\mleft. \frac{\mathrm{d}}{\mathrm{d}t} \mright|_{t=0}\bigl(
	h \circ
	\underbrace{\pi \circ \mleft(
		\alpha f_x + \beta f_y
	\mright)}_{= \pi \circ f_x}
\bigr)
\\
&=
\mathrm{d}_p h \underbrace{\mleft( \mathrm{D}_x\pi (\xi_x) \mright)}
_{ \mathclap{ \stackrel{\text{Eq.~\eqref{LiftingVectorFieldsByLinearOnes}}}{=} \mathrm{D}_{\alpha x + \beta y} \pi (\xi_{\alpha x + \beta y}) } }
\\
&=
\xi_{\alpha x + \beta y} (\pi^*h).
\eas
This proves the claim by Remark \ref{rem:CoordinateOnTangentStuffFOrLinearVectorFields}; that is, fix additionally to the coordinates $\pi^*x^i$ fibre coordinates $y^j \in C^\infty_{\mathrm{lin}}(V)$, then express $\xi$ in those coordinates by
\bas
\xi_{\alpha x + \beta y}
&=
\xi_{\alpha x + \beta y}\mleft( \pi^*x^i \mright) ~ \mleft.\pi^*\mleft(\frac{\partial}{\partial x^i}\mright)\mright|_{\alpha x + \beta y}
	+ \xi_{\alpha x + \beta y}\mleft( y^j \mright) \mleft.\frac{\partial}{\partial y^j}\mright|_{\alpha x + \beta y}
\\
&=
\mleft(
	\alpha \boldsymbol{\cdot} \xi_x
	\RPlus \beta \boldsymbol{\cdot} \xi_y
\mright)\mleft(\pi^*x^i\mright) 
~ \mleft.\pi^*\mleft(\frac{\partial}{\partial x^i}\mright)\mright|_{\alpha x + \beta y}
	+ \mleft(
	\alpha \boldsymbol{\cdot} \xi_x
	\RPlus \beta \boldsymbol{\cdot} \xi_y
\mright)\mleft(y^j\mright)
 \mleft.\frac{\partial}{\partial y^j}\mright|_{\alpha x + \beta y}
\\
&=
\alpha \boldsymbol{\cdot} \xi_x
	\RPlus \beta \boldsymbol{\cdot} \xi_y.
\eas
\end{proof}

As vector fields the linear vector fields carry a natural Lie algebroid structure when they are a closed algebra, and this is trivial to check.
%; with Prop.~\ref{prop:ActionOfLinearVecFields} we have shown that linear vector fields form a subspace of $\mathfrak{X}(V)$, giving rise to a subbundle of $\mathrm{T}V\to V$.\footnote{Observe that for this the double vector bundle structure plays a role to achieve the needed compatibility with Eq.~\eqref{LiftingVectorFieldsByLinearOnes}.}

\begin{corollaries}{Linear vector fields are a subalgebra, \newline \cite[Corollary 3.4.3; page 114]{mackenzieGeneralTheory}}{LinFieldsAsClosedSubalge}
Let $V \stackrel{\pi}{\to} N$ be a vector bundle over a smooth manifold $N$, and $\xi, \varsigma \in \mathfrak{X}(V)$ linear vector fields on $V$ lifting vector fields $X, Y \in \mathfrak{X}(N)$, respectively. Then $[\xi, \varsigma]$ is a linear vector field lifting $[X, Y]$.
\end{corollaries}

\begin{proof}
\leavevmode\newline
That $[\xi, \varsigma]$ is a linear vector field trivially follows by Prop.~\ref{prop:ActionOfLinearVecFields}, that is, compositions of linear vector fields like $\xi \circ \varsigma$ are clearly also lineary vector fields by Prop.~\ref{prop:ActionOfLinearVecFields}, thus, also $[\xi, \varsigma] = \xi \circ \varsigma - \varsigma \circ \xi$.

We also have $\mathrm{D}\pi (\xi) = \pi^*X$ and $\mathrm{D}\pi (\varsigma) = \pi^*Y$. That immediately implies
\bas
\mathrm{D}\pi\mleft( [\xi, \varsigma] \mright)
&=
\pi^*\mleft( [X, Y] \mright),
\eas
which is a well-known fact, as also given in \cite[Proposition A.1.49; page 615]{hamilton}. 

In case this is unknown for the reader:
It can be quickly shown by first observing that 
\bas
\mathcal{L}_\xi(\pi^*f)
&=
\mathcal{L}_\xi \mleft( f \circ \pi \mright)
=
\pi^*\mleft(
	\mathrm{d}f\mleft( \mathrm{D}\pi (\xi) \mright)
\mright)
=
\pi^*\mleft(
	\mathcal{L}_X(f)
\mright)
\eas
for all $f \in C^\infty(N)$,
as also given in \cite[Lemma A.1.48; page 615]{hamilton}; basically the same as for pullback connections. By definition we also clearly have $\mathrm{D}\pi(\xi)(f) = \mathcal{L}_\xi(\pi^*f)$. Therefore altogether
\bas
\pi^*\mleft( \mleft( \mathcal{L}_{X} \circ \mathcal{L}_Y \mright) (f) \mright)
&=
\mathcal{L}_{\xi}\mleft( \pi^* \mleft( \mathcal{L}_Y(f) \mright) \mright)
=
\mleft( \mathcal{L}_\xi \circ \mathcal{L}_\varsigma \mright)(\pi^*f),
\eas
thus,
\bas
\pi^*\mleft( [X, Y](f) \mright)
&=
\pi^*\bigl( \mleft(
	\mathcal{L}_{X} \circ \mathcal{L}_Y 
	- \mathcal{L}_{Y} \circ \mathcal{L}_X 
\mright) (f) \bigr)
=
\mathcal{L}_{[\xi, \varsigma]}(\pi^*f)
=
\mathrm{D}\pi\bigl([\xi, \varsigma]\bigr)(f),
\eas
which finishes the proof.
\end{proof}

Finally we can relate it to the derivations of $V$, denoting the Lie algebra of linear vector fields by $\saut(V)$; the notation comes from that one can motivate that linear vector fields are the Lie algebra of $\sAut(V)$, but we are neither going to prove nor use this, see \textit{e.g.}~the beginning of \cite{meinrenkensplitting} for a short motivation.

\begin{theorems}{Derivations as linear vector fields, \newline \cite[Theorem 3.4.5; page 116]{mackenzieGeneralTheory}}{DerivationsSindEigentlichLineareVektorfelderKrass}
Let $V \stackrel{\pi}{\to} N$ be a vector bundle over a smooth manifold $N$, and let $D$ be a map defined by
\ba
\saut(V) &\to \Gamma(\mathcal{D}(V)),
\nonumber\\
\xi &\mapsto D_\xi,
\ea
where $D_\xi \in \Gamma(\mathcal{D}(V))$ is given by
\ba
\lambda\mleft( D_\xi v \mright)
&\coloneqq
X \bigl( \lambda(v) \bigr)
	- \xi_v(\lambda)
\ea
for all $v \in \Gamma(V)$ and $\lambda \in \Gamma(V^*) = C^\infty_{\mathrm{lin}}(V)$, and where $X \in \mathfrak{X}(N)$ is the vector field lifted by $\xi$.

Then $D$ is a bracket-preserving isomorphism of $C^\infty(N)$-modules.
\end{theorems}

\begin{remark}
\leavevmode\newline
Let us show that $D$ is well-defined. Observe
\bas
\lambda\bigl(
	D_\xi (\alpha v + \beta w)
\bigr)
&=
X \bigl( \lambda(\alpha v + \beta w) \bigr)
	- \xi_{\alpha v + \beta w}(\lambda)
\\
&=
\alpha \Bigl( X \bigl( \lambda(v) \bigr) - \xi_v(\lambda) \Bigr)
	+ \beta \Bigl( X \bigl( \lambda(w) \bigr) - \xi_w(\lambda) \Bigr)
\\
&=
\alpha ~ \lambda(D_\xi v)
	+ \beta ~ \lambda(D_\xi w)
\\
&=
\lambda\mleft( \alpha D_\xi v + \beta D_\xi w \mright)
\eas
for all $v, w \in \Gamma(V)$, $\lambda \in \Gamma(V^*)$, $\xi \in \saut(V)$ (lifting $X \in \mathfrak{X}(N)$) and $\alpha, \beta \in \mathbb{R}$, using $\pi(v) = \mathds{1}_N = \pi(w)$ and Prop.~\ref{prop:ActionOfLinearVecFields}, that is, $\xi(\lambda)$ is linear. Similarly one shows for all $f \in C^\infty(N)$ that
\bas
\lambda \mleft( D_\xi (fv) \mright)
&=
X \underbrace{\bigl( \lambda(fv) \bigr)}
	_{=f \lambda(v)}
	- \xi_{fv}(\lambda)
\\
&=
f ~ \mleft( X \bigl( \lambda(v) \bigr) - \xi_v(\lambda) \mright)
	+ \mathcal{L}_X(f) ~ \lambda(v)
\\
&=
f ~ \lambda(D_\xi v)
	+ \mathcal{L}_X(f) ~ \lambda(v)
\\
&=
\lambda \mleft( f D_\xi v + \mathcal{L}_X(f) ~ v \mright).
\eas
Hence, $D_\xi \in \Gamma(\mathcal{D}(V))$.
\end{remark}

\begin{proof}[Very short sketch for the proof of Thm.~\ref{thm:DerivationsSindEigentlichLineareVektorfelderKrass}]
\leavevmode\newline
We are not going to show this because we will not need this statement, please see the reference; the proof is relatively straightforward, but using several tricks. One first shows that $\saut(V)$ are sections of a certain Lie algebroid isomorphic to $\mathcal{D}(V^*)$ such that one essentially needs to show that $\mathcal{D}(V) \cong \mathcal{D}(V^*)$. For all $L \in \Gamma(\mathcal{D}(V))$ one can define a $T \in \Gamma(\mathcal{D}(V^*))$ as usual by forcing the Leibniz rule as in
\bas
\bigl( T(\lambda) \bigr)(v)
&\coloneqq
a(L)\bigl( \lambda(v) \bigr)
	- \lambda\bigl(L(v)\bigr)
\eas
for all $\lambda \in \Gamma(V^*)$ and $v \in \Gamma(V)$. This defines also an isomorphism of Lie algebroids $\mathcal{D}(V) \cong \mathcal{D}(V^*)$; see more in \cite[discussion after Corollary 3.4.3; page 114ff.]{mackenzieGeneralTheory}.
\end{proof}

%With this we have an immediate natural Lie algebroid structure; for this recall that the base manifold $N$ is an embedded submanifold of the vector bundle $V$, embedded by the zero section of $V$. By the previous corollary and the Frobenius theorem. The set of linear vector fields restricted to $N$ is denoted by 
%\ba
%\mathfrak{X}_{\mathrm{lin}}
%&\coloneqq
%\left\{
%~\middle|~
%\right\}
%\ea
%
%\begin{propositions}{Lie algebroid of linear vector fields, \newline \cite[part of Theorem 3.4.5; page 116]{mackenzieGeneralTheory}}{LinVecFieldsAreSuperDuperAlgoids}
%Let $V \to N$ be a vector bundle over a smooth manifold $N$. Then the set of all linear vector fields on $V$ carry a Lie algebroid structure over $N$ whose anchor $\rho_{\mathrm{lin}}$ is given by
%\ba
%\rho_{\mathrm{lin}}(\xi)
%&\coloneqq
%X
%\ea
%for all linear vector fields $\xi$ lifting $X \in \mathfrak{X}(N)$, and its Lie bracket is $\mleft[ \cdot, \cdot \mright]_{\mathrm{lin}}$ is given by
%\ba
%\mleft[ \xi, \varsigma \mright]_{\mathrm{lin}}
%&=
%\mleft.\mleft[ \xi, \varsigma \mright]\mright|_{N}
%\ea
%\end{propositions}
%
\section{Lie algebroid connections}\label{SubsectionEDiffstuff}

In the following we will introduce the notion of $E$-connections, following partially \cite[\S 2]{basicconn}. See also \cite[\S 2.5]{ELeviCivita} \textit{e.g.} for a discussion about an $E$-Levi-Civita connection and other similar terms similar to Riemannian geometry. However, we want to introduce connections using the previous section, as in \cite{mackenzieGeneralTheory}.

\begin{definitions}{$E$-connection, $E$-curvature and $E$-torsion, \newline \cite[variation of Definition 5.2.5; page 186]{mackenzieGeneralTheory} \newline \cite[variation of Definition 5.2.9; page 187]{mackenzieGeneralTheory} \newline \cite[\S 4.1, trivial generalization of Equation (14); page 154]{mackenzieGeneralTheory}}{Econnection}
Let $E \to N$ be a Lie algebroid over a smooth manifold $N$ and $V \to N$ be a vector bundle over $N$. 
\begin{enumerate}
\item An $E$-connection on the vector bundle $V$ is a base- and anchor-preserving vector bundle morphism $\gls{0nablaE}: E \to \mathcal{D}(V)$, $\nu \mapsto {}^E\nabla_\nu$.
\item The $E$-curvature $\gls{RnablaE}$ of $^E\nabla$ is defined as in Def.~\ref{def:GeneralDefOfCurvMorphisms} by
\ba
R_{{}^E\nabla}(\mu, \nu) 
&\coloneqq 
\mleft[{}^E\nabla_\mu, {}^E\nabla_\nu\mright]_{\mathcal{D}(V)}
	- {}^E\nabla_{[\mu, \nu]_E}
\ea
for all $\mu, \nu \in \Gamma(E)$. ${}^E\nabla$ is called \textbf{flat} if its curvature vanishes.
\item In the special case of $V = E$ we can define also the $E$-torsion $\gls{tEnabla}$ as an element of $\mathcal{T}^1_2(E)$ given by
\ba
t_{{}^E\nabla}(\mu, \nu) \coloneqq {}^E\nabla_\mu \nu - {}^E\nabla_\nu \mu - [\mu, \nu]_E
\ea
for all $\mu, \nu \in \Gamma(E)$.
\end{enumerate}
\end{definitions}

\begin{remark}
\leavevmode \newline
\indent $\bullet$ The base- and anchor-preservation in the definition of an $E$-connection especially means
\bas
a \circ {}^E\nabla
&=
\rho,
\eas
so, for all $\mu \in E$ we have that ${}^E\nabla_\mu$ is $\mathbb{R}$-linear and
\bas
^E\nabla_\mu (fv) &= f ~{}^E\nabla_\mu v + \mathcal{L}_{\rho(\mu)}(f)~ v,
\eas
for all $f \in C^\infty(N)$ and $v \in \Gamma(V)$.
That it is a base-preserving vector bundle morphism, implies that one can extend ${}^E\nabla$ to sections, giving rise to an $\mathbb{R}$-linear map $\Gamma(E) \to \Gamma\bigl( \mathcal{D}(V) \bigr)$, with
\bas
^E\nabla_{f\nu} (v) &= f ~{}^E\nabla_\nu v
\eas
for all $\nu \in \Gamma(E)$, $f \in C^\infty(N)$ and $v \in \Gamma(V)$. This is precisely the typical definition of a connection, besides that the Leibniz rule is along a more general anchor. In the case of $E = \mathrm{T}N$, especially $\rho_E = \mathds{1}_{\mathrm{T}N}$, we have a typical vector bundle connection, and it is trivial to see that both definitions are equivalent in that situation.

$\bullet$ As noted at the end of the introduction, when write "connection" or "vector bundle connection", then we always mean typical $\mathrm{T}N$-connections.

$\bullet$ This clearly generalizes the concept of Lie algebra connections as in Def.~\ref{def:FirstStepLieDerivativeOfAnchors}, for example look at an action Lie algebroid, but now with the tensorial behaviour again due to the bundle structure.

$\bullet$ As for vector bundle connections, one can view the curvature as a map 
\bas
R_{{}^E\nabla}: \Gamma(E) \times \Gamma(E) \times \Gamma(V) &\to \Gamma(V),
\\
(\mu,\nu,v) &\mapsto R_{{}^E\nabla}(\mu,\nu)v
=
{}^E\nabla_\mu {}^E\nabla_\nu v
	- {}^E\nabla_\nu {}^E\nabla_\mu v
	- {}^E\nabla_{\mleft[ \mu, \nu \mright]_E} v.
\eas
In Lemma \ref{lem:KruemmungenSindTensorenMitAnkerErhaltung} we have that it is tensorial the first two arguments. For the third it is as for vector bundle connections,
\bas
R_{{}^E\nabla}(\mu,\nu) (fv)
&=
f ~ R_{{}^E\nabla}(\mu,\nu)v
	+ \underbrace{\mleft(
		\mathcal{L}_{\rho(\mu)}\mleft(\mathcal{L}_{\rho(\nu)} (f) \mright)
		- \mathcal{L}_{\rho(\nu)}\mleft(\mathcal{L}_{\rho(\mu)} (f) \mright)
		- \mathcal{L}_{\mleft[ \rho(\mu), \rho(\nu) \mright]_E (f)}
	\mright)}_{=0} ~ v
\\
&=
f ~ R_{{}^E\nabla}(\mu,\nu)v
\eas
for all $f \in C^\infty(N)$, $\mu, \nu \in \Gamma(E)$ and $v \in \Gamma(V)$, using that $\rho$ is a homomorphism of Lie brackets. To summarize, $a \circ R_{{}^E\nabla} = 0$, and $R_{{}^E\nabla}$ can be viewed as an element of $\mathcal{T}^1_3(E)$.

$\bullet$ As in the situation of vector bundle connections it is trivial and straightforward to check that $t_{{}^E\nabla}$ is an anti-symmetric tensor because of the fact the Leibniz rules in the connections and the Lie bracket cancel each other.
\end{remark}

In Ex.~\ref{ex:LieAlgActionIsAConnection} we had a canonical Lie algebra connection, induced by a Lie algebra action and vector bundle connection. We can generalize this connection.

\begin{examples}{Canonically induced $E$-connection, \newline \cite[first example in Example 2.8]{ELeviCivita}}{NablaRhoConnection}
Let $E \to N$ be a Lie algebroid over a smooth manifold $N$ and $V \to N$ be a vector bundle over $N$, equipped with a vector bundle connection $\nabla$. Then define ${}^E\nabla$ on $V$ by 
\ba
{}^E\nabla_\mu
&\coloneqq 
\nabla_{\rho(\mu)}
\ea
for all $\mu \in \Gamma(E)$. This is a canonical example of an $E$-connection which we will denote as $\gls{0nablarho}$.
\end{examples}

As for vector bundle connections, we can extend a given $E$-connection to $\mathcal{T}^r_s(V)$ ($r, s \in \mathbb{N}_0$).

\begin{examples}{Dual Lie algebroid connections,\newline very typical construction forcing the Leibniz rule as in \cite[Definition 2.1.36, but using connections; page 96]{hamilton}}{DualEConns}
Let $E \to N$ be a Lie algebroid over a smooth manifold $N$ and $V \to N$ be a vector bundle over $N$, equipped with an $E$-connection ${}^E\nabla$. Then we define its \textbf{dual $E$-connection} on $V^*$, still denoted as ${}^E\nabla$, by
\ba
\mleft({}^E\nabla_\nu \omega\mright)(v)
&\coloneqq
\mathcal{L}_\nu\bigl( \omega(v) \bigr)
	- \omega \mleft( {}^E\nabla_\nu v \mright)
\ea
for all $\nu \in \Gamma(E)$, $\omega \in \Gamma(V^*)$ and $v \in \Gamma(V)$. It is trivial to prove that ${}^E\nabla_\nu \omega \in \Gamma(V^*)$ and that this ${}^E\nabla$ is an $E$-connection on $V^*$. Similarly, as for vector bundle connections, one extends ${}^E\nabla$ to $\mathcal{T}^r_s(V)$ for all $r, s \in \mathbb{N}_0$, always denoted by ${}^E\nabla$.
\end{examples}

Flatness just means trivially the following by definition.

\begin{corollaries}{Flat connections, \cite[\S 5.2, Definition 5.2.9; page 187]{mackenzieGeneralTheory}}{FlatConnectionsAreLieAlgebroidMorphisms}
Let $E\to N$ be a Lie algebroid over a smooth manifold $N$ and $V \to N$ a vector bundle. Then an $E$-connection ${}^E\nabla: E \to \mathcal{D}(V)$ on $V$ is flat if and only if it is a (base-preserving) morphism of Lie algebroids.
\end{corollaries}

\begin{proof}
\leavevmode\newline
This simply follows by definition.
\end{proof}

Of special importance regarding curvatures are of course the Bianchi identities.

\begin{theorems}{Bianchi identities, \newline \cite[Satz 8.3, generalization of second statement there; page 90]{LangeIstEsHerMitDerRaumzeit} \newline \cite[reformulation of Proposition 7.1.9; page 265]{mackenzieGeneralTheory}}{1stBianchi}
%\leavevmode\newline
Let $E\to N$ be a Lie algebroid over a smooth manifold $N$, and ${}^E\nabla$ be an $E$-connection on $E$. Then the curvature $R_{{}^E\nabla}$ satisfies both Bianchi identities, \textit{i.e.}~for all $\mu, \nu, \eta \in \Gamma(E)$ we have the \textbf{first Bianchi identity}
\ba\label{eq:firstBianchi}
&R_{{}^E\nabla}(\mu, \nu) \eta + R_{{}^E\nabla}(\nu, \eta) \mu + R_{{}^E\nabla}(\eta, \mu) \nu 
\nonumber \\
&=
t_{{}^E\nabla}\mleft(t_{{}^E\nabla}(\mu, \nu), \eta\mright) + t_{{}^E\nabla}(t_{{}^E\nabla}(\nu, \eta), \mu) + t_{{}^E\nabla}(t_{{}^E\nabla}(\eta, \mu), \nu)
\nonumber \\
&\hspace{1cm}
+ \left({}^E\nabla_\mu t_{{}^E\nabla}\right)(\nu, \eta) 
+ \left({}^E\nabla_\nu t_{{}^E\nabla}\right)(\eta, \mu) + \left({}^E\nabla_\eta t_{{}^E\nabla}\right)(\mu, \nu),
\ea
and we also get the \textbf{second Bianchi identity}
\ba
0&=
\left( {}^E\nabla_\mu R_{{}^E\nabla}\right)(\nu, \eta) + \left( {}^E\nabla_\nu R_{{}^E\nabla}\right)(\eta, \mu) + \left( {}^E\nabla_\eta R_{{}^E\nabla}\right)(\mu, \nu)
\nonumber\\
&\hspace{1cm}
+ R_{{}^E\nabla}\left( t_{{}^E\nabla}(\mu, \nu), \eta \right) 
	+R_{{}^E\nabla}\left( t_{{}^E\nabla}(\nu, \eta), \mu \right)
	+ R_{{}^E\nabla}\left( t_{{}^E\nabla}(\eta, \mu), \nu \right).
\ea
\end{theorems}

\begin{remark}
\leavevmode\newline
Eq. \eqref{eq:firstBianchi} implies that $t_{{}^E\nabla}$ satisfies the Jacobi identity if ${}^E\nabla$ is flat and $t_{{}^E\nabla}$ is covariantly constant with respect to ${}^E\nabla$. Thence, it would define another Lie bracket on $\Gamma(E)$ which is $C^\infty$-bilinear. Moreover, this Lie bracket then also defines a Lie bracket on each fibre $E_p$.
\end{remark}

\begin{proof}[Proof of the first Bianchi identity]
\leavevmode\newline
The second Bianchi identity we will prove later by its generalization (see Thm.~\ref{thm:2ndBianchi} and Remark \ref{rem:FinallyTheOtherBianchiStuff}). The former statement we can prove now by showing that it is equivalent to the Jacobi identity for $[\cdot, \cdot]_E$. First observe for $\mu, \nu, \eta \in \Gamma(E)$ that
\bas
\left[\mu, \left[\nu, \eta\right]_E\right]_E
&=
\left[ \mu, -t_{{}^E\nabla}(\nu, \eta) + {}^E\nabla_\nu \eta - {}^E\nabla_\eta \nu \right]_E \\
&=
t_{{}^E\nabla}( \mu, t_{{}^E\nabla}(\nu, \eta) )
- {}^E\nabla_\mu \left( t_{{}^E\nabla} (\nu,\eta) \right)
+ {}^E\nabla_{t_{{}^E\nabla} (\nu,\eta)} \mu \\
&\hspace{1cm}-
t_{{}^E\nabla}\left( \mu, {}^E\nabla_\nu \eta \right)
+ {}^E\nabla_\mu {}^E\nabla_\nu \eta
- {}^E\nabla_{{}^E\nabla_\nu \eta} \mu \\
&\hspace{1cm}+
t_{{}^E\nabla}\left( \mu, {}^E\nabla_\eta \nu \right)
- {}^E\nabla_\mu {}^E\nabla_\eta \nu
+ {}^E\nabla_{{}^E\nabla_\eta \nu} \mu \\
&=
-t_{{}^E\nabla}( t_{{}^E\nabla}(\nu, \eta), \mu )
- {}^E\nabla_\mu \left( t_{{}^E\nabla} (\nu,\eta) \right)
+ t_{{}^E\nabla}\left( {}^E\nabla_\nu \eta, \mu \right)
+ t_{{}^E\nabla}\left( \mu, {}^E\nabla_\eta \nu \right) \\
&\hspace{1cm}+
{}^E\nabla_\mu {}^E\nabla_\nu \eta
- {}^E\nabla_\mu {}^E\nabla_\eta \nu
- {}^E\nabla_{[\nu, \eta]_E} \mu.
\eas
With $\sigma$ we will denote the cyclic sum and thence by the Jacobi identity (and the cyclic property of the total sum)
\bas
&&0&=
\sigma\left( \left[\mu, \left[ \nu, \eta \right]_E\right]_E \right) \\
&&&=
\sigma\big( -t_{{}^E\nabla}( t_{{}^E\nabla}(\nu, \eta), \mu )
- {}^E\nabla_\mu \left( t_{{}^E\nabla} (\nu,\eta) \right)
+ t_{{}^E\nabla}\left( {}^E\nabla_\nu \eta, \mu \right)\\
&&&\qquad~+
t_{{}^E\nabla}\left( \mu, {}^E\nabla_\eta \nu \right)
+ {}^E\nabla_\mu {}^E\nabla_\nu \eta
- {}^E\nabla_\mu {}^E\nabla_\eta \nu
- {}^E\nabla_{[\nu, \eta]_E} \mu \big) \\
&&&=
\sigma\big( -t_{{}^E\nabla}( t_{{}^E\nabla}(\mu, \nu), \eta )
- {}^E\nabla_\mu \left( t_{{}^E\nabla} (\nu,\eta) \right)
+ t_{{}^E\nabla}\left( {}^E\nabla_\mu \nu, \eta \right)\\
&&&\qquad~+
t_{{}^E\nabla}\left( \nu, {}^E\nabla_\mu \eta \right)
+ {}^E\nabla_\mu {}^E\nabla_\nu \eta
- {}^E\nabla_\nu {}^E\nabla_\mu \eta
- {}^E\nabla_{[\mu, \nu]_E} \eta \big) \\
&\Leftrightarrow&
\sigma\big( R_{{}^E\nabla}(\mu, \nu)\eta \big)
&=
\sigma\left( t_{{}^E\nabla}( t_{{}^E\nabla}(\mu, \nu), \eta)
+ \left( {}^E\nabla_\mu t_{{}^E\nabla} \right)(\nu, \eta) \right).
\eas
\end{proof}

In Section \ref{NewInfGaugeTrafoTrafos} we have seen that pullbacks of Lie algebra connections were important to define the infinitesimal gauge transformation. Hence, let us turn to pullbacks of Lie algebroid connections.

\section{Pullbacks of Lie algebroid connections}\label{PullbacksAlsoGeneral}

As in the discussion around Def.~\ref{def:LieAlgebraPfadeKurvi} we need to be careful about how and when we can make a pullback of Lie algebroid connections. We want to generalize Prop.~\ref{prop:FirstEPullBACkConnectionFormula}, especially recall its proof and Remark \ref{rem:ImportantRemarkAboutPullbacks}. For simplicity let us first look again at curves.

\begin{definitions}{$E$-paths, \cite[\S 2, Definition 2.4]{ELeviCivita}}{EPaths}
Let $\mleft(E, \rho, \mleft[ \cdot, \cdot \mright]_E\mright) \stackrel{\pi}{\to} N$ be a Lie algebroid and $I \subset \mathbb{R}$ an open interval. Then an \textbf{$E$-path} is a smooth map $\alpha: I \to E$ with
\ba
(\gamma^*\rho) ( \alpha)
&=
\frac{\mathrm{d}}{\mathrm{d}t} \gamma,
\ea
where the curve $\gamma: I \to N$, $t \mapsto \pi(\alpha(t))$, is the \textbf{base path of $\alpha$}. We also say that \textbf{$\gamma$ is lifted by $\alpha$}.
\end{definitions}

\begin{remark}\label{SectionsAlongCurvesAreCurvePullbacksections}
\leavevmode\newline
Recall that for a vector bundle $V \stackrel{\mathrm{pr}}{\to} N$ we say that a section of $V$ along $\gamma$ is a smooth map $v: I \to V$ with $\mathrm{pr} \circ v = \gamma$, and that we identify sections of $\gamma^*V$ with sections of $V$ along $\gamma$. That means that an $E$-path $\alpha$ can be viewed as a section of $\gamma^*E$.
\end{remark}

Using this we can define a pullback $E$-connection and a derivation along an $E$-path.

\begin{propositions}{Pull-back of an $E$-connection along an $E$-path, \newline \cite[\S 2, comment before Definition 2.4]{ELeviCivita}}{PullBackEconnAlongEPaths}
Let $E \to N$ be a Lie algebroid, $V \to N$ a vector bundle and ${}^E\nabla$ an $E$-connection on $V$. Fix an $E$-path $\alpha$, $I \ni t \mapsto \alpha(t) \in E$, with base path $\gamma$. Then there is a unique vector bundle connection $\gamma^*\mleft( {}^E\nabla \mright)$ on $\gamma^*V \to I$ with
\ba\label{eqPullbackEconnectioncondition}
\gamma^*\mleft( {}^E\nabla \mright)_{c  \frac{\mathrm{d}}{\mathrm{d}t}} \mleft( \gamma^* v \mright)
&=
\gamma^*\mleft( {}^E\nabla_{c  \alpha} v \mright)
\ea
for all $v \in \Gamma(V)$ and $c \in \mathbb{R}$.
\end{propositions}

\begin{remark}\label{RemarkNotationvonPullbackConnection}
\leavevmode\newline
As introduced, we will view ($E$-)connections as base- and anchor-preserving morphisms, and, when acting on sections, as 1-forms. In the latter case, ${}^E\nabla v \in \Omega^1(E; V)$, and the pull-back as a section gives then $\gamma^*\mleft( {}^E\nabla v \mright) \in \Gamma\mleft( \mleft(\gamma^*E\mright)^* \otimes \gamma^*V \mright)$, therefore we define $\mleft(\gamma^*\mleft( {}^E\nabla v \mright)\mright)(c\alpha) \eqqcolon \gamma^*\mleft( {}^E\nabla_{c\alpha} v \mright)$ when viewing $\alpha$ as a section of $\gamma^*E$. One could also just write ${}^E\nabla_{c\alpha} v$ when using the interpretation of connections as morphisms, because ${}^E\nabla_{c\alpha(t)}$ is then a derivation of $V$ at $\gamma(t)$ such that it is immediate that we have a section along $\gamma$ and, hence, of $\gamma^*V$. However, most of the time we prefer to write the pull-back as an accentuation.

When $\alpha = \gamma^* \nu$ for $\nu \in \Gamma(V)$, then we write $\gamma^*\mleft( {}^E\nabla_{c\nu} v \mright)$, although it looks ambiguous with the notation just discussed previously,
\bas
\gamma^*\mleft( {}^E\nabla_{c~ \gamma^*\nu} v \mright)
&=
\mleft(\gamma^*\mleft( {}^E\nabla v \mright)\mright)\mleft(c~ \gamma^*\nu\mright)
=
\gamma^*\mleft(\mleft( {}^E\nabla v \mright)(c\nu)\mright)
=
\gamma^*\mleft( {}^E\nabla_{c \nu} v \mright),
\eas
but the notation should be clear by the context.
\end{remark}

\begin{proof}[Proof of Prop.~\ref{prop:PullBackEconnAlongEPaths}]
\leavevmode\newline
As usual, the condition \eqref{eqPullbackEconnectioncondition} uniquely defines $\gamma^*\mleft( {}^E\nabla \mright)$ by using that $\gamma^*(\Gamma(V))$ generates $\Gamma(\gamma^*V)$ and extending Eq.~\eqref{eqPullbackEconnectioncondition} by forcing the Leibniz rule, \textit{i.e.}~we define
\bas
\gamma^*\mleft( {}^E\nabla \mright)_{c  \mleft.\frac{\mathrm{d}}{\mathrm{d}t}\mright|_t} \mleft( f^i ~ \gamma^* v_i \mright)
\coloneqq
c  \mleft.\frac{\mathrm{d} f^i}{\mathrm{d}t}\mright|_t ~ \mleft.\gamma^*\mleft( v_i \mright)\mright|_t
	+ f^i(t) ~ \mleft.\gamma^*\mleft({}^E\nabla_{c  \alpha} v_i\mright)\mright|_t
\eas
for all $v_i \in \Gamma(V)$, $f^i \in C^\infty(I)$, $t \in I$ and $c \in \mathbb{R}$, where the index $i$ runs over an arbitrary range; recall Def.~\eqref{FullPulbackGConnection} in the proof of Prop.~\ref{prop:FirstEPullBACkConnectionFormula}. Every other connection satisfying Eq.~\eqref{eqPullbackEconnectioncondition} has the same form by the Leibniz rule, and, so, uniqueness follows if existence is given. Hence, it is only left to prove that this gives a well-defined connection, that is, we need to prove that it is independent of the choice of generators $v_i$ as in the proof of Prop.~\ref{prop:FirstEPullBACkConnectionFormula} and that it is a connection satisfying Eq.~\eqref{eqPullbackEconnectioncondition}. Recall Remark \ref{rem:ImportantRemarkAboutPullbacks}, we especially need to check whether the Leibniz rule inherited by ${}^E\nabla$ is compatible with the Leibniz rule of connections of $\gamma^*V \to I$, for this we need to calculate
\bas
\mleft. \gamma^*\mleft({}^E\nabla_{c  \alpha} (h v)\mright)\mright|_t
&=
\underbrace{\mathcal{L}_{c\rho(\alpha(t))}}_{=~ \mathcal{L}_{c\dot{\gamma}(t)}}(h) ~ v|_{\gamma(t)}
	+ h(\gamma(t)) ~ \mleft. \gamma^*\mleft({}^E\nabla_{c  \alpha}  v\mright)\mright|_t
\\
%&=
%\gamma^*\mleft( h {}^E\nabla_{c  \alpha(t)}  v \mright)
	%+ \gamma^*\mleft(c ~ \mleft.\frac{\mathrm{d}}{\mathrm{d}t}\mright|_t(h \circ \gamma) ~ v|_{\gamma(t)} \mright) \\
&=
\mleft.\mleft(
	c ~ \frac{\mathrm{d} (h \circ \gamma)}{\mathrm{d}t} ~ \gamma^*v 
	+ ( h \circ \gamma ) ~ \gamma^*\mleft( {}^E\nabla_{c \alpha} v \mright)
\mright)\mright|_{t}
\eas
for all $v \in \Gamma(V)$ and $h \in C^\infty(N)$. Thus, the proof is then the same as for Prop.~\ref{prop:FirstEPullBACkConnectionFormula}; linearity and the Leibniz rule follow by construction, and Eq.~\eqref{eqPullbackEconnectioncondition} and the independence of the taken generators follows by the previous calculation.
\end{proof}

As usual, one can use this to define parameter derivatives.

\begin{propositions}{Derivations of sections along $E$-paths, \newline \cite[\S 2, beginning of subsection 2.3; there $\mathrm{D}/\mathrm{d}t$ is denoted as $\nabla^\alpha$]{ELeviCivita}}{DerivationAlongEPath}
Let $E \to N$ be a Lie algebroid, $V \to N$ a vector bundle and ${}^E\nabla$ an $E$-connection on $V$. Fix an $E$-path $\alpha$, $I \ni t \mapsto \alpha(t) \in E$, with base path $\gamma$. Then there is a unique differential operator $\gls{Ddt}: \Gamma\mleft(\gamma^*V\mright) \to \Gamma\mleft(\gamma^*V\mright)$ with
\ba
\frac{\mathrm{D}}{\mathrm{d}t} &\text{ is linear over } \mathbb{R}, \\
\frac{\mathrm{D}}{\mathrm{d}t}(f s)
&=
\frac{\mathrm{d}f}{\mathrm{d}t} ~ s
	+ f ~ \frac{\mathrm{D}}{\mathrm{d}t} s, \\
\mleft.\frac{\mathrm{D}}{\mathrm{d}t}\mright|_t \mleft( \gamma^* v \mright)
&=
\mleft. \gamma^*\mleft({}^E\nabla_{\alpha} v \mright)\mright|_t \label{ParameterderivativeonCurvePullbackSections}
\ea
for all $s \in \Gamma\mleft(\gamma^*V\mright)$, $v \in \Gamma(V)$, $f \in C^\infty(I)$ and $t \in I$. 
\end{propositions}

\begin{proof}
\leavevmode\newline
Uniqueness will follow again by using that $\gamma^*(\Gamma(V))$ generates $\Gamma(\gamma^*V)$ and extending Eq.~\eqref{ParameterderivativeonCurvePullbackSections} by forcing the Leibniz rule, this is given by choosing
\bas
\frac{\mathrm{D}}{\mathrm{d}t}
&\coloneqq
\gamma^*\mleft( {}^E\nabla \mright)_{\frac{\mathrm{d}}{\mathrm{d}t}}
\eas
and then everything follows by Prop.~\ref{prop:PullBackEconnAlongEPaths}.
\end{proof}

\begin{remark}\label{DdtGleichddt}
\leavevmode\newline
%\indent $\bullet$ We often write
%\bas
%\frac{\mathrm{D}_{\mleft(\alpha, ^E\nabla\mright)}}{\mathrm{d}t}
%\eas
%instead of $\frac{\mathrm{D}}{\mathrm{d}t}$. 
When $V = N \times \mathbb{R}$, then we clearly have $\mathrm{D}/\mathrm{d}t = \mathrm{d}/\mathrm{d}t$, for this use the uniqueness and define ${}^E\nabla \coloneqq \nabla^0_\rho$, where $\nabla^0 = \mathrm{d}$ is the canonical flat connection, and
\bas
\mleft.\frac{\mathrm{d}}{\mathrm{d}t}\mright|_t\underbrace{(\gamma^*v)}_{\mathclap{= v \circ \gamma: I \to \mathbb{R}}}
&=
\mathrm{d}_{\gamma(t)}v \underbrace{\mleft( \mleft.\frac{\mathrm{d}}{\mathrm{d}t}\mright|_t \gamma \mright)}
_{\mathclap{= (\gamma^*\rho)(\alpha(t))}}
=
\mleft.\gamma^*\Bigl(
	\mathrm{d}v\bigl( (\gamma^*\rho)(\alpha) \bigr)
\Bigr)\mright|_t
=
\mleft.\gamma^*\mleft( {}^E\nabla_\alpha v \mright)\mright|_t.
\eas
\end{remark}

Prop.~\ref{prop:PullBackEconnAlongEPaths} can be generalized, using the notion defined in Def.~\ref{def:DefOfAnchorPreservingStuff}.

\begin{corollaries}{Pullbacks of Lie algebroid connections by anchor-preserving morphisms}{GeneralPullbackAnchorPreserving}
Let $E_i \to N_i$ ($i \in\{1,2\}$) be two Lie algebroids over smooth manifolds $N_i$, $V \to N_2$ a vector bundle, and ${}^{E_2}\nabla$ an $E_2$-connection on $V$. Also fix an anchor-preserving vector bundle morphism $\xi: E_1 \to E_2$ over a smooth map $f: N_1 \to N_2$. Then there is a unique $E_1$-connection $f^*\mleft( {}^{E_2}\nabla \mright)$ on $f^*V$ with
\ba\label{GeneralPullbackDef}
\mleft(f^*\mleft( {}^{E_2}\nabla \mright)\mright)_\nu (f^*v)
&=
f^*\mleft(
	{}^{E_2}\nabla_{\xi(\nu)} v
\mright)
\ea
for all $v \in \Gamma(V)$ and $\nu \in \Gamma(E_1)$.
\end{corollaries}

\begin{remark}
\leavevmode\newline
This result is motivated by \cite[Example 7.7]{meinrenkenlie} where it is shown that there is a 1:1 correspondence of Lie algebroid paths and anchor-preserving morphisms. That is, let $E_1 = \mathrm{T}I$, where $I \subset \mathbb{R}$ is an open interval. Then define 
\ba\label{1to1AnchorPresAndEpAth}
\alpha
&\coloneqq
\xi\mleft( \frac{\mathrm{d}}{\mathrm{d}t} \mright),
\ea
which is a map $I \to E_2, t \mapsto \xi\mleft( \mathrm{d}/\mathrm{d}t|_t \mright)$,
such that the anchor-preservation implies
\bas
\mleft(f^*\rho_{E_2}\mright)(\alpha)
&=
\mathrm{D}f\mleft( \frac{\mathrm{d}}{\mathrm{d}t} \mright)
=
\frac{\mathrm{d}}{\mathrm{d}t} f.
\eas
Hence, $\alpha$ is an $E_2$-path lifting $f$. Vice versa one can define $\xi$ by Eq.~\eqref{1to1AnchorPresAndEpAth} if $\alpha$ is given, and then extending $\xi$ canonically to a tensor.

Furthermore, as one can see, the presented definitions of connections and their pullbacks can also be extended to vector bundles with just an anchor, without the need of a Lie bracket ($\Rightarrow$ anchored vector bundle). But as we have seen before, for example recall Remark \ref{WecombineeverythingToAvoidStrictPullbacks}, one can even generalize it further which we will do in the next statement.
\end{remark}

\begin{proof}[Proof of Cor.~\ref{cor:GeneralPullbackAnchorPreserving}]
\leavevmode\newline
We only give a sketch because the proof is exactly as in Prop.~\ref{prop:PullBackEconnAlongEPaths}, and all other similar statements as in Section \ref{NewInfGaugeTrafoTrafos}; instead of $\mathrm{d}/\mathrm{d}t$ one has essentially $\mathcal{L}_{\rho_{E_1}(\nu)}$ for $\nu \in \Gamma(E_1)$ which does neither change the structure nor the arguments of the proof. Making use of Def.~\ref{def:DefOfAnchorPreservingStuff} we get
\bas
f^*\mleft({}^{E_2}\nabla_{\xi(\nu)} (h v)\mright)
&=
(h\circ f) ~ f^*\mleft({}^{E_2}\nabla_{\xi(\nu)}  v\mright)
	+ f^*\bigl( \underbrace{\mathcal{L}_{(\rho_{E_2}\circ\xi)(\nu)}}
	_{\mathclap{ =~ \mathcal{L}_{ \mleft(\mathrm{D}f \circ \rho_{E_1}\mright)(\nu)}  }}
	(h)\bigr) ~ f^*v 
\\
%&=
%\gamma^*\mleft( h {}^E\nabla_{c  \alpha(t)}  v \mright)
	%+ \gamma^*\mleft(c ~ \mleft.\frac{\mathrm{d}}{\mathrm{d}t}\mright|_t(h \circ \gamma) ~ v|_{\gamma(t)} \mright) \\
&=
(h\circ f) ~ f^*\mleft({}^{E_2}\nabla_{\xi(\nu)}  v\mright)
	+ \mathcal{L}_{\rho_{E_1}(\nu)}(h \circ f) ~ f^*\mleft( v \mright)
\eas
for all $h \in C^\infty(N_2)$, $v \in \Gamma(V)$ and $\nu \in \Gamma(E_1)$, using
\bas
f^*\Bigl(
	\mathcal{L}_{ \mleft(\mathrm{D}f \circ \rho_{E_1}\mright)(\nu)} (h)
\Bigr)
&=
\mleft(f^*\mathrm{d}h\mright)\bigl(\mleft(\mathrm{D}f \circ \rho_{E_1}\mright)(\nu)\bigr)
=
\underbrace{\mleft(f^!\mathrm{d}h\mright)}_{\mathclap{ = \mathrm{d}f^! h }}\bigl( \rho_{E_1}(\nu) \bigr)
=
\mathcal{L}_{\rho_{E_1}(\nu)}(h \circ f).
\eas
As mentioned in the proof of Prop.~\ref{prop:PullBackEconnAlongEPaths} and Remark \ref{rem:ImportantRemarkAboutPullbacks}, this proves that the inherited Leibniz rule of ${}^{E_2}\nabla$ is compatible with the Leibniz rule of $E_1$-connections on $f^*V$. Hence, the remaining proof is then precisely as in Prop.~\ref{prop:PullBackEconnAlongEPaths} and \ref{prop:FirstEPullBACkConnectionFormula}; locally, $f^*\mleft( {}^{E_2}\nabla \mright)$ is defined by
\bas
\mleft(f^*\mleft( {}^{E_2}\nabla \mright)\mright)_\nu \mu
&\coloneqq
\mathcal{L}_{\rho_{E_1}(\nu)}\mleft(\mu^a\mright) ~ f^*e_a
	+ \mu^a ~ f^*\mleft( {}^{E_2}\nabla_{\xi(\nu)} e_a \mright)
\eas
for all $\mu = \mu^a ~ f^*e_a$, where $\mleft( e_a \mright)_a$ is a local frame of $V$. Linearity and the Leibniz rule follow by construction, and the well-definedness and Eq.~\eqref{GeneralPullbackDef} additionally by the first calculation about the compatibility of Leibniz rules.
\end{proof}

What we need is an even more general statement as in Section \ref{NewInfGaugeTrafoTrafos}, with still precisely the same proof as before; recall Prop.~\ref{prop:ClassicFunctionDerivativesAlongPsiEpsilon}.

\begin{corollaries}{Pullbacks of connections just differentiating along one vector field}{VeryGeneralPullbackConnection}
Let $E_i \to N_i$ ($i \in\{1,2\}$) be two Lie algebroids over smooth manifolds $N_i$, $V \to N_2$ a vector bundle, and ${}^{E_2}\nabla$ an $E_2$-connection on $V$. Moreover, let $f \in C^\infty(N_1;N_2)$, $\nu_1 \in \Gamma(E_1)$ and $\nu_2 \in \Gamma(f^*E_2)$ such that
\ba\label{WeakAnchorPreserv}
\mathrm{D}f\bigl(\rho_{E_1}(\nu_1)\bigr)
&=
\mleft(f^*\rho_{E_2}\mright)(\nu_2).
\ea

Then there is a unique $\mathbb{R}$-linear operator $\delta_{\nu_1}: \Gamma(f^*V) \to \Gamma(f^*V)$ with
\ba
\delta_{\nu_1}(h s)
&=
\mathcal{L}_{\nu_1}(h) ~ s
	+ h ~ \delta_{\nu_1} s,
\\
\delta_{\nu_1} (f^*v)
&=
f^*\mleft(
	{}^{E_2}\nabla_{\nu_2} v
\mright)
\ea
for all $s \in \Gamma(f^*V)$, $v \in \Gamma(V)$ and $h \in C^\infty(N_1)$.
\end{corollaries}

\begin{remarks}{Commutating diagram behind pullbacks}{CommutingDiagramOfPullbacks}
Recall Remark \ref{rem:SomeExtraNotationForAnchorBundleMorphs}, the pullback in $\mleft(f^*\rho_{E_2}\mright)(\nu_2)$ in Eq.~\eqref{WeakAnchorPreserv} is just for emphasizing that $\nu_2$ is a section along $f$; one can omit this in the notation, especially if one views sections like $\nu_2$ as a map $N_1 \to E_2$. Then we can equivalently write
\ba
\mathrm{D}f \circ \rho_{E_1}(\nu_1)
&=
\rho_{E_2} \circ \nu_2,
\ea
that is equivalent to that the following diagram commutes
\begin{center}
	\begin{tikzcd}
		N_1 \arrow{r}{\nu_2} \arrow[d, "\rho_{E_1}(\nu_1)", swap]	& E_2 \arrow{d}{\rho_{E_2}} 
		\\
		\mathrm{T}N_1 \arrow{r}{\mathrm{D}f} & \mathrm{T}N_2
	\end{tikzcd}
\end{center}
\end{remarks}

\begin{remark}\label{JustLieDerivativeForGeneralPullbackAndlineBundle}
\leavevmode\newline
\indent $\bullet$ In general one may want to write $\delta_{\nu_1} = \mleft(f^*\mleft( {}^{E_2}\nabla \mright)\mright)_{\nu_1}$, because it is precisely this by uniqueness if a general pullback is possible. But to avoid confusion about the existence of a general pullback we will stick with $\delta_{\nu_1}$, and it will be clear by context which connection and $\nu_2$ is used for the definition of $\delta_{\nu_1}$.

$\bullet$ As in Remark \ref{DdtGleichddt}, in the case of $V = \mathbb{R} \times N_2$, the trivial line bundle over $N_2$, we canonically use ${}^{E_2}\nabla \coloneqq \nabla^0_{\rho_{E_2}}$, where $\nabla^0 \coloneqq \mathrm{d}$. Then one can similarly show as before that 
\bas
\delta_{\nu_1}
&=
\mathcal{L}_{\nu_1}.
\eas
\end{remark}

\begin{proof}[Proof of Cor.~\ref{cor:VeryGeneralPullbackConnection}]
\leavevmode\newline
That is precisely the same proof as in the previous statements and as in Section \ref{ClassicGaugeTheory}; the only difference is just the meaning, $\nu_i$ are fixed sections, but that does not matter in the calculations. Eq.~\eqref{WeakAnchorPreserv} is just the condition about anchor-preservation in the case of a fixed pair of sections, and one uses this equation in the same fashion to how we used an anchor-preserving morphism in the previous proofs. Essentially replace $\nu$ with $\nu_1$ and $\xi(\nu)$ with $\nu_2$ in the proof of Cor.~\ref{cor:GeneralPullbackAnchorPreserving}.
\end{proof}

The advantage of this weak formulation is that we do not need to know whether or not $f$ can be lifted to any morphism with certain properties like anchor-preservation. Eq.~\eqref{WeakAnchorPreserv} states what one needs to make a pullback of a Lie algebroid connection to just differentiate along one direction. That was precisely the idea in the discussion around Prop.~\ref{prop:ClassicFunctionDerivativesAlongPsiEpsilon}, but now more compactly written down, not using flows of the involved vector fields.

\section{\texorpdfstring{Conjugated $E$-connections}{Conjugated Lie algebroid connections}}\label{ConjugateConnections}

Later we will introduce a Lie algebroid connection known as basic connection, and it has a special form which we want to study in a more general sense of conjugated $E$-connections; the name is motivated by \cite[paragraph after Proposition 2.12]{basicconn}, while we especially refer to \cite{blaomTangentBundleAsLieGroup} where the conjugate connections are called dual connections.

\begin{definitions}{Conjugated $E$-connections, \newline \cite[beginning of \S 4.6]{blaomTangentBundleAsLieGroup}}{ConjugationOfConnections}
Let $E \to N$ be a Lie algebroid over a smooth manifold $N$, and $\overline{\nabla}$ be an $E$-connection on $E$. We define its \textbf{conjugated $E$-connection} $\widehat{\nabla}$ by
\ba
\widehat{\nabla}_\mu \nu &\coloneqq \mleft[ \mu, \nu \mright]_E + \overline{\nabla}_\nu \mu
\ea
for all $\mu, \nu \in \Gamma(E)$. We also say that \textbf{$\widehat{\nabla}$ and $\overline{\nabla}$ are conjugate to each other}.
\end{definitions}

\begin{remark}
\leavevmode\newline
It is straightforward to check that the conjugate is an $E$-connection on $E$, linearity over $\mathbb{R}$ is clear, and we have
\bas
\widehat{\nabla}_\mu \mleft( f \nu \mright)
&=
\mleft[ \mu, f \nu \mright]_E + \overline{\nabla}_{f\nu} \mu
=
f ~\widehat{\nabla}_\mu \nu
	+ \mathcal{L}_{\rho(\mu)}(f) ~ \nu, \\
\widehat{\nabla}_{f\mu} \nu
&=
\mleft[ f \mu, \nu \mright]_E + \overline{\nabla}_\nu \mleft( f \mu \mright)
=
f ~\widehat{\nabla}_\mu \nu
	- \mathcal{L}_{\rho(\nu)}(f) ~ \mu
	+ \mathcal{L}_{\rho(\nu)}(f) ~ \mu
=
f ~\widehat{\nabla}_\mu \nu
\eas
for all $\mu, \nu \in \Gamma(E)$ and $f \in C^\infty(N)$,
using the Leibniz rule of the Lie bracket, and that $\overline{\nabla}$ is an $E$-connection. It also makes sense to say that both $E$-connections are conjugate to each other because $\overline{\nabla}$ is also the conjugate to $\widehat{\nabla}$ by definition, that is,
\bas
\mleft[ \mu, \nu \mright]_E + \widehat{\nabla}_\nu \mu
&=
\overline{\nabla}_\mu \nu,
\eas
and the conjugate of a connection is unique, that follows trivially by definition.
\end{remark}

We need several relations between their curvatures and torsions throughout this work.

\begin{corollaries}{Torsion of conjugated $E$-connections \newline \cite[first statement in the first proposition of \S 4.6]{blaomTangentBundleAsLieGroup}}{TorsionOfDualTorsions}
Let $\widehat{\nabla}$ and $\overline{\nabla}$ be two $E$-connections, conjugate to each other, on a Lie algebroid $E \to N$ over a smooth manifold $N$. Then we get for their torsions
\ba
t_{\widehat{\nabla}}(\mu, \nu)
&=
- t_{\overline{\nabla}}(\mu, \nu)
\ea
for all $\mu, \nu \in \Gamma(E)$.
\end{corollaries}

\begin{proof}
\leavevmode\newline
We have
\bas
t_{\widehat{\nabla}}(\mu, \nu)
&=
\widehat{\nabla}_\mu \nu
	- \widehat{\nabla}_\mu \nu
	- \mleft[ \mu, \nu \mright]_E
\\
&=
\mleft[ \mu, \nu\mright]_E
	+ \overline{\nabla}_\nu \mu
	- \mleft[ \nu, \mu\mright]_E
	- \overline{\nabla}_\mu \nu
	- \mleft[ \mu, \nu \mright]_E
\\
&=
\mleft[ \mu, \nu\mright]_E
	+ \overline{\nabla}_\nu \mu
	- \overline{\nabla}_\mu \nu
\\
&=
- t_{\overline{\nabla}}(\mu, \nu)
\eas
for all $\mu, \nu \in \Gamma(E)$.
\end{proof}

\begin{lemmata}{Curvature of conjugated $E$-connections, \newline the first identity comes from \cite[second statement of the first proposition in \S 4.6]{blaomTangentBundleAsLieGroup}}{CurvatureOfDualConnectionsGeneral}
Let $\widehat{\nabla}$ and $\overline{\nabla}$ be two $E$-connections, conjugate to each other, on a Lie algebroid $E \to N$ over a smooth manifold $N$. Then we have for their curvatures
\ba
R_{\overline{\nabla}}(\mu, \nu) \eta
&=
\left(\widehat{\nabla}_\eta t_{\widehat{\nabla}}\right)(\mu, \nu)
	+ R_{\widehat{\nabla}}(\mu, \eta) \nu
	- R_{\widehat{\nabla}}(\nu, \eta) \mu \label{DualCurvaRemainin1} \\
&=
- \mleft(
\widehat{\nabla}_{\eta} \left( \left[ \mu, \nu \right]_E \right)
	- \left[ \widehat{\nabla}_{\eta} \mu, \nu \right]_E
	- \left[ \mu, \widehat{\nabla}_{\eta} \nu \right]_E
	- \widehat{\nabla}_{\overline{\nabla}_\nu \eta} \mu
	+ \widehat{\nabla}_{\overline{\nabla}_\mu \eta} \nu \mright) \label{DualCurvaRemainin2} 
	%\\
%R_{\widehat{\nabla}}(\mu, \nu) \eta
%&=
%\left(\overline{\nabla}_\eta t_{\overline{\nabla}}\right)(\mu, \nu)
	%+ R_{\overline{\nabla}}(\mu, \eta) \nu
	%- R_{\overline{\nabla}}(\nu, \eta) \mu \label{DualCurvaRemainin1} \\
%&=
%- \mleft(
%\overline{\nabla}_{\eta} \left( \left[ \mu, \nu \right]_E \right)
	%- \left[ \overline{\nabla}_{\eta} \mu, \nu \right]_E
	%- \left[ \mu, \overline{\nabla}_{\eta} \nu \right]_E
	%- \overline{\nabla}_{\widehat{\nabla}_\nu \eta} \mu
	%+ \overline{\nabla}_{\widehat{\nabla}_\mu \eta} \nu \mright)
\ea
for all $\mu, \nu, \eta \in \Gamma(E)$.
\end{lemmata}

\begin{remark}
\leavevmode\newline
The second statement is a generalization of what is shown for a special type of connection in \cite[Proposition 2.12]{basicconn}.
\end{remark}

\begin{proof}[Proof of Lemma \ref{lem:CurvatureOfDualConnectionsGeneral}]
\leavevmode\newline	
We will show Eq.~\eqref{DualCurvaRemainin1} by first showing Eq.~\eqref{DualCurvaRemainin2}, but the latter for $R_{\widehat{\nabla}}$ instead of $R_{\overline{\nabla}}$; this does not matter of course, because when we know the formula for one connection, then also for the conjugated connection. Just by the definition of duality and the Jacobi identity we have
\bas
&\overline{\nabla}_{\mu} \left( \left[ \eta, \nu \right]_E \right)
	- \left[ \overline{\nabla}_{\mu} \eta, \nu \right]_E
	- \left[ \eta, \overline{\nabla}_{\mu} \nu \right]_E
	- \overline{\nabla}_{\widehat{\nabla}_\nu \mu} \eta
	+ \overline{\nabla}_{\widehat{\nabla}_\eta \mu} \nu \\
&=
\mleft[ \mu, \mleft[ \eta, \nu \mright]_E \mright]_E
	+ \mleft[ \nu, \mleft[ \mu, \eta \mright]_E \mright]_E 
	+ \mleft[ \eta, \mleft[ \nu, \mu \mright]_E \mright]_E 
\\
&\hspace{1cm}
	- \mleft[ \widehat{\nabla}_{\eta} \mu , \nu \mright]_E
	- \mleft[ \eta, \widehat{\nabla}_{\nu} \mu \mright]_E
	- \mleft[ \widehat{\nabla}_\nu \mu, \eta  \mright]_E
	+ \mleft[ \widehat{\nabla}_\eta \mu, \nu \mright]_E 
\\
&\hspace{1cm}
	+\widehat{\nabla}_\nu \widehat{\nabla}_\eta \mu
	- \widehat{\nabla}_\eta \widehat{\nabla}_\nu \mu
	+ \widehat{\nabla}_{\mleft[ \eta, \nu \mright]_E} \mu
\\
&=
R_{\widehat{\nabla}}(\nu, \eta) \mu
\\
&=
- R_{\widehat{\nabla}}(\eta, \nu) \mu
\eas
for all $\mu, \nu, \eta \in \Gamma(E)$. Eq.~\eqref{DualCurvaRemainin2} is therefore shown, and using this and Cor.~\ref{cor:TorsionOfDualTorsions} we also have
\bas
\left(\widehat{\nabla}_\eta t_{\widehat{\nabla}}\right)(\mu, \nu)
&=
-\left(\widehat{\nabla}_\eta t_{\overline{\nabla}}\right)(\mu, \nu)\\
&=
-\widehat{\nabla}_\eta \left( t_{\overline{\nabla}}(\mu, \nu) \right)
	+ t_{\overline{\nabla}}\left( \widehat{\nabla}_\eta \mu, \nu \right)
	+ t_{\overline{\nabla}}\left( \mu, \widehat{\nabla}_\eta \nu \right) \\
&=
\left[ \eta, \left[\mu, \nu\right]_E - \overline{\nabla}_{\mu} \nu + \overline{\nabla}_{\nu} \mu \right]_E
+ \overline{\nabla}_{\underbrace{ \left[\mu, \nu\right]_E - \overline{\nabla}_{\mu} \nu + \overline{\nabla}_{\nu} \mu }_{=\overline{\nabla}_{\nu} \mu - \widehat{\nabla}_\nu \mu}} \eta \\
&\hspace{1cm}
	+\overline{\nabla}_{\widehat{\nabla}_\eta \mu} \nu
	-\overline{\nabla}_{\nu} \left( \left[ \eta, \mu \right]_E + \overline{\nabla}_{\mu} \eta \right)
	- \left[ \left[ \eta, \mu \right]_E + \overline{\nabla}_{\mu} \eta, \nu \right]_E \\
&\hspace{1cm}+
\overline{\nabla}_{\mu} \left( \left[ \eta, \nu \right]_E + \overline{\nabla}_{\nu} \eta \right)
	- \overline{\nabla}_{\widehat{\nabla}_\eta \nu} \mu
	- \left[ \mu, \left[ \eta, \nu \right]_E + \overline{\nabla}_{\nu} \eta \right]_E \\
&=
\underbrace{\mleft[ \eta, \mleft[\mu, \nu \mright]_E \mright]_E
	+ \left[ \mu, \left[\nu, \eta \right]_E \right]_E
	+ \left[ \nu, \left[\eta, \mu \right]_E \right]_E}_{=0} \\
&\hspace{1cm}+
\underbrace{\overline{\nabla}_{\mu} \left( \left[ \eta, \nu \right]_E \right)
	- \left[ \overline{\nabla}_{\mu} \eta, \nu \right]_E
	- \left[ \eta, \overline{\nabla}_{\mu} \nu \right]_E
	- \overline{\nabla}_{\widehat{\nabla}_\nu \mu} \eta
	+ \overline{\nabla}_{\widehat{\nabla}_\eta \mu} \nu}_{= R_{\widehat{\nabla}}(\nu, \eta) \mu} \\
&\hspace{1cm}
\underbrace{-\overline{\nabla}_{\nu} \left( \left[ \eta, \mu \right]_E \right)
	+ \left[ \overline{\nabla}_{\nu} \eta, \mu \right]_E
	+ \left[ \eta, \overline{\nabla}_{\nu} \mu \right]_E
	+ \overline{\nabla}_{\widehat{\nabla}_\mu \nu} \eta
	- \overline{\nabla}_{\widehat{\nabla}_\eta \nu} \mu}_{= -R_{\widehat{\nabla}}(\mu, \eta) \nu} \\
&\hspace{1cm}
\underbrace{-\overline{\nabla}_{\widehat{\nabla}_\mu \nu} \eta
	+ \overline{\nabla}_{\overline{\nabla}_{\nu} \mu} \eta}_{= - \overline{\nabla}_{\left[ \mu, \nu \right]_E} \eta}
	+ \overline{\nabla}_{\mu} \overline{\nabla}_{\nu} \eta
	- \overline{\nabla}_{\nu} \overline{\nabla}_{\mu} \eta \\
&=
R_{\overline{\nabla}}(\mu, \nu) \eta
	+ R_{\widehat{\nabla}}(\nu, \eta) \mu
	- R_{\widehat{\nabla}}(\mu, \eta) \nu.
\eas
This gives Eq.~\eqref{DualCurvaRemainin1}.
\end{proof}

We are especially interested into the curvature if the conjugated $E$-connection is flat.

\begin{corollaries}{Curvature of conjugated $E$-connections where one connection is flat, \newline \cite[second and third statement of the first proposition in \S 4.6]{blaomTangentBundleAsLieGroup}}{LemmaCurvatureOfDualConnections}
Let $\widehat{\nabla}$ and $\overline{\nabla}$ be two $E$-connections, conjugate to each other, on a Lie algebroid $E \to N$ over a smooth manifold $N$. If $\widehat{\nabla}$ is flat, then
\ba
R_{\overline{\nabla}}(\mu, \nu) \eta
&=
\left(\widehat{\nabla}_\eta t_{\widehat{\nabla}}\right)(\mu, \nu),
\ea
also written as
\ba
R_{\overline{\nabla}}
&=
\widehat{\nabla} t_{\widehat{\nabla}}.
\ea
\end{corollaries}

\begin{proof}
\leavevmode\newline
This simply follows by Lemma \ref{lem:CurvatureOfDualConnectionsGeneral}.
\end{proof}

If both connections conjugate to each other are flat, then we have another Lie bracket by the first Bianchi identity.

\begin{corollaries}{Torsion as Lie bracket}{TOrsionCanBeLieBracketIfFlat}
Let $\widehat{\nabla}$ and $\overline{\nabla}$ be two flat $E$-connections, conjugate to each other, on a Lie algebroid $E \to N$ over a smooth manifold $N$. Then their torsions are Lie brackets for $\Gamma(E)$ which restrict to Lie brackets on the fibres, giving rise to a BLA structure on $E$.
\end{corollaries}

\begin{proof}
\leavevmode\newline
This follows by the flatness of both connections first Bianchi identity in Thm.~\ref{thm:1stBianchi} and Cor.~\ref{cor:LemmaCurvatureOfDualConnections}, the latter implies 
\bas
\widehat{\nabla} t_{\widehat{\nabla}}
&=
0,
\eas
and the former, the first Bianchi identity, then gives
\bas
t_{\widehat{\nabla}}\mleft(t_{\widehat{\nabla}}(\mu, \nu), \eta\mright) 
	+ t_{\widehat{\nabla}}\mleft(t_{\widehat{\nabla}}(\nu, \eta), \mu\mright) 
	+ t_{\widehat{\nabla}}\mleft(t_{\widehat{\nabla}}(\eta, \mu), \nu\mright)
&=
0
\eas
for all $\mu, \nu, \eta \in \Gamma(E)$. Bilinearity and antisymmetry is given, thus, $t_{\widehat{\nabla}}$ is a Lie bracket for $\Gamma(E)$, therefore also $t_{\overline{\nabla}}$ by Cor.~\ref{cor:TorsionOfDualTorsions}. Since torsions are tensors we can conclude that the torsion describes a Lie bracket on each fibre, too.
\end{proof}

\section{Basic connection and the basic curvature}\label{SectionOfBasicConnStuff}

As mentioned and already introduced in a simplified form in Ex.~\ref{ex:ClassicAdRepIsAConnection}, there is also another canonical example of $E$-connection, the \textbf{basic connection} $\nabla^{\text{bas}}$. We follow mainly \cite[\S 2.3]{basicconn}; however, in \cite[\S 3.4]{fernandes} the basic connection is introduced as a certain Bott connection along certain leaves given by the anchor, but we will neither use nor introduce that notion. The basic connection is actually the conjugate connection of $\nabla_\rho$.
%
%\begin{definition}[Bott $E$-connection $\gls{0nablaL}$ on $E$ and $\mathrm{T}M$ along a leaf $L$]
%\leavevmode
%\begin{enumerate}
%\item (Bott $E$-connection on $E$)
%\newline Let $p \in L$ and $\mu \in E_p$ where $\gls{Ep}$ is the fiber of $E$ at $p$. Also let $\eta \in \Gamma(\gls{Ker}\left(\rho\middle|_L\right))$ and (local) sections $\widetilde{\mu}, \widetilde{\eta} \in \Gamma(E)$ such that $\widetilde{\mu}_p = \mu$ and $\left.\widetilde{\eta}\middle|_L \right. = \eta$. Then we define
%\ba
%\nabla^L_\mu \eta \coloneqq \left. \left[ \widetilde{\mu}, \widetilde{\eta} \right]_E\middle|_p \right..
%\label{def:EBott}
%\ea
%\item (Bott $E$-connection on $\mathrm{T}M$)
%\newline Let $\gls{1nu*}(L) \coloneqq \left\{ v \in \left. \mathrm{T}^*M\middle|_L \right. ~ \middle| ~ \left. v\middle|_{\mathrm{T}L} = 0 \right. \right\}$ be the conormal bundle over the leaf $L$, $\omega \in \Gamma(\nu^*(L))$ and $\mu \in E_p$ for $p \in L$. Take (local) sections $\widetilde{\mu} \in \Gamma(E)$ and a (local) 1-form $\widetilde{\omega} \in \Omega^1(M)$ such that $\widetilde{\mu}_p = \mu$ and $\left. \widetilde{\omega}\middle|_L \right. = \omega$. Then we define
%\ba
%\nabla^L_\mu \omega \coloneqq \left(\mathcal{L}_{\rho\mleft(\widetilde{\mu}\mright)} \widetilde{\omega} \middle)\right|_p.
%\label{def:TMBott}
%\ea
%\end{enumerate}
%\end{definition}
%
%In \cite{fernandes} the following is then shown.
%
%\begin{lemma}[Properties of $\nabla^L$]
%\leavevmode\newline
%$\nabla^L$ on $E$ and on $\mathrm{T}M$ associates canonically to each $\mu \in \Gamma(\left. E\middle|_L \right.)$ a linear operator \linebreak $\nabla^L_\mu: \Gamma\left( \mathrm{Ker}\left(\rho|_L\right) \right) \to \Gamma\left( \mathrm{Ker}\left(\rho|_L\right) \right)$ and $\nabla^L_\mu: \Gamma\left( \nu^*\left(L\right) \right) \to \Gamma\left( \nu^*\left(L\right) \right)$, respectively. Both also satisfy similar properties as in the first part of Def. \ref{def:Econnection}.
%\end{lemma}
%
%\begin{remark}
%\leavevmode\newline
%Observe that both $\nabla^L$ are in general not $E$ connections since they are only defined along $E|_L$ and $E|_L$ itself is in general not a Lie algebroid as it can be seen by the Splitting Thm. \ref{thm:splittingLiealgebroid}.
%\end{remark}
%
%Using that we can now define a specific type of $E$-connection, the basic connection as in \cite{fernandes}, Definition 3.3.
%
%\begin{definition}[Basic connection]\label{def:basicconn}
%\leavevmode\newline
%A basic connection is a pair of two $E$-connections, on $E$ and on $\mathrm{T}M$ (and extended canonically to $\mathrm{T}^*M$), both denoted by ${}^E\nabla$ if
%\begin{enumerate}
%\item \textbf{(Compatibility of Lie algebroid structures)} 
%\ba 
%{}^E\nabla \circ \rho = \rho \circ {}^E\nabla, 
%\ea
%\item \textbf{(${}^E\nabla$ restricts to $\nabla^L$ on $E$ on each leaf $L$)}
%\newline For all $\mu, \eta \in \Gamma(E)$ with $\rho(\eta|_L) = 0$ and $p \in L$
%\ba 
%\left({}^E\nabla_\mu \eta\middle)\right|_L = \nabla^L_{\mu|_L} \left(\eta\middle|_L\right).
%\ea
%\item \textbf{(${}^E\nabla$ restricts to $\nabla^L$ on $\mathrm{T}M$ on each leaf $L$)}
%\newline For all $\mu \in \Gamma(E)$, $\omega \in \Omega^1(M)$ with $\omega|_{\mathrm{T}L} = 0$ and $p \in L$
%\ba
%\left({}^E\nabla_\mu \omega\middle)\right|_L = \nabla^L_{\mu|_L} \left(\omega\middle|_L\right).
%\ea
%\end{enumerate}
%\end{definition}
%
%Every Lie algebroid has a basic connection and we will work with the following canonical one, following now \cite[Subsection 2.3]{basicconn}. This connection will be later very important for gauge theory on Lie algebroids as in \cite{CurvedYMH}.

\begin{definitions}{Basic connection, \cite[Definition 2.9]{basicconn}}{CanonicalBasicConnection}
%\leavevmode\newline
Let $E \to N$ be a Lie algebroid over a smooth manifold $N$, and let $\nabla$ be a vector bundle connection on $E$. We then define the \textbf{basic connection (induced by $\nabla$)} as a pair of $E$-connections, one on $E$ itself and the other one on $\mathrm{T}N$, both denoted by $\gls{0nablabas}$.
\begin{enumerate}
\item \textbf{(Basic $E$-connection on $E$)}
\newline The basic connection on $E$ is defined as the conjugate of $\nabla_\rho$, that is,
\ba
\nabla^{\mathrm{bas}}_\mu \nu \coloneqq [\mu, \nu]_E + \nabla_{\rho(\nu)} \mu
\ea
for all $\mu, \nu \in \Gamma(E)$
\item \textbf{(Basic $E$-connection on $\mathrm{T}N$)}
\newline The basic connection on $\mathrm{T}N$ is defined by
\ba
\nabla^{\mathrm{bas}}_\mu X \coloneqq [\rho(\mu), X] + \rho\left( \nabla_X \mu \right)
\ea
for all $\mu \in \Gamma(E)$ and $X \in \mathfrak{X}(N)$
\end{enumerate}
\end{definitions}

\begin{remark}
\leavevmode\newline
It is trivial to see that these are $E$-connections.

In the physics' part, Chapter \ref{GeneralizedGTfas}, we will discuss the use of this connection in physics, as also arising in \cite[discussion around Equation (17)]{CurvedYMH}. Nevertheless one can see here already that one gets the adjoint representation for bundle of Lie algebras, \textit{i.e.}~$\rho \equiv 0$, because then the basic connection on $E$ is just the field of Lie brackets.

In the following we often just write of the "basic connection" or $\nabla^{\mathrm{bas}}$, while we then always mean both connections. It should be clear by context which of both connections we mean then. Similar for its curvature $R_{\nabla^\mathrm{bas}}$; but the torsion $t_{\nabla^{\mathrm{bas}}}$ will only denote the torsion for the basic connection on $E$ since only on $E$ the torsion is formulated.
\end{remark}
%
%One can easily check that $\nabla^{\mathrm{bas}}$ is indeed a basic connection.
%
%\begin{corollary}
%\leavevmode\newline
%$\nabla^{\mathrm{bas}}$ is a basic connection.
%\end{corollary}
%
%\begin{proof}
%\leavevmode\newline
%It is clear by definition that $\nabla^{\mathrm{bas}} \circ \rho = \rho \circ \nabla^{\mathrm{bas}}$ since $\rho$ is a homomorphism. Now let $L$ be a leaf, $p \in L$ and $\mu, \nu \in \Gamma(E)$ with $\rho(\nu|_L) = 0$, then
%\bas
%\nabla^{\mathrm{bas}}_\mu \nu (p) = [\mu, \nu]_E(p) = \nabla^L_{\mu(p)} (\nu|_L),
%\eas
%since $\left( \nabla_{\rho(\nu)} \mu\middle) \right|_L  = 0$ due to the tensorial behaviour of $\nabla$ w.r.t. $\mathfrak{X}(M)$ and $\rho(\nu|_L) = 0$. The definition of $\nabla^{\mathrm{bas}}$ on $\mathrm{T}^*M$ is canonically induced by ($\omega \in \Omega^1(M)$, $X \in \mathfrak{X}(M)$)
%\bas
%\left( \nabla^{\mathrm{bas}}_\mu \omega \right)(X)
%&\coloneqq \mathcal{L}_{\rho(\mu)}(\omega(X)) - \omega\left( \nabla^{\mathrm{bas}}_\mu X \right)
%= \mathcal{L}_{\rho(\mu)}(\omega(X)) - \omega\left( [\rho(\mu), X] + \rho\left( \nabla_X \mu \right) \right) \\
%&= \mathcal{L}_{\rho(\mu)}(\omega(X)) - \omega( \underbrace{[\rho(\mu), X]}_{= \mathcal{L}_{\rho(\mu)}(X)}) - \omega\left(\rho\left( \nabla_X \mu \right) \right)
%= \left( \mathcal{L}_{\rho(\mu)} \omega \right) (X) - \omega\left(\rho\left( \nabla_X \mu \right) \right).
%\eas
%When $w|_{\mathrm{T}L} = 0$ then the last summand vanishes clearly at $p \in L$ and therefore also
%\bas
%\left( \nabla^{\mathrm{bas}}_\mu \omega \right) (p)
%&= \left( \mathcal{L}_{\rho(\mu)}\omega \middle) \right|_p
%= \nabla^L_{\mu(p)} (\omega|_L).
%\eas
%\end{proof}
%
%\section{Basic curvature}
%
We will use the following essential property of the basic connection very often.

\begin{corollaries}{Compatibility of the basic connection with the anchor, \newline \cite[comment after Definition 2.9]{basicconn}}{ENablaMitRhoVertauschung}
Let $E \to N$ be a Lie algebroid over a smooth manifold $N$, and let $\nabla$ be a vector bundle connection on $E$. Then
\ba
\rho\circ\nabla^{\mathrm{bas}}
&=
\nabla^{\mathrm{bas}} \circ \rho.
\ea
\end{corollaries}

\begin{proof}
\leavevmode\newline
We have
\bas
\rho \mleft(
	\nabla^{\mathrm{bas}}_\mu \nu
\mright)
&=
\rho \mleft(
	\mleft[ \mu, \nu \mright]_E
	+ \nabla_{\rho(\nu)} \mu
\mright)
=
\mleft[ \rho(\mu), \rho(\nu) \mright]_E
	+ \rho\mleft( \nabla_{\rho(\nu)} \mu \mright)
=
\nabla^{\mathrm{bas}}_\mu \bigl( \rho(\nu) \bigr)
\eas
for all $\mu,\nu \in \Gamma(E)$, using that the anchor is a homomorphism of Lie brackets.
\end{proof}

As in \cite{CurvedYMH}, we will later see that $\nabla^{\mathrm{bas}}$ should be flat for a given $\nabla$ in order to formulate a gauge theory (among other conditions). Thence, it is important to study the curvature of $\nabla^{\mathrm{bas}}$. Its curvature is encoded in another tensor, the \textbf{basic curvature}.
%With $\gls{Hom}$ we denote the space of Homomorphisms (here of vector bundles).

\begin{definitions}{Basic curvature, \cite[Definition 2.10]{basicconn}}{basiccurvature}
Let $E \to N$ be a Lie algebroid over a smooth manifold $N$, and let $\nabla$ be a connection on $E$. The \textbf{basic curvature $\gls{Rnablabas}$} is then defined as an element of $\Gamma\left(\bigwedge^2E^* \otimes \mathrm{T}^*N \otimes E \right)$ by
\ba
R^{\mathrm{bas}}_\nabla(\mu, \nu) X
&\coloneqq
\nabla_X\mleft(\mleft[\mu, \nu\mright]_E\mright) 
	- \mleft[ \nabla_X \mu, \nu \mright]_E 
	- \mleft[ \mu, \nabla_X \nu \mright]_E 
	- \nabla_{\nabla^{\mathrm{bas}}_\nu X} \mu 
	+ \nabla_{\nabla^{\mathrm{bas}}_\mu X} \nu,
\ea
where $\mu, \nu \in \Gamma(E)$ and $X \in \mathfrak{X}(N)$.
\end{definitions}

\begin{remark}
\leavevmode\newline
\indent $\bullet$ As stated in \cite{basicconn} one may think of this as $\nabla_X([\mu, \nu]_E) - [ \nabla_X \mu, \nu ]_E - [ \mu, \nabla_X \nu ]_E$ which is a measure of the derivation property of $\nabla$ w.r.t. $[\cdot, \cdot]_E$, but corrected in such a way that it is tensoriel in all arguments. For a zero anchor the basic curvature would be equivalent to $\nabla_X([\mu, \nu]_E) - [ \nabla_X \mu, \nu ]_E - [ \mu, \nabla_X \nu ]_E$ since then the basic connection on $\mathrm{T}N$ is identically zero.

$\bullet$ Compare the form of the basic curvature also with Lemma \ref{lem:CurvatureOfDualConnectionsGeneral}.

$\bullet$ It is trivial to see that the basic curvature is antisymmetric in the Lie algebroid arguments and that it is trilinear. Also let $f \in C^\infty(N)$ and observe
\bas
R^{\mathrm{bas}}_\nabla(\mu, \nu) (fX)
&=
\nabla_{fX}\mleft(\mleft[\mu, \nu\mright]_E\mright) 
	- \underbrace{\mleft[ \nabla_{fX} \mu, \nu \mright]_E}
		_{\mathclap{ = f \mleft[ \nabla_{X} \mu, \nu \mright]_E - \mathcal{L}_\nu(f)~ \nabla_X \mu }}
	- \mleft[ \mu, \nabla_{fX} \nu \mright]_E 
	- \underbrace{\nabla_{\nabla^{\mathrm{bas}}_\nu (fX)} \mu}
		_{\mathclap{ = f \nabla_{\nabla^{\mathrm{bas}}_\nu X} \mu + \mathcal{L}_\nu(f) ~ \nabla_X \mu }}
	+ \nabla_{\nabla^{\mathrm{bas}}_\mu (fX)} \nu
\\
&=
f ~ R^{\mathrm{bas}}_\nabla(\mu, \nu) X
\eas
for all $\mu, \nu \in \Gamma(E)$ and $X \in \mathfrak{X}(N)$, and
\bas
R^{\mathrm{bas}}_\nabla(\mu, f \nu) X
&=
\nabla_X\mleft(\mleft[\mu, f\nu\mright]_E\mright) 
	- \mleft[ \nabla_X \mu, f\nu \mright]_E 
	- \mleft[ \mu, \nabla_X (f\nu) \mright]_E 
	- \nabla_{\nabla^{\mathrm{bas}}_{f\nu} X} \mu 
	+ \nabla_{\nabla^{\mathrm{bas}}_\mu X} (f\nu)
\\
&=
f R^{\mathrm{bas}}_\nabla(\mu, \nu) X
\\
&\hspace{1cm}
	+ \mathcal{L}_X(f) ~ \mleft[ \mu, \nu \mright]_E
	+ \mathcal{L}_{\rho(\mu)}(f) ~ \nabla_X (\nu )
	+ \mathcal{L}_X \mathcal{L}_{\rho(\mu)}(f) ~ \nu
	- \mathcal{L}_{\rho(\nabla_X \mu)}(f) ~ \nu
\\
&\hspace{1cm}
	- \mathcal{L}_{\rho(\mu)}(f) ~ \nabla_X \nu 
	- \mathcal{L}_X(f) ~ \mleft[ \mu, \nu \mright]_E
	- \mathcal{L}_{\rho(\mu)} \mathcal{L}_X(f) ~ \nu
	+ \underbrace{\mathcal{L}_{\nabla^{\mathrm{bas}}_\mu X}(f)}
		_{\mathclap{ = \mathcal{L}_{[\rho(\mu), X] + \rho(\nabla_X \mu)}(f) }} 
		~ \nu
\\
&=
f R^{\mathrm{bas}}_\nabla(\mu, \nu) X
	+ \underbrace{\mathcal{L}_X \mathcal{L}_{\rho(\mu)}(f) ~ \nu
	- \mathcal{L}_{\rho(\mu)} \mathcal{L}_X(f) ~ \nu
	- \mathcal{L}_{[X, \rho(\mu)]}(f) ~ \nu}_{= ~ 0}
\\
&=
f R^{\mathrm{bas}}_\nabla(\mu, \nu) X,
\eas
that the basic curvature is also tensorial in $\mu$ follows by the antisymmetry.
\end{remark}

Do not confuse this tensor with $R_{\nabla^{\mathrm{bas}}}$, the curvature of the basic connection, either on $E$ or $\mathrm{T}N$. However, the curvatures are encoded in the basic curvature.

\begin{propositions}{Relations between the curvatures, \newline \cite[Proposition 2.11]{basicconn}, \cite[Equation (9)]{CurvedYMH}, \cite[generalization of second statement of the first proposition in \S 4.6]{blaomTangentBundleAsLieGroup}}{SnablamitREnabla}
Let $E \to N$ be a Lie algebroid over a smooth manifold $N$, and let $\nabla$ be a connection on $E$. Then one has:
\begin{enumerate}
\item The curvature of $\nabla^{\mathrm{bas}}$ on $E$ is equal to $- R_\nabla^{\mathrm{bas}}(\cdot, \cdot) \circ \rho$.
\item The curvature of $\nabla^{\mathrm{bas}}$ on $\mathrm{T}N$ is equal to $- \rho \circ R^{\mathrm{bas}}_\nabla$.
\end{enumerate}
We also have an important relation to the curvature $R_\nabla$ of $\nabla$,
\ba\label{eq:compcondfast}
R_\nabla^{\mathrm{bas}}(\mu, \nu)X 
&= \left( \nabla_X t_{\nabla^{\mathrm{bas}}} \right)(\mu, \nu) 
- R_\nabla(\rho(\mu), X) \nu + R_\nabla(\rho(\nu), X) \mu
\ea
for all $\mu, \nu \in \Gamma(E)$ and $X \in \mathfrak{X}(N)$,
where $t_{\nabla^{\mathrm{bas}}}$ is the $E$-torsion of the basic connection on $E$. 
\end{propositions}

\begin{remark}\label{rem:vanishingbasicconn}
\leavevmode\newline
This implies that both $\nabla^{\mathrm{bas}}$ are flat if $R_\nabla^{\mathrm{bas}} \equiv 0$. The converse is in general not true. But for invertible $\rho$ the converse would hold.
For $R_\nabla^{\mathrm{bas}} \equiv 0$ one also gets 
\ba
( \nabla_X t_{\nabla^{\mathrm{bas}}})(\mu, \nu)
&= R_\nabla(\rho(\mu), X) \nu - R_\nabla(\rho(\nu), X) \mu,
\ea
and by Cor.~\ref{cor:TorsionOfDualTorsions} we also have $t_{\nabla^{\mathrm{bas}}} = - t_{\nabla_\rho}$ such that one can rewrite this with the torsion of $\nabla_\rho$.
\end{remark}

\begin{proof}[Proof of Prop.~\ref{prop:SnablamitREnabla}]
\leavevmode\newline
For the curvature of $\nabla^{\mathrm{bas}}$ on $E$ observe, using Cor.~\ref{cor:ENablaMitRhoVertauschung},
\bas
-R_\nabla^{\mathrm{bas}}(\mu, \nu) \bigl( \rho(\eta)\bigr)
&=
-\mleft(
	\nabla_{\rho(\eta)}([\mu, \nu]_E) 
	- [ \nabla_{\rho(\eta)} \mu, \nu ]_E 
	- [ \mu, \nabla_{\rho(\eta)} \nu ]_E 
	- \nabla_{\nabla^{\mathrm{bas}}_\nu \rho(\eta)} \mu 
	+ \nabla_{\nabla^{\mathrm{bas}}_\mu \rho(\eta)} \nu
\mright)
\\
&=
-\mleft(
	\nabla_{\rho(\eta)}\mleft(\mleft[\mu, \nu\mright]_E\mright) 
	- \mleft[ \nabla_{\rho(\eta)} \mu, \nu \mright]_E 
	- \mleft[ \mu, \nabla_{\rho(\eta)} \nu \mright]_E 
	- \nabla_{\rho\mleft(\nabla^{\mathrm{bas}}_\nu \eta\mright)} \mu 
	+ \nabla_{\rho\mleft(\nabla^{\mathrm{bas}}_\mu \eta\mright)} \nu
\mright)
\\
&\stackrel{\mathclap{ \text{Lem.~\ref{lem:CurvatureOfDualConnectionsGeneral}} }}{=}\quad~
R_{\nabla^{\mathrm{bas}}}(\mu,\nu)\eta
\eas
for all $\mu, \nu, \eta \in \Gamma(E)$.
%\bas
%S_\nabla(\mu, \nu) \rho(\eta)
%&= \nabla_{\rho(\eta)}([\mu, \nu]_E) - [ \nabla_{\rho(\eta)} \mu, \nu ]_E - [ \mu, \nabla_{\rho(\eta)} \nu ]_E - \nabla_{\nabla^{\mathrm{bas}}_\nu \rho(\eta)} \mu + \nabla_{\nabla^{\mathrm{bas}}_\mu \rho(\eta)} \nu \\
%&= \nabla_{\rho(\eta)}([\mu, \nu]_E) - [ \nabla_{\rho(\eta)} \mu, \nu ]_E - [ \mu, \nabla_{\rho(\eta)} \nu ]_E - \nabla_{\rho\left(\nabla^{\mathrm{bas}}_\nu \eta\right)} \mu + \nabla_{\rho\left(\nabla^{\mathrm{bas}}_\mu \eta\right)} \nu \\ 
%&\quad\underbrace{- [\mu, [\nu, \eta]_E]_E - [\nu, [\eta, \mu]_E]_E - [\eta, [\mu, \nu]_E]_E}_{= 0 \text{ (Jacobi)}} \\
%&= \underbrace{\left[ [\mu, \nu]_E, \eta \right]_E + \nabla_{\rho(\eta)}([\mu, \nu]_E)}_{= \nabla^{\mathrm{bas}}_{[\mu, \nu]_E} \eta} - [ \underbrace{[\mu, \eta]_E + \nabla_{\rho(\eta)} \mu}_{= \nabla^{\mathrm{bas}}_\mu \eta}, \nu ]_E + \nabla_{\rho\left(\nabla^{\mathrm{bas}}_\mu \eta\right)} \nu \\
%&\quad - [ \mu, \underbrace{[\nu, \eta]_E + \nabla_{\rho(\eta)} \nu}_{= \nabla^{\mathrm{bas}}_\nu \eta} ]_E - \nabla_{\rho\left(\nabla^{\mathrm{bas}}_\nu \eta\right)} \mu \\
%&= - \left( \nabla^{\mathrm{bas}}_\mu \nabla^{\mathrm{bas}}_\nu \eta - \nabla^{\mathrm{bas}}_\nu \nabla^{\mathrm{bas}}_\mu \eta - \nabla^{\mathrm{bas}}_{[\mu, \nu]_E} \eta \right)
%= - R_{\nabla^{\mathrm{bas}}}(\mu, \nu) \eta
%\eas
In the same fashion as in the proof of Lemma \ref{lem:CurvatureOfDualConnectionsGeneral}, using the Jacobi identity and that $\rho$ is a homomorphism, we also have
\bas
\rho\left( R_\nabla^{\mathrm{bas}}(\mu, \nu) X \right)
&= \rho\left( \nabla_X([\mu, \nu]_E) - [ \nabla_X \mu, \nu ]_E - [ \mu, \nabla_X \nu ]_E - \nabla_{\nabla^{\mathrm{bas}}_\nu X} \mu + \nabla_{\nabla^{\mathrm{bas}}_\mu X} \nu \right) \\
&\hspace{1cm}
	+ [[\rho(\mu), \rho(\nu)], X] + [[\rho(\nu), X], \rho(\mu)] + [[X, \rho(\mu)], \rho(\nu)] \\
&= \underbrace{[\rho([\mu, \nu]_E), X] + \rho\left( \nabla_X([\mu, \nu]_E)\right)}_{= \nabla^{\mathrm{bas}}_{[\mu, \nu]_E}X}
\\
&\hspace{1cm}
+ [\rho(\nu), \underbrace{[\rho(\mu), X] + \rho(\nabla_X\mu)}_{= \nabla^{\mathrm{bas}}_\mu X}]
%&\hspace{1cm} 
	+ \rho\left( \nabla_{\nabla^{\mathrm{bas}}_\mu X} \nu \right)
\\
&\hspace{1cm}
	- [\rho(\mu), \underbrace{[\rho(\nu), X] + \rho(\nabla_X \nu)}_{= \nabla^{\mathrm{bas}}_\nu X}]
	- \rho\left( \nabla_{\nabla^{\mathrm{bas}}_\nu X} \mu \right) 
\\
&=
\nabla^{\mathrm{bas}}_{[\mu, \nu]_E}X
	+ \nabla^{\mathrm{bas}}_\nu \nabla^{\mathrm{bas}}_\mu X
	- \nabla^{\mathrm{bas}}_\mu \nabla^{\mathrm{bas}}_\nu X
\\
&= - R_{\nabla^{\mathrm{bas}}}(\mu, \nu)X
\eas
for all $X \in \mathfrak{X}(N)$.
By Cor.~\ref{cor:TorsionOfDualTorsions} we know that that $t_{\nabla^{\mathrm{bas}}} = - t_{\nabla_\rho}$, thus,
\bas
\left( \nabla_X t_{\nabla^{\mathrm{bas}}} \right)(\mu, \nu) 
&=
-(\nabla_X t_{\nabla_\rho})(\mu, \nu)
\\
&= - \nabla_X \left(t_{\nabla_\rho}(\mu, \nu)\right)
+ t_{\nabla_\rho}(\nabla_X \mu, \nu)
+ t_{\nabla_\rho}(\mu,\nabla_X \nu) \\
&= \nabla_X \left([\mu, \nu]_E - \nabla_{\rho(\mu)} \nu + \nabla_{\rho(\nu)} \mu\right)
\\
&\hspace{1cm} 
+ \nabla_{\rho\left( \nabla_X \mu \right)} \nu - \nabla_{\rho(\nu)} \nabla_X \mu - \left[ \nabla_X \mu, \nu \right]_E \\
&\hspace{1cm}
	+ \nabla_{\rho(\mu)} \nabla_X \nu - \nabla_{\rho\left( \nabla_X \nu \right)} \mu - [\mu, \nabla_X \nu]_E 
\\
&= 
\nabla_X ([\mu, \nu]_E) 
	- \left[ \nabla_X \mu, \nu \right]_E 
	- [\mu, \nabla_X \nu]_E 
	+ \nabla_{\rho\left( \nabla_X \mu \right)} \nu 
	- \nabla_{\rho\left( \nabla_X \nu \right)} \mu \\
&\hspace{1cm}
	+ R_\nabla(\rho(\mu), X) \nu + \nabla_{[\rho(\mu), X]} \nu - R_\nabla(\rho(\nu), X) \mu - \nabla_{[\rho(\nu), X]} \mu \\
&= R^{\mathrm{bas}}_\nabla(\mu, \nu)X + R_\nabla(\rho(\mu), X) \nu - R_\nabla(\rho(\nu), X) \mu.
\eas
\end{proof}

The basic connection on $E$ is conjugate to $\nabla_\rho$ by definition, and it will be later very important that the basic connection is flat for gauge theory as we will see. By our discussion about conjugate Lie algebroid connections we can immediately derive the following by Cor.~\ref{cor:LemmaCurvatureOfDualConnections}.

\begin{theorems}{Curvature of $\nabla_\rho$ for a vanishing basic curvature}{modBianchithm}
Assume $R_\nabla^{\mathrm{bas}} (\cdot, \cdot) \circ \rho = 0$, then we have 
\ba
R_{\nabla_\rho} 
&= \nabla^{\mathrm{bas}} t_{\nabla^{\mathrm{bas}}},\label{eq:BaufOrbit}
\ea
\textit{i.e.}
\bas
R_{\nabla_\rho}(\mu, \nu) \eta = R_\nabla(\rho(\mu), \rho(\nu)) \eta 
&= \left(\nabla^{\mathrm{bas}}_\eta t_{\nabla^{\mathrm{bas}}}\right)(\mu, \nu)
\eas
for all $\mu, \nu, \eta \in \Gamma(E)$.
\end{theorems}

\begin{proof}
\leavevmode\newline
By Prop.~\ref{prop:SnablamitREnabla} we know that the assumption implies that $\nabla^{\mathrm{bas}}$ on $E$ is flat. Thence, we can use Cor.~\ref{cor:LemmaCurvatureOfDualConnections} because of that $\nabla^{\mathrm{bas}}$ on $E$ and $\nabla_\rho$ are conjugate to each other. This concludes the proof.
\end{proof}

\section{Exterior covariant derivatives}\label{ExteriorCovariantDerivativesAoids}

As for standard connections one can now define exterior covariant derivatives related to Lie algebroid connections. 

\begin{definitions}{Exterior covariant derivatives using Lie algebroid connections, \newline \cite[the discussion after Def. 2.2]{basicconn}}{AllgemeineExteriorCovariantDerivativeSch}
Let $E \to N$ be a Lie algebroid over a smooth manifold $N$, ${}^E\nabla$ an $E$-connection on a vector bundle $V \to N$. Then we define the \textbf{exterior covariant derivative $\gls{dEnabla}$} as an operator $\Omega^q(E;V) \to \Omega^{q+1}(E;V)$ ($q \in \mathbb{N}_0$) by
\ba
\left(\mathrm{d}^{{}^E\nabla} \omega \right) (\nu_0, \dots, \nu_q)
&\coloneqq 
\sum_{i = 0}^{q} (-1)^i ~ {}^E\nabla_{\nu_i}\left(\omega\left( \nu_0, \dots, \widehat{\nu}_i, \dots, \nu_q \right) \right) \nonumber \\
&\hspace{1cm}
	+ \sum_{0 \leq i < j \leq q} (-1)^{i+j} \omega( [\nu_i, \nu_j]_E, \nu_0, \dots, \widehat{\nu}_i, \dots, \widehat{\nu}_j, \dots, \nu_q)
\ea
for all $\omega \in \Omega^q(E;V)$ and $\nu_0, \dots, \nu_q \in \Gamma(E)$.
\end{definitions}

\begin{remark}
\leavevmode\newline
That this is a well-defined operator can be shown as in the case of vector bundle connections.
\end{remark}

Moreover, in the case of a connection $\nabla$ on $E$ one has also the previously discussed basic connection $\nabla^{\mathrm{bas}}$ as $E$-connection on $E$ and $\mathrm{T}N$. $\nabla$ is typical vector bundle connection and $\nabla^{\mathrm{bas}}$ a pair of $E$-connections. Hence, it may make sense to look at forms with two degrees, one for $\mathrm{T}N$ and the other one with respect to $E$. 

The following space is also developed and studied by Alexei Kotov, communicated to me in private communication, his studies are planned to be published in 2021.

\begin{definitions}{$(p,q)$-$E$-forms}{ExteriorCovariantDerivatives}
Let $E \to N$ be a Lie algebroid over a smooth manifold $N$, and $V \to N$ a vector bundle.
Then the \textbf{space of $(p,q)$-$E$-forms with values in $V$} ($p, q \in \mathds{N}_0$), will is defined by
\ba
\gls{1ZOmegapq(NEV)} \coloneqq \Gamma\left(\bigwedge^p \mathrm{T}^*N \otimes \bigwedge^q E^* \otimes V\right).
\ea
\end{definitions}

Let us study possible exterior covariant derivatives on this space in the case of $E =V$.

\begin{remarks}{Exterior covariant derivatives induced by $\nabla$}{ExteriorStuffRemark}
Let $E \to N$ be a Lie algebroid over a smooth manifold $N$ and $\nabla$ a connection on $E$.
\newline\newline
$\bullet$ For $q = 0$ one gets the space of $p$-forms with values in $E$, $\Omega^p(N;E)$, or more general, those are forms on $N$ with values in $\bigwedge^q E^* \otimes E$, \textit{i.e.}~
\ba\label{FirstInterpretationOfDoubleDegree}
\Omega^{p, q}(N, E; E) \cong \Omega^p\left(N; \bigwedge^q E^* \otimes E\right).
\ea
\newline\newline
$\bullet$ Analogously 
\ba\label{SecondInterpretationOfDoubleDegree}
\Omega^{p, q}(N, E; E) \cong \Omega^q\left(E; \bigwedge^p \mathrm{T}N^* \otimes E\right).
\ea
\newline\newline
$\bullet$ Using Eq.~\eqref{FirstInterpretationOfDoubleDegree}, denote with $\nabla$ also the canonically induced connection on $\bigwedge^q E^* \otimes E$; then we have a canonical definition of $\mathrm{d}^\nabla$ on $\Omega^{p,q}(N,E;E)$. Since the canonically induced connection on $\bigwedge^q E^* \otimes E$ is defined by using the Leibniz rule, one can rewrite the exterior covariant derivative $\mathrm{d}^\nabla$ of $\omega \in \Omega^{p, q}(N, E; E)$ as an element of $\Omega^{p+1, q}(N, E; E)$ by
\ba
&\mleft(\mathrm{d}^\nabla \omega\mright)\mleft( X_0, \dots, X_p, \nu_1, \dots, \nu_q \mright) 
\nonumber \\
&= 
\sum_{i=0}^p (-1)^i \biggl( \nabla_{X_i} \Bigl( \omega\left(X_0, \dots, \widehat{X}_i, \dots, X_p, \nu_1, \dots, \nu_q\right) \Bigr) 
\nonumber \\
&\hphantom{\sum_{i=0}^p (-1)^i \biggl(} \hspace{1cm}
	- \sum_{j=1}^q \omega\mleft( X_0, \dots, \widehat{X}_i, \dots, X_p, \nu_1, \dots, \nabla_{X_i} \nu_j, \dots, \nu_q \mright) \biggr) 
\nonumber \\
&\hspace{1cm}
	+ \sum_{0 \leq i < j \leq p} (-1)^{i+j} \omega\mleft( [X_i, X_j], X_0, \dots, \widehat{X}_i, \dots, \widehat{X}_j, \dots, X_p, \nu_1, \dots, \nu_q \mright),
\ea
where $X_0, \dots, X_p \in \mathfrak{X}(N)$ and $\nu_1, \dots, \nu_q \in \Gamma(E)$.
\newline\newline
$\bullet$ Similarly one proceeds with $\nabla^{\mathrm{bas}}$, using that the basic connection acts on both, $E$ and $\mathrm{T}N$, such that there is a canonically induced notion of $\nabla^{\mathrm{bas}}$ on $\bigwedge^p \mathrm{T}N^* \otimes E$. By Eq.~\eqref{SecondInterpretationOfDoubleDegree} we have $\mathrm{d}^{\nabla^{\mathrm{bas}}}: \Omega^{p, q}(N, E; E) \to \Omega^{p, q+1}(N, E; E)$ given by
\ba
&\mleft(\mathrm{d}^{\nabla^{\mathrm{bas}}} \omega\mright)\mleft( X_1, \dots, X_p, \nu_0, \dots, \nu_q \mright) 
\nonumber \\
&=
\sum_{i=0}^q (-1)^i \biggl( \nabla^{\mathrm{bas}}_{\nu_i} \bigl( \omega\left(X_1, \dots, X_p, \nu_0, \dots, \widehat{\nu}_i, \dots \nu_q\right) \bigr) 
\nonumber \\
&\hphantom{\sum_{i=0}^q (-1)^i \biggl(} \hspace{1cm}
	- \sum_{j=1}^p \omega\mleft( X_1, \dots, \nabla^{\mathrm{bas}}_{\nu_i} X_j, \dots, X_p, \nu_0, \dots, \widehat{\nu}_i, \dots, \nu_q \mright) \biggr) 
\nonumber \\
&\hspace{1cm}
	+ \sum_{0 \leq i < j \leq q} (-1)^{i+j} \omega\mleft( X_1, \dots, X_p, [\nu_i, \nu_j]_E, \nu_0, \dots, \widehat{\nu}_i, \dots, \widehat{\nu}_j, \dots, \nu_q \mright),
\ea
where $\omega \in \Omega^{p, q}(N, E; E)$, $X_1, \dots, X_p \in \mathfrak{X}(N)$ and $\nu_0, \dots, \nu_q \in \Gamma(E)$.
\newline\newline
$\bullet$ For LABs one can see that $\mathrm{d}^{\nabla^{\mathrm{bas}}}$ acts as the Chevalley-Eilenberg differential $\mathrm{d}_{\mathrm{CE}}$ because the basic connection on $\mathrm{T}N$ is then identically to zero and the one on $E$ is just the adjoint.
\end{remarks}

The commutation of the basic curvature with the anchor carries over to the differential.

\begin{lemmata}{Differential of basic curvature commutes with anchor}{commutationanchordifferential}
Let $E \to N$ be a Lie algebroid over a smooth manifold $N$ and $\nabla$ a connection on $E$.
Then
\ba
\Big( \nabla^{\mathrm{bas}}_\mu \big( \omega \circ \underbrace{(\rho, \dots, \rho)}_{p \text{ times}} \big) \Big) (\nu_1, \dots, \nu_p)
&=
\left(\mathrm{d}^{\nabla^{\mathrm{bas}}} \omega \right)( \rho(\nu_1), \dots, \rho(\nu_p), \mu),
\ea
for all $\omega \in \Omega^{p}(N;E)$ ($p \in \mathbb{N}_0$) and $\mu, \nu_1, \dots \nu_p \in \Gamma(E)$;
in short
\ba
\nabla^{\mathrm{bas}} \left( \omega \circ (\rho, \dots, \rho) \right)
&=
\left(\mathrm{d}^{\nabla^{\mathrm{bas}}} \omega \right) \circ (\rho, \dots, \rho, \mathds{1}_E).
\ea
\end{lemmata}

\begin{proof}
\leavevmode\newline
Recall $\rho \circ \nabla^{\mathrm{bas}} = \nabla^{\mathrm{bas}} \circ \rho$ by Cor.~\ref{cor:ENablaMitRhoVertauschung}, then
\bas
\left( \nabla^{\mathrm{bas}}_\mu \left( \omega \circ (\rho, \dotsc, \rho) \right) \right)(\nu_1, \dots, \nu_p)
&=
\nabla^{\mathrm{bas}}_\mu \bigl( \omega\mleft( \rho(\nu_1), \dotsc, \rho(\nu_p) \mright) \bigr) \\
&\hspace{1cm}
	- \sum_{j=1}^p \omega\Bigl( \rho(\nu_1), \dotsc ,
	\underbrace{\rho\mleft(\nabla^{\mathrm{bas}}_\mu \nu_j\mright)}
		_{= \nabla^{\mathrm{bas}}_\mu (\rho( \nu_j ))},
		\dotsc, \rho(\nu_p) \Bigr) 
\\
&= 
\left( \nabla^{\mathrm{bas}}_\mu \omega \right) (\rho(\nu_1), \dotsc, \rho(\nu_p))
\\
&= 
\left(\mathrm{d}^{\nabla^{\mathrm{bas}}} \omega \right)( \rho(\nu_1), \dotsc, \rho(\nu_p), \mu).
\eas
\end{proof}

Recall that we did not prove the second Bianchi identity in Thm.~\ref{thm:1stBianchi}. We are going to prove the second Bianchi identity using the following theorem.

\begin{theorems}{Second Bianchi identity, \newline \cite[reformulation of Proposition 7.1.9; page 265]{mackenzieGeneralTheory}}{2ndBianchi}
Let $E \to N$ be a Lie algebroid over a smooth manifold $N$, $V \to N$ a vector bundle, and let ${}^E\nabla$ be an $E$-connection on $V$, while we denote its naturally induced definition on $\mathrm{End}(V)$ also ${}^E\nabla$. Viewing its curvature $R_{{}^E\nabla}$ as an element of $\Omega^2(E; \mathrm{End}(V))$ we then have
\ba
\mathrm{d}^{{}^E\nabla} R_{{}^E\nabla} = 0.
\ea
\end{theorems}

\begin{proof}[Proof of Thm.~\ref{thm:2ndBianchi}]
\leavevmode\newline
Let $\mu, \nu, \eta \in \Gamma(E)$ and $v \in \Gamma(V)$, then
\bas
\left(\left(\mathrm{d}^{{}^E\nabla} R_{{}^E\nabla}\right)(\mu, \nu, \eta)\right) (v)
&=
\Bigl( {}^E\nabla_\mu \left( R_{{}^E\nabla}(\nu, \eta) \right) - {}^E\nabla_\nu \left( R_{{}^E\nabla}(\mu, \eta) \right) + {}^E\nabla_\eta \left( R_{{}^E\nabla}(\mu, \nu) \right) \\
&\hphantom{\Bigl(}\hspace{1cm}
	- R_{{}^E\nabla}([\mu, \nu]_E, \eta) + R_{{}^E\nabla}([\mu, \eta]_E, \nu) - R_{{}^E\nabla}([\nu, \eta]_E, \mu) \Bigr)(v)
\\
&=
{}^E\nabla_\mu\left( R_{{}^E\nabla}(\nu, \eta)v \right) - R_{{}^E\nabla}(\nu, \eta) \left( {}^E\nabla_\mu v\right)
\\
&\hspace{1cm}
- {}^E\nabla_\nu\left( R_{{}^E\nabla}(\mu, \eta)v \right) + R_{{}^E\nabla}(\mu, \eta) \left( {}^E\nabla_\nu v\right) 
\\
&\hspace{1cm}
	+ {}^E\nabla_\eta\left( R_{{}^E\nabla}(\mu, \nu)v \right) - R_{{}^E\nabla}(\mu, \nu) \left( {}^E\nabla_\eta v\right) 
\\
&\hspace{1cm}
	- R_{{}^E\nabla}([\mu, \nu]_E, \eta)v + R_{{}^E\nabla}([\mu, \eta]_E, \nu)v - R_{{}^E\nabla}([\nu, \eta]_E, \mu)v 
\\
&=
{}^E\nabla_\mu {}^E\nabla_\nu {}^E\nabla_\eta v - {}^E\nabla_\mu {}^E\nabla_\eta {}^E\nabla_\nu v - {}^E\nabla_\mu {}^E\nabla_{[\nu, \eta]_E} v 
\\
&\hspace{1cm}
	- {}^E\nabla_\nu {}^E\nabla_\eta {}^E\nabla_\mu v + {}^E\nabla_\eta {}^E\nabla_\nu {}^E\nabla_\mu v + {}^E\nabla_{[\nu, \eta]_E} {}^E\nabla_\mu v 
\\
&\hspace{1cm}
	- {}^E\nabla_\nu {}^E\nabla_\mu {}^E\nabla_\eta v + {}^E\nabla_\nu {}^E\nabla_\eta {}^E\nabla_\mu v + {}^E\nabla_\nu {}^E\nabla_{[\mu, \eta]_E} v 
\\
&\hspace{1cm}
	+ {}^E\nabla_\mu {}^E\nabla_\eta {}^E\nabla_\nu v - {}^E\nabla_\eta {}^E\nabla_\mu {}^E\nabla_\nu v - {}^E\nabla_{[\mu, \eta]_E} {}^E\nabla_\nu v 
\\
&\hspace{1cm}
	+ {}^E\nabla_\eta {}^E\nabla_\mu {}^E\nabla_\nu v - {}^E\nabla_\eta {}^E\nabla_\nu {}^E\nabla_\mu v - {}^E\nabla_\eta {}^E\nabla_{[\mu, \nu]_E} v 
\\
&\hspace{1cm}
	- {}^E\nabla_\mu {}^E\nabla_\nu {}^E\nabla_\eta v + {}^E\nabla_\nu {}^E\nabla_\mu {}^E\nabla_\eta v + {}^E\nabla_{[\mu, \nu]_E} {}^E\nabla_\eta v 
\\
&\hspace{1cm}
	- {}^E\nabla_{[\mu, \nu]_E} {}^E\nabla_\eta v + {}^E\nabla_\eta {}^E\nabla_{[\mu, \nu]_E} v + {}^E\nabla_{\left[ [\mu, \nu]_E, \eta \right]_E}v 
\\
&\hspace{1cm}
	+ {}^E\nabla_{[\mu, \eta]_E} {}^E\nabla_\nu v - {}^E\nabla_\nu {}^E\nabla_{[\mu, \eta]_E} v - {}^E\nabla_{\left[ [\mu, \eta]_E, \nu \right]_E}v 
\\
&\hspace{1cm}
	- {}^E\nabla_{[\nu, \eta]_E} {}^E\nabla_\mu v + {}^E\nabla_\mu {}^E\nabla_{[\nu, \eta]_E} v + {}^E\nabla_{\left[ [\nu, \eta]_E, \mu \right]_E}v
\\
&= 
0,
\eas
where we also used the Jacobi identity.
\end{proof}

\begin{remarks}{Proof of the second Bianchi identity of Thm.~\ref{thm:1stBianchi}}{FinallyTheOtherBianchiStuff}
We can now finally prove the second statement of Thm.~\ref{thm:1stBianchi} by showing that it is equivalent to Thm.~\ref{thm:2ndBianchi} if $V=E$; for $\mu, \nu, \eta \in \Gamma(E)$ we have
\bas
&\mleft( {}^E\nabla_\mu R_{{}^E\nabla}\mright)(\nu, \eta) + \mleft( {}^E\nabla_\nu R_{{}^E\nabla}\mright)(\eta, \mu) + \mleft( {}^E\nabla_\eta R_{{}^E\nabla}\mright)(\mu, \nu)
\\
&\hspace{1cm}
	+ R_{{}^E\nabla}\left( t_{{}^E\nabla}(\mu, \nu), \eta \right) 
	+ R_{{}^E\nabla}\left( t_{{}^E\nabla}(\nu, \eta), \mu \right)
	+ R_{{}^E\nabla}\left( t_{{}^E\nabla}(\eta, \mu), \nu \right) \\
&=
{}^E\nabla_\mu \left( R_{{}^E\nabla} (\nu, \eta) \right) - R_{{}^E\nabla} \left( {}^E\nabla_\mu \nu, \eta \right) - R_{{}^E\nabla}\left(\nu, {}^E\nabla_\mu \eta\right) 
\\
&\hspace{1cm}
	+ {}^E\nabla_\nu \left( R_{{}^E\nabla} (\eta, \mu) \right) - R_{{}^E\nabla} \left( {}^E\nabla_\nu \eta, \mu \right) - R_{{}^E\nabla}\left(\eta, {}^E\nabla_\nu \mu\right) \\
&\hspace{1cm}
	+ {}^E\nabla_\eta \left( R_{{}^E\nabla} (\mu, \nu) \right) - R_{{}^E\nabla} \left( {}^E\nabla_\eta \mu, \nu \right) - R_{{}^E\nabla}\left(\mu, {}^E\nabla_\eta \nu\right) \\
&\hspace{1cm}
	+ R_{{}^E\nabla}\left( {}^E\nabla_\mu \nu - {}^E\nabla_\nu \mu - [\mu, \nu]_E, \eta \right)
+ R_{{}^E\nabla}\left( {}^E\nabla_\nu \eta - {}^E\nabla_\eta \nu - [\nu, \eta]_E, \mu \right) \\
&\hspace{1cm}
	+ R_{{}^E\nabla}\left( {}^E\nabla_\eta \mu - {}^E\nabla_\mu \eta - [\eta, \mu]_E, \nu \right) \\
&=
{}^E\nabla_\mu \left( R_{{}^E\nabla} (\nu, \eta) \right) - {}^E\nabla_\nu \left( R_{{}^E\nabla} (\mu, \eta) \right) + {}^E\nabla_\eta \left( R_{{}^E\nabla} (\mu, \nu) \right)
\\
&\hspace{1cm}
	- R_{{}^E\nabla}([\mu, \nu]_E, \eta)
	+ R_{{}^E\nabla}([\mu, \eta]_E, \nu) - R_{{}^E\nabla}([\nu, \eta]_E, \mu) \\
&=
\mleft( \mathrm{d}^{{}^E\nabla} R_{{}^E\nabla} \mright)(\mu, \nu, \eta) 
\\
&\stackrel{\mathclap{ \text{Thm.~\ref{thm:2ndBianchi}} }}{=}\quad~~
0.
\eas
So, both formulations are equivalent for $V=E$, but Thm.~\ref{thm:2ndBianchi} is valid for any vector bundle $V$ and, thus, more general.
\end{remarks}

\begin{remark}
\leavevmode\newline
With a similar calculation as in Remark \ref{rem:FinallyTheOtherBianchiStuff} one can also rewrite the first Bianchi identity of Thm.~\ref{thm:1stBianchi} to
\bas
R_{{}^E\nabla}(\mu, \nu) \eta + R_{{}^E\nabla}(\nu, \eta) \mu + R_{{}^E\nabla}(\eta, \mu) \nu 
&=
\mleft( \mathrm{d}^{{}^E\nabla} t_{{}^E\nabla} \mright)(\mu, \nu, \eta)
\eas
for all $\mu, \nu, \eta \in \Gamma(E)$. Be careful, the right hand side is not the same as \textit{e.g.}~in Thm.~\ref{thm:modBianchithm}, \textit{i.e.}~not the same as ${}^E\nabla t_{{}^E\nabla}$ because the torsion is an element of $\Omega^{0,2}(N,E;E)$ such that ${}^E\nabla$ and $\mathrm{d}^{{}^E\nabla}$ do act differently.
\end{remark}

It is now natural to ask whether there is some usable commutation relation between both differentials, $\mathrm{d}^\nabla$ and $\mathrm{d}^{\nabla^{\mathrm{bas}}}$ for a fixed connection $\nabla$.

\begin{propositions}{Commutation relation}{commutationrelation}
Let $E \to N$ be a Lie algebroid over a smooth manifold $N$ and $\nabla$ a connection on $E$. Then
\ba
&\mleft(\mathrm{d}^\nabla \mathrm{d}^{\nabla^{\mathrm{bas}}} \omega\mright)\mleft( X_0, \dots, X_p, \nu_0, \dots, \nu_q \mright) \nonumber \\
&= \mleft(\mathrm{d}^{\nabla^{\mathrm{bas}}} \mathrm{d}^\nabla \omega\mright) \mleft( X_0, \dots, X_p, \nu_0, \dots, \nu_q \mright) \nonumber \\
&\hspace{0.8cm}
	+ \sum_{i=0}^p \sum_{k=0}^q (-1)^{i+k} R_\nabla^{\mathrm{bas}}\mleft(\nu_k, \omega\mleft( X_0, \dots, \widehat{X}_i, \dots, X_p, \nu_0, \dots, \widehat{\nu}_k, \dots, \nu_q \mright)\mright)X_i \nonumber \\
&\hspace{0.8cm}
	+ \sum_{i=0}^p \sum_{k=0}^q (-1)^{i+k} R_\nabla\Big( X_i, \rho\mleft( \omega \mleft( X_0, \dots, \widehat{X}_i, \dots, X_p, \nu_0, \dots, \widehat{\nu}_k, \dots, \nu_q \mright) \mright)\Big) \nu_k \nonumber \\
&\hspace{0.8cm}
	+ \sum_{\substack{ i,j=0 \\ i < j }}^p \sum_{k=0}^q (-1)^{i+j+k} \omega\mleft( \rho\mleft( R_\nabla\mleft( X_i, X_j \mright)\nu_k \mright), X_0, \dots, \widehat{X}_i, \dots, \widehat{X}_j, \dots, X_p, \nu_0, \dots, \widehat{\nu}_k, \dots, \nu_q \mright) \nonumber \\
&\hspace{0.8cm}
	+ \sum_{i=0}^p \sum_{\substack{ k,l=0 \\ k < l  }}^q (-1)^{i+k+l} \omega\mleft( X_0, \dots, \widehat{X}_i, \dots, X_p, R_\nabla^{\mathrm{bas}}(\nu_k, \nu_l)X_i, \nu_0, \dots, \widehat{\nu}_k, \dots, \widehat{\nu}_l, \dots, \nu_q \mright)
\ea
for all $\omega \in \Omega^{p, q}(N, E; E)$ ($p, q \in\mathbb{N}_0$), $X_0, \dots, X_p \in \mathfrak{X}(N)$ and $\nu_0, \dots, \nu_q \in \Gamma(E)$.
\end{propositions}

\begin{remark}
\leavevmode\newline
If $\nabla$ is flat and if $R_\nabla^{\mathrm{bas}} = 0$, then one has simply
\ba\label{eq:flatcommutation}
\mathrm{d}^\nabla \mathrm{d}^{\nabla^{\mathrm{bas}}} \omega
&= \mathrm{d}^{\nabla^{\mathrm{bas}}} \mathrm{d}^\nabla \omega.
\ea
Both differentials, $\mathrm{d}^\nabla$ and $\mathrm{d}^{\nabla^{\mathrm{bas}}}$, square to zero (recall Prop. \ref{prop:SnablamitREnabla})\footnote{As for vector bundle connections, one can also show for general Lie algebroid connections that the square of their exterior covariant derivatives is directly related to their curvature. We will not need this and the statements about $\mathrm{d}_1$ and $\mathrm{d}_2$, hence, we do not show this. But the calculation is precisely the same.} and, so, also the differentials
\ba
\omega &\mapsto 
\mathrm{d}_1 \omega
\coloneqq
\left( \mathrm{d}^\nabla + (-1)^p \mathrm{d}^{\nabla^{\mathrm{bas}}} \right) \omega, 
\label{def:differential1} \\
\omega &\mapsto 
\mathrm{d}_2 \omega
\coloneqq
\left( (-1)^q \mathrm{d}^\nabla + \mathrm{d}^{\nabla^{\mathrm{bas}}} \right) \omega 
\label{def:differential2}
\ea
for all $\omega \in \Omega^{p,q}(N,E;E)$, that can be seen by
\bas
\mathrm{d}_1^2 \omega
&= \mathrm{d}_1 \left( \mathrm{d}^\nabla + (-1)^p \mathrm{d}^{\nabla^{\mathrm{bas}}} \right) \omega \\
&= \left( \mathrm{d}^\nabla + (-1)^{p+1} \mathrm{d}^{\nabla^{\mathrm{bas}}} \right) \mathrm{d}^\nabla \omega
+ \left( \mathrm{d}^\nabla + (-1)^p \mathrm{d}^{\nabla^{\mathrm{bas}}} \right) (-1)^p \mathrm{d}^{\nabla^{\mathrm{bas}}} \omega \\
&= \underbrace{\left(\mathrm{d}^\nabla\right)^2}_{=0} \omega + \underbrace{\left( \mathrm{d}^{\nabla^{\mathrm{bas}}} \right)^2}_{=0} \omega + (-1)^{p+1} \mathrm{d}^{\nabla^{\mathrm{bas}}} \mathrm{d}^\nabla \omega + (-1)^p \mathrm{d}^\nabla \mathrm{d}^{\nabla^{\mathrm{bas}}} \omega \\
&= (-1)^p \left( \mathrm{d}^\nabla \mathrm{d}^{\nabla^{\mathrm{bas}}} - \mathrm{d}^{\nabla^{\mathrm{bas}}} \mathrm{d}^\nabla \right) \omega \\
&\stackrel{\mathclap{\text{Eq. } \eqref{eq:flatcommutation}}}{=} \quad~ 
0,
\eas
similarly with $\mathrm{d}_2$.

For $\nu \in \Gamma(E)$ one gets
\ba
\mleft[ \mathrm{d}^{\nabla^{\mathrm{bas}}}, \mathrm{d}^\nabla \mright] \nu
&=
\iota_\nu R_\nabla^{\mathrm{bas}} + \iota_{\rho(\nu)} R_\nabla,
\ea
here, $\gls{1jota}$ denotes the contraction. Especially for flat $\nabla$, $R_\nabla^{\mathrm{bas}}$ describes the commutation relation of both exterior covariant derivatives.
\end{remark}

\begin{proof}[Proof of Prop.~\ref{prop:commutationrelation}]
\leavevmode\newline
That is an extremely long and tedious but completely straightforward calculation. There is no trick to use, "just" insert the definitions of all tensors and exterior covariant derivatives on both sides of the equation and compare.
\end{proof}

We can immediately conclude the following.

\begin{corollaries}{Commutation for vanishing basic curvature}{commutationS=0}
Let $E \to N$ be a Lie algebroid over a smooth manifold $N$ and $\nabla$ a connection on $E$.
Then $R_\nabla^{\mathrm{bas}} = 0$ if and only if
\ba\label{EasyDifferentialCommutationSGleichNuuull}
&\mleft(\mathrm{d}^\nabla \mathrm{d}^{\nabla^{\mathrm{bas}}} \omega\mright)\mleft( X_0, \dots, X_p, \nu_0, \dots, \nu_q \mright) \nonumber \\
&= \mleft(\mathrm{d}^{\nabla^{\mathrm{bas}}} \mathrm{d}^\nabla \omega\mright) \mleft( X_0, \dots, X_p, \nu_0, \dots, \nu_q \mright) \nonumber \\
&\hspace{0.8cm}
	+ \sum_{i=0}^p \sum_{k=0}^q (-1)^{i+k} R_\nabla\Big( X_i, \rho\mleft( \omega \mleft( X_0, \dots, \widehat{X}_i, \dots, X_p, \nu_0, \dots, \widehat{\nu}_k, \dots, \nu_q \mright) \mright)\Big) \nu_k \nonumber \\
&\hspace{0.8cm}
	+ \sum_{\substack{ i,j=0 \\ i<j }}^p \sum_{k=0}^q (-1)^{i+j+k} \omega\mleft( \rho\mleft( R_\nabla\mleft( X_i, X_j \mright)\nu_k \mright), X_0, \dots, \widehat{X}_i, \dots, \widehat{X}_j, \dots, X_p, \nu_0, \dots, \widehat{\nu}_k, \dots, \nu_q \mright)
\ea
for all $\omega \in \Omega^{p, q}(N, E; E)$ ($p,q \in \mathbb{N}_0$), $X_0, \dots, X_p \in \mathfrak{X}(N)$ and $\nu_0, \dots, \nu_q \in \Gamma(E)$.
\end{corollaries}

\begin{remark}
\leavevmode\newline
The "$\Rightarrow$"-direction was also found by Alexei Kotov. While I have derived it with the more general previous proposition, Alexei Kotov has directly shown it from the point of view of differentialgraded manifolds. This was communicated in a personal communication but there is a paper planned about that by Alexei Kotov and Thomas Strobl, planned for 2021.
\end{remark}

\begin{proof}[Proof of Cor.~\ref{cor:commutationS=0}]
\leavevmode\newline
The "$\Rightarrow$" direction, \textit{i.e.}~we assuming a vanishing basic curvature, is clear by Prop.~\ref{prop:commutationrelation}. For the "$\Leftarrow$" direction we want to use Eq.~\eqref{eq:compcondfast} in Prop.~\ref{prop:SnablamitREnabla}. Observe that
\bas
\mleft(\mathrm{d}^{\nabla^{\mathrm{bas}}} \mathds{1}_E\mright) (\mu, \nu)
&=
\nabla^{\mathrm{bas}}_\mu \nu - \nabla^{\mathrm{bas}}_\nu \mu - [\mu, \nu]_E
=
t_{\nabla^{\mathrm{bas}}}(\mu, \nu)
\eas
for all $\mu, \nu \in \Gamma(E)$,
and
\bas
\mleft( \mathrm{d}^\nabla \mathds{1}_E \mright)(X, \mu)
&= \mleft( \nabla_X \mathds{1}_E\mright) (\mu)
= \nabla_X \mu - \nabla_X \mu
= 0
\eas
for all $X \in \mathfrak{X}(N)$ and $\mu \in \Gamma(E)$.
Using these and by choosing $\omega = \mathds{1}_E \in \Omega^{0,1}(M,E;E)$ we have by Eq.~\eqref{EasyDifferentialCommutationSGleichNuuull}
\bas
&&\mleft(\mathrm{d}^\nabla t_{\nabla^{\mathrm{bas}}} \mright)(X, \mu, \nu)
&=
\mleft(\mathrm{d}^\nabla \mathrm{d}^{\nabla^{\mathrm{bas}}} \mathds{1}_E \mright)(X, \mu, \nu)
= 
R_\nabla(X, \rho(\nu))\mu - R_\nabla(X, \rho(\mu))\nu \\
&\stackrel{\text{Eq. } \eqref{eq:compcondfast}}{\Leftrightarrow}&
R_\nabla^{\mathrm{bas}}(\mu, \nu)X 
&= \underbrace{\mleft( \nabla_X t_{\nabla^{\mathrm{bas}}} \mright)(\mu, \nu)}_{= \mleft(\mathrm{d}^\nabla t_{\nabla^{\mathrm{bas}}} \mright)(X, \mu, \nu)}
- R_\nabla(\rho(\mu), X) \nu + R_\nabla(\rho(\nu), X) \mu
= 0.
\eas
\end{proof}
%
%\section{Implications of vanishing basic curvature}
%
%We already have the discussed the first implication of $R_\nabla^{\mathrm{bas}} = 0$ in Rem. \ref{rem:vanishingbasicconn}.
%
%\begin{corollary}[Vanishing basic connection]\label{cor:basicisflat}
%\leavevmode\newline
%Let $\nabla$ be a connection on $E$ such that $R_\nabla^{\mathrm{bas}} = 0$, then
%\ba
%R_{\nabla^{\mathrm{bas}}} = 0 \quad \mathrm{(both)}.
%\ea
%\end{corollary}
%
%\begin{proof}
%\leavevmode\newline
%That both basic connections are flat follows by Prop. \ref{prop:SnablamitREnabla} and $R_\nabla^{\mathrm{bas}} = 0$.
%\end{proof}
%
%The condition about the existence of the $B$ function is already satisfied along the leafs $L$ when one knows that the basic curvature vanishes.
%
%We prove Thm. \ref{thm:modBianchithm} by showing the following; for this we need to define that two $E$-connections $\widehat{\nabla}$ and $\overline{\nabla}$ are dual to each other when we have $\widehat{\nabla}_\mu \nu = \mleft[ \mu, \nu \mright]_E + \overline{\nabla}_\nu \mu$; it is then obvious that we have $t_{\widehat{\nabla}} = - t_{\overline{\nabla}}$ for their $E$-torsions.
%
%\begin{lemma}[Curvature of dual $E$-connections]\label{LemmaCurvatureOfDualConnections}
%\leavevmode\newline
%Let $\widehat{\nabla}$ and $\overline{\nabla}$ be two dual $E$-connections. When $\widehat{\nabla}$ is flat then
%\ba
%R_{\overline{\nabla}}(\mu, \nu) \eta
%&=
%\left(\widehat{\nabla}_\eta t_{\widehat{\nabla}}\right)(\mu, \nu).
%\ea
%\end{lemma}
%
%\begin{proof}
%\leavevmode\newline	
%We have
%\bas
%\left(\widehat{\nabla}_\eta t_{\widehat{\nabla}}\right)(\mu, \nu)
%&=
%-\left(\widehat{\nabla}_\eta t_{\overline{\nabla}}\right)(\mu, \nu)\\
%&=
%-\widehat{\nabla}_\eta \left( t_{\overline{\nabla}}(\mu, \nu) \right)
	%+ t_{\overline{\nabla}}\left( \widehat{\nabla}_\eta \mu, \nu \right)
	%+ t_{\overline{\nabla}}\left( \mu, \widehat{\nabla}_\eta \nu \right) \\
%&=
%\left[ \eta, \left[\mu, \nu\right]_E - \overline{\nabla}_{\mu} \nu + \overline{\nabla}_{\nu} \mu \right]_E
%+ \overline{\nabla}_{\underbrace{ \left[\mu, \nu\right]_E - \overline{\nabla}_{\mu} \nu + \overline{\nabla}_{\nu} \mu }_{=\overline{\nabla}_{\nu} \mu - \widehat{\nabla}_\nu \mu}} \eta \\
%&\quad+ 
%\overline{\nabla}_{\widehat{\nabla}_\eta \mu} \nu
	%-\overline{\nabla}_{\nu} \left( \left[ \eta, \mu \right]_E + \overline{\nabla}_{\mu} \eta \right)
	%- \left[ \left[ \eta, \mu \right]_E + \overline{\nabla}_{\mu} \eta, \nu \right]_E \\
%&\quad+
%\overline{\nabla}_{\mu} \left( \left[ \eta, \nu \right]_E + \overline{\nabla}_{\nu} \eta \right)
	%- \overline{\nabla}_{\widehat{\nabla}_\eta \nu} \mu
	%- \left[ \mu, \left[ \eta, \nu \right]_E + \overline{\nabla}_{\nu} \eta \right]_E \\
%&=
%\underbrace{\mleft[ \eta, \mleft[\mu, \nu \mright]_E \mright]_E
	%+ \left[ \mu, \left[\nu, \eta \right]_E \right]_E
	%+ \left[ \nu, \left[\eta, \mu \right]_E \right]_E}_{=0} \\
%&\quad+
%\underbrace{\overline{\nabla}_{\mu} \left( \left[ \eta, \nu \right]_E \right)
	%- \left[ \overline{\nabla}_{\mu} \eta, \nu \right]_E
	%- \left[ \eta, \overline{\nabla}_{\mu} \nu \right]_E
	%- \overline{\nabla}_{\widehat{\nabla}_\nu \mu} \eta
	%+ \overline{\nabla}_{\widehat{\nabla}_\eta \mu} \nu}_{= \mleft[ \mu, \mleft[ \eta, \nu \mright]_E \mright]_E + \mleft[ \nu, \mleft[ \mu, \eta \mright]_E \mright]_E + \mleft[ \eta, \mleft[ \nu, \mu \mright]_E \mright]_E + R_{\widehat{\nabla}}(\nu, \eta) \mu = 0} \\
%&\quad
%\underbrace{-\overline{\nabla}_{\nu} \left( \left[ \eta, \mu \right]_E \right)
	%+ \left[ \overline{\nabla}_{\nu} \eta, \mu \right]_E
	%+ \left[ \eta, \overline{\nabla}_{\nu} \mu \right]_E
	%+ \overline{\nabla}_{\widehat{\nabla}_\mu \nu} \eta
	%- \overline{\nabla}_{\widehat{\nabla}_\eta \nu} \mu}_{= 0} \\
%&\quad
%\underbrace{-\overline{\nabla}_{\widehat{\nabla}_\mu \nu} \eta
	%+ \overline{\nabla}_{\overline{\nabla}_{\nu} \mu} \eta}_{= - \overline{\nabla}_{\left[ \mu, \nu \right]_E} \eta}
	%+ \overline{\nabla}_{\mu} \overline{\nabla}_{\nu} \eta
	%- \overline{\nabla}_{\nu} \overline{\nabla}_{\mu} \eta \\
%&=
%R_{\overline{\nabla}}(\mu, \nu) \eta.
%\eas
%\end{proof}
%
%\begin{remark}
%\leavevmode\newline
%As one can see in the proof, with similar calculations, one has in general
%\ba
%R_{\overline{\nabla}}(\mu, \nu) \eta
%&=
%\left(\widehat{\nabla}_\eta t_{\widehat{\nabla}}\right)(\mu, \nu)
	%+ R_{\widehat{\nabla}}(\mu, \eta) \nu
	%- R_{\widehat{\nabla}}(\nu, \eta) \mu \\
%&=
%- \mleft(
%\widehat{\nabla}_{\eta} \left( \left[ \mu, \nu \right]_E \right)
	%- \left[ \widehat{\nabla}_{\eta} \mu, \nu \right]_E
	%- \left[ \mu, \widehat{\nabla}_{\eta} \nu \right]_E
	%- \widehat{\nabla}_{\overline{\nabla}_\nu \eta} \mu
	%+ \widehat{\nabla}_{\overline{\nabla}_\mu \eta} \nu \mright), \\
%R_{\widehat{\nabla}}(\mu, \nu) \eta
%&=
%- \mleft(
%\overline{\nabla}_{\eta} \left( \left[ \mu, \nu \right]_E \right)
	%- \left[ \overline{\nabla}_{\eta} \mu, \nu \right]_E
	%- \left[ \mu, \overline{\nabla}_{\eta} \nu \right]_E
	%- \overline{\nabla}_{\widehat{\nabla}_\nu \eta} \mu
	%+ \overline{\nabla}_{\widehat{\nabla}_\mu \eta} \nu \mright),
%\ea
%especially compare the right hand sides with Eq. \eqref{eq:compcondfast} and with the definition of the basic curvature, Def. \ref{def:basiccurvature}. One could say that the curvature of one $E$-connection is the (reduced) basic curvature of its dual $E$-connection.
%\end{remark}
%
%We can also apply the first Bianchi identity to get an expression of $R_{\nabla_\rho}$ without $\nabla^{\mathrm{bas}}$.
%
%\begin{proposition}[First version of modified Bianchi identity]\label{prop:modBianchiprop}
%\leavevmode\newline
%Assume $R_\nabla^{\mathrm{bas}} = 0$, then for all $\mu, \nu, \eta \in \Gamma(E)$
%\ba
%R_{\nabla_\rho}(\mu, \nu)\eta
%&=
%t_{\nabla_\rho}\left( t_{\nabla_\rho}(\mu, \nu), \eta  \right)
%+ t_{\nabla_\rho}\left( t_{\nabla_\rho}(\nu, \eta), \mu  \right)
%+ t_{\nabla_\rho}\left( t_{\nabla_\rho}(\eta, \mu), \nu  \right)
%+ \left( \nabla_{\rho(\mu)} t_{\nabla_\rho} \right)(\nu, \eta) \nonumber \\
%&\quad+
%\left( \nabla_{\rho(\nu)} t_{\nabla_\rho} \right)(\eta, \mu).
%\ea
%\end{proposition}
%
%\begin{proof}
%\leavevmode\newline
%This follows by combining Eq. \eqref{eq:compcondfast} for $R_\nabla^{\mathrm{bas}} = 0$ with the first Bianchi identity \eqref{eq:firstBianchi} (\textit{i.e.} one uses the Bianchi identity for $R_{\nabla_\rho}$ and cancels the $\nabla_{\rho(\eta)} t_{\nabla_{\rho}} = - \nabla_{\rho(\eta)} t_{\nabla^{\mathrm{bas}}}$ with two curvature terms).
%\end{proof}
%
%Comparing Prop. \ref{prop:modBianchiprop} and Thm. \ref{thm:modBianchithm} one gets the following.
%
%\begin{corollary}[Simplified Bianchi identity on the leaves]
%\leavevmode\newline
%Assume $R_\nabla^{\mathrm{bas}} = 0$, then
%\ba
%0&=
%2t_{\nabla_\rho}\left( t_{\nabla_\rho}(\mu, \nu), \eta  \right)
%+ 2t_{\nabla_\rho}\left( t_{\nabla_\rho}(\nu, \eta), \mu  \right)
%+ 2t_{\nabla_\rho}\left( t_{\nabla_\rho}(\eta, \mu), \nu  \right) \\
%&\quad+
%\left( \nabla_{\rho(\mu)} t_{\nabla_\rho} \right)(\nu, \eta)
%+ \left( \nabla_{\rho(\nu)} t_{\nabla_\rho} \right)(\eta, \mu)
%+ \left( \nabla_{\rho(\eta)} t_{\nabla_\rho} \right)(\mu, \nu)
%\ea
%and therefore, by the first Bianchi identity \eqref{eq:firstBianchi},
%\ba
%&R_{\nabla_\rho}(\mu, \nu) \eta + R_{\nabla_\rho}(\nu, \eta) \mu + R_{\nabla_\rho}(\eta, \mu) \nu \nonumber &&&&&&&& \\
%&=
%-t_{\nabla_\rho}(t_{\nabla_\rho}(\mu, \nu), \eta) - t_{\nabla_\rho}(t_{\nabla_\rho}(\nu, \eta), \mu) - t_{\nabla_\rho}(t_{\nabla_\rho}(\eta, \mu), \nu)  \nonumber \\
%&=
%\frac{1}{2} \left(\left(\nabla_{\rho(\mu)} t_{\nabla_\rho}\right)(\nu, \eta)
%+ \left(\nabla_{\rho(\nu)} t_{\nabla_\rho}\right)(\eta, \mu) + \left(\nabla_{\rho(\eta)} t_{\nabla_\rho}\right)(\mu, \nu) \right).
%\ea
%\end{corollary}
%
%\begin{proof}
%\leavevmode\newline
%Observe that ($\mu, \nu, \eta \in \Gamma(E)$)
%\bas
%\nabla^{\mathrm{bas}}_\mu \nu 
%&=
%\nabla_{\rho(\mu)} \nu
%+ t_{\nabla^{\mathrm{bas}}}(\mu, \nu),
%\eas
%then ($t_{\nabla^{\mathrm{bas}}} = - t_{\nabla_\rho}$)
%\bas
%&&\left( \nabla^{\mathrm{bas}}_\eta t_{\nabla^{\mathrm{bas}}} \right)(\mu, \nu)
%&=
%\nabla^{\mathrm{bas}}_\eta \left( t_{\nabla^{\mathrm{bas}}}(\mu, \nu) \right)
%- t_{\nabla^{\mathrm{bas}}}\left(\nabla^{\mathrm{bas}}_\eta \mu, \nu\right)
%- t_{\nabla^{\mathrm{bas}}}\left(\mu, \nabla^{\mathrm{bas}}_\eta \nu\right) \\
%&&&=
%\nabla_{\rho(\eta)} (t_{\nabla^{\mathrm{bas}}}(\mu, \nu))
%+ t_{\nabla^{\mathrm{bas}}}(\eta, t_{\nabla^{\mathrm{bas}}}(\mu, \nu))
%- t_{\nabla^{\mathrm{bas}}}\left( \nabla_{\rho(\eta)}\mu, \nu \right) \\
%&&&\quad-
%t_{\nabla^{\mathrm{bas}}}(t_{\nabla^{\mathrm{bas}}}(\eta, \mu), \nu)
%- t_{\nabla^{\mathrm{bas}}}\left(\mu, \nabla_{\rho(\eta)}\nu\right)
%- t_{\nabla^{\mathrm{bas}}}(\mu, t_{\nabla^{\mathrm{bas}}}(\eta, \nu)) \\
%&&&=
%- t_{\nabla_\rho}\left( t_{\nabla_\rho}(\mu, \nu), \eta  \right)
%- t_{\nabla_\rho}\left( t_{\nabla_\rho}(\nu, \eta), \mu  \right)
%- t_{\nabla_\rho}\left( t_{\nabla_\rho}(\eta, \mu), \nu  \right) \\
%&&&\quad-
%\left( \nabla_{\rho(\eta)} t_{\nabla_\rho} \right) (\mu, \nu) \\
%&&&\stackrel{\mathclap{\text{Thm. } \ref{thm:modBianchithm}}}{=}\quad~
%R_{\nabla_\rho}(\mu,\nu)\eta\\
%&&&\stackrel{\mathclap{\text{Prop. } \ref{prop:modBianchiprop}}}{=}\quad~
%t_{\nabla_\rho}\left( t_{\nabla_\rho}(\mu, \nu), \eta  \right)
%+ t_{\nabla_\rho}\left( t_{\nabla_\rho}(\nu, \eta), \mu  \right)
%+ t_{\nabla_\rho}\left( t_{\nabla_\rho}(\eta, \mu), \nu  \right) \\
%&&&\quad+
%\left( \nabla_{\rho(\mu)} t_{\nabla_\rho} \right)(\nu, \eta)
%+ \left( \nabla_{\rho(\nu)} t_{\nabla_\rho} \right)(\eta, \mu) \\
%&\Leftrightarrow&
%0&=
%2t_{\nabla_\rho}\left( t_{\nabla_\rho}(\mu, \nu), \eta  \right)
%+ 2t_{\nabla_\rho}\left( t_{\nabla_\rho}(\nu, \eta), \mu  \right)
%+ 2t_{\nabla_\rho}\left( t_{\nabla_\rho}(\eta, \mu), \nu  \right) \\
%&&&\quad+
%\left( \nabla_{\rho(\mu)} t_{\nabla_\rho} \right)(\nu, \eta)
%+ \left( \nabla_{\rho(\nu)} t_{\nabla_\rho} \right)(\eta, \mu)
%+ \left( \nabla_{\rho(\eta)} t_{\nabla_\rho} \right)(\mu, \nu).
%\eas
%\end{proof}
\section{Direct product of Lie algebroids}\label{DirectProdsOfLieAlgoids}

We will also need to know how to define the direct products of Lie algebroids where we especially refer to \cite[Lemma 6.25]{meinrenkenlie} or \cite[beginning of \S 4.2; page 155]{mackenzieGeneralTheory}.

In the following we will have two Lie algebroids $(E_1, \mleft[ \cdot, \cdot \mright]_{E_1}, \rho_1) \to N_1$ and $(E_2, \mleft[ \cdot, \cdot \mright]_{E_2}, \rho_2) \to N_2$ over two smooth manifolds $N_1$ and $N_2$. With $\gls{pri}: N_1 \times N_2 \to N_i$ we will denote in the following part of this section the projection onto the $i$-th factor ($i \in \{1,2\}$), and $\mathrm{T}\mleft( N_1 \times N_2 \mright)$ can be regarded as the Whitney sum of vector bundles $\mathrm{pr}_1^*\mleft( \mathrm{T}N_1 \mright) \oplus \mathrm{pr}_2^*\mleft( \mathrm{T}N_2 \mright)$, as usual and as mentioned in \cite{mackenzieGeneralTheory}. We want to define a Lie algebroid structure on $\mathrm{pr}_1^*\mleft( E_1 \mright) \oplus \mathrm{pr}_2^*\mleft( E_2 \mright) \to N_1 \times N_2$ (Whitney sum of $\mathrm{pr}_i^*\mleft( E_i \mright)$), and, thus, a canonical candidate of the anchor is immediately given by $\mathrm{pr}_1^*\rho_{E_1} \oplus \mathrm{pr}_2^*\rho_{E_2}$.

Sections of $\mathrm{pr}_i^*\mleft( E_i \mright)$ can be viewed as compositions of the form $\mu^a ~ \mathrm{pr}_i^*\mleft( V^i_a \mright)$, where $V^i_a \in \Gamma(E_i)$ and $\mu^a \in C^\infty\mleft( N_1 \times N_2 \mright)$, simply using that pullbacks of sections generate all sections. Using such decompositions has the advantage that the frames are given by (pullbacks of) frames of $E_i$, especially, $\mathrm{pr}_i^*\mleft( V^i_a \mright)$ (no sum over $i$) is constant along $N_j$, $j \neq i$. We then say that we take a \textbf{frame induced by $E_1$ and $E_2$}.

\begin{lemmata}{Uniqueness of the Lie algebroid structure on $E_1 \times E_2$, \newline \cite[Lemma 6.25]{meinrenkenlie} \newline \cite[beginning of \S 4.2; page 155]{mackenzieGeneralTheory}}{LemmaUniquenessOfDirectProductStructure}
Let $(E_1, \mleft[ \cdot, \cdot \mright]_{E_1}, \rho_1) \to N_1$ and $(E_2, \mleft[ \cdot, \cdot \mright]_{E_2}, \rho_2) \to N_2$ be two Lie algebroids over two smooth manifolds $N_1$ and $N_2$, and let $E_1 \times E_2 \coloneqq \mathrm{pr}_1^*\mleft( E_1 \mright) \oplus \mathrm{pr}_2^*\mleft( E_2 \mright) \to N_1 \times N_2$ be the Whitney sum of vector bundles, equipped with the direct product of anchors. Then there is a unique Lie algebroid structure on $E_1 \times E_2$ such that
\ba\label{defGenerationOfDirectProductSections}
\Gamma(E_1) \oplus \Gamma(E_2) &\to \Gamma\mleft(E_1 \times E_2\mright),
\nonumber \\
(\mu, \nu) &\mapsto \mathrm{pr}_1^*\mu \oplus \mathrm{pr}_2^*\nu
=
\mleft( \mathrm{pr}_1^*\mu, \mathrm{pr}_2^*\nu \mright)
\ea
is a Lie algebra homomorphism, where $\Gamma(E_i)$ are viewed as (infinite-dimensional) Lie algebras.
\end{lemmata}

\begin{remark}
\leavevmode\newline
With the direct product of anchors we mean here
\bas
\rho_{E_1 \times E_2}
\coloneqq
\rho_{E_1} \times \rho_{E_2}
&\coloneqq
\mathrm{pr}_1^*\rho_{E_1} \oplus \mathrm{pr}_2^*\rho_{E_2}.
\eas
\end{remark}

\begin{proof}[Sketch of the proof of Lemma \ref{lem:LemmaUniquenessOfDirectProductStructure}]
\leavevmode\newline
We just give a sketch of the proof since the calculations are all very straightforward, but tedious to write down explicitly; the construction is as usual, making use of that some certain subset of sections generate all sections and that one knows how to define structures on that subset given by the map in \eqref{defGenerationOfDirectProductSections}. The full structure then uniquely follows by forcing the Leibniz rule on the Lie bracket.

In the following we will also omit all the pullback notations, so, when we write for example that we take a section of $\Gamma(E_1)$, then we actually mean a pullback of that section along $\mathrm{pr}_1$. Especially, we understand $\Gamma(E_1) \oplus \Gamma(E_2)$ as embedded in the sense of \eqref{defGenerationOfDirectProductSections}.

$\bullet$ For the existence we define the Lie bracket $\mleft[ \cdot, \cdot \mright]_{E_1 \times E_2}$ as in the following: Let $\mleft( f_a^{(i)} \mright)_a$ be a frame of $E_i$ ($i \in \{1,2\}$) and their pullbacks give combined a frame of $E_1 \times E_2$ which we denote by $\mleft( e_a \mright)_a$; note that $e_a \in \Gamma(E_1) \oplus \Gamma(E_2)$. The bracket $\mleft[ e_a, e_b \mright]_{E_1 \times E_2}$ of this frame is then canonically defined as direct product of the brackets $\mleft[ \cdot, \cdot \mright]_{E_1}$ and $\mleft[ \cdot, \cdot \mright]_{E_2}$ given by the direct product of Lie algebras $\Gamma(E_1) \oplus \Gamma(E_2)$. Making use of that $\Gamma(E_1) \oplus \Gamma(E_2)$ generates $\Gamma(E_1 \times E_2)$, we then write for two sections $\mu = \mu^a e_a, \nu = \nu^a e_a \in \Gamma(E_1 \times E_2)$, and we then apply the typical construction to force the Leibniz rule on the full set of sections,
\ba\label{1defLieBracketOfDirectProductAlgebroids}
\mleft[ \mu, \nu \mright]_{E_1 \times E_2}
&\coloneqq
\mu^a \nu^b ~ \mleft[ e_a, e_b \mright]_{E_1 \times E_2}
	+ \mu^a \mathcal{L}_{\rho_{E_1 \times E_2}(e_a)}\mleft(\nu^b\mright) ~ e_b
	- \nu^b \mathcal{L}_{\rho_{E_1 \times E_2}(e_b)}\mleft(\mu^a\mright) ~ e_a,
\ea
where $\rho_{E_1 \times E_2} = \rho_{E_1} \times \rho_{E_2}$ is the direct product of anchors. This is well-defined, because any other frames $\mleft( f_a^{(i)} \mright)_a$ are locally related by a matrix on $N_i$, so, a change constant along $N_j$ ($j \in \{1,2\}$, $i \neq j$). Hence, $E_1$-$E_2$-mixed terms of $\mleft[ e_a, e_b \mright]_{E_1 \times E_2}$ are unaffected by a change of such frames, and, so, it is still a direct product of Lie brackets for another frame. Especially, it follows that the bracket is the direct product of the brackets on $\Gamma(E_1) \oplus \Gamma(E_2)$. That the whole bracket is independent of the chosen frame is also trivial and straightforward to check; that essentially follows by construction since the Lie derivatives $\mathcal{L}_{\rho_{E_1 \times E_2}(e_a)}$ will cancel the Leibniz rule of $\mleft[ e_a, e_b \mright]_{E_1 \times E_2}$ when changing the frame.

The calculations that this gives a Lie algebroid structure is now straightforward, similar to the proof of Prop.~\ref{prop:ActionLieoidsAreOids}. That is, the curvature of $\rho_{E_1 \times E_2}$ is trivially the direct product of the curvature of $\rho_{E_1}$ and $\rho_{E_2}$
\bas
R_{\rho_{E_1 \times E_2}}
&=
R_{\rho_{E_1}} \times R_{\rho_{E_2}}
\eas
recall Def.~\ref{def:GeneralDefOfCurvMorphisms}. That simply follows by the fact that the anchor is a direct product and that the Lie bracket is a direct product on $\Gamma(E_1) \oplus \Gamma(E_2)$, so, the curvature is a direct product in the frame $(e_a)$, and therefore always because the curvature is a tensor (Lemma \ref{lem:KruemmungenSindTensorenMitAnkerErhaltung}) and $\Gamma(E_1) \oplus \Gamma(E_2)$ generates $\Gamma(E_1 \times E_2)$. Since $E_i$ are Lie algebroids, the curvature is zero.

The Lie bracket clearly satisfies the Leibniz rule with respect to $\rho_{E_1 \times E_2}$, and hence by Prop.~\ref{prop:MeasureofJacobiandHomom}, we can test the Jacobi identity in a given frame; by construction, with respect to the frame $(e_a)$ the bracket is a direct product of Lie brackets given by the direct product of Lie algebras $\Gamma(E_1) \oplus \Gamma(E_2)$. So, Jacobi identity immediately follows.

$\bullet$ That the map defined in \eqref{defGenerationOfDirectProductSections} is a Lie algebra homomorphism follows by construction since the anchor and the Lie bracket are defined as direct products on $\Gamma(E_1) \oplus \Gamma(E_2)$.

$\bullet$ Uniqueness will follow by using that $\Gamma(E_1 \times E_2)$ is generated by $\Gamma(E_1) \oplus \Gamma(E_2)$ as a module over $C^\infty(N_1 \times N_2)$ using the map defined in \eqref{defGenerationOfDirectProductSections}, now denoted by $\Phi$. Since $\Phi$ shall be a homomorphism, the bracket on $\Gamma(E_1) \oplus \Gamma(E_2)$ embedded into $\Gamma(E_1 \times E_2)$ is given by the direct product of $\mleft[ \cdot, \cdot \mright]_{E_1}$ and $\mleft[ \cdot, \cdot \mright]_{E_2}$ in sense of Lie algebras; similarly as for $\mathfrak{X}(N_1) \oplus \mathfrak{X}(N_2)$. Then take any Lie algebroid bracket on $E_1 \times E_2$ such that $\Phi$ is a homomorphism and express sections with respect to $(e_a)_a$. Using the Leibniz rule, every other possible Lie bracket has then the form of \eqref{1defLieBracketOfDirectProductAlgebroids}, therefore uniqueness is given.
\end{proof}
%
%\begin{definition}[Direct product of Lie algebroids]\label{DefDirectProductOfLiealgebroids}
%\leavevmode\newline
%Let $(E_1, \mleft[ \cdot, \cdot \mright]_{E_1}, \rho_1) \to M_1$ and $(E_2, \mleft[ \cdot, \cdot \mright]_{E_2}, \rho_2) \to M_2$ be two Lie algebroids, then their direct product $E_1 \times E_2$ is defined as the Whitney sum $\mathrm{pr}_1^*\mleft( E_1 \mright) \oplus \mathrm{pr}_2^*\mleft( E_2 \mright)$ equipped with the direct product of the anchors as anchor, \textit{i.e.}
%\ba\label{DefDirectProductOfAnchors}
%\rho_{E_1 \times E_2} \mleft( \mleft\{ \mu^a ~ \mathrm{pr}_1^*\mleft( V^1_a \mright) \mright\} \oplus \mleft\{ \nu^b ~ \mathrm{pr}_2^*\mleft( V^2_b \mright) \mright\} \mright)
%&\coloneqq
%\mleft\{ \mu^a ~ \mathrm{pr}_1^*\mleft( \rho_1\mleft( V^1_a\mright) \mright) \mright\} \oplus \mleft\{ \nu^b ~ \mathrm{pr}_2^*\mleft( \rho_2\mleft( V^2_b \mright) \mright) \mright\},
%\ea
%and the Lie bracket for two elements $\xi \coloneqq \mleft\{ \mu^a ~ \mathrm{pr}_1^*\mleft( V^1_a \mright) \mright\} \oplus \mleft\{ \nu^b ~ \mathrm{pr}_2^*\mleft( V^2_b \mright) \mright\}$ and $\psi \coloneqq \mleft\{ \eta^c ~ \mathrm{pr}_1^*\mleft( W^1_c \mright) \mright\} \oplus \mleft\{ \gamma^d ~ \mathrm{pr}_2^*\mleft( W^2_d \mright) \mright\}$ is given by
%\ba\label{defLieBracketOfDirectProductAlgebroids}
%\mleft[ \xi, \psi \mright]_{E_1 \times E_2}
%&\coloneqq\quad~~
%\bigg\{ \mu^a \eta^c ~ \mathrm{pr}_1^*\mleft(\mleft[ V^1_a, W^1_c \mright]_{E_1}\mright) \nonumber \\
	%&\qquad\quad~~
	%+ \mathcal{L}_{\rho_{E_1 \times E_2}\mleft( \xi\mright)} (\eta^c) ~ \mathrm{pr}_1^*\mleft( W^1_c \mright)
	%- \mathcal{L}_{\rho_{E_1 \times E_2}\mleft( \psi\mright)} (\mu^a) ~ \mathrm{pr}_1^*\mleft( V^1_a \mright) \bigg\} \nonumber \\
	%&\qquad
	%\oplus 
		%\bigg\{ \nu^b \gamma^d ~ \mathrm{pr}_2^*\mleft(\mleft[ V^2_b, W^2_d \mright]_{E_2}\mright) \nonumber \\
		%&\qquad\quad~~
		%+ \mathcal{L}_{\rho_{E_1 \times E_2}\mleft( \xi\mright)} (\gamma^d) ~ \mathrm{pr}_2^*\mleft( W^2_d \mright)
		%- \mathcal{L}_{\rho_{E_1 \times E_2}\mleft( \psi\mright)} (\nu^b) ~ \mathrm{pr}_2^*\mleft( V^2_b \mright) \bigg\}.
%\ea
%\end{definition}
%
%\begin{remark}
%\leavevmode\newline
%\indent $\bullet$ The idea of the Lie bracket is that the mixed terms for sections of $\mathrm{pr}_1^*(E_1)$ which are constant along $N_2$ (these sections we denote by $\Gamma(E_1)$) and sections of $\mathrm{pr}_2^*(E_2)$ which are constant along $N_1$ (which we also denote by $\Gamma(E_2)$)\footnote{Similarly we define the same notation for $\mathfrak{X}(N_i)$.} vanish, while the Leibniz rule is then forced for the remaining sections. Due to the desired Leibniz rule one can not make all mixed terms to zero; see also the following Lemma \ref{LemmaUniquenessOfDirectProductStructure}.
%
%$\bullet$ In the following, especially for calculations in explicit examples, we will often omit the $\mathrm{pr}_i^*$, and we are going to write "$+$" instead of "$\oplus$". As in the proof of \cite[Lemma 2.6]{meinrenkenlie}, taking a local frame $\mleft(e_a^1\mright)_a$ of $E_1$ and a local frame $\mleft(e_a^2\mright)_a$ of $E_2$ one can canonically construct a local frame $\mleft( e_a \mright)_a$ of $E_1 \times E_2$ as usual for direct products of vector bundles, then we get for all $f, h \in C^\infty(N_1 \times N_2)$
%\ba\label{MeinrenkensNotationForDirectProductBracket}
%\mleft[ f^a e_a, h^b e_b \mright]_{E_1 \times E_2}
%&=
%f^a h^b ~ \mleft[ e_a, e_b \mright]_{E_1 \times E_2}
	%+ f^a \mathcal{L}_{\rho_{E_1 \times E_2}(e_a)}(h^b) ~ e_b
	%- h^b \mathcal{L}_{\rho_{E_1 \times E_2}(e_b)}(f^a) ~ e_a,
%\ea
%where $\mleft[ e_a, e_b \mright]_{E_1 \times E_2}$ calculates as for a direct product of Lie algebras.
%\end{remark}
%
%Moreover, as shown in \cite[Lemma 2.6]{meinrenkenlie} this is a unique Lie algebroid structure given some natural condition.

Hence, we define:

\begin{definitions}{Direct product of Lie algebroids}{DirecProductOfLieAlgebroids}
Let $(E_1, \mleft[ \cdot, \cdot \mright]_{E_1}, \rho_1) \to N_1$ and $(E_2, \mleft[ \cdot, \cdot \mright]_{E_2}, \rho_2) \to N_2$ be two Lie algebroids over two smooth manifolds $N_1$ and $N_2$, and let $E_1 \times E_2 \coloneqq \mathrm{pr}_1^*\mleft( E_1 \mright) \oplus \mathrm{pr}_2^*\mleft( E_2 \mright) \to N_1 \times N_2$ be the Whitney sum of vector bundles.

Then we call the Lie algebroid structure as given in Lemma \ref{lem:LemmaUniquenessOfDirectProductStructure} the \textbf{direct product of Lie algebroids}.
\end{definitions}

There are some examples of direct products, especially also the Higgs mechanism of the standard model.

\begin{examples}{Examples of direct products of Lie algebroids}{ExamplesOfDirectProductsLieAoids}
We provide two canonical examples; the first one directly comes by the construction for which we viewed $\mathrm{T}\mleft( N_1 \times N_2 \mright)$ as the Whitney sum $\mathrm{pr}_1^*\mleft( \mathrm{T}N_1 \mright) \oplus \mathrm{pr}_2^*\mleft( \mathrm{T}N_2 \mright)$.
\begin{enumerate}
	\item The first example is the direct product of two tangent bundles, $E_i \coloneqq \mathrm{T}N_i$ where the Lie brackets are the ones from the tangent bundles and $\rho_i \coloneqq \mathds{1}_{\mathrm{T}N_i}$. Then $E_1 \times E_2 = \mathrm{T}\mleft( N_1 \times N_2 \mright)$.
	\item Let $E_1$ be the action Lie algebroid of the electroweak interaction, see Ex. \ref{ex:electroweakinteractionasLiealgoid}, and $E_2$ be the Lie algebra $\mathrm{su}(3) \to \{*\}$ over a point set $\{*\}$ (with zero anchor). Then $E_1 \times E_2$ is called the \textbf{Higgs mechanism of the standard model}.
\end{enumerate}
\end{examples}

As usual, if we have several structures given on both factors, then we can often take their product to define a similar structure on the whole product of Lie algebroids. For tensors and connections this is straightforward, however, we also have Lie algebroid connections and we have seen that pullbacks of those may not always been given; especially recall Cor.~\ref{cor:GeneralPullbackAnchorPreserving}, that is, anchor-preserving vector bundle morphisms are needed.

\begin{lemmata}{Projections have lifts to anchor-preserving morphisms}{LiftsOfProjections}
Let $(E_1, \mleft[ \cdot, \cdot \mright]_{E_1}, \rho_1) \to N_1$ and $(E_2, \mleft[ \cdot, \cdot \mright]_{E_2}, \rho_2) \to N_2$ be two Lie algebroids over two smooth manifolds $N_1$ and $N_2$, and let $E_1 \times E_2$ be the direct product of Lie algebroids.

Then the projections $\pi_i: E_1 \times E_2 \to E_i$ ($i \in \{1,2\}$) are anchor preserving vector bundle morphisms over $\mathrm{pr}_i: N_1 \times N_2 \to N_i$.
\end{lemmata}

\begin{remark}
\leavevmode\newline
To clarify: $\pi_i$ project to $E_i \to N_i$ as Lie algebroid, not onto $\mathrm{pr}_i^*E_i \to N_1 \times N_2$. However, extended to sections, $\pi_i$ maps to $\Gamma(\mathrm{pr}_i^*E_i)$; recall Remark \ref{rem:SomeExtraNotationForAnchorBundleMorphs}.
\end{remark}

\begin{proof}[Proof of Lemma \ref{lem:LiftsOfProjections}]
\leavevmode\newline
$\pi_i$ are clearly vector bundle morphisms by definition. Denote with $p_i$ the projection of the bundle $E_i \stackrel{p_i}{\to} N_i$, similarly $p$ the projection of $E_1 \times E_2 \stackrel{p}{\to} N_1 \times N_2$, then
\bas
p_i \circ \pi_i
&=
\mathrm{pr}_i \circ p
\eas
by definition, \textit{i.e.}~using that $E_1 \times E_2 = \mathrm{pr}_1^*E_1 \oplus \mathrm{pr}_2^*E_2$. Hence, $\pi_i$ are vector bundle morphisms over $\mathrm{pr}_i$. Therefore we only need to check the anchor-preservation, that is, observe that with precisely the same arguments
\bas
\mathrm{Dpr}_1: \mathrm{pr}_1^*\mathrm{T}N_1 \oplus \mathrm{pr}_2^*\mathrm{T}N_2
&\to
\mathrm{T}N_1,
\\
(X, Y)
&\mapsto
X
\eas
is a vector bundle morphism over $\mathrm{pr}_1$ as it is also well-known, similarly for $\mathrm{Dpr}_2$.\footnote{Essentially, the $\mathrm{Dpr}_i$ are the "$\pi_i$ for $E_i = \mathrm{T}N_i$".} Then
\bas
\mleft(\mathrm{Dpr}_i \circ
	\rho_{E_1 \times E_2}
\mright)(\mu_1, \mu_2)
&=
\mathrm{Dpr}_i \mleft(
	\rho_{E_1 \times E_2}(\mu_1, \mu_2)
\mright)
\\
&=
\mathrm{Dpr}_i \Bigl(
	\bigl( (\mathrm{pr}_1^*\rho_{E_1})(\mu_1), (\mathrm{pr}_2^*\rho_{E_2})(\mu_2) \bigr)
\Bigr)
\\
&=
(\mathrm{pr}_i^*\rho_{E_i})(\mu_i)
\\
&=
(\mathrm{pr}_i^*\rho_{E_i}\circ\pi_i)(\mu_1, \mu_2)
\eas
for all $(\mu_1, \mu_2) \in \Gamma(E_1 \times E_2)$. Thus, $\pi_i$ is anchor-preserving; also recall Remark \ref{rem:SomeExtraNotationForAnchorBundleMorphs}.
\end{proof}

By Cor.~\ref{cor:GeneralPullbackAnchorPreserving} we can therefore also make pullbacks of Lie algebroid connections along those projections. As a conclusion of this section, let us summarize and introduce the following.

\begin{remarks}{Products of inherited structures}{NotationAboutProductStructures}
Let $(E_1, \mleft[ \cdot, \cdot \mright]_{E_1}, \rho_1) \to N_1$ and $(E_2, \mleft[ \cdot, \cdot \mright]_{E_2}, \rho_2) \to N_2$ be two Lie algebroids over two smooth manifolds $N_1$ and $N_2$, and let $E_1 \times E_2$ be the direct product of Lie algebroids. 
Furthermore, let $\pi_i$ ($i \in \{1,2\}$) be the projections $E_1 \times E_2 \to E_i$ as in Lemma \ref{lem:LiftsOfProjections}.

Then, roughly in general, if we have some object $B_i$ on $E_i$, then we define their product by
\ba\label{ProductOfObjects}
B_1 \times B_2
&\coloneqq
\mathrm{pr}_1^*B_1 \oplus \mathrm{pr}_2^*B_2,
\ea
in case there is a well-defined notion for $\mathrm{pr}_i^*B_i$. This is of course well-defined for tensors, \textit{i.e.}~$B_i \in \mathcal{T}^r_s(E_i)$ ($r,s \in \mathbb{N}_0$).

Another examples are vector bundle connections $B_i \coloneqq \nabla^i$ on $E_i$, or $E_i$-connections $B_i \coloneqq {}^{E_i}\nabla$ on vector bundles $V_i \to N_i$ by using Cor.~\ref{cor:GeneralPullbackAnchorPreserving}. Especially the latter means that we always canonically use $\pi_i$ for the pullbacks of $E_i$-connections, and observe
\bas
\mleft(\mathrm{pr}_i^*\mleft( {}^{E_i}\nabla \mright)\mright)_{(\mu_1, \mu_2)} (\mathrm{pr}_i^*v)
&=
\mathrm{pr}_i^*\mleft(
	{}^{E_i}\nabla_{\mu_i} v
\mright)
\eas
for all $v \in \Gamma(V_i)$ and $(\mu_1, \mu_2) \in \Gamma(E_1 \times E_2)$. Thence, exactly what one naturally expects, for example "mixed terms are zero", that is, for example 
\bas
\mleft(\mathrm{pr}_1^*\mleft( {}^{E_1}\nabla \mright)\mright)_{(0, \mu_2)} (\mathrm{pr}_1^*v)
&=
0.
\eas

That is of special usage if one uses that $\Gamma(E_1) \oplus \Gamma(E_2)$ generates $\Gamma(E_1 \times E_2)$ and that the mentioned structures are uniquely given by how they act on $\Gamma(E_1) \oplus \Gamma(E_2)$; also recall Lemma \ref{lem:LemmaUniquenessOfDirectProductStructure}. So, one just needs to take a frame induced by frames of $E_i$, and if a given structure restricts in that frame to a structure on $E_i$, if just using the part of the frame induced by $E_i$, and has no "mixed terms", then one knows that this object can be written as direct product. 

All of that above similarly for structures given by $\mathrm{T}N_i$, and structures involving the tangent bundles and the $E_i$ as in the case of the anchors.

For example, let us have vector bundle connections $\nabla^i$ on $E_i$, then we have the induced basic connections $\nabla^{i,\mathrm{bas}}$. We have a vector bundle connection on $E_1 \times E_2$ by
\bas
\nabla
&\coloneqq
\nabla^1 \times \nabla^2,
\eas
whose curvature also splits as it is well-known (trivial to check with a frame induced by frames of $E_1$ and $E_2$). With $\nabla^{1,\mathrm{bas}} \times \nabla^{2,\mathrm{bas}}$ one has a pair of $E_1 \times E_2$-connections on $E_1 \times E_2$ and $\mathrm{T}N_1 \times \mathrm{T}N_2$. Taking a frame induced by frames of $E_1$ and $E_2$ and $\mathrm{T}N_1$ and $\mathrm{T}N_2$, all of those connections and Lie algebroid connections restrict to the factors in $E_1 \times E_2$ by definition. Using Lemma \ref{lem:LemmaUniquenessOfDirectProductStructure}, also the Lie bracket and anchor are a direct product on such a frame, for both $E_1 \times E_2$ and $\mathrm{T}N_1 \times \mathrm{T}N_2$, hence, 
\bas
\mleft( \nabla^1 \times \nabla^2 \mright)^{\mathrm{bas}}
&=
\nabla^{1,\mathrm{bas}} \times \nabla^{2,\mathrm{bas}}
\eas
and
\bas
R_{\nabla^1 \times \nabla^2}^{\mathrm{bas}}
&=
R_{\nabla^1}^{\mathrm{bas}}
\times
R_{\nabla^2}^{\mathrm{bas}}.
\eas
Similarly the exterior covariant derivatives of $\nabla^1 \times \nabla^2$ and $\mleft( \nabla^1 \times \nabla^2 \mright)^{\mathrm{bas}}$ split on products of forms $\omega_i \in \Omega^{p_i, q_i}(N, E;E)$ ($p_i, q_i \in \mathbb{N}_0$) given by
\ba
\omega_1 \times \omega_2
&\coloneqq
\mathrm{pr}_1^!\omega_1 \oplus \mathrm{pr}_2^!\omega_2.
\ea
The differentials of $\mathrm{Dpr}_i$ are projections $\mathrm{T}(N_1 \times N_2) \to \mathrm{T}N_i$ such that there is not really a significant distinction between $\mathrm{pr}_i^*$ and $\mathrm{pr}_i^!$. This is why we are not going to clarify in such situations whether the product is using pullbacks in sense of sections or forms. It will be clear by context.
\end{remarks}

\section{Splitting theorem for Lie algebroids}\label{SectionAboutSplitting}

Using the last section, one can locally formulate Lie algebroids as direct products of certain Lie algebroids. Let us study that, but first we need some basic notions; we are mainly following \cite{DaSilva} now.

\begin{definitions}{Singular and regular points of vector bundle morphisms, \newline \cite[\S 4; generalization of third remark after Theorem 4.1; page 17]{DaSilva}}{RegularPointsOfVectorBundleMorphisms}
Let $V_1 \stackrel{\pi_1}{\to} N_1$ and $V_2 \stackrel{\pi_2}{\to} N_2$ be vector bundles over smooth manifolds $N_1$ and $N_2$, respectively. Also let $P: V_1 \to V_2$ be a continuous vector bundle morphism over some continuous map $f: N_1 \to N_2$, \textit{i.e.}~$\pi_2 \circ P = f \circ \pi_1$.

We call a point $p \in N_1$ a \textbf{regular point} if there is an open neighbourhood around $p$ onto which $\gls{Rzk}(P)$, the rank of $P$, is constant. \textbf{Singular points} are points $p \in N_1$ which are not regular.
\end{definitions}

In our case $P$ will be the anchor $\rho$, and since $\rho$ is a homomorphism we know that the image of $\rho$, $\gls{Im}(\rho)$, is closed under the Lie bracket of the tangent bundle such that we expect a foliation related to the image of $\rho$ by the Frobenius Theorem; however, since the rank of an anchor is not constant as we pointed out earlier, the foliation induced by the image of the anchor is a singular foliation. Formally, this is proven as a more general Frobenius theorem as also discussed in \cite[discussion after the definition in \S 16.1; page 113]{DaSilva}; also see \cite[beginning of \S 3.1]{meinrenkensplitting}. Essentially, one gets still a foliation if a subset of the tangent bundle is closed under the Lie bracket, but the foliation is singular (non-constant dimension of the leaves). We are interested into those leaves of the anchor, also called orbits, such that we need to study the rank of $\rho$. There is a statement about that the amount of singular points is "small".

\begin{propositions}{Amount of singular and regular points, \newline \cite[generalization of second remark after Theorem 4.1; page 17]{DaSilva}}{RegularPointsAreDense}
Let the situation be as in Def.~\ref{def:RegularPointsOfVectorBundleMorphisms}. Then the set of all regular points is dense in $N_1$.
\end{propositions}

\begin{proof}
\leavevmode\newline
Let $S_{\text{reg}}$ and $S_{\text{locmax}}$ be the sets of regular points and of local maxima of $\text{rk}(P)$ in $N_1$, respectively. It is clear that $S_{\text{reg}} \subset S_{\text{locmax}}$ but we can also show $S_{\text{locmax}} \subset S_{\text{reg}}$: Let $p \in N_1$ be a local maximum of $\text{rk}(P)$ with value $k \in \mathds{N}_0$ and let $k \geq 1$ w.l.o.g. (since for $k=0$ it is clear that then $p \in S_{\text{reg}}$). Then there is a minor $m$ of order $k$ of $P$ such that $m(p) \neq 0$. By continuity of $P$ there is an open neighbourhood $U \subset N_1$ containing $p$ such that $\left. m\middle|_{ U} \right. \neq 0$ and, thus, $\left.rk(P)\middle|_U\right. \geq k$. Therefore also $\left.rk(P)\middle|_U\right. = k$ due to $p \in S_{\text{locmax}}$. Thence, $p \in S_{\text{reg}}$ and so $S_{\text{reg}} = S_{\text{locmax}} \eqqcolon S$.

Now let $x_0 \in N_1 \setminus S$ and $U$ an open neighbourhood of $x_0$. $\mathrm{rk}(P)$ reaches its upper bound on $U$, \textit{i.e.}
\bas
\exists y \in U: ~ \forall x \in U: ~ (\text{rk}(P))(x) \leq (\text{rk}(P))(y).
\eas
This follows by the fact that $\sup_{x \in U} (\mathrm{rk}(P))(x) \eqqcolon l < \infty$ by the boundedness of $\mathrm{rk}(P)$ and w.l.o.g.~we can say that $l \in \mathds{N}_0$ by the $\mathds{N}_0 \text{-valuedness}$ of $\mathrm{rk}(P)$; there must be a $y \in U$ such that $l = (\mathrm{rk}(P))(y)$ since for any other upper bound $l' \in \mathds{N}_0$ of the rank on $U$, for which there is no $y \in U$ with $l' = (\mathrm{rk}(P))(y)$, one can lower $l'$ by 1 such that $l' - 1$ is still an upper bound (follows again by the $\mathds{N}_0\text{-valuedness}$). This procedure is repeated until one gets an upper bound which is the value of some element in $U$. Thus, the supremum is also a maximum. Thence
\bas
&\forall x_0 \in N_1 \setminus S: ~ \forall \text{ open neighbourhoods } U \text{ of } x_0: \exists y \in U: ~ y \in S_{\text{locmax}} = S_{\text{reg}} \\
&\Rightarrow x_0 \text{ is an accumulation point of } S_{\text{reg}}
\\
&\Rightarrow N_1 \setminus S \subset \overline{S}
\\
&\Rightarrow \overline{S_{\text{reg}}} = N_1,
\eas
where $\overline{S}$ denotes the closure of $S = S_{\mathrm{reg}}$.
\end{proof}

\begin{remark}
\leavevmode\newline
This means, assuming $N_1$ is connected, one has "walls of measure zero" of singular points between the connected components of the set of all regular points, \textit{i.e.}~between zones of different rank of $P$. By the previous proof one can also see that the rank of $P$ is locally not maximal at a singular point.
\end{remark}

Around regular points of $\rho$, its distribution is also an integrable foliation since the rank is constant. In general the natural question arises if one can split the Lie algebroid structure locally along this distribution, in sense of "orbital plus transversal structure". Indeed, there are several statements about such \textbf{splitting theorems}, starting with the important splitting theorem of Poisson manifolds by Weinstein as in \cite[Theorem 4.2; page 19]{DaSilva}, another splitting theorem for Lie algebroids can be found in \cite[Theorem 1.1]{fernandes}. If you are interested into a more general approach and theorem then see \cite{meinrenkensplitting}; in this paper the locality is just along the foliation while it can be "arbitrary big" along the transversal direction.

To discuss the splitting theorem for Lie algebroids would certainly exceed the work of this thesis. Hence, we will just state the most simplified statement around regular points without further proof; see the listed references for a thorough discussion. Recall the discussion after Def.~\ref{def:IsotropyForLieAlgeoids}, the kernel of the anchor at a point is a Lie algebra. Around regular points this means that the kernel is a bundle of Lie algebras, $\mathrm{Ker}(\rho) \to N$, one makes use of that in the following statement. For the following statement also recall that two submanifolds $M_1, M_2$ of $N$ are transversal if
\bas
\mathrm{T}_pM_1 + \mathrm{T}_p M_2
&=
\mathrm{T}_p N
\eas
for all $p \in M_1 \cap M_2$. We speak of a \textbf{direct transversal} if the sum is a direct sum/product.
%
%\begin{theorem}[Splitting theorem for Lie algebroids]\label{thm:splittingLiealgebroid}
%\leavevmode\newline
%Let $E \to M$ be a Lie algebroid over a smooth manifold $M$ with anchor $\rho$ and Lie bracket $[\cdot, \cdot]_E$. Also let be $p \in M$ with $\mathrm{rk}(\rho_p) = q \in \mathds{N}_0$. Then there are coordinates $(x^i, y^j)$ (with $i \in \{1, \dots, q\}$, $j \in \{q + 1, \dots, m \coloneqq \mathrm{dim}(M)\}$) defined in some open neighbourhood $U$ around $p$ with a corresponding frame of sections $(e_a)_{a=1}^{r \coloneqq \mathrm{rk}(E)}$ of $E$ such that
%\ba
%\rho(e_i) &= \frac{\partial}{\partial x^i}, ~ i \in \{1, \dots, q\}, \\
%\rho(e_c) &= \sum_{j=q+1}^m \rho^{j}_c \frac{\partial}{\partial y^j}, ~ c \in \{q+1, \dots, r\},
%\ea
%where $\rho^j_c \in C^\infty(U)$ which only depend on the $y$-coordinates and vanish at $p$, \textit{i.e.} $\rho^j_c = \rho^j_c(y^{q+1}, \dots, y^{m})$ and $\rho^j_c(0) = 0$. Also
%\ba
%[e_a, e_b] &= \sum_{c = q+1}^r C_{ab}^c e_c,
%\label{eq:Liesplitting}
%\ea
%so the structure functions $C_{ab}^c \in C^\infty(U)$ vanish for $c \leq q$; they also satisfy
%\ba
%\forall i \in \{1, \dots, q\}: ~ \forall j \in \{q+1, \dots, m\}: ~ \forall a,b \in \{1, \dots, r\} : ~ \sum_{c = q+1}^r \frac{\partial C^c_{ab}}{\partial x^i} \rho_c^j &= 0.
%\label{eq:diffeqforC}
%\ea
%\end{theorem}
%
%\begin{remark}
%\leavevmode\newline
%When $\gls{Leaf}$ is the leaf of the singular foliation going through $p$ then one sees that for $y^j=0$ one stays in this leaf and its coordinates are exactly the $x^i$ and the $e_i$ form the tangent bundle of $L$. Due to the latter and due to that $\rho$ is a homomorphism it is clear that Eq. \eqref{eq:Liesplitting} has to hold on $L$. The $y^j$ are coordinates transversal to $L$.
%
%For regular points $p$ and suitable "small" $U$ one knows that $\rho^j_c \equiv 0$ and so the discussion above extends to the whole of $U$.
%
%Eq. \eqref{eq:diffeqforC} comes from combining the Jacobi identity with the homomorphism property of $\rho$.
%\end{remark}

\begin{theorems}{Splitting theorem around regular points, \cite[Corollary 4.2]{meinrenkensplitting}}{DirectSplitting}
Let $E \to N$ be a Lie algebroid over a connected manifold $N$ such that $N$ only consists of regular points of the anchor $\rho$. Fix a point $p \in N$, and denote with $L$ the leaf through $p$, given by the foliation of $\rho$. Furthermore, take a submanifold $S$ with $p \in S$ and which is transversal to the foliation of the anchor and which is a direct transversal of $L$. Then
\ba
E
&\stackrel{\text{locally around } p}{\cong}
\mathrm{T}L \times \mathrm{Ker}(\rho)|_S,
\ea
where $\mathrm{T}L \times \mathrm{Ker}(\rho)|_S$ is the direct product of Lie algebroids $\mathrm{T}L \to L$ and $\mathrm{Ker}(\rho)|_S \to S$ (the bundle of Lie algebras given by the $\mathrm{Ker}(\rho)$ restricted to $S$).
\end{theorems}

\begin{remarks}{Local frame of the splitting theorem}{LocalSplittingFrame}
This theorem implies that around regular points $p \in N$ are coordinate vector field $\mleft( \partial_i \mright)_i$ of $L$, and a frame $\mleft( e_a \mright)_a$ of $\mathrm{Ker}(\rho)|_S$ such that
\bas
\rho(\partial_i)
&=
\partial_i,
\\
\rho(e_a)
&=
0,
\\
\mleft[ \partial_i, e_a \mright]_E
&=
0,
\eas
using Lemma \ref{lem:LemmaUniquenessOfDirectProductStructure}. We will later define the field of gauge bosons $A$ as a form on the spacetime with values in (the pullback of) a Lie algebroid; the components of $A$ along $e_a$ are then the massless gauge bosons, while the other ones may get mass. The Higgs field will be a smooth map of the spacetime to $N$, and its components along $L$ are then the Nambu-Goldstone bosons, while the transversal components are the Higgs bosons; for this recall the discussion about the Higgs mechanism after Def.~\ref{def:ClassicYMHLagrangian} and the isotropy around Def.~\ref{def:IsotropyForLieAlgeoids}.
\end{remarks}

Using such a frame we conclude this section with a short statement about the existence of parallel frames of Lie algebroid connections.

\begin{lemmata}{Parallel frames of flat Lie algebroid connections around regular points, \newline \cite[Lemma 2.9]{parallelFrameEconn}}{ParallelFramesForEConnections}
Let $E\to N$ be a Lie algebroid over a smooth manifold $N$, and ${}^E\nabla$ be an $E$-connection on a vector bundle $V \to N$. Moreover, assume that ${}^E\nabla_\nu = 0$ for all $\nu \in E$ with $\rho(\nu) = 0$. Then there is locally around each regular point a frame $\mleft( e_a \mright)_a$ of $E$ such that
\bas
{}^E\nabla e_a 
&=
0.
\eas
\end{lemmata}

\begin{proof}[Sketch of the proof]
\leavevmode\newline
Fix a regular point $p \in N$. We just give a short sketch of the proof, using a frame around $p$ as given in Remark \ref{rem:LocalSplittingFrame}, denoted by $\mleft( f_a \mright)_a$, such that a subset of the frame, denoted as $\mleft(g_i\mright)_i$, satisfies $\rho(g_i) = \partial_i$ for some local coordinate vector fields $\mleft( \partial_i \mright)_i$ of the leaf through $p$. The remaining part of the frame, denoted as $\mleft( h_\alpha \mright)_\alpha$, spans the kernel of the anchor, that is, $\rho(h_\alpha) = 0$. Then 
\bas
{}^E\nabla_{f_b} v
&=
\mathcal{L}_{\rho(f_a)}(v^a) ~ f_a
	+ v^a ~ {}^E\nabla_{f_b} f_a
=
\mathcal{L}_{\rho(f_a)}(v^a) ~ f_a
	+ v^a \omega_{ab}^c f_c
\eas
for all $v = v^a f_a \in \Gamma(V)$ and $\mu \in \Gamma(E)$, where $\omega_{ab}^c$ are smooth functions locally on $N$ given by $\omega_{ab}^c f_c = {}^E\nabla_{f_b} f_a$. Let us study the equation ${}^E\nabla v = 0$. If $f_b = g_i$, then
\bas
0
&=
\partial_i v^a ~ f_a
	+ v^a \omega_{ai}^c f_c,
\eas
that is just the standard well-known PDEs, which we can solve. However, if $f_b = h_\alpha$, then
\bas
0
&=
v^a \omega_{a\alpha}^c f_c,
\eas
and that is an algebraic equation, which may or may not be solvable. By the condition ${}^E\nabla_\nu = 0$ for all $\nu \in E$ with $\rho(\nu) = 0$ we know that ${}^E\nabla_{h_\alpha} = 0$ and, so, $\omega_{a\alpha}^c=0$. This resolves the problem of the algebraic equations which are now trivially satisfied. Hence, the remaining proof of the existence of the parallel frame is then similar to flat vector bundle connections, making use of the vanishing mixed components of the Lie bracket as given in the third equation in Remark \ref{rem:LocalSplittingFrame} when studying the curvature with respect to such statements, in order to allow similar arguments about parallel transport as for vector bundle connections; see the reference for the remaining proof.
\end{proof}

Especially the proof emphasizes why one cannot expect in general to have a parallel frame for flat Lie algebroid connections. For example take an action Lie algebroid $E = N \times \mathfrak{g}$ over a smooth manifold $N$, related to a Lie algebra $\mathfrak{g}$, and denote with $\nabla$ its canonical flat connection. Then the basic connection on $E$ gives
\bas
\nabla^{\mathrm{bas}}_{\mu} \nu
&=
\mleft[ \mu, \nu \mright]_{\mathfrak{g}}
\eas
for all constant sections $\mu, \nu \in \Gamma(N \times \mathfrak{g})$. Therefore the basic connection is also flat because it is just the Lie bracket (by the Jacobi identity); but it is a canonical flat connection if and only if $\mathfrak{g}$ is abelian. If the basic connection on $E$ has a parallel frame $\mleft( e_a \mright)_a$, then
\bas
\nabla_{\rho(e_a)} e_b
&=
\mleft[ e_a, e_b \mright]_E,
\eas
which may not necessarily hold for any frame. Since the left hand side is tensorial in $e_a$ we could then derive for all sections $\nu$ with (in that neighbourhood) $\rho(\nu)=0$ that
\bas
0
&=
\nu^a ~ \mleft[ e_a, e_b \mright]_E.
\eas
 However, the important piece of information in this work is to know that the basic connection is in general not the canonical flat connection for action Lie algebroids if $\nabla$ is already the canonical flat connection.

\section{Lie algebra bundles}\label{SectionOfLABStuff}

Of special importance are the Lie algebra bundles (LABs), defined in Def.~\ref{def:LAB}. As Lie algebroids they are rather easy since the anchor is zero. But they will still play an important role later; also the kernel of each anchor is a bundle of Lie algebras around regular points, which is why it is important to study those. LABs are a special case of bundle of Lie algebras, but we will see later why we are mainly interested into those.

We will summarise the most important results of this section in Ex.~\ref{ex:BigCoolDiagramOfMackenzieAboutLABsStuff}.

\subsection{Notions similar to Lie algebras}

Many constructions related to Lie algebras carry over to LABs. We will explain why.

\begin{propositions}{sub-LABs, \cite[Proposition 3.3.9; page 105]{mackenzieGeneralTheory}}{SubLABS}
Let $K\to N$ be an LAB over a smooth manifold $N$ with fibre type $\mathfrak{g}$ as Lie algebra. Moreover, let $\mathfrak{h}$ be a Lie \textbf{characteristic subalgebra of $\mathfrak{g}$}, that is, a subalgebra of $\mathfrak{g}$ such that $\varphi(\mathfrak{h}) = \mathfrak{h}$ for all Lie algebra automorphism $\varphi: \mathfrak{g} \to \mathfrak{g}$. 

Then there is a well-defined \textbf{sub-LAB $L$ of $K$}, that is, a subbundle $L$ of $K$ which is also an LAB such that each LAB chart $\psi: K|_U \to U \times \mathfrak{g}$ restricts to an LAB chart $L|_U \to U \times \mathfrak{h}$, where $U$ is an open subset of $N$ on which an LAB chart is defined.
\end{propositions}

\begin{remark}
\leavevmode\newline
It is an immediate consequence that the field of Lie brackets of $L$ is given by the field of Lie brackets of $K$ restricted to $L$.
\end{remark}

\begin{proof}[Proof of Prop.~\ref{prop:SubLABS}]
\leavevmode\newline
That is trivial. The essential thing to note is that we need $\varphi(\mathfrak{h}) = \mathfrak{h}$ for all Lie algebra automorphisms $\varphi: \mathfrak{g} \to \mathfrak{g}$ as a condition for gluing the canonical construction of a sub-LAB in given a trivialization, \textit{i.e.}~it is trivial to construct a sub-LAB for a trivial LAB, and for gluing those constructions it is important that each LAB chart can restrict to a Lie algebra isomorphism $L|_U \to U \times \mathfrak{h}$ corresponding to the same subalgebra $\mathfrak{h}$. To make this possible, the local images/restrictions must be stable under transition maps in case two LAB charts of $K$ overlap in some open neighbourhood. The transition maps are Lie algebra automorphisms, and, so, if two overlapping LAB charts of $K$ restrict as stated, then their transition map is in alignment with this due to $\varphi(\mathfrak{h}) = \mathfrak{h}$ for all Lie algebra automorphisms $\varphi: \mathfrak{g} \to \mathfrak{g}$.

Hence, restricting the inverse of each LAB chart of $K$ to $U \times \mathfrak{h}$ defines a subbundle $L$ of $K$, such that each fibre is essentially the subalgebra $\mathfrak{h}$ and its bracket is canonically the restriction of the field of Lie brackets of $K$; all of that is well-defined by the previous paragraph, and that gives an LAB structure on $L$.
\end{proof}

\begin{examples}{Centres of LABs, \newline \cite[first parapgraph after Proposition 3.3.9; page 105]{mackenzieGeneralTheory}}{CentreOfLABK}
With this proposition we can quickly generalize certain constructions of Lie algebras to the level of LABs. For example, possible subalgebras $\mathfrak{h}$ of a Lie algebra $\mathfrak{g}$ with $\varphi(\mathfrak{h}) = \mathfrak{h}$ for all Lie algebra automorphisms $\varphi: \mathfrak{g} \to \mathfrak{g}$ are trivially, due to that $\varphi$ is a homomorphism of brackets, the centre $Z(\mathfrak{g})$ of $\mathfrak{g}$ and $\mleft[ \mathfrak{g}, \mathfrak{g} \mright]_{\mathfrak{g}}$, the corresponding sub-LABs are denoted by $\gls{ZLAB}$ and $\mleft[ K, K \mright]_K$, respectively; we especially need the former. Moreover, the sections of $Z(K)$ are also the centre of the Lie algebra $\Gamma(K)$.
\begin{center}
	\begin{tikzcd}
	Z(\mathfrak{g}) \arrow{r} & Z(K) \arrow{d} \\
			& N
	\end{tikzcd}
\end{center}
\end{examples}

\begin{examples}{Derivations of LABs, \newline \cite[second and third parapgraph after Proposition 3.3.9, and discussion around Proposition 3.3.10; page 105]{mackenzieGeneralTheory}}{DerivationsOFLABSK}
Another important LABs will be related to Lie bracket derivations $\mathrm{Der}(\mathfrak{g})$ of a Lie algebra $\mathfrak{g}$; those are as usual defined as those endomorphisms $T \in \mathrm{End}(\mathfrak{g})$ of $\mathfrak{g}$ such that
\bas
T\mleft( \mleft[ x, y \mright]_{\mathfrak{g}} \mright)
&=
\mleft[ T(x), y \mright]_{\mathfrak{g}}
	+ \mleft[ x, T(y) \mright]_{\mathfrak{g}}
\eas
for all $x, y \in \mathfrak{g}$. Recall, that we derived the derivations of a vector bundle $V \to N$, denoted by $\mathcal{D}(V)$, whose anchor was denoted by $a$ and its kernel is trivially given by $\mathrm{End}(V)$. Since the rank of $\mathrm{End}(V)$ is constant, so, $a$ has constant rank, and the kernel of anchors around regular points is a bundle of Lie algebras, we can conclude that $\mathrm{End}(V)$ is an LAB, also because of that the Lie algebra fibre type is trivially given by $\mathrm{End}(W)$ where $W$ is the fibre type of of $V$.

In case of $V=K$ an LAB over $N$, we have an LAB with fibre type $\mathrm{End}(\mathfrak{g})$, and $\mathrm{Der}(\mathfrak{g})$ is a subalgebra as it is well-known and trivial to check. Now let $\varphi \in \mathrm{Aut}(\mathfrak{g})$, then take $T \in \mathrm{Der}(\mathfrak{g})$, and observe for $\varphi \circ T \circ \varphi^{-1}$ that
\bas
\mleft(\varphi \circ T \circ \varphi^{-1}\mright)\mleft( \mleft[ x, y \mright]_{\mathfrak{g}} \mright)
&=
(\varphi \circ T)\mleft(
	\mleft[ \varphi^{-1}(x),\varphi^{-1}(y) \mright]_{\mathfrak{g}}
\mright)
\\
&=
\varphi\mleft(
	\mleft[ T\mleft(\varphi^{-1}(x)\mright),\varphi^{-1}(y) \mright]_{\mathfrak{g}}
	+ \mleft[ \varphi^{-1}(x),T\mleft(\varphi^{-1}(y)\mright) \mright]_{\mathfrak{g}}
\mright)
\\
&=
\mleft[ \mleft(\varphi \circ T \circ \varphi^{-1}\mright)(x), y \mright]_{\mathfrak{g}}
	+ \mleft[ x, \mleft(\varphi \circ T \circ \varphi^{-1}\mright)(y) \mright]_{\mathfrak{g}}
\eas
for all $x, y \in \mathfrak{g}$. Thus, $\varphi \circ T \circ \varphi^{-1} \in \mathrm{Der}(\mathfrak{g})$; similar for the inverse of $\varphi$ such that $\varphi \circ \mathrm{Der}(\mathfrak{g}) \circ \varphi^{-1} = \mathrm{Der}(\mathfrak{g})$. The conjugation with $\varphi$ is just a certain type of elements in $\mathrm{Aut}\bigl(\mathrm{End}(\mathfrak{g})\bigr)$ such that it looks like that we cannot yet use Prop.~\ref{prop:SubLABS}. However, the proof of Prop.~\ref{prop:SubLABS} was just about transition maps and in case of $\mathrm{End}$-bundles the typical atlas\footnote{This is also clearly its LAB atlas.} has such transition maps as we know in general, which is why we can conclude similarly as in the proof of Prop.~\ref{prop:SubLABS} that there is a well-defined sub-LAB $\gls{DAerK}$ of $\mathrm{End}(K)$ with fibre type $\mathrm{Der}(\mathfrak{g})$.
\begin{center}
	\begin{tikzcd}
	\mathrm{Der}(\mathfrak{g}) \arrow{r} & \mathrm{Der}(K) \arrow{d} \\
			& N
	\end{tikzcd}
\end{center}
There is a special set of derivations, the \textbf{ideal of inner derivations $\mathrm{ad}(\mathfrak{g})$ of $\mathfrak{g}$}; that is, an inner derivation is of the form $\mathrm{ad}(x)$ for an $x \in \mathfrak{g}$. It is trivially a derivation by the Jacobi identity, and an ideal of $\mathrm{Der}(\mathfrak{g})$ by
\bas
\mleft( \mleft[ \mathrm{ad}(x), T \mright]_{\mathrm{Der}(K)} \mright)(y)
&=
\mleft[ x, T(y) \mright]_{\mathfrak{g}}
	- \underbrace{T\mleft( \mleft[ x, y \mright]_{\mathfrak{g}} \mright)}
	_{\mathclap{ = \mleft[ T(x), y \mright]_{\mathfrak{g}} + \mleft[ x, T(y) \mright]_{\mathfrak{g}} }}
\\
&=
- \mleft(\mathrm{ad}\bigl(T(x)\bigr)\mright)(y)
\eas
for all $x, y \in \mathfrak{g}$ and $T \in \mathrm{Der}(\mathfrak{g})$. As above, observe that for all $\varphi \in \mathrm{Aut}(\mathfrak{g})$ we have
\bas
\mleft(\varphi \circ \mathrm{ad}(x) \circ \varphi^{-1}\mright)(y)
&=
\varphi\mleft(
	\mleft[ x, \varphi^{-1}(y) \mright]_{\mathfrak{g}}
\mright)
=
\mleft(\mathrm{ad}\bigl( \varphi(x) \bigr)\mright)(y),
\eas
hence, the discussed conjugation above restricts to inner derivations. Therefore we can apply the same argument as above to derive that $\mathrm{ad}(\mathfrak{g})$ gives rise to a sub-LAB of $\mathrm{Der}(K)$ and of $\mathrm{End}(K)$, denoted by $\gls{adK}$, the \textbf{ideal of inner derivations of $K$}.
\end{examples}

\begin{remark}
\leavevmode\newline
As shown in \cite[discussion around Proposition 3.3.10; page 105]{mackenzieGeneralTheory}, one can quickly derive that $\mathrm{ad}(K)$ is the image of $\mathrm{ad}: K \to \mathrm{Der}(K)$, which is just defined as the fibre-wise extended adjoint map of $\mathrm{ad}$ on $\mathfrak{g}$. Since it is a tensor, the adjoint extends to sections.
\end{remark}

$\mathrm{ad}(K)$ is trivially an ideal in the following sense.

\begin{definitions}{Ideals of LABs, \cite[Definition 3.3.11; page 106]{mackenzieGeneralTheory}}{IdealsOfLABSK}
Let $K \to N$ be an LAB over a smooth manifold $N$ and $L$ a sub-LAB of $K$. Then $L$ is an \textbf{ideal of $K$} if each fibre of $L_p$ is an ideal of $K_p$ for all $p \in N$.
\end{definitions}

One can construct a quotient of $\mathrm{Der}(K)$ over $\mathrm{ad}(K)$ in the usual way, but we need such quotients a bit more general. For this we need to discuss extensions of tangent bundles where LABs play an important role. Those are best described as certain short exact sequences.

\subsection{Extensions of tangent bundles with Lie algebra bundles}

\begin{definitions}{Extension of tangent bundles by LABs and transversals, \newline \cite[\S 7.1, Definition 7.1.11; page 266; and Definition 7.3.1; page 277]{mackenzieGeneralTheory}}{ExtensionOfTNByLABs}
Let $K \to N$ be an LAB. Then an \textbf{extension of $\mathrm{T}N$ by $K$} is a short exact sequence of Lie algebroids over $N$
\begin{center}
	\begin{tikzcd}
		0 \arrow{r} & K \arrow{r}{\iota} & E \arrow{r}{\pi} & \mathrm{T}N \arrow{r} & 0,
	\end{tikzcd}
\end{center}
where $E \to N$ is a Lie algebroid and 
the sequence is exact as a sequence of vector bundles but each arrow represents a Lie algebroid morphism,
equivalently denoted as\footnote{The hooked arrow emphasizes the inclusion, and the two-headed arrow the surjectivity.}
\be\label{defShortExactSeqExtensionOfTNByK}
	\begin{tikzcd}
		K \arrow[hook]{r}{\iota} & E \arrow[two heads]{r}{\pi} & \mathrm{T}N.
	\end{tikzcd}
\ee
A transversal of \eqref{defShortExactSeqExtensionOfTNByK} is a vector bundle morphism $\chi: \mathrm{T}N \to E$ such that $\pi \circ \chi = \mathds{1}_{\mathrm{T}N}$.
\end{definitions}

\begin{remark}
\leavevmode\newline
\indent $\bullet$ As in this definition, we will use those sequences also to define the corresponding notation of the Lie algebroid morphisms, in order to avoid separately writing "[$\dotsc$] where $\iota: K \to E$ is a Lie algebroid morphism [$\dotsc$]". We also only give the sequence, implicitly meaning that $K$ will be an LAB and $E$ a Lie algebroid over $N$ without mentioning it further. 

$\bullet$ Furthermore, $\iota$ is an injective Lie algebroid morphism, especially an embedding since it is also vector bundle morphism. Hence, $\iota$ is up to Lie algebroid isomorphisms the inclusion in this work and can be thought as such, which is why we often omit it. These notations normally emphasize that a change of the explicit description of $K$ is possible, in that case the inclusion would be replaced by a composition of the corresponding inclusion with a Lie algebroid isomorphism; however, we will not need this.

$\bullet$ We will, as usual, denote the Lie bracket of $E$ by $\mleft[ \cdot, \cdot \mright]_E$, and $\pi$ is its anchor $\rho$ due to that $\pi$ is anchor-preserving and that the anchor of $\mathrm{T}N$ is the identity. Therefore we will use the typical notation of anchors in the following instead of $\pi$; we also clearly have $\iota(K) = \mathrm{Ker}(\rho)$ by the exactness of the sequence.

$\bullet$ $E$ is a transitive Lie algebroid because $\rho = \pi$ is surjective in that case; in fact, by \cite[Theorem 6.5.1; page 248]{mackenzieGeneralTheory} each transitive Lie algebroid $E$ is such a short exact sequence. The rank of the anchor is constant for transitive Lie algebroids such that there are only regular points and, so, the kernel of the anchor, $\mathrm{Ker}(\rho)$, is a bundle of Lie algebras. One can show that $\mathrm{Ker}(\rho)$ is also a Lie algebra bundle by Thm.~\ref{thm:BLALAB}; the essential trick is to take a vector bundle morphism $\chi: \mathrm{T}N \to E$ with $\rho \circ \chi = \mathds{1}_{\mathrm{T}N}$, and then to define a connection $\nabla$ on $\mathrm{Ker}(\rho)$ by $\mathrm{ad} \circ \chi$, \textit{i.e.}~$\nabla_X \nu \coloneqq \mleft[ \chi(X), \nu \mright]_E$ for all $X \in \mathfrak{X}(N)$ and $\nu \in \mathrm{Ker}(\rho)$. This connection will be a Lie bracket derivation of $\mathrm{Ker}(\rho)$ such that Thm.~\ref{thm:BLALAB} can be used. We will not prove this, since we are not going to need it, hence, see the reference; however, the essential calculations will be done later in Section \ref{ObstrLAB}. Moreover, it is useful for the following constructions to keep this information in mind, in order to understand why it is a useful simplification to assume transitive Lie algebroids.

$\bullet$ So, in our case, extensions are equivalent to transitive Lie algebroids, such that one may wonder about the different name. Often, especially in Section \ref{ObstrLAB}, we will have a given $K$ and $N$, then there is the question whether there is an $E$ in the sense of an extension involving $K$ and $\mathrm{T}N$. Thence, the idea is that $E$ \textbf{extends} $\mathrm{T}N$ by $K$ in sense of Lie algebroids. The different name here is especially to emphasize a different context. Moreover, the idea of extensions can be generalized in the sense of replacing $\mathrm{T}N$ by an arbitrary Lie algebroid as in \cite[Definition 3.3.19; page 109]{mackenzieGeneralTheory}.
\end{remark}

\begin{examples}{Derivations as extension and connections as transversal, \newline \cite[second statement of Corollary 3.6.11; page 140]{mackenzieGeneralTheory}}{DerivationsAreExtensions}
Let $V \to N$ be a vector bundle over a smooth manifold $N$. Then $\mathcal{D}(V)$ with anchor $a$ describes an extension as a transitive Lie algebroid as we have seen,
\be\label{ExtensioNofDerivations}
	\begin{tikzcd}
		\mathrm{End}(V) \arrow[hook]{r} & \mathcal{D}(V) \arrow[two heads]{r}{a} & \mathrm{T}N.
	\end{tikzcd}
\ee
By definition, a vector bundle connection $\nabla$ of $V$ is then a transversal of \eqref{ExtensioNofDerivations}, and each transversal a connection.

In the case of $V = K$ an LAB, we can define $\gls{DAVDerK}$ as the subset of those derivations generated by sections $T \in \Gamma(\mathcal{D}(K))$ with
\bas
T\mleft( \mleft[ \mu, \nu \mright]_{K} \mright)
&=
\mleft[ T(\mu), \nu \mright]_{K}
	+ \mleft[ \mu, T(\nu) \mright]_{K}
\eas
for all $\mu, \nu \in \Gamma(K)$. Since $\mleft[ \cdot, \cdot \mright]_{\mathcal{D}(K)}$ is just defined as a commutator, it follows as trivial as for $\mathrm{Der}(\mathfrak{g})$ of a Lie algebra $\mathfrak{g}$ that $\Gamma\mleft(\mathcal{D}_{\mathrm{Der}}(K)\mright)$ is a subalgebra of $\Gamma(\mathcal{D}(K))$; and at each point $p\in N$ we have that $\mathcal{D}_{\mathrm{Der}}(K)$ is a subspace of $\mathcal{D}(K)$. It is also a Lie algebroid, whose structure is inherited by $\mathcal{D}(K)$; for this take a connection $\nabla$ on $K$ which is a Lie bracket derivation, see Thm.~\ref{thm:BLALAB} for its existence later. Then define a map 
\bas
\mathrm{T}N \times \mathrm{Der}(K) &\to \mathcal{D}_{\mathrm{Der}}(K),
\\
(X, A) &\mapsto \nabla_X + A,
\eas
which is clearly well-defined because of the fact that the difference of two connections is always an element $L$ of $\Omega^1(N; \mathrm{End}(K))$; if then both of these connections are Lie bracket derivations, then so also $L$ such that $L \in \Omega^1(N; \mathrm{Der}(K))$. Hence, $\nabla_X + A \in \mathcal{D}_{\mathrm{Der}}(K)$. As in the proof of Prop.~\ref{prop:IsomorphismofDerivationonVectorbundleatabasepoint}, see also Lemma \ref{lem:LemmaVectorbundlestructureofDV}, this defines an isomorphism of vector spaces at each point, and as for $\mathcal{D}(K)$ this leads to that $\mathcal{D}_{\mathrm{Der}}(K)$ has constant rank and it admits a transitive Lie algebroid structure with precisely the same arguments as for general derivations; since this structure is inherited by $\mathcal{D}(K)$, we may say that $\mathcal{D}_{\mathrm{Der}}(K)$ is a transitive Lie subalgebroid. The kernel of its anchor, $a|_{\mathcal{D}_{\mathrm{Der}}(K)}$, consists by definition of those elements of $\mathrm{End}(K)$ which are also Lie bracket derivations, so, the kernel is $\mathrm{Der}(K)$. Therefore we arrive at another extension, basically the restriction of \eqref{ExtensioNofDerivations} onto $\mathcal{D}_{\mathrm{Der}}(K)$,
\be\label{SequenceForBracketDerivations}
	\begin{tikzcd}
		\mathrm{Der}(K) \arrow[hook]{r} & \mathcal{D}_{\mathrm{Der}}(K) \arrow[two heads]{r}{a} & \mathrm{T}N,
	\end{tikzcd}
\ee
and also here, a vector bundle connection of $K$ which is also a Lie bracket derivation is equivalent to a transversal for \eqref{SequenceForBracketDerivations}.
\end{examples}

As for Lie algebras, we want to take the quotient of $\mathrm{Der}(K)$ and $\mathcal{D}_{\mathrm{Der}}(K)$ over $\mathrm{ad}(K)$. That is, as usual, done over ideals of Lie algebroids, which shall be subsets of the kernel of the anchor; the reason behind this is to avoid problems in quotients with respect to the anchor. The typical constructions for quotients will then apply because the anchor of an equivalence class is going to be independent of the chosen representative.

\begin{definitions}{Ideals of transitive Lie algebroids, \newline \cite[Definition 6.5.6; page 250]{mackenzieGeneralTheory}}{IdealsOfTransitiveLieAlgebroids}
Let
\begin{center}
	\begin{tikzcd}
		K \arrow[hook]{r}{\iota} & E \arrow[two heads]{r}{\rho} & \mathrm{T}N.
	\end{tikzcd}
\end{center}
be an extension. Then an \textbf{ideal $L$ of $E$} is a sub-LAB of $K$ with
\ba
\mleft[ \nu, \mu \mright]_E &\in \Gamma(L)
\ea
for all $\nu \in \Gamma(E)$ and $\mu \in \Gamma(L)$.
\end{definitions}

\begin{remark}
\leavevmode\newline
As we know, the kernel of $\rho$, $K$, is a canonical example of an ideal.
\end{remark}

\begin{propositions}{Quotient Lie algebroids of transitive Lie algebroids, \newline \cite[Proposition 6.5.8]{mackenzieGeneralTheory}}{QuotientsOfTransitiveLAOids}
Let
\begin{center}
	\begin{tikzcd}
		K \arrow[hook]{r}{\iota} & E \arrow[two heads]{r}{\rho} & \mathrm{T}N.
	\end{tikzcd}
\end{center}
be an extension and $L$ an ideal of $E$. Furthermore, we denote with $E\Big/\iota(L)$ and $K \Big/ L$ the quotient bundle as vector bundles, whose natural projections we denote by $\sharp: E \to E\Big/\iota(L), \mu \mapsto \mu + \iota(L)$, and $\sharp|_K$, respectively. Then naturally define
\ba
K \Big/ L &\stackrel{\overline{\iota}}{\to} E \Big/ \iota(L),
\\
\sharp|_K(\mu) &\mapsto \overline{\iota}\bigl(\sharp|_K(\mu)\bigr) \coloneqq \sharp\bigl(\iota(\mu)\bigr),\label{SharpRestricted}
\ea
and
\ba
E \Big/ \iota(L) &\stackrel{\overline{\rho}}{\to} \mathrm{T}N,
\\
\sharp(\nu) &\mapsto \overline{\rho}\bigl( \sharp(\nu) \bigr) \coloneqq \rho(\nu),
\ea
and finally equip $E \Big/ \iota(L)$ with the bracket $\mleft[ \cdot, \cdot \mright]_{E \big/ \iota(L)}$
\ba
\mleft[ \sharp(\nu), \sharp(\eta) \mright]_{E \big/ \iota(L)}
&\coloneqq
\sharp\mleft(\mleft[ \nu, \eta \mright]_E\mright)
\ea
for all $\nu, \eta \in \Gamma(E)$. Then
\begin{center}
	\begin{tikzcd}
		K \Big/ L \arrow[hook]{r}{\overline{\iota}} & E \Big/ \iota(L) \arrow[two heads]{r}{\overline{\rho}} & \mathrm{T}N
	\end{tikzcd}
\end{center}
is an extension such that $\sharp$ is a surjective submersion with kernel $\iota(L)$.
\end{propositions}

\begin{remarks}{}{LabelOfQuotient}
We call $E \Big/ \iota(L)$ the \textbf{quotient (transitive) Lie algebroid of $E$ over $L$}. By definition $\sharp$ is a Lie algebroid morphism, as is $\sharp|_K$ by Eq.~\eqref{SharpRestricted} since $\overline{\iota}$ and $\iota$ are injective Lie algebroid morphisms and embeddings.
\end{remarks}

\begin{proof}[Sketch of the proof of Prop.~\ref{prop:QuotientsOfTransitiveLAOids}]
\leavevmode\newline
The proof is straightforward because the constructions are the typical ones for such structures. We just give a sketch, one essentially needs to check that everything is well-defined, that we have a Lie bracket in combination with an anchor and that the sequence of the quotients is exact. First of all, everything has constant rank such that the taken quotients as vector bundles are valid. Moreover, $\overline{\iota}$ is well-defined because $\iota$ is injective by the exactness of the sequence, hence, let $\mu, \mu^\prime \in K$ with $\sharp|_K(\mu) = \sharp|_K(\mu^\prime)$
\bas
\sharp\bigl(  \iota(\mu) \bigr)
&=
\sharp\bigl( \underbrace{\iota(\mu - \mu^\prime)}_{\in \iota(L)} + \iota(\mu^\prime) \bigr)
=
\sharp\bigl( \iota(\mu^\prime) \bigr),
\eas
such that $\overline{\iota}\bigl( \sharp|_K(\mu) \bigr) =\overline{\iota}\bigl( \sharp|_K(\mu^\prime) \bigr)$;
similarly for $\hat{\nu}, \hat{\nu}^\prime \in E$ with $\sharp(\hat{\nu}) = \sharp(\hat{\nu}^\prime)$
\bas
\rho(\hat{\nu})
&=
\rho(\underbrace{\hat{\nu} - \hat{\nu}^\prime}_{\mathclap{ \in ~ \iota(L)~ \subset ~ \iota(K) }} + \hat{\nu}^\prime)
=
\rho(\hat{\nu}^\prime),
\eas
thus, $\overline{\rho}\bigl( \sharp(\hat{\nu}) \bigr) = \overline{\rho}\bigl( \sharp(\hat{\nu}^\prime) \bigr)$, and, finally for $\nu, \nu^\prime, \eta, \eta^\prime \in \Gamma(E)$ with $\sharp(\nu) = \sharp(\nu^\prime)$ and $\sharp(\eta) = \sharp(\eta^\prime)$,
\bas
\sharp\bigl(
	\mleft[ \nu, \eta \mright]_E
\bigr)
&=
\sharp\bigl(
	[ \underbrace{\nu - \nu^\prime}_{\mathclap{ \in \iota(L) \subset \iota(K) }} + \nu^\prime,
	\underbrace{\eta - \eta^\prime}_{\mathclap{ \in \iota(L) \subset \iota(K) }} + \eta^\prime ]_E
\bigr)
=
\sharp\bigl(
	\mleft[ \nu^\prime, \eta^\prime \mright]_E
\bigr),
\eas
using that the kernel of the anchor is an ideal of the Lie bracket, therefore also $\mleft[ \sharp(\nu), \sharp(\eta) \mright]_{E \big/ \iota(L)} = \mleft[ \sharp(\nu^\prime), \sharp(\eta^\prime) \mright]_{E \big/ \iota(L)}$. The (bi-)linearity of all those maps follows trivially, the bracket is also clearly anti-symmetric, and 
\bas
[ \sharp(\nu), \underbrace{f ~ \sharp(\eta)}_{\mathclap{ =\sharp(f \eta) }} ]_{E \big/ \iota(L)}
&=
\sharp\bigl(
	\mleft[ \nu, f \eta \mright]_E
\bigr)
\\
&=
\sharp\bigl(
	f \mleft[ \nu, \eta \mright]_E
	+	\underbrace{\mathcal{L}_{\rho(\nu)}}_{\mathclap{ = \mathcal{L}_{\overline{\rho}( \sharp(\nu) )} }}(f) ~ \eta
\bigr)
\\
&=
f ~ \sharp\bigl(
	\mleft[ \nu, \eta \mright]_E
\bigr)
	+ \mathcal{L}_{\overline{\rho}( \sharp(\nu) )}(f) ~ \sharp(\eta)
\\
&=
f \mleft[ \sharp(\nu), \sharp(\eta) \mright]_{E \big/ \iota(L)}
	+ \mathcal{L}_{\overline{\rho}( \sharp(\nu) )}(f) ~ \sharp(\eta)
\eas
for all $f\in C^\infty(N)$. The Jacobi identity is clearly inherited by $\mleft[ \cdot, \cdot \mright]_E$, so, it is a Lie bracket and $\overline{\rho}$ is the anchor by Prop.~\ref{prop:MeasureofJacobiandHomom}.
By construction, $\overline{\iota}$ is still injective, that is, assume
\bas
\overline{\iota}\bigl(\sharp|_K(\mu)\bigr)
&=
\overline{\iota}\bigl(\sharp|_K(\mu^\prime)\bigr)
\eas
for two fixed $\mu, \mu^\prime \in K$, then
\bas
0
&=
\sharp\bigl(
	\iota(\mu - \mu^\prime)
\bigr),
\eas
thus, $\mu - \mu^\prime \in L$ such that $\sharp|_K(\mu) = \sharp|_K(\mu^\prime)$, which proves the injectivity of $\overline{\iota}$.
Moreover,
\bas
\overline{\rho}\mleft(
	\overline{\iota}\bigl(\sharp|_K(\mu)\bigr)
\mright)
&=
\overline{\rho}\mleft(
	\sharp\bigl( \iota(\mu) \bigr)
\mright)
=
\rho\bigl(
	\iota(\mu)
\bigr)
=
0
\eas
for all $\mu \in K$; the anchor $\overline{\rho}$ is clearly surjective by $\overline{\rho} \circ \sharp= \rho$ and because the quotient is just over a subbundle of $K = \mathrm{Ker}(\rho)$, that is, for all $X \in \mathfrak{X}(N)$ let $\nu \in \Gamma(E)$ such that $X = \rho(\nu)$, then
\bas
\overline{\rho}\bigl( \sharp(\nu)\bigr)
&=
\rho(\nu)
=
X.
\eas
Thence, the sequence of the quotients is exact. That $\sharp$ is a surjective submersion with kernel $\iota(L)$ follows trivially by construction as natural projection of quotient spaces.
\end{proof}

\begin{examples}{Outer bracket derivations of K, \newline \cite[Definition 7.2.1 and Equation (7); page 271]{mackenzieGeneralTheory}}{OuterDerivationsOfK}
Let $K \to N$ be an LAB over a smooth manifold $N$. Then we have the following quotient
\be
	\begin{tikzcd}
		\mathrm{Der}(K)\Big/\mathrm{ad}(K) \arrow[r, hook] & \mathcal{D}_{\mathrm{Der}}(K)\Big/ \mathrm{ad}(K) \arrow[two heads]{r}{\overline{a}} & \mathrm{T}N,
	\end{tikzcd}
\ee
which we denote by
\be
	\begin{tikzcd}
		\gls{OutKA} \arrow[r, hook] & \gls{OutKDDerK} \arrow[two heads]{r}{\overline{a}} & \mathrm{T}N,
	\end{tikzcd}
\ee
where $\mathrm{Out}(K) \coloneqq \mathrm{Der}(K)\Big/\mathrm{ad}(K)$ are the \textbf{outer bracket derivations of $K$}, and $\mathrm{Out}\mleft(\mathcal{D}_{\mathrm{Der}}(K)\mright) \coloneqq \mathcal{D}_{\mathrm{Der}}(K)\Big/ \mathrm{ad}(K)$ are those derivations in $\mathcal{D}(K)$ which are also outer bracket derivations. This quotient is possible because exactly as in Ex.~\ref{ex:DerivationsOFLABSK} one can show that $\mathrm{ad}(K)$ is also an ideal of $\mathcal{D}_{\mathrm{Der}}(K)$ and not just of $\mathrm{Der}(K)$, that is, we get again as in Ex.~\ref{ex:DerivationsOFLABSK}
\ba
\mleft[ T, \mathrm{ad}(\nu) \mright]_{\mathcal{D}_{\mathrm{Der}(K)}}
&=
\mathrm{ad}\bigl( T(\nu) \bigr)
\ea
for all $\nu \in \Gamma(K)$ and $T \in \Gamma\mleft(\mathcal{D}_{\mathrm{Der}(K)}\mright)$.
\end{examples}

Let us finish this chapter with a summary of this section, also recall Remark \ref{rem:LabelOfQuotient}.

\begin{examples}{Summary of Section \ref{SectionOfLABStuff}, \newline \cite[\S 7.2, Figure 7.1; page 272; we omit the labels of the inclusion arrows]{mackenzieGeneralTheory}}{BigCoolDiagramOfMackenzieAboutLABsStuff}
Let $K \to N$ be an LAB over a smooth manifold $N$. Then the main results of Section \ref{SectionOfLABStuff} can be summarized in the following commuting diagram
\be
	\begin{tikzcd}
		Z(K) \arrow[hook]{d} \arrow[equal]{r} & Z(K) \arrow[hook]{d} \\
		K \arrow{d}{\mathrm{ad}} \arrow[equal]{r} & K \arrow{d} \\
		\mathrm{Der}(K) \arrow[two heads]{d}{\sharp^+} \arrow[hook]{r} & \mathcal{D}_{\mathrm{Der}}(K) \arrow[two heads]{d}{\sharp} \arrow[two heads]{r}{a} & \mathrm{T}N \arrow[equal]{d} \\
		\mathrm{Out}(K) \arrow[hook]{r} & \mathrm{Out}\mleft(\mathcal{D}_{\mathrm{Der}}(K)\mright) \arrow[two heads]{r}{\overline{a}} & \mathrm{T}N
	\end{tikzcd}
\ee
where both rows and columns are short exact sequences of Lie algebroid morphisms, especially the last two rows are extensions, and the diagram serves as a definition of the notation of the new Lie algebroid morphisms, for example $\sharp^{(+)}$ denotes the projection of derivations into the space of outer derivations.
\end{examples}