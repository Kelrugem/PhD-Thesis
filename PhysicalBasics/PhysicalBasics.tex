\chapter{Generalized gauge theory}\label{GeneralizedGTfas}
%
%\textbf{THAT WILL BE SOMEWHERE ELSE LATER}
%
%\begin{definitions}{Wedge product of forms acting on each other}{WedgeProductOfVbFormsMitEndormophormisms}
%
%\end{definitions}
%
%\begin{propositions}{Several useful identities}{SeveralIdentitiesFortheCalculusWithPullbackandBlah}
%Let $M$ and $N$ be two smooth manifolds, $K \to N$ a vector bundle, $X: M \to N$ a smooth map, $\nabla$ a connection on $K$, and $k,l, m \in \mathbb{N}_0$. Then we have
%\ba\label{EqGeilePullBackCommuteFormel}
%\mathrm{d}^{X^*\nabla}\mleft( X^!\omega \mright)
%&=
%X^! \mleft( \mathrm{d}^\nabla \omega \mright), \\
%\mathrm{d}^{\nabla+D} \omega
%&=
%\mathrm{d}^\nabla \omega + D \wedge \omega, \label{eqDifferentialSplit}, \\
%\mathrm{d}^\nabla \mleft( T \wedge \omega \mright)
%&=
%\mathrm{d}^\nabla T \wedge \omega
	%+ (-1)^m ~ T \wedge \mathrm{d}^\nabla \omega \label{TypischerSplitdesDifferentialsaufdasWedgeProdukt}
%\ea
%for all $\omega \in \Omega^l(N; K)$, $\psi \in \Omega^k(N;K)$, $D \in \Omega^1(N; \mathrm{End}(K))$, and $T \in \Omega^m(N; \mathrm{End}(K))$.
%\newline
%
%When $K$ is additionally an LAB, then we also have
%\ba
%\mleft( \mathrm{ad} \circ \omega \mright) \wedge \psi
%&=
%\mleft[ \omega \stackrel{\wedge}{,} \psi \mright]_K, \label{wedgeproduktmitadLambdaergibtLieklammer} \\
%X^!\mleft( \mleft[ \omega \stackrel{\wedge}{,} \psi \mright]_K \mright)
%&=
%\mleft[ X^!\omega \stackrel{\wedge}{,} X^!\psi \mright]_{X^*K}, \label{eqPullbackofLiebracketStuff} \\
%\mleft[ \omega \stackrel{\wedge}{,} \psi \mright]_K
%&=
%- (-1)^{lk} ~ \mleft[ \psi \stackrel{\wedge}{,} \omega \mright]_K, \label{VertauschungsregelForKKlammerAufFormen}\\
%\mleft[ \omega \stackrel{\wedge}{,} \mleft[ \omega \stackrel{\wedge}{,} \omega \mright]_K \mright]_K
%&=
%0 \label{JacobiIdentityForFormBracket}, \\
%\mathrm{ad}^* \circ X^!\omega
%&=
%X^!\mleft( \mathrm{ad} \circ \omega \mright) \label{EqCommutationRelation}
%\ea
%for all $\omega \in \Omega^l(N; K)$, $\psi \in \Omega^k(N;K)$, and smooth maps $X: M \to N$, where we write $\mathrm{ad}^*$ for the adjoint representation with respect to $\mleft[ \cdot, \cdot \mright]_{X^*K}$.
%\end{propositions}

The purpose of the following sections is now to introduce a new and more general formulation of gauge theory which we have introduced in Chapter \ref{ClassicGaugeTheory}. Especially recall the section about the infinitesimal gauge transformation using Lie algebra connections, Section \ref{NewInfGaugeTrafoTrafos}. Again, we do not want to assume integrability, and so we only compare the new theory with a classical gauge theory whose principal bundle is trivial and can thus be avoided completely by fixing a global gauge.\footnote{We will use Lie algebroids; their integration is more complicated than the integrability of Lie algebras, see \textit{e.g.}~\cite[\S 16.4; page 117]{DaSilva}.}

In that chapter we have used a "bookkeeping trick", denoted by $\iota$;\footnote{Recall the discussion about $\iota$ after Cor.~\ref{cor:ClassicFLowsOfXgMg}.} that is, generalized, that we had a spacetime $M$ and the Higgs field $\Phi$ is a smooth map $M \to N$. The physical quantities like the field strength then had values in $\mathrm{ev}^*K$ and hence in $\Phi^*K$ after point evaluation at $\Phi$, where $\mathrm{ev}$ was the evaluation map of Def.~\ref{def:FirstAttemptOfEvaluationMap} and $K$ was some vector bundle over $N$ (like the Lie algebra); also recall Remark \ref{rem:BosonsAsFunctionalies} where we argued that one can do something similar for the field of gauge bosons and its infinitesimal gauge transformation, we are going to do so, thus, viewing the field of gauge bosons of the classical formulation as forms with values in a $\Phi$-pullback of a trivial Lie algebra bundle. Moreover, we used $\mathfrak{g}$-connections, where $\mathfrak{g}$ is a Lie algebra acting on $N$ via a Lie algebra action $\gamma$. By Prop.~\ref{prop:ActionLieoidsAreOids} action Lie algebroids as bundle over $N$ are a good candidate describing that notion, or more general, Lie algebroids and the notion of Lie algebroid connections.

This is why we are going to define the following physical quantities as having values in some pullback using the evaluation map and $\Phi$ as for the field of gauge bosons, why we are going to use a Lie algebroid $E$ over $N$ instead of a Lie algebra $\mathfrak{g}$, and why we will compare the following definitions with action Lie algebroids in order to allow a comparison with Chapter \ref{ClassicGaugeTheory}. We will see that action Lie algebroids with their canonical flat connection will be the standard formulation of gauge theory.

Although we speak of $\Phi$ as the Higgs field it can be of course any other field with a similar Lagrangian, since we never really discuss the potential term. The Higgs field is just a main example.

If you are interested into the calculations of this and the following chapter, then read Appendix \ref{CalculusIdentitiesNeeded} first and the proofs listed there; certain steps of calculations are explained there which will be simply used in the following without further explanation. We also need a similar notation as in Def.~\ref{def:GradedExtensionOfBracket}, but extended to more than two arguments.

\begin{definitions}{Graded extension of products, \newline \cite[generalization of Definition 5.5.3; page 275]{hamilton}}{GradingOfProducts}
Let $l \in \mathbb{N}$ and $E_1, \dots E_{l+1} \to N$ be vector bundles over a smooth manifold $N$, and $F \in \Gamma\left( \left(\bigotimes_{m=1}^{l} E_m^*\right) \otimes E_{l+1} \right)$. Then we define the \textbf{graded extension of $F$} as
	\bas
\Omega^{k_1}(N; E_1) \times \dots \times \Omega^{k_l}(N; E_l)
&\to \Omega^{k}(N; E_{l+1}), \\
(A_1, \dots, A_l)
&\mapsto
F\mleft(A_1\stackrel{\wedge}{,} \dotsc \stackrel{\wedge}{,} A_l\mright),
\eas
where $k := k_1+\dots k_l$ and $k_i \in \mathbb{N}_0$ for all $i\in \{1, \dots, l\}$. $F\mleft(A_1\stackrel{\wedge}{,} \dotsc \stackrel{\wedge}{,} A_l\mright)$ is defined as an element of $\Omega^{k}(N; E_{l+1})$ by
\bas
&F\mleft(A_1\stackrel{\wedge}{,} \dotsc \stackrel{\wedge}{,} A_l\mright)\mleft(Y_1, \dots, Y_{k}\mright)
\coloneqq \\
&\frac{1}{k_1! \cdot \dots \cdot k_l!} \sum_{\sigma \in S_{k}} \mathrm{sgn}(\sigma) ~ F\left( A_1\left( Y_{\sigma(1)}, \dots, Y_{\sigma(k_1)} \right), \dots, A_l\left( Y_{\sigma(k-k_l+1)}, \dots, Y_{\sigma(k)} \right) \right)
\eas
for all $Y_1, \dots, Y_{k} \in \mathfrak{X}(N)$, where $S_{k}$ is the group of permutations of $\{1, \dots, k\}$ and $\mathrm{sgn}(\sigma)$ the signature of a given permutation $\sigma$. 

$\stackrel{\wedge}{,}$ may be written just as a comma when a zero-form is involved.

Locally, with respect to given frames $\mleft( e^{(i)}_{a_i} \mright)_{a_i}$ of $E_i$, this definition has the form
\ba\label{CoordExprOfGradedExtension}
F\mleft(A_1\stackrel{\wedge}{,} \dotsc \stackrel{\wedge}{,} A_l\mright)
&=
F\mleft(e^{(1)}_{a_1}, \dotsc, e^{(l)}_{a_l}\mright) \otimes A_1^{a_1} \wedge \dotsc \wedge A_l^{a_l}
\ea
for all $A_i = A_i^{a_i} \otimes e^{(i)}_{a_i}$, where $A_i^{a_i}$ are $k_i$-forms on $N$.
\end{definitions}

\begin{remark}
\leavevmode\newline
Using this notation, one has a useful way to compare pullbacks of forms, denoted by an exclamation mark, and pullbacks of sections, denoted by a star. That is, let $\Phi \in C^\infty(M;N)$ and $F \in \Omega^l(N;W)$ for $W \to N$ a vector bundle, then
\ba\label{EqPullBackFormelFuerVerschiedeneDefinitionen}
\Phi^!F 
&=
\frac{1}{l!}~
\mleft(\Phi^*F\mright) ( \underbrace{\mathrm{D}\Phi \stackrel{\wedge}{,} \dotsc \stackrel{\wedge}{,} \mathrm{D}\Phi}_{l \text{ times}} )
\ea
by using the anti-symmetry of $F$ and Def. \ref{def:GradingOfProducts}, \textit{i.e.}
\bas
&\mleft.\frac{1}{l!}~
\Big(\mleft(\Phi^*F\mright) ( \mathrm{D}\Phi \stackrel{\wedge}{,} \dotsc \stackrel{\wedge}{,} \mathrm{D}\Phi ) \Big) (Y_1, \dots, Y_l)\mright|_p \\
&\hspace{1cm}
=
\frac{1}{l!}~
\sum_{\sigma \in S_{l}} \mathrm{sgn}(\sigma) ~ \underbrace{(\Phi^*F)\mleft(\mathrm{D}\Phi\mleft(Y_{\sigma(1)}\mright), \dots, \mathrm{D}\Phi\mleft(Y_{\sigma(l)}\mright)\mright)}_{\mathclap{= \mathrm{sgn}(\sigma) ~ (\Phi^*F)\mleft(\mathrm{D}\Phi\mleft(Y_{1}\mright), \dots, \mathrm{D}\Phi\mleft(Y_{l}\mright)\mright)}}\Big|_p \\
&\hspace{1cm}
=
\frac{1}{l!}~ \underbrace{\mleft( \sum_{\sigma \in S_{l}} 1 \mright)}_{= l!} ~
F_{\Phi(p)}\mleft(\mathrm{D}_p\Phi\mleft(\mleft.Y_{1}\mright|_p\mright), \dots, \mathrm{D}_p\Phi\mleft(\mleft.Y_{l}\mright|_{p}\mright)\mright) \\
&\hspace{1cm}
= \mleft.\mleft(\Phi^!F\mright)(Y_1, \dots, Y_l)\mright|_p
\eas
for all $p \in M$ and $Y_1, \dots, Y_l \in \mathfrak{X}(M)$.
\end{remark}

In case of antisymmetric tensors we of course preserve that.

\begin{propositions}{Graded extensions of antisymmetric tensors}{GradedExtensionPlusAntiSymm}
Let $E_1, E_2 \to N$ be real vector bundles of finite rank over a smooth manifold $N$, $F \in \Omega^2(E_1; E_2)$. Then
\ba
F \mleft( A \stackrel{\wedge}{,} B \mright)
&=
-\mleft( -1 \mright)^{km}
F \mleft( B \stackrel{\wedge}{,} A \mright)
\ea 
for all $A \in \Omega^k(N; E_1)$ and $B \in \Omega^m(N; E_2)$ ($k,m \in \mathbb{N}_0$). Similarly extended to all $F \in \Omega^l(E_1; E_2)$.
\end{propositions}

\begin{remark}
\leavevmode\newline
This is a generalization of similar relations just using the Lie algebra bracket $\mleft[ \cdot, \cdot\mright]_{\mathfrak{g}}$ of a Lie algebra $\mathfrak{g}$, see \cite[\S 5, first statement of Exercise 5.15.14; page 316]{hamilton}.
\end{remark}

\begin{proof}
\leavevmode\newline
Trivial by using Eq.~\eqref{CoordExprOfGradedExtension}.
\end{proof}

\section{Space of fields}\label{SpaceOfFieldsSection}

Before we can define quantities like the field strength, we need to define and study the infinite-dimensional manifold of the arising fields as we did in the classical situation; recall Def.~\ref{def:ClassicSpaceofFieldsAgain}. Because of the non-triviality of the following bundles we need to take a closer look at this space. Recall that we assume convenient settings when treating infinite-dimensional objects.

\begin{definitions}{Space of fields}{SpaceOfFields}
Let $M, N$ be two smooth manifolds and $E\to N$ a Lie algebroid. Then we denote the \textbf{space of fields} by
\ba
\gls{MSpaceOfFields}
&\coloneqq
\mathfrak{M}_E(M; N)
\coloneqq
\left\{ (\Phi, A)
~\middle|~
\Phi \in C^\infty(M;N) \text{ and } A \in \Omega^1(M; \Phi^*E)
\right\}
\ea
which we sometimes view as a fibration over $C^\infty(M;N)$
\begin{center}
	\begin{tikzcd}
		\mathfrak{M}_E(M; N) \arrow{d} \\
		C^\infty(M;N)
	\end{tikzcd}
\end{center}
where the projection is given by $\mathfrak{M}_E(M; N) \ni (\Phi, A) \mapsto \Phi$.
%Thus for $\mathfrak{M}_E(M; N)$ we sometimes write
%\begin{center}
	%\begin{tikzcd}
		%\Omega^1(M;{}^*E) \arrow{d} \\
		%C^\infty(M;N)
	%\end{tikzcd}
%\end{center}

We will refer to $\gls{a0} \in \Omega^1(M; \Phi^*E)$ as the \textbf{field of gauge bosons} and $\gls{1vhi}$ just as a \textbf{physical field} of this theory.
\end{definitions}

Let us look at the tangent space of $\mathfrak{M}_E(M; N)$; we are interested into that because of the identification of infinitesimal gauge transformations as tangent vectors. Also recall the discussion about the double vector bundle structure before Def.~\ref{def:LinearVectorFieldsOnVectorBundles} which we need now again.

\begin{propositions}{Tangent space of $\mathfrak{M}_E(M; N)$}{TangentSpaceOfSpaceOfFields}
Let $M, N$ be two smooth manifolds and $E \stackrel{\pi}{\to} N$ a Lie algebroid. Then the tangent space $\mathrm{T}_{(\Phi_0,A_0)} \bigl(\mathfrak{M}_E(M; N)\bigr)$ of $\mathfrak{M}_E(M; N)$ at $(\Phi_0, A_0)$ consists of pairs $(\mathcal{v}, \mathcal{a})$ with $\mathcal{v} \in \Gamma\mleft(\Phi_0^*\mathrm{T}N\mright)$ and $\mathcal{a} \in \Omega^1\mleft(M; \mathcal{v}^*\mathrm{T}E\mright)$, where $\mathcal{v}^*\mathrm{T}E$ is the pullback of $\mathrm{T}E \stackrel{\mathrm{D}\pi}{\to}\mathrm{T}N$ as a vector bundle, viewing $\mathcal{v}$ as a map $M \to \mathrm{T}N$. This pair also satisfies
\ba
\pi_{\mathrm{T}E}(\mathcal{a})
&=
A_0,
\ea
where $\pi_{\mathrm{T}E}$ denotes the projection of the vector bundle $\mathrm{T}E \to E$.
\end{propositions}

\begin{remarks}{Total situation as commuting diagram}{TangentCommutingDiagram}
This implies that we have in total\footnote{Recall that we view sections of pullback bundles also as sections along maps; see Section \ref{StandardNotation}.}
\begin{center}
	\begin{tikzcd}
		 \mathrm{T}E \arrow{rr}{\mathrm{D}\pi} \arrow[dd, "\pi_{\mathrm{T}E}", swap] && \mathrm{T}N \arrow{dd}{\pi_{\mathrm{T}N}} \\
		&M \arrow[ld, "A_0(Y)", pos=0.3] \arrow{rd}{\Phi_0} \arrow{ru}{\mathcal{v}} \arrow[lu, "\mathcal{a}(Y)", pos=0.2]  \\
		E \arrow[rr, "\pi", swap]&& N
	\end{tikzcd}
\end{center}
for all $(\Phi_0,A_0) \in \mathfrak{M}_E(M;N)$, $(\mathcal{v}, \mathcal{a}) \in \mathrm{T}_{(\Phi_0, A_0)}\bigl( \mathfrak{M}_E(M;N) \bigr)$ and $Y \in \mathfrak{X}(M)$, that is,
\ba
\pi\bigl(A_0(Y)\bigr)
&=
\Phi_0,
\\
\pi_{\mathrm{T}N} (\mathcal{v})
&=
\Phi_0,
\\
\pi_{\mathrm{T}E}(\mathcal{a})
&=
A_0,\label{AGaugeTrafoIsOverA}
\\
\mathrm{D}\pi \bigl( \mathcal{a}(Y) \bigr)
&=
\mathcal{v}\label{HorizontalCompOfDeltaA}
\ea
for all $Y \in \mathfrak{X}(M)$, where the projections of the vector bundles $\mathrm{T}E \to E$ and $\mathrm{T}N \to N$ are denoted by $\pi_{\mathrm{T}E}$ and $\pi_{\mathrm{T}N}$, respectively.
\end{remarks}
\newpage
\begin{remark}\label{RemarkAboutThatWeStillHaveLinearStructureinDeltaA}
\leavevmode\newline
Especially for Eq.~\eqref{HorizontalCompOfDeltaA} recall the discussion about the double vector bundle structure before Def.~\ref{def:LinearVectorFieldsOnVectorBundles}. That is,
\bas
\mathcal{a}(f Y + h Z)
&=
f \boldsymbol{\cdot} \mathcal{a}(Y)
	\RPlus h \boldsymbol{\cdot} \mathcal{a}(Z)
\eas
for all $Y, Z \in \mathfrak{X}(M)$ and $f, h \in C^\infty(M)$, because $\mathcal{a}$ has values in $\mathrm{T}E$ viewed as a vector bundle over $\mathrm{T}N$. Therefore also
\bas
\mathrm{D}\pi\bigl( \mathcal{a}(f Y + h Z) \bigr)
&=
\mathrm{D}\pi\bigl( \mathcal{a}(Y) \bigr).
\eas
This is also in alignment with Eq.~\eqref{AGaugeTrafoIsOverA} although it is about the vector bundle $\mathrm{T}E \to E$, so,
\bas
\pi_{\mathrm{T}E}\bigl( \mathcal{a}(f Y + h Z) \bigr)
&=
\pi_{\mathrm{T}E}\bigl( f \boldsymbol{\cdot} \mathcal{a}(Y)
	\RPlus h \boldsymbol{\cdot} \mathcal{a}(Z) \bigr)
\\
&=
f ~ \pi_{\mathrm{T}E}\bigl( \mathcal{a}(Y) \bigr)
	+ h ~ \pi_{\mathrm{T}E}\bigl( \mathcal{a}(Z) \bigr)
\\
&=
A_0\bigl( f Y + h Z \bigr).
\eas
\end{remark}

\begin{proof}[Proof of Prop.~\ref{prop:TangentSpaceOfSpaceOfFields}]
\leavevmode\newline
We identify the tangent spaces of $(\Phi_0, A_0) \in \mathfrak{M}_E(M; N)$ with the set consisting of elements of the form
\bas
\mleft.\frac{\mathrm{d}}{\mathrm{d}t}\mright|_{t=0} \gamma,
\eas
where $\gamma: I \to \mathfrak{M}_E(M; N)$ is a curve with $\gamma(0) = (\Phi_0, A_0)$ and $I$ an open interval of $\mathbb{R}$ around 0. Since we do not have any conditions on $\mathfrak{M}_E(M; N)$ besides that $A_0$ has values in $\Phi_0^*E$, we will see that we just need to describe where the "velocity" of the curves live, and surjectivity will then just follow by that we always can find curves with arbitrary initial conditions on position and velocity. Let us write $\gamma = (\Phi, A)$, $t \mapsto \gamma(t)=(\Phi_t, A_t)$, with
\bas
\Phi_t &\in C^\infty(M;N), 
&
\Phi_{t=0} &= \Phi_0,
\\
A_t &\in \Omega^1(M; \Phi^*_t E),
&
A_{t=0} &= A_0
\eas
for all $t \in I$. As usual, the tangent space consists of elements of the form
\bas
\mleft( \mleft.\frac{\mathrm{d}}{\mathrm{d}t}\mright|_{t=0} \mleft[t\mapsto\Phi_t\mright], 
	\mleft.\frac{\mathrm{d}}{\mathrm{d}t}\mright|_{t=0} \mleft[t\mapsto A_t\mright]
\mright).
\eas
Hence, for all $p\in M$ we have a curve $\Phi(p) \coloneqq \mleft[ t \mapsto \Phi_t(p) \mright]$ in $N$ with
\bas
\mleft.\frac{\mathrm{d}}{\mathrm{d}t}\mright|_{t=0} (\Phi(p)) &\in \mathrm{T}_{\Phi_0(p)}N,
\eas
such that for all curves $\Phi$
\bas
\mleft.\frac{\mathrm{d}}{\mathrm{d}t}\mright|_{t=0} [t\mapsto \Phi_t]
&\in
\Gamma(\Phi_0^*\mathrm{T}N),
\eas
and besides $\Phi_{t=0}(p) = \Phi_0(p)$ there is no other condition on $\Phi(p)$, thus, for all $v \in \mathrm{T}_{\Phi_0(p)}N$ there is a curve $\Phi(p)$ such that its "initial velocity" is $v$, \textit{i.e.}~
\bas
v &= \mleft.\frac{\mathrm{d}}{\mathrm{d}t}\mright|_{t=0} \bigl(\Phi(p)\bigr),
\eas
and extending this argument we can achieve that for all $\mathcal{v} \in \Gamma(\Phi_0^*\mathrm{T}N)$ there is a curve $\Phi$ such that
\bas
\mathcal{v}
&=
\mleft.\frac{\mathrm{d}}{\mathrm{d}t}\mright|_{t=0} [t\mapsto\Phi_t],
\eas

Now we fix such a curve $\Phi$ for a fixed $\mathcal{v}$. Let us look at the curve $A(Y) \coloneqq \mleft[ t \mapsto A_t(Y) \mright]$ for all $Y \in \mathfrak{X}(M)$, that is $A(Y): I \times M \to E$, $(t,p) \mapsto A_{t,p}(Y_p)$ with $\pi \circ A(Y) = \Phi$, where $\pi$ is the projection of $E$ onto N. So,
\bas
\mathrm{T}_{\Phi_0(p)}N \ni
\mathcal{v}_p
&=
\mleft.\frac{\mathrm{d}}{\mathrm{d}t}\mright|_{t=0} \bigl(\pi ( A_p(Y_p))\bigr)
=
\mathrm{D}_{A_{0}(Y)|_p} \pi \mleft( \mleft.\frac{\mathrm{d}}{\mathrm{d}t}\mright|_{t=0} A_p(Y_p) \mright)
=
\mathrm{D}_{A_{0}(Y)|_p} \pi (\mathcal{a}_p(Y_p)),
\eas
where 
\bas
\mathcal{a}_p(Y_p)
&\coloneqq
\mleft.\frac{\mathrm{d}}{\mathrm{d}t} \mright|_{t=0} [ t\mapsto A_{t,p}(Y_p) ]
\in \mathrm{T}_{A_{0}(Y)|_p} E
\eas
for all $p \in M$. Hence, we can also see $\mathcal{a}$ equivalently as a form on $M$ with values in $\mathrm{T}E$ such that
\bas
\pi_{\mathrm{T}E}(\mathcal{a})
&=
A,
\\
\mathrm{D}\pi\bigl( \mathcal{a}(Y) \bigr)
&=
\mathcal{v}
\eas
for all $Y\in\mathfrak{X}(M)$, and we view $\mathcal{a}$ as an element of $\Omega^1(M; \mathcal{v}^*\mathrm{T}E)$, too, where we view $\mathrm{T}E$ as the vector bundle $\mathrm{T}E \stackrel{\mathrm{D}\pi}{\to} \mathrm{T}N$; that is because of the following: Let $Z \in \mathfrak{X}(M)$ be another vector field and $f,h \in C^\infty(M)$, then
\bas
\mathcal{a}_p\mleft(f(p) ~ Y_p + h(p) ~ Z_p\mright)
&=
\mleft.\frac{\mathrm{d}}{\mathrm{d}t} \mright|_{t=0} \mleft[ t\mapsto 
	A_{t,p}\mleft(f(p) ~ Y_p + h(p) ~ Z_p\mright) 
\mright]
\\
&=
\mleft.\frac{\mathrm{d}}{\mathrm{d}t} \mright|_{t=0} \mleft[ t\mapsto 
	f(p) ~ A_{t,p}\mleft( Y_p \mright) 
	+ h(p) ~ A_{t,p}\mleft( Z_p \mright) 
\mright]
\\
&=
f(p) \boldsymbol{\cdot} \mathcal{a}_p(Y_p)
	\RPlus h(p) \boldsymbol{\cdot} \mathcal{a}_p(Z_p),
\eas
because of
\bas
\mathrm{D}_{A_p(Y_p)}\pi\mleft(\mathcal{a}_p(Y_p)\mright)
&=
\mathcal{v}_p
=
\mathrm{D}_{A_p(Z_p)}\pi\mleft(\mathcal{a}_p(Z_p)\mright)
\eas
and since $[ t\mapsto A_{t,p}(Y_p) ]$ and $[ t\mapsto A_{t,p}(Z_p) ]$ are the representing curves of $\mathcal{a}_p(Y_p)$ and $\mathcal{a}_p(Z_p)$ as tangent vectors, respectively, satisfying
\bas
\pi\mleft(A_{t,p}(Y_p)\mright)
&=
\Phi_t(p)
=
\pi\mleft(A_{t,p}(Z_p)\mright),
\eas
such that we precisely get the definitions of $\boldsymbol{\cdot}$ and $\RPlus$.

As before, we can conclude that we can find a curve $A$ for all $\mathcal{a} \in \Omega^1(M; \mathcal{v}^*\mathrm{T}E)$ such that
\bas
\mathcal{a}
&=
\mleft.\frac{\mathrm{d}}{\mathrm{d}t} \mright|_{t=0} A.
\eas
(In this proof, we make use of the homotopy lifting property of fibrations such that we can find an $A(Y): I \times M \to E$ for each $\Phi: I \times M \to N$ for all $(\Phi_0, A_0) \in \mathfrak{M}_E(M; N)$ with the suitable properties.)
\end{proof}

Think of $(\mathcal{v}, \mathcal{a})$ again as candidates for the infinitesimal gauge transformations, for which we wrote $(\delta \Phi, \delta A)$ in Chapter \ref{ClassicGaugeTheory}; also recall Remark \ref{rem:TangentSpaceOfMathfrakMg}. But other than in Remark \ref{rem:TangentSpaceOfMathfrakMg} we cannot assume canonical flat connections now which is why the last result shows that we cannot view $(\mathcal{v}, \mathcal{a})$ as an element of $\mathfrak{M}_E(M;N)$ in general, thus, we changed the notation to $(\mathcal{v}, \mathcal{a})$ for now. So, we do not have any canonical horizontal distribution given, and therefore let us study the vertical structure first.

Recall that there is the notion of a \textbf{vertical bundle} for fibre bundles $F \stackrel{\pi}{\to} N$ (as \textit{e.g.}~introduced in \cite[\S 5.1.1, for principal bundles, but it is straightforward to extend the definitions; page 258ff.]{hamilton}), which is defined as a subbundle $\gls{VF} \to F$ of the tangent bundle $\mathrm{T}F \to F$ given as the kernel of $\mathrm{D}\pi : \mathrm{T}F \to \mathrm{T}N$. The fibres $\mathrm{V}_eF$ of $F$ at $e \in F$ are then given by 
\bas
\mathrm{V}_e F
&=
\mathrm{T}_e F_p,
\eas
where $p \coloneqq \pi(e) \in N$ and $F_p$ is the fibre of $F$ at $p$.
Now consider a vector bundle $E \stackrel{\pi}{\to} N$, then $\mathrm{V}_e E = \mathrm{T}_e E_p \cong E_p$ because the fibres are vector spaces.
%This can be emphasized by looking at the pullback vector bundle $\pi^*E \to E$ (\textbf{COMMENT: Reference?})
%\begin{center}
	%\begin{tikzcd}
		 %\pi^*E \arrow{d} & E \arrow{d}{\pi} \\
		%E \arrow{r}{\pi} & N
	%\end{tikzcd}
%\end{center}
%Elements of $\pi^*E$ are pairs $(e, e^\prime)$ defined by $e, e^\prime \in E$ and $\pi(e)= \pi(e^\prime)$, hence, $e^\prime \in E_{p}$ for $p = \pi(e)$. Thus, each fibre $(\pi^*E)_e$ of $\pi^*E$ is given by $E_p$, and therefore there is a very natural vector bundle isomorphism $\pi^*E \cong \mathrm{V}E$ because the tangent bundle $\mathrm{T}E$ also consists of pairs $(e, v)$ with $e \in E$ and $v \in \mathrm{T}_e E$, for the latter we then make use of the natural isomorphism $\mathrm{T}_e E_p \cong E_p$ when looking at $\mathrm{V}E$.

\begin{propositions}{Vertical bundle of $\mathfrak{M}_E(M; N)$}{VerticalBundleOfFracM}
Let $M, N$ be two smooth manifolds and $E \stackrel{\pi}{\to} N$ a Lie algebroid. Then the vertical bundle of $\mathfrak{M}_E(M; N)$, viewed as a fibration over $C^\infty(M;N)$, is given by
\ba
\mathrm{V}_{(\Phi,A)}\bigl(\mathfrak{M}_E(M; N)\bigr)
&\cong
\left\{
	(\mathcal{v}, \mathcal{a})
~\middle|~
	\mathcal{v}= 0 \in \Gamma(\Phi^*\mathrm{T}N), ~
	\mathcal{a} \in \Omega^1(M; \Phi^*E)
\right\}
\cong
\Omega^1(M; \Phi^*E).
\ea
%In total, $\mathrm{V}\mathfrak{M}_E(M; N)$ is isomorphic to
%\begin{center}
	%\begin{tikzcd}
		%\Omega^1(M; {}^*E) \arrow{d} \\
		%\mathfrak{M}_E(M; N)
	%\end{tikzcd}
%\end{center}
\end{propositions}

%\begin{remark}
%\leavevmode\newline
%We can quickly motivate the result of the proposition by using what we discussed before this statement, \textit{i.e.}~we identify $\mathrm{V}\mathfrak{M}_E(M; N)$ with $\varpi^*\mathfrak{M}_E(M; N)$, where $\varpi$ is the projection of $\mathfrak{M}_E(M; N) \to C^\infty(M;N)$,
%\begin{center}
	%\begin{tikzcd}
		 %\varpi^*\mathfrak{M}_E(M; N) \arrow{d} & \mathfrak{M}_E(M; N) \arrow{d}{\varpi} \\
		 %\mathfrak{M}_E(M; N) \arrow{r}{\varpi} & C^\infty(M; N)
	%\end{tikzcd}
%\end{center}
%Therefore the fibre of $\mathrm{V}\mathfrak{M}_E(M; N) = \varpi^*\mathfrak{M}_E(M; N)$ at $(\Phi,A)\in \mathfrak{M}_E(M; N)$ is given by the fibre of $\mathfrak{M}_E(M; N)$ at
%\bas
%\varpi(\Phi, A) &=\Phi,
%\eas
%which is $\Omega^1(M; \Phi^*E)$.
%\end{remark}

\begin{proof}[Proof of Prop.~\ref{prop:VerticalBundleOfFracM}]
\leavevmode\newline
We have the fibration $\mathfrak{M}_E(M; N) \stackrel{\varpi}{\to} C^\infty(M;N)$, where $\varpi(\Phi, A) \coloneqq \Phi$ for all $(\Phi, A) \in \mathfrak{M}_E(M; N)$. Hence,
\bas
\mathrm{D}_{(\Phi, A)}\varpi(\mathcal{v}, \mathcal{a})
=
\mathcal{v}
\eas
for all $(\mathcal{v}, \mathcal{a}) \in \mathrm{T}_{(\Phi,A)}\mathfrak{M}_E(M; N)$. The kernel of $\mathrm{D}\varpi$ at $(\Phi,A) \in \mathfrak{M}_E(M; N)$ is then given by 
\bas
\mathrm{Ker}\mleft( \mathrm{D}_{(\Phi,A)} \varpi \mright)
&=
\left\{
(\mathcal{v}, \mathcal{a}) \in \mathrm{T}_{(\Phi,A)} \mathfrak{M}_E(M; N)
~\middle|~
\mathcal{v}=0
\right\}.
\eas
By Prop.~\ref{prop:TangentSpaceOfSpaceOfFields}, we then know that $\mathcal{a}$ has values in the vertical bundle $\mathrm{V}E$, that is, for $\mathcal{a}_p(Y_p) \in \mathrm{T}_{A_p(Y_p)} E$ ($p \in M$, $Y \in \mathfrak{X}(M)$) we have
\bas
&&
\mathrm{D}_{A_p(Y_p)}\pi\mleft(\mathcal{a}_p(Y_p)\mright)
&=
0
\\
&\Leftrightarrow&
\mathcal{a}_p(Y_p)
&\in
\mathrm{V}_{A_p(Y_p)} E
\cong
E_{\Phi(p)}.
\eas
Thus, we can view $\mathcal{a}$ equivalently as an element of $\Omega^1(M; \Phi^*E)$, so,
\bas
\mathrm{V}_{(\Phi,A)}\mathfrak{M}_E(M; N)
&\cong
\left\{
	(\mathcal{v}, \mathcal{a})
~\middle|~
	\mathcal{v}= 0 \in \Gamma(\Phi^*\mathrm{T}N), ~
	\mathcal{a} \in \Omega^1(M; \Phi^*E)
\right\}
\cong
\Omega^1(M; \Phi^*E).
\eas
\end{proof}

That is, we can in general only expect to have $(\mathcal{v}, \mathcal{a}) \in \mathfrak{M}_E(M;N)$ if at least $\mathcal{v}=0$. Recall that we identified this component with the infinitesimal gauge transformation of the Higgs field which was proportional to the Lie algebra representation, see Def.~\ref{def:ClassicTrafos}. Even when we do not have yet the general definition of that infinitesimal gauge transformation, it is natural to assume that this transformation is therefore only zero when there is no coupling of the gauge bosons to the Higgs field (= zero action), but in general there will be of course a coupling. As already mentioned, we circumvented that problem in Chapter \ref{ClassicGaugeTheory} by choosing canonical flat connections; moreover, observe that this condition about $\mathcal{v}=0$ comes from that the field of gauge bosons $A$ has values in $\Phi^*E$, as if we would have applied the "bookkeeping trick" to $A$ in Section \ref{NewInfGaugeTrafoTrafos}, too. Thus, we are going to treat the infinitesimal gauge transformation of $A$ similar to how we defined the infinitesimal gauge transformation for functionals in Section \ref{NewInfGaugeTrafoTrafos}, then we also achieve that its transformation can be viewed again as an element of $\Omega^1(M; \Phi^*E)$, simplifying further calculations, without really loosing information about the transformation of $A$; we will explain this later. That the infinitesimal gauge transformation of the Higgs field is in general not a smooth map $M \to N$ will be on the other hand actually less of a problem.

But before we can make that mathematical precise, we need to define at what type of functionals we are going to look at. One key step is to look at $M \times \mathfrak{M}_E(M;N)$ as we did in Def.~\ref{def:FirstAttemptOfEvaluationMap} and afterwards.

\begin{definitions}{Evaluation map of $M \times \mathfrak{M}_E$}{EvaluationMap}
Let $M, N$ be manifolds, and $E \to N$ a Lie algebroid over $N$.
Then we define the \textbf{evaluation map} $\mathrm{ev}$ by
\ba
M \times \mathfrak{M}_E(M;N) &\to N
\nonumber\\
(\Phi,A)
&\mapsto
\mathrm{ev}(p, \Phi, A)
\coloneqq
\Phi(p)
\ea
for all $p\in M$ and $(\Phi, A) \in \mathfrak{M}_E$.
\end{definitions}

\begin{remarks}{Bigrading of forms on $M \times \mathfrak{M}_E$}{Bigrading}
Let $\pi_i$ ($i \in \{1,2\}$) be the projection onto the $i$-th factor in $M \times \mathfrak{M}_E$, then
\ba
\mathrm{T}\mleft( M \times \mathfrak{M}_E \mright)
\cong
\pi_1^*\mathrm{T}M \oplus \pi_2^*\mathrm{T}\mathfrak{M}_E.
\ea
Gives rise to a bigrading of $\bigwedge^k \mathrm{T}^*\mleft( M \times \mathfrak{M}_E \mright)$ ($k \in \mathbb{N}_0$),
\ba
\bigwedge^k \mathrm{T}^*\mleft( M \times \mathfrak{M}_E \mright)
\cong
\bigoplus_{\substack{p,q \in \mathbb{N}_0 \\ p+q = k}} \mleft(
	\bigwedge^{p,q} \mathrm{T}^*\mleft( M \times \mathfrak{M}_E \mright)
\mright),
\ea
where
\ba
\bigwedge^{p,q} \mathrm{T}^*\mleft( M \times \mathfrak{M}_E \mright)
&\coloneqq
\pi_1^*\mleft(\bigwedge^p \mathrm{T}^*M\mright) \otimes \pi_2^*\mleft(\bigwedge^q \mathrm{T}^*\mathfrak{M}_E\mright).
\ea
Similarly, for $V$ a vector bundle over $M \times \mathfrak{M}_E$,
\ba
\Omega^k\mleft( M \times \mathfrak{M}_E; V \mright)
\cong
\bigoplus_{\substack{p,q \in \mathbb{N}_0 \\ p+q = k}} \bigl(
	\Omega^{p,q} \mleft( M \times \mathfrak{M}_E; V \mright)
\bigr),
\ea
with 
\ba
\Omega^{p,q} \mleft( M \times \mathfrak{M}_E; V \mright)
\coloneqq
\Gamma\mleft(
	\pi_1^*\mleft(\bigwedge^p \mathrm{T}^*M\mright) \otimes \pi_2^*\mleft(\bigwedge^q \mathrm{T}^*\mathfrak{M}_E\mright) \otimes V
\mright).
\ea
When $V$ is the trivial line bundle, then we just write $\Omega^{p,q}(M \times \mathfrak{M}_E)$.

If $V$ is instead a vector bundle over $N$, then we have $\mathrm{ev}^*V$ naturally as bundle over $M \times \mathfrak{M}_E$. Then, when taking a slice through $(\Phi, A) \in \mathfrak{M}_E$, \textit{i.e.}~evaluating a form at points $M \times \{\Phi, A\}$ while $(\Phi, A) \in \mathfrak{M}_E$ is fixed,
\ba\label{SliceOfBIiiigManifold}
\mleft.L\mright|_{M \times \{\Phi, A\}}
&\in
\Omega^p(M; \Phi^*V)
\ea
for all $L \in \Omega^{p,0} \mleft( M \times \mathfrak{M}_E; \mathrm{ev}^*V \mright)$.
Similarly, the de-Rham differential splits on $\Omega^k(M \times \mathfrak{M}_E)$ as a differential along $M$ and $\mathfrak{M}_E$, $\mathrm{d}_{\text{total}} = \mathrm{d}_M + \mathrm{d}_{\mathfrak{M}_E}$. When using exterior derivatives, then we focus on directions along $M$, and we will denote that de-Rham differential by $\mathrm{d}$, \textit{i.e.}~$\mathrm{d}= \mathrm{d}_M$.
\end{remarks}

\begin{remark}
\leavevmode\newline
Do not confuse notations like $\Omega^{p,q} \mleft( M \times \mathfrak{M}_E; V \mright)$ with the notation given in Def.~\ref{def:ExteriorCovariantDerivatives}; it will be clear by the context which we mean, and, besides the next paragraphs, we actually will not really use $\Omega^{p,q} \mleft( M \times \mathfrak{M}_E; V \mright)$ as notation anymore because we only want to motivate the next and some following definitions with this notation.
\end{remark}

Eq.~\eqref{SliceOfBIiiigManifold} is precisely the space our functionals should take values in when evaluated at $(\Phi, A) \in \mathfrak{M}_E$. This leads to the following definition.

\begin{definitions}{Space of functionals in gauge theory}{FunctionalsAsForms}
Let $M, N$ be two smooth manifolds, $E\to N$ a Lie algebroid, and $V \to N$ a vector bundle. Then the \textbf{space of functionals $\gls{Fk}(M; {}^*V)$} ($k \in \mathbb{N}_0$) is defined as
\ba
\mathcal{F}^k_E(M; {}^*V)
&\coloneqq
\Omega^{k,0}\bigl(M \times \mathfrak{M}_E(M;N); \mathrm{ev}^*V\bigr).
\ea

If $V = N \times \mathbb{R}$ is the trivial line bundle over $N$, then we just write $\mathcal{F}_E^k(M)$ instead of $\mathcal{F}^k_E(M;{}^*V)$.
\end{definitions}

\begin{remark}
\leavevmode\newline
We often write for $L \in \mathcal{F}^k_E(M; {}^*V)$
\bas
\mathfrak{M}_E \ni (\Phi, A)
&\mapsto
L(\Phi, A) 
\coloneqq 
\mleft.L\mright|_{M \times \{\Phi, A\}}
\in \Omega^k(M; \Phi^*V)
\eas
especially when we do not evaluate at $p \in M$; recall Eq.~\eqref{SliceOfBIiiigManifold}. Observe that $L$ acts non-trivially only on $\mathrm{T}M$.
\end{remark}

\begin{examples}{Projection onto the field of gauge bosons}{ProjectionOntoGaugeBosonies}
Besides the physical quantities which we will define later, we have an important and trivial functional $\gls{1pivar} \in \mathcal{F}^1_E(M; {}^*E)$ given as the projection onto the field of gauge bosons, that is
\ba
\varpi_2(\Phi,A)
&\coloneqq
A
\ea
for all $(\Phi, A) \in \mathfrak{M}_E$.
We will especially need this functional to define the infinitesimal gauge transformation of $A$ and in several combinations with other functionals.
\end{examples}

\begin{examples}{Tangent map, total differential as functional}{DAsFunctional}
Also the total differential $\gls{D}$ can be viewed as a functional. That is $\mathrm{D} \in \mathcal{F}^1_E(M; {}^*\mathrm{T}N)$ by
\ba
\mathrm{D}(\Phi, A)
&\coloneqq
\mathrm{D}\Phi
\in
\Omega^1(M; \Phi^*\mathrm{T}N).
\ea
Hence, when we just write $\mathrm{D}$, then we mean precisely that. 
\end{examples}

For the following discussion and definitions we use a similar convention of notation as in Section \ref{DirectProdsOfLieAlgoids}. That is, we have $\mathrm{T}(M \times \mathfrak{M}_E) \cong \pi_1^*\mathrm{T}M \oplus \pi_2^*\mathrm{T}\mathfrak{M}_E$ as in Remark \ref{rem:Bigrading}. If we speak for example about $\mathrm{T}M$, especially sections thereof, $\mathfrak{X}(M)$, then we mean their canonical embedding as a subalgebra of $\mathfrak{X}(M \times \mathfrak{M}_E)$; so, $X \in \mathfrak{X}(M)$ is also viewed as an element of $\mathfrak{X}(M\times \mathfrak{M}_E)$ but constant along $\mathfrak{M}_E$. For vector bundle morphisms defined on $\mathrm{T}(M \times \mathfrak{M}_E)$ we then also mean that forms restricted onto $\mathrm{T}M$ extend to maps acting on $\mathfrak{X}(M)$.

\begin{remarks}{Notions on $\mathcal{F}^k_E$ and further pullbacks with $\mathrm{ev}$}{NotionsOnFunctionals}
By Def.~\ref{def:FunctionalsAsForms}, we recover typical notions on the space of functionals, notions like wedge products, Def.~\ref{def:GradingOfProducts} and contractions \textit{etc.}~by restricting notions on $\Omega^\bullet(M \times \mathfrak{M}_E)$ and $\Omega^\bullet(M \times \mathfrak{M}_E; \mathrm{ev}^*V)$ to $\Omega^{\bullet,0}(M \times \mathfrak{M}_E)$ and $\Omega^{\bullet,0}(M \times \mathfrak{M}_E; \mathrm{ev}^*V)$ ($\bullet$ as placeholder for the degree), respectively. Hence, we will not need to define all those notions in that setting, and, especially, $\Gamma(\mathrm{ev}^*V)$ is therefore generated by elements of the form $\mathrm{ev}^*v$, where $v \in \Gamma(V)$. 

\hspace{0.3cm} Now assume we have a vector bundle connection $\nabla$ on $V$, then $\mathrm{ev}^*\nabla$ is a connection on $\mathrm{ev}^*V$. We want to restrict the exterior covariant derivative related to that connection just to vector fields on $M$. Observe for all $X\in \mathfrak{X}(M) \subset \mathfrak{X}(M \times \mathfrak{M}_E)$, with flow $\gamma$ in $M$ through a $p \in M$, $(t, p) \mapsto \gamma_t(p)$ ($t \in I$ for some open interval in $\mathbb{R}$ containing 0),
\ba\label{DGleichDev}
\mathrm{D}_{(p, \Phi, A)} \mathrm{ev} (X)
&=
\mleft.\frac{\mathrm{d}}{\mathrm{d}t}\mright|_{t=0}
\bigl(
	\mathrm{ev} \circ (\gamma(p), \Phi, A)
\bigr)
=
\mleft.\frac{\mathrm{d}}{\mathrm{d}t}\mright|_{t=0}
\bigl(
	(\Phi \circ \gamma )(p) 
\bigr)
=
\mathrm{D}_p \Phi (X)
\ea
for all $(p, \Phi, A) \in M \times \mathfrak{M}_E$, where $(\gamma(p), \Phi, A)$ is the flow of $X \in \mathfrak{X}(M)$ at $(p, \Phi, A)$, viewed as an element of $\mathfrak{X}(M \times \mathfrak{M}_E)$. So, the pushforward of $X$ with $\mathrm{ev}$ at $(\Phi,A)$ is the same as the pushforward of $X$ with $\Phi$, thus
\bas
\mleft(\mathrm{ev}^* \nabla\mright)_{X_{(p, \Phi, A)}}
&=
\mleft( \Phi^*\nabla \mright)_{X_p}
\eas
for all $(p, \Phi, A)$,
viewing $X$ as an element of $\mathfrak{X}(M \times \mathfrak{M}_E)$ on the left hand side and as an element of $\mathfrak{X}(M)$ on the right hand side. Hence, we then also have
\bas
\mleft.\bigl(\mleft(\mathrm{ev}^* \nabla\mright)_{X} v\bigr)\mright|_{(p, \Phi, A)}
=
\mleft.\mleft(\mleft( \Phi^*\nabla \mright)_{X_p} v|_{(\Phi, A)} \mright) \mright|_p
\eas
for all $v \in \Gamma(\mathrm{ev}^*V)$,
since $X$ does not differentiate along $\mathfrak{M}_E$, and viewing $v|_{(\Phi, A)} \coloneqq [p \mapsto v|_{(p, \Phi, A)}]$ as an element of $\Gamma(\Phi^*V)$ on the right hand side.
Therefore this naturally leads on one hand to an exterior covariant derivative on the space of functionals by restricting $\mathrm{ev}^*\nabla$ to $\mathrm{T}M$ because then the exterior covariant derivative of $\mleft.\mleft(\mathrm{ev}^*\nabla\mright)\mright|_{\mathrm{T}M}$ clearly restricts to $\mathcal{F}^\bullet_E(M; {}^*V)$, and on the other hand
\bas
\mleft.\mleft(\mathrm{d}^{\mleft.\mleft(\mathrm{ev}^*\nabla\mright)\mright|_{\mathrm{T}M}} L\mright)\mright|_{(\Phi,A)} 
&=
\mathrm{d}^{\Phi^*\nabla}\bigl( L(\Phi, A) \bigr),
\eas
also recall Eq.~\eqref{SliceOfBIiiigManifold}.

\hspace{0.3cm} Similarly, one shows for the pullback $\mathrm{ev}^!\omega$ of forms $\omega \in \Omega^k(N; V)$ that
\bas
\mleft. \mleft(\mathrm{ev}^! \omega\mright)\mright|_{(p,\Phi,A)}
\mleft( X_1, \dotsc, X_k \mright)
&=
\mleft. \mleft(\Phi^! \omega\mright)\mright|_{p}
\mleft( X_1, \dotsc, X_k \mright)
\eas
for all $X_1, \dotsc, X_k \in \mathfrak{X}(M)$. Hence, also the $\mathrm{ev}$-pullback of forms restricts to a $\Phi$-pullback of forms when fixing $(\Phi,A)$ and just evaluating at vector fields along $M$.
\end{remarks}

Therefore we define pullback functionals as in the following definition.

\begin{definitions}{Pullbacks as functionals}{PullbacksAsFunctionals}
Let $M, N$ be smooth manifolds, $E \to N$ a Lie algebroid, and $V \to N$ a vector bundle. For all $\omega \in \Gamma\mleft( V \mright)$ we define its \textbf{pullback functional $\gls{0*omega}$} as an element of $\mathcal{F}^0_E(M; {}^*V)$ by
\ba
{}^*v
&\coloneqq
\mathrm{ev}^*v.
\ea

For a vector bundle connection $\nabla$ on $V$ we define the \textbf{pullback connection $\gls{0*nabla}$ (to functionals)} by
\ba
{}^*\nabla
&\coloneqq
\mleft.\mleft(\mathrm{ev}^*\nabla\mright)\mright|_{\mathrm{T}M}.
\ea
Its induced exterior covariant derivative $\mathrm{d}^{{}^*\nabla}$ we view as an exterior covariant derivative on the space of functionals, especially
\ba
\mathrm{d}^{{}^*\nabla}:
\mathcal{F}^k_E(M; {}^*V)
&\to
\mathcal{F}^{k+1}_E(M; {}^*V)
\ea
for all $k \in \mathbb{N}_0$.

For all $\omega \in \Omega^k(N;V)$  ($k \in \mathbb{N}_0$) we define similarly its \textbf{form-pullback functional $\gls{0!omega}$} as an element of $\mathcal{F}_E^k(M; {}^*V)$ by
\ba
{}^!\omega
&\coloneqq
\mleft.\mleft(\mathrm{ev}^!\omega\mright)\mright|_{\bigwedge^k \mathrm{T}M}.
\ea
\end{definitions}

\begin{remarks}{}{}
Observe that
\ba
\mleft.({}^*v)(\Phi,A)\mright|_p
&\coloneqq
(\mathrm{ev}^*v)|_{(p, \Phi, A)}
=
\Phi^*v|_p
\ea
for all $(p, \Phi, A) \in M \times \mathfrak{M}_E$. Especially, $({}^*v)(\Phi, A) = \Phi^*v$, similarly to what we already pointed out for ${}^!w$ and ${}^*\nabla$ in Remark \ref{rem:NotionsOnFunctionals}. By construction, and as argued in Rem.~\ref{rem:NotionsOnFunctionals}, we also get
\ba
\mleft( \mathrm{d}^{{}^*\nabla} L \mright)(\Phi,A)
&=
\mathrm{d}^{\Phi^*\nabla}\bigl( L(\Phi,A) \bigr)
\ea
for all $L \in \mathcal{F}^k_E(M; {}^*V)$ ($k \in \mathbb{N}_0$) and $(\Phi, A) \in \mathfrak{M}_E(M;N)$. 

We can also locally write, using a frame $\mleft( e_a \mright)_a$ of $V$,
\ba
L
&=
L^a \otimes {}^*e_a,
\ea
using that $\mathrm{ev}$-pullbacks generate $\Gamma(\mathrm{ev}^*V)$,
where $L^a \in \mathcal{F}^k_E(M) = \Omega^{k,0}(M \times \mathfrak{M}_E)$ (restriction on open neighbourhood omitted).
\newline\newline
The first calculation of Remark \ref{rem:NotionsOnFunctionals} also shows that we have
\bas
\mathrm{D}
&=
\mathrm{Dev}|_{\mathrm{T}M}
\eas
as functionals, where we view $\mathrm{Dev}|_{\mathrm{T}M}$ as an element of $\mathcal{F}^1_E(M; {}^*\mathrm{T}N)$ given by Eq.~\eqref{DGleichDev}. This implies that we can apply Eq.~\eqref{EqPullBackFormelFuerVerschiedeneDefinitionen}, that is,
\bas
{}^!\omega
&=
\mleft.\mleft(\mathrm{ev}^!\omega\mright)\mright|_{\bigwedge^k \mathrm{T}M}
\stackrel{\eqref{EqPullBackFormelFuerVerschiedeneDefinitionen}}{=}
\frac{1}{k!}~
\mleft(\mathrm{ev}^*\omega\mright)\mleft(\mathrm{Dev}|_{\mathrm{T}M} \stackrel{\wedge}{,} \dotsc \stackrel{\wedge}{,} \mathrm{Dev}|_{\mathrm{T}M} \mright)
=
\frac{1}{k!}~
\mleft({}^*\omega\mright)\mleft(\mathrm{D} \stackrel{\wedge}{,} \dotsc \stackrel{\wedge}{,} \mathrm{D} \mright)
\eas
for all $\omega \in \Omega^k(N;V)$ ($k \in \mathbb{N}_0$). We are going to use this very often by just giving reference to Eq.~\eqref{EqPullBackFormelFuerVerschiedeneDefinitionen}.
\end{remarks}

\begin{examples}{Anchor as functional}{AnchorAsFunctional}
Recall Ex.~\ref{ex:ProjectionOntoGaugeBosonies}; the anchor gives also rise to a functional, especially needed for the minimal coupling. $({}^*\rho)(\varpi_2)$ is a functional in $\mathcal{F}^1_E(M; {}^*\mathrm{T}N)$, that is
\bas
\bigl(({}^*\rho)(\varpi_2)\bigr)(\Phi, A)
&=
(\Phi^*\rho)(A)
\eas
for all $(\Phi, A) \in \mathfrak{M}_E(M;N)$.
\end{examples}

We have now the setup to finally define the physical quantities.

\section{Physical Quantities}\label{NewPhysicQuants}
%
%The structure will be the following:
%\begin{itemize}
	%\item We define of the physical quantities using two smooth manifolds $M, N$, and $E \to N$ a Lie algebroid equipped with a connection $\nabla$. Additionally, we will also sometimes have a Riemannian metric $g$ on $\mathrm{T}N$, and a fibre metric $\kappa$ on $E$.
	%\item We want to define the Lagrangian of this new theory, generalizing the Lagrangian defined in Chapter \ref{ClassicGaugeTheory}. We are going to formulate infinitesimal gauge invariance of that Lagrangian, using the framework introduced in the previous section. Hence, the functionals we will vary in the following will depend on a pair of fields, $(\Phi, A) \in \mathfrak{M}_E(M; N)$.
	%\item $A$ is going to be the field of gauge bosons while the major example of $\Phi$ will be the Higgs field, hence, the important example of $M$ will be the spacetime and of $N$ will be a vector space $W$, in which the Higgs field $\Phi$ lives; \textit{e.g.}~for the latter $N=\mathbb{C}^2$ as for the electroweak interaction. The example to think about for the Riemannian metric $g$ on $N$ is then a scalar product on $W$, canonically extended to $\mathrm{T}W \cong W \times W$.
	%\item The major example of $E$ will be an action Lie algebroid as defined in Def.~\ref{def:ActionLieAlgebroids}, \textit{i.e.}~$E= N \times \mathfrak{g}$ for a Lie algebra action $\mathfrak{g}$ with a Lie algebra action $\gamma: \mathfrak{g} \to \mathrm{T}N$; its Lie algebroid structure is uniquely defined by the conditions that its anchor and its Lie bracket restricts on constant sections to $\gamma$ and to the Lie bracket of $\mathfrak{g}$, respectively. We want this behaviour on constant structures such that we can hope for recovering the standard formulation of gauge theory. The most important example of $\gamma$ will be an action coming from a Lie algebra representation $\psi: \mathfrak{g} \to \mathrm{End}(W)$ on a vector space $N = W$; for that recall Prop.~\ref{prop:LieRepAndLieAct}. 
	%%
%%Another major example of $E$ would be a Lie algebra bundle (LAB), but that implies a zero anchor which would therefore not contain any information about a Lie algebra action. Due to the uniqueness of the Lie algebroid structure, the only Lie algebroid structure of the vector bundle $E = N \times \mathfrak{g}$ containing information about both, a Lie algebra action and a Lie algebra bracket, is the action Lie algebroid. Hence, LABs should be seen as a simplified example, which we will discuss separately later.
	%\item After each definition we will restrict the corresponding definition to the following: $E = N \times \mathfrak{g}$ an action Lie algebroid of a Lie algebra $\mathfrak{g}$ whose Lie algebra action $\gamma$ is induced by a Lie algebra representation $\psi$ on a vector space $N=W$; $\nabla$ the canonical flat connection on $E$ , that is $\nabla \nu = 0$ when $\nu$ is a constant section; the fibre metric $\kappa$ on $E$ is a canonically extended scalar product of $\mathfrak{g}$, and the Riemannian metric $g$ of $N$ is a canonically extended scalar product of $W$. We will refer to this setup as the \textbf{standard/classic formulation of gauge theory} because we are going to see that our definitions will restrict to the classical definitions when restricting on this setup.
	%%\item We will provide some motivational ideas about the following constructions. In order to understand these motivations with less struggle, we have put these motivations at the end of this section, see Remark \ref{rem:WhyAPullback} and \ref{rem:WhyDoesTheNewLiebRacketHaveALeibnizRule}.
%\end{itemize}
%
Let us first start with the definition of the field strength. The following definitions essentially are motivated by \cite{CurvedYMH}, however, we completely reformulated it with the previously-introduced notation in order to allow coordinate-free versions, also "free" with respect to $(\Phi, A) \in \mathfrak{M}_E$.

\begin{definitions}{Field of gauge bosons and their field strength, \newline \cite[especially Eq.~(11); $\Phi$ is denoted as $X$ there]{CurvedYMH}}{EichbosonenUndFeldstaerke}
Let $M, N$ be smooth manifolds, and $E \to N$ a Lie algebroid equipped with a connection $\nabla$ on $E$. We define the \textbf{field strength $\gls{F}$} as an element of $\mathcal{F}_E^2(M; {}^*E)$ by
\ba
F
&\coloneqq
\mathrm{d}^{{}^*\nabla}\varpi_2
	- \frac{1}{2} ({}^*t_{\nabla_\rho})\mleft( \varpi_2 \stackrel{\wedge}{,} \varpi_2 \mright),
\ea
that is
\ba\label{DefOfCovariantizedFieldStrengthF}
F(\Phi, A)
\coloneqq
\mathrm{d}^{\Phi^*\nabla} A
	- \frac{1}{2} \mleft( \Phi^* t_{\nabla_\rho} \mright)\mleft( A \stackrel{\wedge}{,} A \mright)
\ea
for all $\Phi \in C^\infty(M;N)$ and $A \in \Omega^1(M; \Phi^*E)$.
\end{definitions}

\begin{remark}\label{RemarkUeberDefinitionVonNormalerFeldUndA}
\leavevmode\newline
\indent $\bullet$ Recall Def.~\ref{def:CanonicalBasicConnection} and Prop.~\ref{prop:SnablamitREnabla} which imply $t_{\nabla_\rho} = -t_{\nabla^{\mathrm{bas}}}$, where $\nabla^{\mathrm{bas}}$ is the basic connection, such that
\bas
F
=
\mathrm{d}^{{}^*\nabla} \varpi_2
	+ \frac{1}{2} \mleft( {}^* t_{\nabla^{\mathrm{bas}}} \mright)\mleft( \varpi_2 \stackrel{\wedge}{,} \varpi_2 \mright).
\eas
We are going to use this often later.

$\bullet$ Let us recall the definition of the standard setting, recall Def.~\ref{def:ClassicFieldStrength}, and recall the bookkeeping trick before Prop.~\ref{prop:ClassicFunctionDerivativesAlongPsiEpsilon}, which we denoted by $\iota$: We then normally have $A \in \Omega^1(M; \mathfrak{g}), \Phi \in C^\infty(M;W)$ for a given Lie algebra $\mathfrak{g}$ and $W$ a vector space, then the field strength is normally defined as
\ba\label{StandardFDef}
F^{\mathrm{clas}}(\Phi, A)
\equiv
F^{\mathrm{clas}}(A)
&=
\mathrm{d}A^a \otimes e_a
	+ \frac{1}{2} \mleft[ A \stackrel{\wedge}{,} A \mright]_{\mathfrak{g}}
\ea
for some given basis $\mleft( e_a \mright)_a$ of $\mathfrak{g}$. $\mathfrak{g}$ is viewed as "trivial bundle" over $M$, $M \times \mathfrak{g}$, and $\mleft( e_a \mright)_a$ is a constant frame.

Now, let us instead restrict Eq.~\eqref{DefOfCovariantizedFieldStrengthF} to an action Lie algebroid $E = N \times \mathfrak{g}$ equipped with $\nabla$ as the canonical flat connection and $\mleft( e_a \mright)_a$ a global frame of constant sections, especially $\nabla e_a = 0$. Then $\mleft( \Phi^*e_a \mright)_a$ trivializes $\Phi^*E$ such that $\Phi^*E \cong M \times \mathfrak{g}$, $\mleft( \Phi^*e_a \mright)_a$ describes a constant frame, especially $(\Phi^* \nabla) (\Phi^*e_a) = \Phi^! (\nabla e_a) = 0$,
and all $\Phi^*E$-valued objects can be viewed as $\mathfrak{g}$-valued.
In that case, write $A = A^a \otimes \Phi^*e_a$, and observe that
\bas
- \frac{1}{2} \mleft( \Phi^* t_{\nabla_\rho} \mright)\mleft( A \stackrel{\wedge}{,} A \mright)
&=
- \frac{1}{2} \underbrace{\mleft( \Phi^* t_{\nabla_\rho} \mright)\mleft( \Phi^*e_a, \Phi^*e_b \mright)}_{\mathclap{= \Phi^*\mleft( t_{\nabla_\rho}(e_a, e_b) \mright)}}~
	A^a \wedge A^b
=
\frac{1}{2} \Phi^*\underbrace{\mleft( \mleft[ e_a, e_b \mright]_E \mright)}_{\mathclap{=\mleft[ e_a, e_b \mright]_{\mathfrak{g}} = \text{const.}}}~
	A^a \wedge A^b
=
\frac{1}{2} \mleft[ A \stackrel{\wedge}{,} A \mright]_{\mathfrak{g}}
\eas
and
\bas
\mathrm{d}^{\Phi^*\nabla} A
&=
\mathrm{d}A^a \otimes \Phi^*e_a
	- A^a \otimes \Phi^!(\nabla e_a)
=
\mathrm{d}A^a \otimes \Phi^*e_a
\eas
for all $A \in \Omega^1(M; \Phi^*E)$. Hence, we get
\bas
F
&=
\iota\mleft( F^{\mathrm{clas}} \mright).
\eas
%\footnote{Strictly spoken, the right hand side is $\iota\mleft( F^{\mathrm{clas}} \mleft( \iota^{-1}(A) \mright) \mright)$ due to that $A$ has values in the pullback bundle and is, thus, the field of gauge bosons of the classical theory with applied bookkeeping trick. But the bookkeeping trick is just that, bookkeeping, trivially $\iota(A) = A$. It just emphasizes the relationship }
%Hence, Eq.~\eqref{DefOfCovariantizedFieldStrengthF} restricts to the standard definition when restricted to the standard setting, comparing it with Eq.~\eqref{StandardFDef}.  That we have $\Phi^*e_a$ instead of $e_a$ as in the standard definition is just a distinction in the notation, both describe of course a global frame of constant sections, and that is all which is needed here. Due to the constancy of $e_a$, one could even argue to write $\Phi^*e_a = e_a$, \textit{i.e.}~one can omit the notation of the pull-back in the standard setting to emphasize that it is not affected by variations of $\Phi$ as it happens under a gauge transformation using the standard definition of gauge transformations in the standard setting, that is, one will use a canonical flat connection for the gauge transformations as argued in the last bullet point of Rem.~\ref{RemLeibnizeRegelaufProdukteWeshalbEConnectionNichtWichtigIst}. A similar argument holds for all other types of constant frames in the following, even when we do not explicitly mention it anymore. However, when one does not want to use canonical flat connections, then the pull-back should not be omitted; this will be important later.
\end{remark}

As we have seen in the definition of the action Lie algebroid, the anchor $\rho$ replaces the notion of Lie algebra actions and representations such that we now use the anchor to define the minimal coupling of $A$ to $\Phi$.

\begin{definitions}{Minimal coupling, \cite[Eq.~(3), $\Phi$ is denoted as $X$ there]{CurvedYMH}}{MinimalCoupling}
Let $M, N$ be smooth manifolds and $E \to N$ a Lie algebroid. Then we define the \textbf{minimal coupling $\mathfrak{D}$} as an element of $\mathcal{F}_E^1(M; {}^*\mathrm{T}N)$ by
\ba\label{MinimalCouplingInKurz}
\mathfrak{D}
&\coloneqq
\mathrm{D}
	- ({}^*\rho)(\varpi_2).
\ea

We also write
\ba
\gls{DAPhi}
&\coloneqq
\mathfrak{D}(\Phi, A)
=
\mathrm{D}\Phi
	- \mleft( \Phi^*\rho\mright)(A)
\ea
for all $\Phi \in C^\infty(M;N)$ and $A \in \Omega^1(M; \Phi^*E)$, and we say that \textbf{$\Phi$ is minimally coupled to $A$}.
\end{definitions}

\begin{remark}
\leavevmode\newline
Restricting this to the standard situation gives back the standard definition: Assume $N = W$ where $W$ is a vector space, $E = W \times \mathfrak{g}$ an action Lie algebroid over $W$,
whose action is induced by a Lie algebra representation $\psi: \mathfrak{g} \to \mathrm{End}(W)$. Then the minimal coupling is
\bas
\mleft.\mathfrak{D}^A \Phi\mright|_p
&=
\mleft.\mathrm{d}_p\Phi^\alpha \otimes \Phi^*\partial_\alpha\mright|_p
	+ \psi\bigl(A_p(Y)\bigr)\bigl(\Phi(p)\bigr)
\eas
for all $(p, \Phi, A) \in M \times \mathfrak{M}_E(M;W)$ and $Y \in \mathrm{T}_pM$,
where we use some global coordinates $\mleft(\partial_\alpha\mright)_\alpha$ of $W$ and Prop.~\ref{prop:LieRepAndLieAct}. Now we make use of the canonical identification of $W$'s tangent spaces with $W$ itself, especially, $v_\alpha = \partial_\alpha$ for some basis $\mleft( v_\alpha \mright)_\alpha$ on $W$. Then the first summand is clearly $\mathrm{d}\Phi^\alpha \otimes \Phi^*\partial_\alpha = \iota(\mathrm{d}\Phi)$. Hence, also here we arrive at the classical definition (under the bookkeeping trick), recall Def.~\ref{def:ClassicMinimalCoupling}.
\end{remark}

Finally we turn to the Lagrangian. 

\begin{definitions}{Yang-Mills-Higgs Lagrangian, \newline \cite[Eq.~(2) and (16); but a different field strength there which we will introduce later]{CurvedYMH}}{CurvedYMHLagrangian}
Let $M$ be a spacetime with a spacetime metric $\eta$, $N$ a smooth manifold, $E \to N$ a Lie algebroid, $\nabla$ a connection on $E$, and let $\kappa$ and $g$ be fibre metrics on $E$ and $\mathrm{T}N$, respectively. Also let $V \in C^\infty(N)$, which we call the \textbf{potential of the Higgs field}. Then we define the \textbf{Yang-Mills-Higgs Lagrangian $\gls{LYMH}$} as an element of $\mathcal{F}_E^{\mathrm{dim}(M)}(M)$ by
\ba
\mathfrak{L}_{\mathrm{YMH}}
&\coloneqq
- \frac{1}{2} \mleft( {}^*\kappa \mright)\mleft(F \stackrel{\wedge}{,} *F\mright)
	+ \mleft( {}^*g \mright)\mleft(\mathfrak{D} \stackrel{\wedge}{,} *\mathfrak{D} \mright)
	- *({}^*V),
\ea
that is
\ba
\mathfrak{L}_{\mathrm{YMH}}(\Phi, A)
&\coloneqq
- \frac{1}{2} \mleft( \Phi^*\kappa \mright)\mleft(F(\Phi, A) \stackrel{\wedge}{,} *F(\Phi, A)\mright)
	+ \mleft( \Phi^*g \mright)\mleft(\mathfrak{D}^A \Phi \stackrel{\wedge}{,} *\mathfrak{D}^A\Phi\mright)
	- *(V \circ \Phi)
\ea
for all $(\Phi, A) \in \mathfrak{M}_E(M;N)$, where $*$ is the Hodge star operator with respect to $\eta$.
\end{definitions}

A short summary:

\begin{corollaries}{Standard theory as action Lie algebroid, as motivated in \cite{CurvedYMH}}{StandardTheory}
Let $M$ be a spacetime with a spacetime metric $\eta$, $N = W$ be a vector space, equipped with a Riemannian metric $g$ on $\mathrm{T}W \cong W \times W$ canonically induced by a scalar product on $W$, and $E = N \times \mathfrak{g}$ an action Lie algebroid for a Lie algebra $\mathfrak{g}$, equipped with its canonical flat connection $\nabla$ and a fibre metric $\kappa$ which constantly extends a scalar product on $\mathfrak{g}$. The $\mathfrak{g}$-action $\gamma$ is induced by a Lie algebra representation $\psi: \mathfrak{g} \to \mathrm{End}(W)$, and we have a potential $V \in C^\infty(W)$.

Then Def.~\ref{def:EichbosonenUndFeldstaerke}, \ref{def:MinimalCoupling} and \ref{def:CurvedYMHLagrangian} are the same as for the standard formulation of gauge theory as introduced in Chapter \ref{ClassicGaugeTheory}.
\end{corollaries}

\begin{proof}[Proof of Cor.~\ref{cor:StandardTheory}]
\leavevmode\newline
By construction; also recall the remarks of Def.~\ref{def:EichbosonenUndFeldstaerke} and \ref{def:MinimalCoupling}. For $\kappa$ take a constant frame $\mleft( e_a \mright)_a$ of $E$ such that $\mleft( \Phi^*e_a \mright)_a$ trivializes $\Phi^*E \cong M \times \mathfrak{g}$ for all $\Phi \in C^\infty(M;N)$ and $\mleft( \Phi^*e_a \mright)_a$ is also a constant frame, and denote the scalar product on $\mathfrak{g}$ by $\widetilde{\kappa}$. Then observe
\bas
(\Phi^*\kappa)(\Phi^*e_a, \Phi^*e_b)
&=
\Phi^*\bigl( \kappa(e_a, e_b) \bigr)
=
\Phi^* \underbrace{(\widetilde{\kappa}(e_a, e_b))}_{\mathclap{= \text{const.}}}
=
\widetilde{\kappa}(e_a, e_b),
\eas
hence, ${}^*\kappa = \iota(\kappa) = \kappa$ a constant extension of $\widetilde{\kappa}$; similarly for $g$. Thence, we arrive at the standard definition of the Lagrangian, using the remarks of Def.~\ref{def:EichbosonenUndFeldstaerke} and \ref{def:MinimalCoupling},
\bas
\mathfrak{L}_{\mathrm{YMH}}(\Phi, A)
&=
- \frac{1}{2} \widetilde{\kappa} \mleft(F(\Phi, A) \stackrel{\wedge}{,} *F(\Phi, A)\mright)
	+ \widetilde{g} \mleft(\mathfrak{D}^A \Phi \stackrel{\wedge}{,} *\mathfrak{D}^A\Phi\mright)
	- *(V \circ \Phi),
\eas
where $\widetilde{g}$ is the scalar product on $W$; recall Def.~\ref{def:ClassicYMHLagrangian}.
\end{proof}

Now let us finally turn to the infinitesimal gauge transformation.

%Let us conclude this section with a final remark about how to motivate the pullbacks and the Leibniz rule of the Lie bracket of $E$ in the definitions above.

%\begin{remarks}{Why a Pullback of a bundle over $N$ instead of assuming initially a bundle directly over $M$?}{WhyAPullback}
%It is natural to ask why we do not define the Lie algebroid $E$ over $M$ directly. Assume we start with a gauge theory, formulated in the classical approach, \textit{i.e.}~we also have a Lie algebra representation $\psi: \mathfrak{g} \to \mathrm{End}(W)$ of a Lie algebra $\mathfrak{g}$ on a vector space $W$. We want to replace $W$ with any manifold $N$ in the mathematical framework, hence, it is better to think of the Lie algebra representation as a Lie algebra action instead, because the notion of the latter is defined on every manifold. For this we use Prop.~\ref{prop:LieRepAndLieAct}, \textit{i.e.}~the canonical action $\gamma: \mathfrak{g} \to \mathfrak{X}(W)$ on $W$ is then given by
%\bas
%\gamma(x)_w 
%&\coloneqq
%- \psi(x)(w)
%\eas
%for all $x \in \mathfrak{g}$ and $w \in W$. When we then have the fields $\Phi \in C^\infty(M; W)$ and $A \in \Omega^1(M; \mathfrak{g})$ as in the classical formulation, then the minimal coupling consists also of the term $\psi(A)(\Phi)$ which then is written as $-\gamma(A)_\Phi \in \Omega^1(M;\Phi^*\mathrm{T}W)$ given by
%\bas
%\mleft(\gamma(A)_\Phi\mright)_p(X_p)
%&=
%\gamma\bigl(A_p(X_p)\bigr)_{\Phi(p)}
%\eas
%for all $p \in M$ and $X_p \in \mathrm{T}_pM$. Let us now define $E \coloneqq M \times \mathfrak{g}$ as a trivial vector bundle over $M$ with fibre type $\mathfrak{g}$, and we now want to extend $\gamma$ to $E$, denoting it by $\rho: E \to \mathrm{T}W$\footnote{The anchor of the previous definitions can be thought of an extension of $\gamma$ to the whole bundle $E$ when $E$ was the action Lie algebroid over $N$, that can be seen as part of the covariantization which immediately leads to a definition allowing more general bundles.}, naturally defined by
%\bas
%\rho(\mu)_p = \gamma(\mu_p)|_{\Phi(p)}
%\eas
%for all $p \in M$ and $\mu \in \Gamma(E)$.
%$\rho$ is linear by $\gamma$, and $\rho$ is a vector bundle morphism over $\Phi$ due to the fact that $\gamma(A)_\Phi$ has values in $\Phi^*\mathrm{T}W$; we cannot expect that there is in general another definition of $\rho$ such that it is a morphism over the identity because we now assume that $E$ is a bundle over $M$ while $\mathrm{T}W$ is a bundle over $W$, even more so when $W$ is replaced with any manifold $N$. Viewing $\rho$ as a morphism over $\Phi$ would imply that $\gamma$ is subject to the calculus of variations since we want to vary with respect to $\Phi$. As in the standard formulation, we want to keep the data coming from the Lie algebra, and of the Lie algebroid in general, independent of the calculus of variations.
%
%To avoid this issue it is better to write $\gamma(A)_\Phi = (\Phi^*\gamma)(A)$, \textit{i.e.}~to view this term as something coming from a pullback of a structure over $N$. This is precisely what we did in the previous definitions, and what we will continue in the following sections. Hence, the pullback, as uncommon as it looks like, is very natural to do here, it is just not obvious in the standard formulation of gauge theory because the structures are trivially enough to omit the pullback.
%\end{remarks}
%
%\begin{remarks}{Why does the Lie bracket needs a Leibniz rule?}{WhyDoesTheNewLiebRacketHaveALeibnizRule}
%Assume precisely the same situation as in the previous remark, Remark \ref{rem:WhyAPullback}. There we have seen that we rather should define $E = N \times \mathfrak{g}$ instead of $M \times \mathfrak{g}$, and $\rho$ is instead defined on that (but in the same fashion). In the last remark we just argued with that $\rho$ should be linear, but, as in the standard formulation, we also want that it is a homomorphism of Lie brackets. At the beginning of this section we shortly argued that there is only one unique Lie algebroid structure on $N \times \mathfrak{g}$ with the information about the Lie bracket of $\mathfrak{g}$ and its action. But this is based on the assumption that we define $E$ as a Lie algebroid. Besides the homomorphism properties of $\rho$ in the definition of Lie algebroids one also has a Leibniz rule in the Lie bracket. In this remark we will see that the Leibniz rule in the Lie bracket of $E$ is actually very natural, just not obvious in the standard formulation. 
%
%Assume we have a Lie bracket $\mleft[ \cdot, \cdot \mright]_E$ on the sections of $E$, and assume that the vector bundle morphism $\rho: E \to \mathrm{T}W$ is a homomorphism of Lie brackets, \textit{i.e.}
%\bas
%\rho\mleft( \mleft[ \mu, \nu \mright]_E \mright)
%&=
%\mleft[ \rho(\mu), \rho(\nu) \mright]
%\eas
%for all $\mu, \nu \in \Gamma(E)$. Let us rewrite the right-hand side a bit, using a global and constant frame $\mleft( e_a \mright)_a$ of $E$ and making use of that $\gamma: \mathfrak{g} \to \mathfrak{X}(W)$ is a homomorphism of Lie brackets,
%\bas
%\rho_p\mleft( \mleft.\mleft[\mu, \nu \mright]_E\mright|_p \mright)
%&=
%\mleft.\mleft[ \rho(\mu), \rho(\nu) \mright]\mright|_p
%\\
%&=
%\mleft.\mleft[ \mu^a \rho_a, \nu^b \rho_b \mright]\mright|_p
%\\
%&=
%\left(\mu^a \nu^b ~ \mleft[ \rho(e_a), \rho(e_b) \mright]
	%+ \mathcal{L}_{\rho_p(\mu_p)}\mleft( \nu^a \mright) ~ \rho_a
	%- \mathcal{L}_{\rho_p(\nu_p)}\mleft( \mu^a \mright) ~ \rho_a \middle) \right|_p 
%\\
%&=
%\left( \mu^a \nu^b ~ \mleft[ \gamma(e_a), \gamma(e_b) \mright]
	%+ \mathcal{L}_{\gamma(\mu_p)|_p}\mleft( \nu^a \mright) ~ \rho_a
	%- \mathcal{L}_{\gamma(\nu_p)|_p}\mleft( \mu^a \mright) ~ \rho_a \middle) \right|_p
%\\
%&=
%\left( \mu^a \nu^b ~ \gamma\mleft( \mleft[ e_a, e_b \mright]_{\mathfrak{g}} \mright)
	%+ \mathcal{L}_{\gamma(\mu_p)}\mleft( \nu^a \mright) ~ \rho_a
	%- \mathcal{L}_{\gamma(\nu_p)}\mleft( \mu^a \mright) ~ \rho_a \middle) \right|_p
%\\
%&=
%\rho_p \mleft( \mleft[ \mu_p, \nu_p \mright]_{\mathfrak{g}}
%+ \mleft.\mathcal{L}_{\gamma(\mu_p)}\mleft( \nu^a \mright)\mright|_p ~ e_a
%- \mleft.\mathcal{L}_{\gamma(\nu_p)}\mleft( \mu^a \mright)\mright|_p ~ e_a \mright)
%\eas
%for all $\mu, \nu \in \Gamma(E)$ and $p \in N$. Thus, under the action of $\rho$ the Lie bracket of $E$ needs to satisfy the Leibniz rule in order to achieve that $\rho$ is a homomorphism of Lie brackets; observe that one precisely gets the bracket of an action Lie algebroid. Since $\rho$ will be arbitrary from a theoretical point of view, it is natural to assume that $E$ carries the structure of a Lie algebroid.
%
%That shows that we get the Leibniz rule due to that we replace the Lie algebra $\mathrm{End}(W)$ with $\mathfrak{X}(W)$, in order to replace the vector space $W$ with any manifold; recall Lemma \ref{lem:LemmaEndGleichMinusVectorField}. The Leibniz rule of vector fields then carries over to the Lie bracket on $E$. The classical formulation just uses a Lie algebra representation $\psi: \mathfrak{g} \to \mathrm{End}(W)$, hence, the Leibniz rule is hidden, especially additionally due to the fact that elements of $\mathfrak{g}$ are essentially constant sections of $E$.
%\end{remarks}

%\section{Calculus of variations}
%
%To define a covariantized gauge theory we need to rethink infinitesimal gauge transformations, especially because a lot of terms have now values in the pull-back of a vector bundle, but the function used for the pull-back gets varied. For this we need to know what $E$-paths are with which it is possible to define several notions on $E$-connections similar to typical connections. 
%
%Finally, we can now define variations.
%
%\begin{definitions}{Variations}{VariationsInVariationalCalc}
%Let $M, N$ be two smooth manifolds, $E, B, C \to N$ Lie algebroids, $V \to N$ a vector bundle, ${}^B\nabla$ a $B$-connection on $V$, ${}^C\nabla$ a $C$-connection on $E$, $\Phi_0: M \to N$ a smooth map, and $A_0 \in \Omega^1(M; \Phi_0^*E)$. Also let $I \subset \mathbb{R}$ with $0 \in I$.
%\begin{itemize}
	%\item We define a \textbf{field of $E$-paths} by a map $\xi: I \times M \to E$, $(t,p) \mapsto \xi_t(p)$ such that $\xi(p) \coloneqq \mleft[ t \mapsto \xi_t(p) \mright]$ is an $E$-path.
	%\item For an open interval $I \subset \mathbb{R}$ with $0 \in I$ assume that we have a fixed field of maps 
%\bas
	%\widetilde{\Phi}: I \times M \to N, (t, p) \mapsto \widetilde{\Phi}_t(p), \text{ such that } \widetilde{\Phi}_{t=0} \equiv \Phi_0. 
%\eas	
%Then we define the \textbf{variation $\delta_{\widetilde{\Phi}} \Phi_0$ of $\Phi_0$ along $\widetilde{\Phi}$} as an element of $\in \Gamma\mleft(\Phi_0^* \mathrm{T}N\mright)$ by
%\ba
%\mleft(\delta_{\widetilde{\Phi}} \Phi_0\mright)_p
%\coloneqq
%\mleft.\frac{\mathrm{d}}{\mathrm{d}t}\mright|_{t=0}\mleft[ t \mapsto \widetilde{\Phi}_t(p) \mright]
%\ea
%for all $p \in M$.
%\item Now let $L \in \mathcal{F}_E^k(M; {}^*V)$ ($k \in \mathbb{N}_0$), and assume we have a $\widetilde{\Phi}$ as before and that we have a map
%\bas
%\widetilde{A} \in \Omega^1\mleft(M; \widetilde{\Phi}^*E\mright), 
%I \ni t \mapsto \widetilde{A}_t \in \Omega^1\mleft( M; \mleft( \widetilde{\Phi}_t \mright)^*E \mright), \text{ such that }
%\widetilde{A}_{t=0} = A.
%\eas 
%Furthermore assume that $\widetilde{\Phi}$ can be lifted to a smooth field of $B$-paths $\Psi: I \times M \to E$. Then we define the \textbf{variation $\mleft(\delta_{\Psi, \widetilde{A}} L\mright)(\Phi_0, A_0)$ of $L$ along $\Psi$ and $\widetilde{A}$ at $\Phi_0$ and $A_0$} as an element of $\Omega^{k}(M; \Phi_0^*V)$ by
%\ba\label{defKrassesVariationsPrinzip}
%\mleft(\mleft(\delta_{\Psi, \widetilde{A}} L\mright)(\Phi_0, A_0)\mright)_p\mleft(Y_1, \dotsc, Y_k\mright)
%&\coloneqq
%\mleft.\frac{\mathrm{D}_{\mleft(\Psi(p), {}^B\nabla\mright)}}{\mathrm{d}t}\mright|_{t=0}
%\mleft[ t \mapsto
%\mleft(L\mleft[ \widetilde{\Phi}_t \mright]\mleft[ \widetilde{A}_t \mright]\mright)_p\mleft(Y_1, \dotsc, Y_{k}\mright)
%\mright]
%\ea
%for all $p \in M$ and $Y_1, \dotsc, Y_k \in \mathrm{T}_pM$. Shortly written as
%\bas
%\mleft(\delta_{\Psi, \widetilde{A}} L\mright)(\Phi_0, A_0)
%&\coloneqq
%\mleft.\frac{\mathrm{D}_{\mleft(\Psi, {}^B\nabla\mright)}}{\mathrm{d}t}\mright|_{t=0}
%\bigg[ t \mapsto
%L\mleft[ \widetilde{\Phi}_t \mright]\mleft[ \widetilde{A}_t \mright]
%\bigg].
%\eas
%\item In the special case of $V=E$ we have $L = \mathds{1}_{\Omega^1(M;{}^*E)} \in \mathcal{F}^1_E(M;{}^*E)$, given by 
%\bas
%\mathds{1}_{\Omega^1(M;{}^*E)}(\Phi, A) 
%&\coloneqq
%A
%\eas
%for all $\Phi \in C^\infty(M;N)$ and $A \in \Omega^1(M; \Phi^*E)$, and we assume that $\widetilde{\Phi}$ can be lifted to a field of $C$-paths $\varpi: I \times M \to C$. Then we define the \textbf{variation $\delta_{\varpi, \widetilde{A}} A_0$ of $A_0$ along $\varpi$ and $\widetilde{A}$} as an element of $\Omega^1\mleft(M; \mleft( \Phi_0 \mright)^*E \mright)$ by
%\ba
%\delta_{\varpi, \widetilde{A}} A_0
%&\coloneqq
%\mleft(\delta_{\varpi, \widetilde{A}} \mathds{1}_{\Omega^1(M;{}^*E)}\mright)(\Phi_0, A_0)
%=
%\mleft.\frac{\mathrm{D}_{\mleft(\varpi(p), {}^C\nabla\mright)}}{\mathrm{d}t}\mright|_{t=0}
%\mleft[ t \mapsto
	%\widetilde{A}_t
%\mright].
%\ea
%%When we are able to do this for all pairs $\Phi_0 \in C^\infty(M;N)$ and $A_0 \in \Omega^1(M; \Phi^*_0E)$, we can then define $\delta_{\Psi, \widetilde{A}} L$ as an element of $\mathcal{F}^k_E(M; {}^*V)$ in that fashion.
%\end{itemize}
%\end{definitions}
%
%\begin{remark}\label{remVariationalbasics}
%\leavevmode\newline
%\indent $\bullet$ In the notation of $\delta$ we omit the used connection. By the context it should be clear which one is used, otherwise we will mention it separately.
%
%$\bullet$ In order to understand Def.~\eqref{defKrassesVariationsPrinzip}, observe that
%\bas
%&&
%\mleft(L\mleft[ \widetilde{\Phi}_t \mright]\mleft[ \widetilde{A}_t \mright]\mright)_p\mleft(Y_1, \dotsc, Y_{k}\mright)
%&\in
%V_{\widetilde{\Phi}_t(p)}\\
%&\Rightarrow&
%\mleft[ t \mapsto
%\mleft(L\mleft[ \widetilde{\Phi}_t \mright]\mleft[ \widetilde{A}_t \mright]\mright)_p\mleft(Y_1, \dotsc, Y_{k}\mright)
%\mright]
%&\in
%\Gamma\mleft( \mleft( \widetilde{\Phi}(p) \mright)^*V \mright),
%\eas
%therefore we can define a derivative $\frac{\mathrm{D}}{\mathrm{d}t}$ as in Prop.~\ref{prop:DerivationAlongEPath}, using ${}^B\nabla$ and the $B$-path $\Psi(p)$ lifting $\widetilde{\Phi}(p)$. Due to the fact that $\widetilde{\Phi}_0(p) = \Phi(p)$, we can conclude
%\bas
%\mleft.\frac{\mathrm{D}_{\mleft(\Psi, {}^B\nabla\mright)}}{\mathrm{d}t}\mright|_{t=0}
%\bigg[ t \mapsto
%L\mleft[ \widetilde{\Phi}_t \mright]\mleft[ \widetilde{A}_t \mright]
%\bigg]
%&\in
%V_{\Phi(p)}.
%\eas
%
%$\bullet$ In the following we will also keep the notations presented here, \textit{e.g.}~$\Psi(p)$ denotes the curve $t \mapsto \Psi_t(p)$. Moreover, we always use the standard notion of variation when dealing with $\mathbb{R}$-valued forms, that is, we then take $V= N \times \mathbb{R} \to N$ and $B = \mathrm{T}N$, $\Psi(t,p) = \mathrm{d}/\mathrm{d}t \mleft[t \mapsto \widetilde{\Phi}_t(p) \mright]$, and we make use of that all pull-backs of $V$ are trivial line bundles, too. In that case we then simply write $\delta_{\Psi, \widetilde{A}}= \delta_{\widetilde{\Phi}, \widetilde{A}}$.
%
%$\bullet$ For dual vector bundles we canonically take dual connections.
%
%$\bullet$ Observe that a field of $B$-paths like $\Psi$ is a section of $B$ along $\widetilde{\Phi}$ and, thus, $\Psi \in \Gamma\mleft(\widetilde{\Phi}^*B\mright)$, not discussing regularity as mentioned earlier.
%
%$\bullet$ For a frame $\mleft( e_a \mright)_a$ of $V$ we do in general \textbf{not} have $\mleft(\delta_{\Psi, \widetilde{A}} L\mright)^a = \delta_{\Psi, \widetilde{A}} \mleft(L^a\mright)$; that would in general only hold when ${}^B\nabla$ is flat and $\mleft( e_a \mright)_a$ is a parallel frame. The same for the variation of $A$.
%\end{remark}
%
%We extend $\delta$ as usual to any other form of tensors by forcing the Leibniz rule, \textit{e.g.}~in contractions of such $L$ as in Def.~\ref{def:VariationsInVariationalCalc}. In order to do so, it is natural to say that we view pullbacks of sections under $\Phi$ as a functional in $\Phi$.
%
%
%
%We know how to vary pullbacks of sections.
%
%\begin{corollaries}{Variations of pullback functionals}{VariationsOfPullBackStuff}
%Let $M, N$ be smooth manifolds, $B \to N$ a Lie algebroid, $V \to N$ a vector bundle, $\Phi_0 \in C^\infty(M;N)$, and ${}^B\nabla$ a $B$-connection on $V$. Furthermore, assume that we have a field of $B$-paths $\Psi$ with a field of base paths $\widetilde{\Phi}$ such that $\widetilde{\Phi}_{t=0} = \Phi_0$, as in Def.~\ref{def:VariationsInVariationalCalc}. Then
%\ba\label{PullBackVariation}
%\bigl( \delta_\Psi ({}^*\omega) \bigr)[\Phi_0]
%&=
%\Phi^*_0\mleft( {}^B\nabla_{\Psi_{t=0}} \omega \mright)
%\ea
%for all $\omega \in \Gamma\mleft( V \mright)$.
%\end{corollaries}
%
%\begin{proof}
%\leavevmode\newline
%This is a direct consequence of the definition of the variation,
%\bas
%\mleft(\bigl( \delta_\Psi ({}^*\omega) \bigr)[\Phi_0]\mright)_p
%&\stackrel{\text{Def.~\ref{def:VariationsInVariationalCalc}}}{=}
%\mleft.\frac{\mathrm{D}_{\mleft(\Psi(p), {}^B\nabla\mright)}}{\mathrm{d}t}\mright|_{t=0} \mleft[ t \mapsto
	%\mleft( \widetilde{\Phi}_t(p) \mright)^*\omega
%\mright]
%\stackrel{\text{Prop.~\ref{prop:DerivationAlongEPath}}}{=}
%\Phi^*_0\mleft( {}^B\nabla_{\Psi_{t=0}(p)} \omega \mright).
%\eas
%\end{proof}
%
%We will often make use of that, although we will not always give a reference to this corollary.
%
%\begin{definitions}{Variations of products}{ProductVariations}
%Let $l \in \mathbb{N}$, $E, E_1, \dots E_{l+1} \to N$ be Lie algebroids and $V_1, \dotsc, V_{l+1} \to N$ real vector bundles of finite rank over a smooth manifold $N$, and let $^{E_i}\nabla$ be $E_i$-connections on $V_i$, where $i \in \{1, \dotsc, l+1\}$. Also let $M$ be another smooth manifold, $\Phi_0 \in C^\infty(M;N)$, $A_0 \in \Omega^1(M; \Phi_0^*E)$, and $L \in \mathcal{F}^k_E\mleft( M; {}^*\mleft( \mleft(\bigotimes_{m=1}^l (V_m)^* \mright) \otimes V_{l+1} \mright) \mright)$ ($k \in \mathbb{N}_0$), and let $\Psi_i: I \times M \to N$ be fields of $E_i$-paths for an open interval $I \ni 0$, all lifting the same field of base paths $\widetilde{\Phi}:I \times M \to N$ with $\Phi = \widetilde{\Phi}_{t=0}$ as in Def.~\ref{def:VariationsInVariationalCalc}. Furthermore, we have an $\widetilde{A}$ as in Def.~\ref{def:VariationsInVariationalCalc} with $\widetilde{A}_0 = A$. 
%
%Then we define the \textbf{variation $\mleft(\delta_{\Psi_1, \dotsc, \Psi_{l+1}, \widetilde{A}}L \mright) (\Phi_0, A_0)$ of $L$ along $\psi_1, \dotsc, \psi_{l+1}$ and $\widetilde{A}$ at $\Phi_0$ and $A_0$} as an element of $\Omega^k\mleft(M;\mleft( \bigotimes_{m=1}^{l} \mleft(\Phi^*V_m\mright)^*\mright) \otimes \Phi^*V_{l+1} \mright)$ by
%\ba
%&\mleft(\mleft(\delta_{\Psi_1, \dotsc, \Psi_{l+1}, \widetilde{A}} L\mright) (\Phi_0, A_0)\mright) 
%\mleft(\Phi_0^*\nu_1, \dotsc, \Phi_0^*\nu_l \mright) \nonumber\\
%&\coloneqq
%\mleft(\delta_{\Psi_{l+1}, \widetilde{A}} \bigl( L \circ \mleft({}^*\nu_1, \dotsc,  {}^*\nu_l \mright) \bigr)\mright)(\Phi_0, A_0)
%\nonumber\\
%&\hspace{1cm}
	%- \sum_{j=1}^l \mleft( L(\Phi_0, A_0) \mright)\mleft( \Phi_0^*\nu_1, \dotsc, \mleft(\delta_{\Psi_j} \mleft( {}^*\nu_j \mright)\mright)[\Phi_0], \dotsc, \Phi_0^*\nu_l \mright) \label{DefKomplexeProduktRegelBeiVariationen}
%\ea
%for all $\nu_j \in \Gamma(V_j)$ ($j \in \{1, \dotsc, l\}$), where
%\bas
%\delta_{\Psi_i}
%&=
%\frac{\mathrm{D}_{\mleft(\Psi_i, ^{E_i}\nabla\mright)}}{\mathrm{d}t},
%\eas
%and $L \circ \mleft({}^*\nu_1, \dotsc,  {}^*\nu_l \mright)$ is an element of $\mathcal{F}^k_E(M; {}^*V_{l+1})$ given by
%\bas
%\bigl(L \circ \mleft({}^*\nu_1, \dotsc,  {}^*\nu_l \mright) \bigr)(\Phi, A)
%&=
%\bigl(L(\Phi, A)\bigr) \mleft(\Phi^*\nu_1, \dotsc,  \Phi^*\nu_l \mright)
%\eas
%for all $\Phi \in C^\infty(M;N)$ and $A\in \Omega^1(M; \Phi^*E)$.
%%When we can do that for all $\Phi_0 \in C^\infty(M;N)$ and $A_0 \in \Omega^1(M; \Phi^*_0E)$, then we can define $\delta_{\Psi_1, \dotsc, \Psi_{l+1}, \widetilde{A}} L$ as an element of $\mathcal{F}^k_E\mleft( M; {}^*\mleft( \mleft(\bigotimes_{m=1}^l (V_m)^* \mright) \otimes V_{l+1} \mright) \mright)$.
%\end{definitions}
%
%\begin{remark}
%\leavevmode\newline
%\indent $\bullet$ Eq.~\eqref{DefKomplexeProduktRegelBeiVariationen} is well-defined: To show this we only need to check whether the equation is tensorial with respect to pull-backs of $\Gamma(V_j)$ since we can then extend it canonically to all sections, by making use of that pull-backs of sections generate all sections. Assume w.l.o.g. that $l=1$ and that $L$ is independent of the fields $A$ in order to simplify the notation, then observe
%\bas
%&\Big(\mleft(\delta_{\Psi_1, \Psi_{2}} L \mright)[\Phi_0]\Big)_p \mleft((f\circ \Phi_0)(p) ~ \mleft.\Phi_0^*\nu_1\mright|_p  \mright) \\
%&=
%\biggl(\mleft(\delta_{\Psi_{2}} \Bigl(L \circ \bigl( {}^*(f\nu_1) \bigr) \Bigr)\mright) [\Phi_0]\biggr)_p
	%- \mleft( L [\Phi_0] \mright)_p\mleft( \biggl( \Bigl(\delta_{\Psi_1} \bigl( {}^*(f\nu_1) \bigr)\Bigr)[\Phi_0] \biggr)_p \mright) 
%\\
%&=
%\mleft.\frac{\mathrm{D}_{\mleft(\Psi_2(p), ^{E_2}\nabla\mright)}}{\mathrm{d}t}\mright|_{t=0}
	%\underbrace{\mleft[ t \mapsto \mleft(L \mleft[ \widetilde{\Phi}_t \mright] \mright)_p\mleft( \mleft. \widetilde{\Phi}_t^*\mleft(f \nu_1\mright) \mright|_p \mright) \mright]}
	%%%%%%%%%%%%%
	%_{
	%\mathclap{
	%= \mleft[ 
	%t ~ \mapsto 
	%\mleft(f\circ \widetilde{\Phi}_t\mright)(p)
	%~ \mleft(L \mleft[ \widetilde{\Phi}_t \mright] \mright)_p\mleft( \mleft. \widetilde{\Phi}_t^*\mleft(f \nu_1\mright) \mright|_p \mright)
	%\mright]
	%}
	%} \\
%&\hspace{1cm}
	%- \mleft( L \mleft[ \Phi_0 \mright] \mright)_p\mleft( 
		%\mleft.\frac{\mathrm{D}_{\mleft(\Psi_1(p), ^{E_1}\nabla\mright)}}{\mathrm{d}t}\mright|_{t=0}
		%\mleft[ t \mapsto \mleft(f\circ \widetilde{\Phi}_t\mright)(p) ~ \mleft.\mleft( \widetilde{\Phi}(p)  \mright)^*\nu_1\mright|_t \mright]
	%\mright) 
%\\
%&=
%(f \circ \Phi_0)(p) ~ \biggl(
	%\mleft(\delta_{\Psi_{2}} \Bigl(L \circ \bigl( {}^*\nu_1 \bigr) \Bigr)\mright) [\Phi_0]
	%- \mleft( L [\Phi_0] \mright)\Bigl( \bigl(\delta_{\Psi_1} ( {}^*\nu_1 )\bigr)[\Phi_0] \Bigr)
%\biggr)_p
%\\
%&\hspace{1cm}
	%+ \mleft( \mleft.\frac{\mathrm{d}}{\mathrm{d}t}\mright|_{t=0} \mleft[ t \mapsto \mleft(f\circ \widetilde{\Phi}_t\mright)(p) \mright] 
		%- \mleft.\frac{\mathrm{d}}{\mathrm{d}t}\mright|_{t=0} \mleft[ t \mapsto \mleft(f\circ \widetilde{\Phi}_t\mright)(p) \mright] \mright) ~
	%\mleft. (L[\Phi_0])(\Phi_0^*\nu_1) \mright|_p \\
%&=
%(f \circ \Phi_0)(p) ~ \mleft.\mleft( \Big(\delta_{\Psi_1, \Psi_{2}} L[\Phi_0]\Big) \mleft( \Phi_0^*\nu_1  \mright) \mright)\mright|_p
%\eas
%for all $\nu_1 \in \Gamma(V_1)$, $f \in C^\infty(N)$ and $p \in M$, where we used the Leibniz rule on $\mleft[ t \mapsto f \circ \widetilde{\Phi}_t\mright]$ and that $L$ is tensorial; the remaining linearity is clear by the linearity of $\frac{\mathrm{D}}{\mathrm{d}t}$.
%
%$\bullet$ Together with the statement about dual vector bundles in Remark \ref{remVariationalbasics} this definition assures as usual the Leibniz rule for variations on tensor products. See also the statement about dual vector bundles in Remark \ref{RemLeibnizeRegelaufProdukteWeshalbEConnectionNichtWichtigIst}.
%\end{remark}
%
%The following will be very useful in discussing the Lagrangian.
%
%\begin{propositions}{Variations of graded extensions of products}{IndependencewithrespecttoEconnectionsinProductVariations}
%Let $l \in \mathbb{N}$, $E, E_1, \dots E_{l+1} \to N$ be Lie algebroids and $V_1, \dotsc, V_{l+1} \to N$ real vector bundles of finite rank over a smooth manifold $N$, and let $^{E_i}\nabla$ be $E_i$-connections on $V_i$, where $i \in \{1, \dotsc, l+1\}$. Also let $M$ be another smooth manifold, $\Phi_0 \in C^\infty(M;N)$, $A_0 \in \Omega^1(M; \Phi_0^*E)$, and $L \in \mathcal{F}^0_E\mleft( M; {}^*\mleft( \mleft(\bigotimes_{m=1}^l (V_m)^* \mright) \otimes V_{l+1} \mright) \mright)$ ($k \in \mathbb{N}_0$), and let $\Psi_i: I \times M \to N$ be fields of $E_i$-paths for an open interval $I \ni 0$, all lifting the same field of base paths $\widetilde{\Phi}:I \times M \to N$ with $\Phi = \widetilde{\Phi}_{t=0}$ as in Def.~\ref{def:VariationsInVariationalCalc}. Furthermore, we have an $\widetilde{A}$ as in Def.~\ref{def:VariationsInVariationalCalc} with $\widetilde{A}_0 = A$, and let $L_j \in \mathcal{F}^{k_j}_E\mleft(M; {}^*V_j\mright)$ ($k_j \in \mathbb{N}_0$) for all $j \in \{1,\dotsc, l\}$.
%
%Then we get
%\ba
%&\delta_{\Psi_{l+1}, \widetilde{A}} \mleft(L \bigl(L_1\stackrel{\wedge}{,} \dotsc \stackrel{\wedge}{,} L_l \bigr) \mright)
%\nonumber \\
%&=
%\mleft(\mleft(\delta_{\Psi_1, \dotsc, \Psi_{l+1}, \widetilde{A}} L \mright) (\Phi_0, A_0)\mright) 
%\bigl(L_1(\Phi_0, A_0)\stackrel{\wedge}{,} \dotsc \stackrel{\wedge}{,} L_l(\Phi_0, A_0) \bigr) 
%\nonumber\\
%&\hspace{1cm}
	%+ \sum_{j=1}^l \bigl( L (\Phi_0, A_0) \bigr)\mleft( L_1(\Phi_0, A_0) \stackrel{\wedge}{,} \dotsc \stackrel{\wedge}{,} \mleft(\delta_{\Psi_j, \widetilde{A}} L_j\mright)(\Phi_0, A_0) \stackrel{\wedge}{,} \dotsc \stackrel{\wedge}{,} L_l(\Phi_0, A_0) \mright), \label{EqVariationVonVALagrangianTermen}
%\ea
%where
%\bas
%\delta_{\Psi_i}
%&=
%\frac{\mathrm{D}_{\mleft(\Psi_i, ^{E_i}\nabla\mright)}}{\mathrm{d}t},
%\eas
%and $L \bigl(L_1\stackrel{\wedge}{,} \dotsc \stackrel{\wedge}{,} L_l \bigr)$ is an element of $\mathcal{F}^{k_1+\dotsc+k_l}_E(M; {}^*V_{l+1})$ given by
%\bas
%\mleft(L \bigl(L_1\stackrel{\wedge}{,} \dotsc \stackrel{\wedge}{,} L_l \bigr)\mright)(\Phi, A)
%&=
%(L(\Phi, A))\bigl(L_1(\Phi, A)\stackrel{\wedge}{,} \dotsc \stackrel{\wedge}{,} L_l(\Phi, A) \bigr)
%\eas
%for all $\Phi \in C^\infty(M;N)$ and $A\in \Omega^1(M; \Phi^*E)$.
%\end{propositions}
%
%\begin{remark}\label{RemLeibnizeRegelaufProdukteWeshalbEConnectionNichtWichtigIst}
%\leavevmode\newline
%$\bullet$ Observe that the value of $\delta_{\Psi_{l+1}, \widetilde{A}} \mleft(L \bigl(L_1\stackrel{\wedge}{,} \dotsc \stackrel{\wedge}{,} L_l \bigr)\mright)$ is independent of the chosen $^{E_j}\nabla$ and $\Psi_j$ ($j \in \{1, \dotsc, l\}$), by its own definition. 
%We can make use of that to simplify calculations because the left hand side will be a part of our Lagrangian, \textit{i.e.}~$L$ will be then $\mathbb{R}$-valued, and for $\mathbb{R}$ we will of course take the standard flat connection in the definition of $\delta_{\Psi_{l+1}, \widetilde{A}}$ (especially then both, $E_{l+1}$ and $V_{l+1}$, are trivial $\mathbb{R}$-bundles). The information on the right hand side will be given, such that we can use Eq.~\eqref{EqVariationVonVALagrangianTermen} to calculate the left hand side; by changing the used Lie algebroid connections we can sometimes simplify this calculation.
%
%$\bullet$ For simplicity let us now assume $l=1$. Take a local frame $\mleft( e_a \mright)_a$ of $V_1$ and $\mleft( h_a \mright)_a$ of $V_2$. Then
%%\bas
%%\bigl(\delta_{\Psi_1} L_1[\Phi]\bigr)_p(Y)
%%&=
%%\mleft.\frac{\mathrm{D}_{\mleft(\Psi_1, ^{E_1}\nabla\mright)}}{\mathrm{d}t}\mright|_{t=0} \mleft[ t \mapsto
%%\mleft( L_1^a\mleft[\widetilde{\Phi}_t\mright] \mright)_p(Y) \otimes \mleft.\mleft(\widetilde{\Phi}(p)\mright)^*e_a\mright|_t
%%\mright]\\
%%&\stackrel{\mathclap{\text{Prop.~\ref{prop:DerivationAlongEPath}}}}{=}~~~~
%%\mleft(\delta_{\widetilde{\Phi}} \bigl[ \phi \mapsto L_1^a[\phi] \bigr] \mright)_p(Y)
	%%\otimes \Phi^*e_a
%%+ \mleft(L_1^a[\Phi]\mright)_p(Y) \otimes {}^{E_1}\nabla_{\Psi_{1, t=0}(p)} e_a
%%\eas
%%for all $p \in M$ and $Y \in \mathrm{T}_pM$. 
%Let $\mleft( f^b \mright)_b$ be the dual frame of $\mleft( e_a \mright)_a$, \textit{i.e.}~$f^b(e_a) = \delta^b_a$, then locally write $L_1 = L_1^a \otimes {}^* e_a$, where $L_1^a \in \mathcal{F}^{k_1}_E(M)$, and $L = L^a_b \cdot\mleft( {}^*f^b \otimes {}^*h_a \mright)$ for $L^a_b \in \mathcal{F}^0_E(M)$. Then equations like Eq.~\eqref{EqVariationVonVALagrangianTermen} imply as usual the Leibniz rule on products/contractions:
%%By Def.~\ref{def:ProductVariations} we have
%%\bas
%%\mleft(\delta_{\Psi_1, \Psi_2} L \mright)[\Phi_0]
%%&=
%%\mleft(\delta_{\widetilde{\Phi}} L^a_b \mright) [\Phi_0] \cdot \Phi_0^*f^b \otimes \Phi_0^*h_a
	%%+ L_b^a [\Phi_0] ~ \mleft(\delta_{\Psi_1}\mleft({}^*f^b\mright)\mright)[\Phi_0] \otimes \Phi_0^*h_a \\
%%&\hspace{1cm}
	%%+	L_b^a [\Phi_0] ~ \Phi_0^*f^b \otimes \bigl(\delta_{\Psi_2}\mleft({}^*h_a\mright)\bigr)[\Phi_0],
%%\eas
%%hence, in total by using Eq.~\eqref{EqVariationVonVALagrangianTermen}
%\ba
%&\mleft(\delta_{\Psi_{2}, \widetilde{A}} \bigl( ( L )(L_1) \bigr)\mright)(\Phi_0, A_0)
%%\\
%%&=
%%\delta_{\widetilde{\Phi}}L^a_b[\Phi_0] \cdot L_1^b[\Phi_0] \otimes \Phi_0^*h_a
	%%+ L_b^a [\Phi_0] \cdot L_1^b[\Phi_0] \otimes \delta_{\Psi_2}\mleft[\phi \mapsto \phi^*h_a\mright][\Phi_0] \\
%%&\hspace{1cm}
	%%+	L_b^a [\Phi_0] \cdot \underbrace{\mleft( \mleft( \delta_{\Psi_1}\mleft[\phi \mapsto\phi^*f^b\mright] \mright) [\Phi_0]\mright)\mleft( L_1[\Phi_0] \mright)}
	%%_{\mathclap{\stackrel{\text{Eq.~\eqref{EqVariationVonVALagrangianTermen}}}{=} \delta_{\widetilde{\Phi}}L_1^b[\Phi_0]
		%%- \mleft(\Phi_0^*f^b\mright)\mleft( \delta_{\Psi_1} L_1[\Phi_0] \mright)}}
	%%\otimes ~ \Phi_0^*h_a
	%%+ (L^a_b \circ \Phi_0) ~ \mleft(\Phi_0^*f^b\mright)\mleft( \delta_{\Psi_1} L_1[\Phi_0] \mright) \otimes \Phi_0^*h_a 
%\nonumber\\
%&\stackrel{\text{Cor.~\ref{cor:VariationsOfPullBackStuff}}}{=}
%\mleft(\delta_{\widetilde{\Phi}, \widetilde{A}} L^a_b\mright) (\Phi_0, A_0) \cdot L_1^b(\Phi_0, A_0) \otimes \Phi_0^*h_a
	%+ L^a_b (\Phi_0, A_0) \cdot \mleft(\delta_{\widetilde{\Phi}, \widetilde{A}}L_1^b\mright)(\Phi_0, A_0) \otimes \Phi_0^*h_a 
%\nonumber\\
%&\hspace{1cm}
	%+ L_b^a (\Phi_0, A_0) \cdot L_1^b(\Phi_0, A_0) \otimes \Phi^*_0\mleft( {}^{E_2}\nabla_{\Psi_2} h_a \mright).\label{eqProduktVariationBrauchtWirklichNurDieKomponenten}
%\ea
%%It is the same calculation for any value of $l$ such that we might introduce the short notation
%%\ba\label{eqProduktVariationBrauchtWirklichNurDieKomponenten}
%%&\delta_{\Psi_{l+1}} \mleft( L^a_{b_1, \dotsc, b_l} [\phi] ~ L^{b_1}_1[\phi] \wedge \dotsc \wedge L^{b_l}_l[\phi] \otimes \phi^*h_a \mright)[\Phi_0] \nonumber \\
%%&=
%%\delta_{\widetilde{\Phi}} L^a_{b_1, \dotsc, b_l} [\Phi_0] \cdot L^{b_1}_1[\Phi_0] \wedge \dotsc \wedge L^{b_l}_l[\Phi_0] \otimes \Phi_0^*h_a \nonumber\\
%%&\hspace{1cm}
	%%+ \sum_{j=1}^{l} L^a_{b_1, \dotsc, b_l} [\Phi_0] ~ L^{b_1}_1[\Phi_0] \wedge \dotsc \wedge \delta_{\widetilde{\Phi}} L^{b_j}_j [\Phi_0] \wedge \dotsc \wedge L^{b_l}_l[\Phi_0] \otimes \Phi_0^*e_a \nonumber \\
%%&\hspace{1cm}
	%%+ L^a_{b_1, \dotsc, b_l} [\Phi_0] \cdot L^{b_1}_1[\Phi_0] \wedge \dotsc \wedge L^{b_l}_l[\Phi_0] \otimes \delta_{\Psi_{l+1}}\mleft(\phi^*h_a \mright)[\Phi_0],
%%\ea
%%where we use frames $\mleft( e^{(j)}_{b_j} \mright)$ for $V_j$ ($j \in \{1, \dotsc, l\}$) and (shorter notation again)
%%\bas
%%\delta_{\Psi_j} L_j[\Phi_0]
%%&=
%%\delta_{\widetilde{\Phi}} L^{b_j}_j [\Phi_0] \otimes \Phi_0^*e^{(j)}_{b_j}
	%%+ L^{b_j}_j[\Phi_0] \otimes \delta_{\Psi_j} \mleft( \phi^*e^{(j)}_{b_j} \mright)[\Phi_0].
%%\eas
%Eq.~\eqref{eqProduktVariationBrauchtWirklichNurDieKomponenten} emphasizes the first remark, \textit{i.e.}~that we just need to know how the components get varied when $L$ is scalar-valued, whose variation is independent of the chosen connections since we take the standard parameter derivative on $\mathbb{R}$-valued tensors; a constant is then chosen for $h_a$ as usual such that ${}^{E_2}\nabla h_a = 0$. This allows us to do the following: In order to achieve gauge invariance of the Lagrangian we just need certain variations of the components, we can then take any connection for the variations which results into the same variations of the components. Such a connection is not unique due to $\mleft(\delta_{\Psi_1, \widetilde{A}} L_1 \mright)^a (\Phi_0, A_0) \neq \mleft(\delta_{\Psi_1, \widetilde{A}} L_1^a\mright)[\Phi_0[A_0]$ in general, and this allows us to explicitly change the definition of the infinitesimal gauge transformations with precisely the same effect on the Lagrangian. In more detail, we have
%\bas
%\mleft(\delta_{\Psi_1, \widetilde{A}} L_1\mright)(\Phi_0, A_0)
%&=
%\mleft(\delta_{\widetilde{\Phi}, \widetilde{A}} L_1^a\mright)(\Phi_0, A_0) \otimes \Phi_0^*e_a
	%+ L_1^b(\Phi_0, A_0) \otimes \bigl( \delta_{\Psi_1}({}^*e_a) \bigr)[\Phi_0]
%\nonumber\\
%&=
%\mleft(\delta_{\widetilde{\Phi}, \widetilde{A}} L_1^a\mright)(\Phi_0, A_0) \otimes \Phi_0^*e_a
	%+ L_1^b(\Phi_0, A_0) \otimes \Phi^*_0\mleft( {}^{E_1}\nabla_{\Psi_{1,t=0}} e_b \mright),
%\eas
%and since only the first summand is important for the variation of the Lagrangian (or for contractions in general), we can freely shape the second summand. That allows us to simplify coordinate-free calculations. Moreover, we will motivate a certain definition of the infinitesimal gauge transformation by studying its curvature.
%
%This argument of course applies to the connection used in the variation of $A$, see Def.~\ref{def:VariationsInVariationalCalc}, if the dependency of $A$ is in sense of a contraction like $C^a_{bc} A^b \wedge A^c$ for structure constants $C^a_{bc}$ as arising in the standard formulation of gauge theory, and in these contractions the differentials $\mathrm{d}A^a$ are allowed, too, see Eq.~\eqref{eqVariationVertauschtMitDifferential} later (basically, $\delta$ commutes with $\mathrm{d}$).
%%do not need to know what $\delta_{\Psi_j} \mleft( \phi^*e^{(j)}_{b_j} \mright)[\Phi_0]$ gives, we just need to know $\delta_{\widetilde{\Phi}} L^{b_j}_j[\Phi_0]$, $\delta_{\widetilde{\Phi}} L^a_{b_1, \dotsc, b_l} [\Phi_0]$ and $\delta_{\Psi_{l+1}}\mleft(\phi^*h_a \mright)[\Phi_0]$. When $L$ is scalar-valued and when taking the standard $\mathrm{D}/\mathrm{d}t$ on $\mathbb{R}$, then we not even need the last one such that everything is independent of the chosen Lie algebroid connections. This shows that for the variation of Lagrangians we could just define how the scalar-valued components with respect to pull-backs of frames are varied, assuming standard conventions like using the dual connection on dual bundles and the standard $\mathrm{D}/\mathrm{d}t$ on $\mathbb{R}$.
%%
%%Finally, while we have in general $\mleft(\delta_{\Psi_j} L_j[\Phi_0] \mright)^{b_j} \neq \delta_{\widetilde{\Phi}}L_j^{b_j}[\Phi_0]$ due to the contribution of the pull-back frame, we can observe that for contractions with scalar-valued $L$ we can calculate as we would have an equality instead, since the contributions of the pull-back frames cancel with the ones of the dual frames.
%\end{remark}
%
%\begin{proof}[Proof of Prop.~\ref{prop:IndependencewithrespecttoEconnectionsinProductVariations}]
%\leavevmode\newline
%%The proof will be independent of $l$, thus, assume $l=1$ for simplicity. Take a local frame $\mleft( e_a \mright)_a$ of $V_1$, and write $L_1[\phi] = \omega^a[\phi] \otimes \phi^*e_a$ with respect to a $k_1$-form $\omega^a[\phi]$. Then, by using Def.~\ref{def:ProductVariations},
%%\bas
%%\mleft(\delta_{\Psi_1, \Psi_{2}} \mleft[ \phi \mapsto \phi^*F \mright]\mright) 
%%\mleft(L_1[\Phi] \mright) 
%%&=
%%\omega^a[\Phi] \otimes \mleft(
	%%\delta_{\Psi_{2}} \Big[ \phi \mapsto \phi^*\mleft( F \mleft(\nu_1 \mright) \mright) \Big]
		%%- \mleft( \Phi^*F \mright)\mleft( \delta_{\Psi_1} \mleft[ \phi \mapsto \phi^*\nu_1 \mright] \mright) \mright)
%%\eas
%By Def.~\ref{def:VariationsInVariationalCalc}, the $\frac{\mathrm{D}}{\mathrm{d}t}$ operators commute with the evaluation at tangent vectors and points. Doing so will result into that form-valued arguments are sections of a pull-back vector bundle. Then the statement is a direct consequence of Def.~\ref{def:ProductVariations}.
%\end{proof}
%
%%We will not just vary with respect to $\Phi$ but also with respect to the field of gauge bosons $A$, this will be an element of $\Omega^1(M; \Phi^*E)$ for a given $\Phi$. It is straightforward to increase the number of arguments of varied fields in the previous definitions, but the pull-back of the vector bundles is still just with respect to $\Phi$.
%%%we have for example maps $\left\{(\phi, \omega) ~ \middle| ~ \phi \in C^\infty(M;N), \omega \in \Omega^1(M; \Phi^*E)\right\} \ni (\Phi, A) \mapsto L[\Phi] [\omega] \in \Omega^k(M; \Phi^*E)$ ($k \in \mathbb{N}_0$). 
%%On one hand $A$ has also values in the pull-back vector $\Phi^*E$ bundle, thus, we need to choose a connection on $E$, too. On the other hand $A$ will not be a map defined on $\Phi$, such that we still need to define some parts of the variation of $A$. For example, with respect to a local frame $\mleft( e_a \mright)_a$ of $E$, we can locally write $A= A^a \otimes \Phi^*e_a$ for 1-forms $A^a$. The variation of $\Phi^*e_a$ is already defined, but not the variation of $A^a$ itself.
%%
%%\begin{definitions}{Variations with gauge bosons}{VariationsInVariationalCalcWithGaugeBosons}
%%Let $M, N$ be two smooth manifolds and $\Phi: M \to N$ a smooth map. Also let $E, B \to N$ be Lie algebroids, $V \to N$ a vector bundle, ${}^B\nabla$ a $B$-connection on $V$, $\Phi \in C^\infty(M;N)$ and $A \in \Omega^1(M; \Phi^*E)$. For an open interval $I \subset \mathbb{R}$ with $0 \in I$ we assume that we have a field of $B$-paths $\Psi: I \times M \to B$, $(t,p) \mapsto \Psi_t(p)$ such that its field of base paths $\widetilde{\Phi}$ satisfies $\widetilde{\Phi}_{t=0} \equiv \Phi$, and we have $\widetilde{A} \in \Omega^1\mleft(M; \widetilde{\Phi}^*E\mright)$, $I \ni t \mapsto \widetilde{A}_{t} \in \Omega^1\mleft(M; \mleft( \widetilde{\Phi}_{t} \mright)^*E\mright)$, with 
%%\bas
%%\widetilde{A}_{t=0} &= A.
%%\eas
%%Now let $L \in \mathcal{F}_E^k(M; V)$ ($k \in \mathbb{N}_0$). Then we define the \textbf{variation $\delta_{\Psi, \widetilde{A}} L(\Phi, A)$ of $L$ along $\Psi$ and $\widetilde{A}$} as an element of $\Omega^{k}(M; \Phi^*V)$ by
%%\ba\label{defNochVielKrassesVariationsPrinzip}
%%\mleft(\delta_{\Psi, \widetilde{A}} L(\Phi, A)\mright)_p\mleft(Y_1, \dotsc, Y_k\mright)
%%&\coloneqq
%%\mleft.\frac{\mathrm{D}_{\mleft(\Psi(p), {}^B\nabla\mright)}}{\mathrm{d}t}\mright|_{t=0}
%%\mleft[ t \mapsto
%%L\mleft[ \widetilde{\Phi}_t \mright] \mleft[ \widetilde{A}_t \mright]_p\mleft(Y_1, \dotsc, Y_{k}\mright)
%%\mright]
%%\ea
%%for all $p \in M$ and $Y_1, \dotsc, Y_k \in \mathrm{T}_pM$. Shortly written as
%%\bas
%%\delta_{\Psi, \widetilde{A}} L[\Phi, A]
%%&\coloneqq
%%\mleft.\frac{\mathrm{D}_{\mleft(\Psi, {}^B\nabla\mright)}}{\mathrm{d}t}\mright|_{t=0}
%%\bigg[ t \mapsto
%%L\mleft[ \widetilde{\Phi}_t \mright]\mleft[ \widetilde{A}_t \mright]
%%\bigg].
%%\eas
%%When we have such a $\widetilde{\Phi}$ and $\Psi$ for all $\Phi \in C^\infty(M;N)$, and an $\widetilde{A}$ for all $A$, then we can define $\delta_{\Psi} L$ as a smooth map, $\phi \to \delta_{\Psi} L[\phi]$, $\Omega^1(M; \phi^*E) \ni \omega \mapsto L[\phi][\omega] \in \Omega^k(M; \phi^*E)$, using \eqref{defNochVielKrassesVariationsPrinzip}.
%%\end{definitions}
%%
%%\begin{remark}\label{UnabhaengigkeitMitAAuchMoeglichAberNurMitKontraktion}
%%\leavevmode\newline
%%Def.~\ref{def:ProductVariations} and Prop.~\ref{prop:IndependencewithrespecttoEconnectionsinProductVariations} can clearly be extended to this definition, also Remark \ref{RemLeibnizeRegelaufProdukteWeshalbEConnectionNichtWichtigIst} if the dependency of $A$ is in sense of a contraction like $C^a_{bc} A^b \wedge A^c$ for structure constants $C^a_{bc}$ as arising in the standard formulation of gauge theory, and in these contractions the differentials $\mathrm{d}A^a$ are allowed, too, see Eq.~\eqref{eqVariationVertauschtMitDifferential} later (basically, $\delta$ commutes with $\mathrm{d}$).
%%\end{remark}
%%
%%As for $\Phi$ we could try to define $\delta_{\widetilde{A}} A$ in a similar manner; but since $A$ has values in a vector bundle we should rather try to preserve this by fixing a ($E$-)connection on $E$. To fix such a connection we need to introduce the gauge transformations.

\section{Infinitesimal gauge transformations}\label{InfinitesimalGaugeTransformation}

\subsection{Infinitesimal gauge transformation of the Higgs field}

We will now do precisely the same, but more general, as in Section \ref{NewInfGaugeTrafoTrafos}. Infinitesimal gauge transformations of a functional $L \in \mathcal{F}^k(M; {}^*V)$ ($k \in \mathbb{N}_0$ and $V \to N$ a vector bundle) are derivatives along certain directions in $\mathfrak{M}_E(M; N)$, while the components of these directions as vector field will be identified with the infinitesimal gauge transformations of the corresponding fields, $\Phi$ and $A$. We want that these transformations satisfy the Leibniz rule, and we want to study the commutator of such two transformations. In order to do that easily, we require that such a derivative keeps a functional vertical, \textit{i.e.}~$\delta L \in \mathcal{F}^k(M; {}^*V)$, where $\delta$ denotes such a transformation, and for this we will use connections, especially ones induced by a Lie algebroid connection on $V$ itself. We will do that by using pull-backs, especially using Cor.~\ref{cor:VeryGeneralPullbackConnection}. That is, since functionals are forms on $M \times \mathfrak{M}_E$, we want to make the pullback along $\mathrm{ev}$, while avoiding the issue of lifting the evaluation map to a suitable vector bundle morphism by restricting to certain vector fields on $\mathfrak{M}_E$ satisfying the condition given in Cor.~\ref{cor:VeryGeneralPullbackConnection}; we will see that this will precisely give the formula of the infinitesimal gauge transformation of the Higgs field.

The arguments are precisely the same as in the discussion before Def.~\ref{def:ClassicGaugeTrafoOfHiggs}. Hence, we start now with a similar definition, but, as we also mentioned in the discussion of Def.~\ref{def:ClassicGaugeTrafoOfHiggs}, the Lie algebroid used for the mentioned Lie algebroid connection on $V$ does not need to be the same Lie algebroid used in the definition of $\mathfrak{M}_E(M;N)$. This is why there is now a second Lie algebroid $B$ over $N$, equipped with a Lie algebroid connection ${}^B\nabla$ on $V$; but when we turn to the infinitesimal gauge transformation of quantities like the minimal coupling, it is useful to have $E=B$, which we are then going to assume. However, one may want to do a similar construction using a typical vector bundle connection on $V$ which implies $B=\mathrm{T}N$; in order to allow those type of constructions we keep it that general for the basic definitions. Also recall Prop.~\ref{prop:TangentSpaceOfSpaceOfFields}.

\begin{definitions}{Vector fields along Lie algebroid paths}{VectorFieldAlongEPaths}
Let $M, N$ be two smooth manifolds and $\mleft(E, \rho_E, \mleft[ \cdot,\cdot \mright]_E \mright)$, $\mleft(B, \rho_B, \mleft[ \cdot,\cdot \mright]_B \mright)$ two Lie algebroids over $N$. For $(\Phi, A) \in \mathfrak{M}_E(M; N)$ we define $\mathrm{T}^B_{(\Phi,A)}\mathfrak{M}_E(M; N)$ as a subspace of $\mathrm{T}_{(\Phi,A)}\mathfrak{M}_E(M; N)$ by
\ba
\mathrm{T}^B_{(\Phi,A)}\mathfrak{M}_E(M; N)
&\coloneqq
\left\{ (\mathcal{v}, \mathcal{a}) \in \mathrm{T}_{(\Phi,A)}\bigl(\mathfrak{M}_E(M; N)\bigr)
~\middle|~
\exists \epsilon \in \Gamma(\Phi^*B):~
\mathcal{v} = - (\Phi^*\rho_B)(\epsilon)
\right\}.
\ea
The set of sections with values in these subspaces, called the set of \textbf{vector fields along $B$-paths}, is denoted by $\gls{XBM}$.
\end{definitions}

\begin{remark}\label{NotASubalgebraXB}
\leavevmode\newline
As images of the pullback of the anchor, it is clear that $\mathrm{T}^B_{(\Phi,A)}\bigl(\mathfrak{M}_E(M; N)\bigr)$ and $\mathfrak{X}^B\bigl(\mathfrak{M}_E(M; N)\bigr)$ are subspaces of $\mathrm{T}_{(\Phi,A)}\bigl(\mathfrak{M}_E(M; N)\bigr)$ and $\mathfrak{X}\bigl(\mathfrak{M}_E(M; N)\bigr)$, respectively.

For all $\Psi \in \mathfrak{X}^B(\mathfrak{M})$ there is by definition then an $\varepsilon \in \mathcal{F}^0_E(M; {}^*B)$ such that 
\ba\label{GaugeTrafoVektor}
\Psi
&=
\mleft( -({}^*\rho_B )(\varepsilon), \mathfrak{a} \mright)
\ea
where $({}^*\rho_B)(\varepsilon)$ is an element of $\mathcal{F}^0_E(M; {}^*\mathrm{T}N)$ given by $\mathfrak{M}_E(M; N) \ni (\Phi, A) \mapsto (\Phi^*\rho_B)(\varepsilon(\Phi, A))$, and $\mathfrak{a}$ is a map defined on $\mathfrak{M}_E(M; N)$ such that $\Psi|_{(\Phi,A)}$ is a tangent vector for all $(\Phi, A) \in \mathfrak{M}_E(M; N)$ as in Prop.~\ref{prop:TangentSpaceOfSpaceOfFields}. We will study $\mathfrak{a}$ in more detail later, but now it will not be important. We will write $\Psi \eqqcolon \Psi_\varepsilon$ to emphasize the relationship with an $\varepsilon \in \mathcal{F}^0_E(M; {}^*B)$. As in Remark \ref{PsiEpsilonDieErste}, for a given $\varepsilon$ there can be several $\Psi_\varepsilon$ as long as we do not fix $\mathfrak{a}$. Moreover, since $\varepsilon \in \mathcal{F}^0_E(M; {}^*B)$ we cannot expect in general that $\mathfrak{X}^B\bigl(\mathfrak{M}_E(M; N)\bigr)$ is a subalgebra of $\mathfrak{X}\bigl(\mathfrak{M}_E(M; N)\bigr)$. One may be able to show that if just allowing $\varepsilon = {}^*b$ ($b \in \Gamma(B)$), but since those more general $\varepsilon$ can have  very general dependencies on $(\Phi,A) \in \mathfrak{M}_E(M;N)$ one cannot expect a sub-algebraic behaviour at this point. We will come back to this after we will have defined the infinitesimal gauge transformation for the field of gauge bosons.
\end{remark}

By construction, the flows of those vector fields carry the structure of Lie algebroid paths which will allow us to do pullbacks of connections along these flows in order to define certain connections on functionals.

\begin{corollaries}{Flows of $\mathfrak{X}^B\bigl(\mathfrak{M}_E(M; N)\bigr)$}{ReasonWhyVectorFieldAlongAnBPath}
Let $M, N$ be two smooth manifolds and $\mleft(E, \rho_E, \mleft[ \cdot,\cdot \mright]_E \mright)$, $\mleft(B, \rho_B, \mleft[ \cdot,\cdot \mright]_B \mright)$ two Lie algebroids over $N$. For a $\Psi \in \mathfrak{X}^B(\mathfrak{M}_E(M; N))$ we denote its flow by $\gamma = (\Phi, A): I \to \mathfrak{M}_E(M; N)$, $t \mapsto \gamma(t) = (\Phi_t, A_t) \in \mathfrak{M}_E(M; N)$ through a fixed point $(\Phi_0, A_0) \in \mathfrak{M}_E(M; N)$ at $t=0$, where $I$ is an open interval of $\mathbb{R}$ containing 0, and we write $\Psi|_{\gamma(t)} = \mleft( - (\Phi_t^*\rho_B)(\epsilon_t), \mathcal{a}_t \mright) \in \mathrm{T}^B_{(\Phi_t, A_t)}\mathfrak{M}_E(M; N)$, where $\epsilon_t \in \Gamma(\Phi_t^*B)$ and $\mathcal{a}_t \in \Omega^1\mleft(M; \epsilon_t^*\mathrm{T}E\mright)$ (recall Prop.~\ref{prop:TangentSpaceOfSpaceOfFields}).

Then $-\epsilon(p) \coloneqq \mleft[ t\mapsto -\epsilon_t|_p\mright]$, viewed as a curve $I \to B$, is a $B$-path with base path $\Phi(p) \coloneqq \mleft[ t \mapsto \Phi_t(p) \mright]$ for all $p \in M$.
\end{corollaries}

\begin{proof}
\leavevmode\newline
For $p \in M$ fixed, it is clear by definition that the base path of $-\epsilon(p)$ is given by $\Phi(p)$ since $\epsilon_t|_p \in B_{\Phi_t(p)}$ for all $t \in I$, where $B_{\Phi_t(p)}$ is the fibre of $B$ at $\Phi_t(p)$. By definition of flows we have
\bas
\mleft.\frac{\mathrm{d}}{\mathrm{d}t}\mright|_t \gamma
&=
\Psi|_{\gamma(t)}
\eas
for all $t \in I$, and, so,
\bas
\mleft. \bigl((\Phi(p))^*\rho_B\bigr)\bigl(-\epsilon(p)\bigr)\mright|_t
&=
-\mleft. (\Phi_t^*\rho_B)(\epsilon_t)\mright|_p
= 
\mleft.\frac{\mathrm{d}}{\mathrm{d}t}\mright|_t \bigl(\Phi(p)\bigr),
\eas
which proves the claim.
\end{proof}

As in Section \ref{NewInfGaugeTrafoTrafos}, the first component of these vector fields also define the infinitesimal gauge transformation of the Higgs field.

\begin{definitions}{Infinitesimal gauge transformation of $\Phi$}{VariationenOfAundPhi}
Let $M, N$ be two smooth manifolds, $\mleft(E, \rho_E, \mleft[ \cdot,\cdot \mright]_E \mright)$, $\mleft(B, \rho_B, \mleft[ \cdot,\cdot \mright]_B \mright)$ two Lie algebroids over $N$, and $\varepsilon \in \mathcal{F}^0_E(M; {}^*B)$. For a $(\Phi, A) \in \mathfrak{M}_E(M; N)$ we define the \textbf{infinitesimal gauge transformation $\delta^B_{\varepsilon(\Phi, A)} \Phi$ of $\Phi$ along $\varepsilon(\Phi, A)$} as an element of $\Gamma (\Phi^*\mathrm{T}N)$ by
\ba\label{EqVariationOfHiggsField}
%\mleft( \gls{1delta0epsilon} \mathds{1}_{C^\infty(M;N)} \mright)(\Phi, A)
%&\coloneqq
\delta^B_{\varepsilon(\Phi, A)} \Phi
&\coloneqq
\bigl( -\mleft({}^*\rho_B\mright)(\varepsilon) \bigr) (\Phi, A)
=
- \mleft( \Phi^* \rho_B \mright)\bigl(\varepsilon(\Phi, A)\bigr),
\ea
shortly denoted as $\delta^B_\varepsilon \Phi \coloneqq - \mleft({}^*\rho_B\mright)(\varepsilon) \in \mathcal{F}^0_E(M; {}^*\mathrm{T}N)$.

In the case of $E=B$ we just write $\delta_\varepsilon \Phi \coloneqq - ({}^*\rho)(\varepsilon)$.
\end{definitions}

\begin{remark}\label{RemUeberVariationVonHiggs}
\leavevmode\newline
\indent $\bullet$ 
%For Eq.~\eqref{EqVariationOfHiggsField} recall Def.~\ref{def:VariationsInVariationalCalc}, \textit{i.e.}
%\bas
%\mleft(\delta_{\widetilde{\Phi}_\varepsilon} \Phi\mright)_p
%&=
%\mleft.\frac{\mathrm{d}}{\mathrm{d}t}\mright|_{t=0}\mleft[ t \mapsto \widetilde{\Phi}_{\varepsilon, t}(p) \mright]
%\stackrel{\text{Def.~\ref{def:EPaths}}}{=}
%\rho_{\Phi(p)}\mleft( \Psi_{\varepsilon, t=0}(p) \mright)
%\stackrel{\text{Eq.~\ref{EPfadWertBeiTGleich0}}}{=}
%- \mleft.\mleft( \Phi^* \rho \mright)(\varepsilon)\mright|_p
%\eas
%for all $p \in M$, where we used the abbreviated notation. 
Eq.~\eqref{EqVariationOfHiggsField} is also a generalization of a similar equation for a gauge transformation given in \cite[paragraph before Equation (10); we have a different sign in $\varepsilon$]{CurvedYMH}.
%
%$\bullet$ In the standard setting the infinitesimal gauge transformations are parametrised by sections $\epsilon$ of $M \times \mathfrak{g}$ for a given Lie algebra $\mathfrak{g}$. When $E = N \times \mathfrak{g}$ is an action Lie algebroid, we then take any constant global frame $\mleft( e_a \mright)_a$ of $E$, such that we have a trivialisation of $\Phi^*E \cong M \times \mathfrak{g}$ by $\mleft( \Phi^*e_a \mright)_a$. Thus, $\epsilon = \epsilon^a ~ \Phi^*e_a \in \Gamma(\Phi^*E)$, and with that notation there is a natural (frame-dependent) interpretation of $\epsilon$ as an element of $\mathcal{F}^0_E(M;{}^*E)$ given by $\epsilon = \epsilon^a ~ \Phi^*e_a \mapsto \varepsilon \coloneqq \epsilon^a ~ {}^*e_a \in \mathcal{F}^0_E(M;{}^*E)$. That identification is surely not surjective, so, hereby we see that $\mathcal{F}^0_E(M;{}^*E)$ is a richer space than $C^\infty(M;\mathfrak{g})$, and, thus, we are going to achieve a gauge invariance with respect to a bigger space, even when we restrict ourselves to the standard setting. 

$\bullet$ Finally let us observe why Eq.~\eqref{EqVariationOfHiggsField} recovers the standard formula of the infinitesimal gauge transformation of $\Phi$, Def.~\ref{def:ClassicTrafos}. As usual, use the setting as in Cor.~\ref{cor:StandardTheory}, \textit{i.e.}~let $W$ be a vector space and $N = W$ such that $\Phi \in C^\infty(M;W)$, and $E = N \times \mathfrak{g}$ an action Lie algebroid for a Lie algebra $\mathfrak{g}$ whose Lie algebra action $\gamma$ is induced by a Lie algebra representation $\psi: \mathfrak{g} \to \mathrm{End}(W)$. Also $E=B$. Then we can simply use Prop.~\ref{prop:LieRepAndLieAct}, using $\epsilon \coloneqq \varepsilon(\Phi, A)$, to get
\bas
\mleft(\delta_\varepsilon \Phi\mright)(p)
&=
- \epsilon^a(p) ~ \rho_{\Phi(p)}(e_a)
=
- \epsilon^a(p) ~ \gamma(e_a)_{\Phi(p)}
=
\epsilon^a(p)  ~ \psi(e_a)\bigl(\Phi(p)\bigr)
=
\psi\mleft(\epsilon_p\mright)\bigl(\Phi(p)\bigr)
\eas
for all $p \in M$ and $\epsilon \in \Gamma(\Phi^*E)$ viewed as an element of $C^\infty(M; \mathfrak{g})$, where $\mleft( e_a \mright)_a$ is a frame of constant sections. This is precisely the standard formula.
\end{remark}

There is a relationship similar to Cor.~\ref{cor:VeryGeneralPullbackConnection}, which summarizes the whole motivation of our construction; also recall Remark \ref{rem:CommutingDiagramOfPullbacks}.

\begin{corollaries}{Infinitesimal gauge transformation as condition for allowing pullbacks}{CoolesCommutingDiagramForHiggsTrafosStuff}
Let $M, N$ be two smooth manifolds and $\mleft(E, \rho_E, \mleft[ \cdot,\cdot \mright]_E \mright)$, $\mleft(B, \rho_B, \mleft[ \cdot,\cdot \mright]_B \mright)$ two Lie algebroids over $N$, and $\varepsilon \in \mathcal{F}^0_E(M; {}^*B)$. Then $\Psi \in \mathfrak{X}\bigl(\mathfrak{M}_E(M;N)\bigr)$ is an element of $\mathfrak{X}^B\bigl(\mathfrak{M}_E(M; N)\bigr)$ if and only if there is an $\varepsilon \in \mathcal{F}^0_E(M; {}^*B)$ such that the following diagram commutes
\begin{center}
	\begin{tikzcd}
		M \times \mathfrak{M}_E(M;N) \arrow{r}{-\varepsilon} \arrow{d}{(0, \Psi)}	& B \arrow{d}{\rho_B} 
		\\
		\mathrm{T}\bigl( M \times \mathfrak{M}_E(M;N) \bigr) \arrow{r}{\mathrm{Dev}} & \mathrm{T}N
	\end{tikzcd}
\end{center}
that is
\ba
\mathrm{Dev} \circ (0, \Psi)
&=
-\rho_B \circ \varepsilon,
\ea
where $(0, \Psi) \in \mathfrak{X}(M) \times \mathfrak{X}\bigl(\mathfrak{M}_E(M; N)\bigr)$ is the canonical embedding of $\Psi$ as a vector field on $M \times \mathfrak{M}_E(M; N)$.
\end{corollaries}

\begin{proof}
\leavevmode\newline
That is by construction. Let $\gamma = (\Phi, A): I \to \mathfrak{M}_E(M;N), t \mapsto \gamma(t)= (\Phi_t, A_t)$ ($I \subset \mathbb{R}$ an open interval containing 0) be the flow of $\Psi$ through $(\Phi_0, A_0) \in \mathfrak{M}_E(M;N)$ at $t=0$, as \textit{e.g.}~in Cor.~\ref{cor:ReasonWhyVectorFieldAlongAnBPath}. Then the local flow of $(0, \Psi)$ through $(p, \Phi_0, A_0) \in M \times \mathfrak{M}_E(M;N)$ is given by $(p, \Phi, A)$. Thus,
\bas
\mathrm{D}_{(p, \Phi_0, A_0)}\mathrm{ev}(0,\Psi)
&=
\mleft. \frac{\mathrm{d}}{\mathrm{d}t} \mright|_{t=0}\mleft(
	\mathrm{ev}(p, \Phi, A)
\mright)
=
\mleft. \frac{\mathrm{d}}{\mathrm{d}t} \mright|_{t=0}\mleft[
	t \mapsto \Phi_t(p)
\mright]
=
\mleft.\mleft(\mleft.\Psi^{(\Phi)}\mright|_{(\Phi_0, A_0)}\mright)\mright|_p
\in \mathrm{T}_{\Phi_0(p)}N,
\eas
where $\Psi^{(\Phi)}$ is the first component of $\Psi$, for this also recall Prop.~\ref{prop:TangentSpaceOfSpaceOfFields}. The commutation of the diagram is then equivalent to say that there is an $\varepsilon \in \mathcal{F}^0_E(M; {}^*B)$
\bas
\Psi^{(\Phi)}
&=
- ({}^*\rho_B)(\varepsilon),
\eas
which is precisely the definition for $\mathfrak{X}^B\bigl(\mathfrak{M}_E(M;N)\bigr)$ of Def.~\ref{def:VectorFieldAlongEPaths}.
\end{proof}

That immediately leads to:

\begin{propositions}{Parametrised variations of functionals}{VariationVonSkalarZeugsEasyPeasy}
Let $M, N$ be two smooth manifolds, $\mleft(E, \rho_E, \mleft[ \cdot,\cdot \mright]_E \mright)$, $\mleft(B, \rho_B, \mleft[ \cdot,\cdot \mright]_B \mright)$ two Lie algebroids over $N$, $V \to N$ a vector bundle, ${}^B\nabla$ a $B$-connection on $V$, and $\Psi_\varepsilon \in\mathfrak{X}^B(\mathfrak{M}_E(M; N))$ for $\varepsilon \in \mathcal{F}^0_E(M; {}^*B)$. Then there is a unique $\mathbb{R}$-linear map $\gls{1delta0Psiepsilon}: \mathcal{F}_E^\bullet(M;{}^*V) \to \mathcal{F}_E^\bullet(M;{}^*V)$ with
\ba\label{PullBackVariation}
\delta_{\Psi_\varepsilon} \mleft( {}^* v \mright)
&=
- {}^*\mleft({}^B\nabla_{\varepsilon} v \mright),
%\\
%\delta_{\Psi_\varepsilon} f
%&=
%\mathcal{L}_
\\
\iota_Y \delta_{\Psi_\varepsilon}
&=
\delta_{\Psi_\varepsilon} \iota_Y
\label{VertauschungMitVerjuengungVonEichtrafo}\\
\delta_{\Psi_\varepsilon}(f \wedge L)
&=
\mathcal{L}_{\Psi_\varepsilon}(f) \wedge L
	+ f \wedge \delta_{\Psi_\varepsilon} (L), \label{LeibnizForGauging}
\ea
for all $Y \in \mathfrak{X}(M)$, $v \in \Gamma(V)$, $L \in \mathcal{F}_E^k(M; {}^*V)$, and $f \in \mathcal{F}^m_E(M)$ ($k, m \in \mathbb{N}_0$), where $\mathcal{F}_E^\bullet(M;{}^*V) \coloneqq \bigoplus_{l\in \mathbb{N}_0} \mathcal{F}^l_E(M; {}^*V)$ while $\delta_{\Psi_\varepsilon}$ keeps a given degree invariant.
\end{propositions}

\begin{remark}
\leavevmode\newline
Since the notation of $\delta_{\Psi_\varepsilon}$ does not emphasize the used connection, we will often roughly write: \textbf{For the functional space $\mathcal{F}^\bullet_E(M;{}^*V)$ let $\delta_{\Psi_\varepsilon}$ be the unique operator of Prop.~\ref{prop:VariationVonSkalarZeugsEasyPeasy}, using ${}^B\nabla$ as a $B$-connection on $V$}, where $\bullet$ denotes an arbitrary degree.
\end{remark}

\begin{proof}[Proof of Prop.~\ref{prop:VariationVonSkalarZeugsEasyPeasy}]
\leavevmode\newline
That is a trivial consequence of Cor.~\ref{cor:CoolesCommutingDiagramForHiggsTrafosStuff} and Cor.~\ref{cor:VeryGeneralPullbackConnection}, that is, we have a unique $\mathbb{R}$-linear operator $\delta_{\Psi_\varepsilon}: \mathcal{F}^0_E(M; {}^*V) \to \mathcal{F}^0_E(M; {}^*V)$ such that
\bas
\delta_{\Psi_\varepsilon}(h s)
&=
\mathcal{L}_{\Psi_\varepsilon}(h) ~ s
	+ h ~ \delta_{\Psi_\varepsilon} s,
\\
\delta_{\Psi_\varepsilon}\underbrace{({}^*v)}_{\mathclap{ = \mathrm{ev}^*v }}
&=
-{}^*\mleft( {}^B\nabla_\varepsilon v \mright)
\eas
for all $s \in \Gamma(\mathrm{ev}^*V) = \mathcal{F}^0_E(M; {}^*V), h \in C^\infty(M \times \mathfrak{M}_E)$, and $v \in \Gamma(V)$. Eq.~\eqref{VertauschungMitVerjuengungVonEichtrafo} and linearity uniquely extends this operator to $\mathcal{F}^\bullet_E(M; {}^*V)$, that is,
\ba\label{DefOfGaugeTrafoWithBookkeep}
\mleft(\delta_{\Psi_\varepsilon} L\mright)(Y_1, \dotsc, Y_k)
&\coloneqq
\delta_{\Psi_\varepsilon}\bigl( L(Y_1, \dotsc, Y_k)\bigr)
\ea
for all $L \in \mathcal{F}^k_E(M; {}^*V)$ and $Y_1, \dotsc, Y_k \in \mathfrak{X}(M)$; similar to Def.~\ref{def:InfinitesimalGaugeTrafoClassicAsConnection} this is well-defined (recall also the remark after Def.~\ref{def:InfinitesimalGaugeTrafoClassicAsConnection}). This is not in violation with the desired Leibniz rule because $\Psi_\varepsilon$ are vector fields on $\mathfrak{M}_E(M;N)$ while $Y_1, \dotsc, Y_k$ are vector fields on $M$, thence, $[\Psi_\varepsilon, Y_i] = 0$ ($i \in \{1, \dotsc, k\}$) in $M \times \mathfrak{M}_E(M;N)$; also recall Eq.~\eqref{SliceOfBIiiigManifold}, the non-trivial information of $L$ is just stored in along $\mathrm{T}M$, then everything follows by the fact that $\mathcal{L}_{\Psi_\varepsilon}$ preserves the factors in $\mathrm{T}M \times \mathrm{T}\mathfrak{M}_E$ and that $\mathfrak{X}(M\times\mathfrak{M}_E)$ is generated by vector fields on $M$ and $\mathfrak{M}_E$ such that one can conclude that $\delta_\varepsilon L$ contains its non-trivial information just along $M$. The Leibniz rule in Eq.~\eqref{LeibnizForGauging} then just follows by this and the Leibniz rule inherited by Cor.~\ref{cor:VeryGeneralPullbackConnection}. In case this is unclear, it follows a more precise explanation why $\delta_\varepsilon L \in \Omega^{k,0}\bigl(M \times \mathfrak{M}_{E}; K\bigr)$ is implied by construction: Usually one would define
\ba\label{OriginalFormula}
\mleft(\delta_\varepsilon L \mright)\mleft( X_1, \dotsc, X_k \mright)
&\coloneqq
\delta_{\Psi_\varepsilon}\bigl(
	\iota(L)\mleft(X_1, \dotsc, X_k\mright)
\bigr)
	- \sum_{i=1}^k L\mleft( X_1, \dotsc, \mathcal{L}_{\Psi_\varepsilon} X_i, \dotsc, X_k \mright)
\ea
for all $X_1, \dotsc, X_k \in \mathfrak{X}\bigl( M \times \mathfrak{M}_{E} \bigr)$. As usual, such a definition leads to $C^\infty\bigl( M \times \mathfrak{M}_{E} \bigr)$-multilinearity, and vector fields of $M \times \mathfrak{M}_{E}$ are generated by vector fields of $M$ and $\mathfrak{M}_{E}$, such that we can restrict ourselves to vector fields of $M$ and $\mathfrak{M}_{E}$. If for example $X_1 = \Psi \in \mathfrak{X}\mleft(\mathfrak{M}_{E} \mright) \subset \mathfrak{X}\bigl( M \times \mathfrak{M}_{E} \bigr)$, then
\bas
\iota(L)\mleft( \Psi, X_2, \dotsc, X_k \mright)
&=
0
\eas
and
\bas
\sum_{i=1}^k L\mleft( X_1, \dotsc, \mathcal{L}_{\Psi_\varepsilon} X_i, \dotsc, X_k \mright)
&=
L( \mathcal{L}_{\Psi_\varepsilon} \Psi, X_2 \dotsc, X_k )
	+ \sum_{i=2}^k \underbrace{L\mleft( \Psi, X_2, \dotsc, \mathcal{L}_{\Psi_\varepsilon} X_i, \dotsc, X_k\mright)}_{=0}
\\
&=
L\bigl( \underbrace{[\Psi_\varepsilon, \Psi]}_{ \mathclap{\in ~ \mathfrak{X}\mleft( \mathfrak{M}_{E} \mright)} }, X_2 \dotsc, X_k \bigr)
\\
&=
0,
\eas
using $L \in \Omega^{k,0}\bigl(M \times \mathfrak{M}_{E}; K\bigr)$ and $\Psi_\varepsilon \in \mathfrak{X}\bigl( \mathfrak{M}_{E} \bigr)$. On the other hand if $X_1 = Y \in \mathfrak{X}(M)$, then 
\bas
L( \mathcal{L}_{\Psi_\varepsilon} Y, X_2 \dotsc, X_k )
&=
L\bigl( \underbrace{[\Psi_\varepsilon, Y]}_{ \mathclap{= 0} }, X_2 \dotsc, X_k \bigr)
=
0.
\eas
Using these relations, we can conclude that the non-trivial information of Def.~\eqref{OriginalFormula} is encoded completely on $\mathfrak{X}(M)$ as a $C^\infty(M)$-module,\footnote{Observe that $\mathcal{L}_{\Psi_\varepsilon}(f) = 0$ for all $f \in C^\infty(M)$.} as given in Def.~\eqref{DefOfGaugeTrafoWithBookkeep}. Therefore one can use Def.~\eqref{DefOfGaugeTrafoWithBookkeep} instead and canonically/trivially extend this definition to the "full" form.

Alternatively, use the flows given by Cor.~\ref{cor:ReasonWhyVectorFieldAlongAnBPath} and prove it in the same manner as in Prop.~\ref{prop:ClassicFunctionDerivativesAlongPsiEpsilon} (in combination with Def.~\ref{def:InfinitesimalGaugeTrafoClassicAsConnection}).
%For a fixed point $(\Phi_0, A_0) \in \mathfrak{M}_E(M; N)$ define $\epsilon_0 \coloneqq \varepsilon(\Phi_0, A_0) \in \Gamma(\Phi_0^*B)$. As in Cor.~\ref{cor:ReasonWhyVectorFieldAlongAnBPath} (using Eq.~\eqref{GaugeTrafoVektor}) we denote the flow of $\Psi_\varepsilon$ through $(\Phi_0, A_0)$ by $\gamma: I \to \mathfrak{M}_E(M; N)$, $t \mapsto \gamma(t) \eqqcolon (\Phi_t, A_t) \in \mathfrak{M}_E(M; N)$, where $I$ is an open interval of $\mathbb{R}$ with $0 \in I$, and we write $\Psi_\varepsilon|_{\gamma(t)} = \mleft( - (\Phi_t^*\rho_B)(\epsilon_t), \mathcal{a}_t \mright) \in \mathrm{T}^B_{(\Phi_t, A_t)}\mathfrak{M}_E(M; N)$, where $\epsilon_t \in \Gamma(\Phi_t^*B)$ satisfies $\epsilon_{t=0} = \epsilon_0$ and $-\epsilon(p) \coloneqq [t \mapsto -\epsilon_t|_p]$ is a $B$-path with base path $\Phi(p) \coloneqq [t \mapsto \Phi_t(p)]$ for all $p\in M$. Then we define $\delta_{\Psi_\varepsilon} L$ for $L \in \mathcal{F}^k_E(M; {}^*V)$ by
%\ba\label{DefVonUnsererCoolenEichung}
%\mleft( \mleft( \delta_{\Psi_\varepsilon} L\mright)(\Phi_0, A_0) \mright)_p(Y_1,\dotsc, Y_k)
%&\coloneqq
%\mleft.\frac{\mathrm{D}_{\mleft( -\epsilon(p), {}^B\nabla \mright)}}{\mathrm{d}t} \mright|_{t=0} \mleft[ t \mapsto
%\bigl( L[\Phi_t][A_t] \bigr)_p(Y_1, \dotsc, Y_k)
%\mright]
%\ea
%for all $(\Phi_0, A_0) \in \mathfrak{M}_E(M; N)$, $p \in M$ and $Y_1, \dotsc, Y_p \in \mathrm{T}_p M$. For simplicity, we will just write $\mathrm{D}/\mathrm{d}t$ in the following proof, and we omit the notation of $[t \mapsto \dotsc]$. By definition we have $\mathbb{R}$-linearity and the commutation with the contraction $\iota$, by Prop.~\ref{prop:DerivationAlongEPath} and Cor.~\ref{cor:ReasonWhyVectorFieldAlongAnBPath} we also get 
%\bas
%\bigl( \mleft(\delta_{\Psi_\varepsilon}( {}^* \omega)\mright)(\Phi_0, A_0) \bigr)_p
%&=
%\mleft.\frac{\mathrm{D}}{\mathrm{d}t} \mright|_{t=0} \underbrace{
%\bigl( \Phi_t^*\omega \bigr)_p
%}_{= \mleft[ t\mapsto \mleft( (\Phi(p))^*\omega \mright)|_t \mright]}
%\\
%&=
%- \mleft.\Phi_0^* \mleft( {}^B\nabla_{\epsilon_0} \omega \mright) \mright|_p
%\\
%&=
%- \mleft( {}^*\mleft( {}^B\nabla_\varepsilon \omega \mright) \mright)(\Phi_0, A_0)
%\eas
%for all $\omega \in \Gamma(V)$, and, using the $\mathbb{R}$-linearity and the commutation with $\iota$,
%\bas
%&\mleft(\delta_{\Psi_\varepsilon}(f \wedge L) \mright)(Y_1, \dotsc, Y_{m+k})
%\\
%&=
%\delta_{\Psi_\varepsilon} \mleft(
%\sum_{\sigma \in S_{m+k}}
%\frac{1}{m! k!}~
%f\mleft(Y_{\sigma(1)}, \dotsc, Y_{\sigma(m)}\mright) ~ L\mleft( Y_{\sigma(m+1)}, \dotsc Y_{\sigma(m+k)} \mright)
%\mright)
%\\
%&=
%\sum_{\sigma \in S_{m+k}} \frac{1}{m! k!}
%\Biggl(\underbrace{\mleft.\frac{\mathrm{d}}{\mathrm{d}t} \mright|_{t=0} \biggl( \bigl(f[\Phi_t][A_t]\bigr)\mleft(Y_{\sigma(1)}, \dotsc, Y_{\sigma(m)}\mright) \biggr)}_{= \mleft(\mathcal{L}_{\Psi_\varepsilon} \mleft(f\mleft(Y_{\sigma(1)}, \dotsc, Y_{\sigma(m)}\mright) \mright)\mright)(\Phi_0, A_0)} ~ L(\Phi_0, A_0)\mleft( Y_{\sigma(m+1)}, \dotsc Y_{\sigma(m+k)} \mright)
%\\
%&\hspace{1cm}\hphantom{\sum_{\sigma \in S_{m+k}} \frac{1}{m! k!} \Biggl(}+
%\bigl( f(\Phi_0, A_0) \bigr)\mleft(Y_{\sigma(1)}, \dotsc, Y_{\sigma(m)}\mright)
%~ \mleft.\frac{\mathrm{D}}{\mathrm{d}t} \mright|_{t=0} \biggl(  
	%L[\Phi_t][A_t]\mleft( Y_{\sigma(m+1)}, \dotsc Y_{\sigma(m+k)} \mright)
%\biggr)
%\\
%&=
%\bigl(\mleft( 
	%\mathcal{L}_{\Psi_\varepsilon}(f) \wedge L
	%+ f \wedge \delta_{\Psi_\varepsilon} L 
%\mright)(\Phi_0, A_0)\bigr)\mleft( Y_1, \dotsc, Y_{m+k} \mright)
%\eas
%for all $f \in \mathcal{F}_E^m(M)$ and $Y_1, \dotsc, Y_{m+k} \in \mathfrak{X}(M)$, where $S_{m+k}$ denotes the group of permutations $\{1,\dotsc, m+k\}$. 
%
%For the uniqueness observe the following: Due to the commutation with the contraction $\iota$, we can conclude the uniqueness for all $k \in \mathbb{N}_0$ when we know the uniqueness for $\mathcal{F}^0_E(M; {}^*V)$ ($k=0$), since then 
%\bas
%\mleft( \delta_{\Psi_\varepsilon} L \mright)\mleft( Y_1, \dotsc, Y_k \mright)
%&=
%\delta_{\Psi_\varepsilon} \underbrace{\mleft(  L\mleft( Y_1, \dotsc, Y_k \mright) \mright)}
%_{\in \mathcal{F}^0_E(M; {}^*V)}
%\eas
%for all $L \in \mathcal{F}^k_E(M; {}^*V)$ and $Y_1, \dotsc, Y_k \in \mathfrak{X}(M)$. The uniqueness on $\mathcal{F}^0_E(M; {}^*V)$ then comes from that this space is generated by pullback functionals, using the desired Leibniz rule of $\delta_{\Psi_\varepsilon}$ and $\delta_{\Psi_\varepsilon} \mleft( {}^* \omega \mright) = - {}^*\mleft({}^B\nabla_{\varepsilon} \omega \mright)$.
\end{proof}

\begin{remark}\label{RemLeibnizeRegelaufProdukteWeshalbEConnectionNichtWichtigIst}
\leavevmode\newline
\indent $\bullet$ Given by Remark \ref{JustLieDerivativeForGeneralPullbackAndlineBundle}, for $V = N \times \mathbb{R}$ we always take the canonical flat $B$-connection, \textit{i.e.}~the canonical flat vector bundle connection $\nabla^0 = \mathrm{d}$ and then ${}^B\nabla \coloneqq \nabla^0_{\rho_B}$ such that
\bas
\delta_{\Psi_\varepsilon}
&=
\mathcal{L}_{\Psi_\varepsilon}.
\eas
Thus, 
\ba
\delta_{\Psi_\varepsilon} \mathrm{d}
&=
\mathcal{L}_{\Psi_\varepsilon} \mathrm{d}
=
\mathrm{d} \mathcal{L}_{\Psi_\varepsilon}
=
\mathrm{d} \delta_{\Psi_\varepsilon}, \label{eqVariationVertauschtMitDifferential}
\ea
since coordinates on $\mathfrak{M}_E(M; N)$ and $M$ are independent; recall the end of Remark \ref{rem:Bigrading} for this. The Leibniz rule for $\delta_{\Psi_\varepsilon}$ can be then rewritten to
\ba
\delta_{\Psi_\varepsilon}(f \wedge L)
&=
\delta_{\Psi_\varepsilon}(f) \wedge L
	+ f \wedge \delta_{\Psi_\varepsilon} (L).
\ea

$\bullet$ For dual bundles $V^*$ we canonically take the dual connection to ${}^B\nabla$ in order to have Leibniz rules as usual. That also means the following (still keeping the same notation): Let $L \in \mathcal{F}^k_E(M; {}^*V)$ and $T \in \mathcal{F}^0_E(M; {}^*(V^*))$, then in a frame $\mleft( e_a \mright)_a$ of $V$ and $\mleft( f^a \mright)_a$ of $V^*$, $f^b(e_a) = \delta^b_a$, we locally write $L = L^a \otimes {}^*e_a$ and $T = T_b \cdot {}^*f^b$, where $L^a \in \mathcal{F}^k_E(M)$ and $T_b \in \mathcal{F}^0_E(M)$. Then with these conventions, including the previous bullet point,
\ba
\delta_{\Psi_\varepsilon} (T(L))
&=
\delta_{\Psi_\varepsilon} \underbrace{\mleft(
	T_a L^a
\mright)}_{\in \mathcal{F}^k_E(M)}
=
\mathcal{L}_{\Psi_\varepsilon} \mleft(
	T_a L^a
\mright)
=
\mathcal{L}_{\Psi_\varepsilon} (T_a) ~ L^a
	+ T_a ~ \mathcal{L}_{\Psi_\varepsilon}(L^a),
\ea
hence, one achieves an independence of the chosen ${}^B\nabla$. This emphasizes what we expect, that we can freely choose the chosen connections for the variations of the tensors involved in contractions, only the variations of their components matter in such situations; this is important for the gauge invariance of the Lagrangian later. As we have discussed at the end of Section \ref{NewInfGaugeTrafoTrafos}, we are going to take the basic connection to define $\delta_{\Psi_\varepsilon}$ for quantities like the field strength, which will not be related to the canonical flat connection when imposing the classical theory; also recall Thm.~\ref{thm:RecoverOfClassicInfgGaugeTrafo}. That is possible because the infinitesimal gauge transformation of the Lagrangian stays untouched by this, it is always just the Lie derivative. The connections only get important in explicit calculations when applying the Leibniz rule as in
\bas
\delta_{\Psi_\varepsilon} (T(L))
&=
\mleft(\delta_{\Psi_\varepsilon} T\mright)(L)
	+ T\mleft(\delta_{\Psi_\varepsilon} L\mright),
\eas
but the result will of course not change. Henceforth, the essential work is in defining $\Psi_\varepsilon$; we did not yet define the infinitesimal gauge transformation of $A$.
%
%$\bullet$ Let us shortly motivate what one does in the standard setting. In the standard setting $V$ will be a trivial vector bundle over $N$. Especially looking at Eq.~\eqref{DefVonUnsererCoolenEichung}, we want that the used $\mathrm{D}/\mathrm{d}t$ is the typical $\mathrm{d}/\mathrm{d}t$, that is, we then have $B = \mathrm{T}N$ and ${}^B\nabla$ is the canonical flat connection of $V$. Fix a global parallel (=constant) frame of $V$, $\mleft(e_a\mright)_a$, \textit{i.e.}~$\nabla e_a = 0$ such that 
%\bas
%(\delta_{\Psi_\varepsilon} {}^*e_a)(\Phi, A)
%&= 
%(\Phi^*\nabla)_{\varepsilon(\Phi, A)} (\Phi^*e_a) 
%= 
%\Phi^!\mleft(\nabla_{\varepsilon(\Phi, A)} e_a\mright) 
%= 0
%\eas
%for all $(\Phi, A ) \in \mathfrak{M}_E(M; N)$. We then get for all $L = L^a \otimes e_a \in \mathcal{F}^k_E(M; {}^*V)$ ($k \in \mathbb{N}_0$)
%\bas
%\delta_{\Psi_\varepsilon} L
%&=
%\delta_{\Psi_\varepsilon} L^a \otimes {}^*e_a,
%\eas
%thus, it is equivalent to a variation in each component with respect to a global parallel frame of the underlying connection (the pullback of $\nabla$), which is precisely what one normally does in the standard framework of gauge theory; see more about this later when we actually introduced more about the physical setting.
\end{remark}

This recovers the classical idea of infinitesimal gauge transformation, \textit{i.e.}~it is a Lie derivative of components with respect to flat connections; also recall Thm.~\ref{thm:RecoverOfClassicInfgGaugeTrafo}.

\begin{theorems}{Parametrised variations in the flat case}{NewFormulaRecoversOldGaugeTrafoYay}
Let $M, N$ be two smooth manifolds, $\mleft(E, \rho_E, \mleft[ \cdot,\cdot \mright]_E \mright)$, $\mleft(B, \rho_B, \mleft[ \cdot,\cdot \mright]_B \mright)$ two Lie algebroids over $N$, and $V \to N$ a trivial vector bundle. Also let $\nabla$ be the canonical flat connection of $V$, $\Psi_\varepsilon \in \mathfrak{X}^B\bigl( \mathfrak{M}_E(M;N) \bigr)$ for an $\varepsilon \in \mathcal{F}^0_E(M; {}^*B)$ and for $\mathcal{F}^\bullet_E(M; {}^*V)$ let $\delta_{\Psi_\varepsilon}$ be the unique operator of Prop.~\ref{prop:VariationVonSkalarZeugsEasyPeasy}, using ${}^B\nabla \coloneqq \nabla_{\rho_B}$ as a $B$-connection on $V$.

Then we have
\ba
\delta_{\Psi_\varepsilon} L
&=
\mleft(\mathcal{L}_{\Psi_\varepsilon}L^a\mright) \otimes {}^*e_a
\ea
for all $L \in \mathcal{F}^\bullet_E(M; {}^*V)$, where $\mleft( e_a \mright)_a$ is a global constant frame of $V$.
\end{theorems}

\begin{proof}
\leavevmode\newline
That is basically the same proof as in Thm.~\ref{thm:RecoverOfClassicInfgGaugeTrafo}. Take a global constant frame $\mleft( e_a \mright)_a$ of $V$, then
\bas
\nabla e_a &= 0,
\eas
and therefore
\bas
(\Phi^*\nabla)(\Phi^*e_a)
&=
\Phi^!(\nabla e_a)
=
0
\eas
for all $\Phi \in C^\infty(M;N)$. Hence, $({}^*\nabla)({}^*e_a) = {}^!(\nabla e_a) = 0$, such that, using the Leibniz rule,
\bas
\delta_{\Psi_\varepsilon} L
&=
\mleft(\mathcal{L}_{\Psi_\varepsilon}L^a\mright) \otimes {}^*e_a.
\eas
\end{proof}

As argued before, we can write $\Psi_\varepsilon = \mleft( -({}^*\rho_B )(\varepsilon), \mathfrak{a} \mright)$ (Eq.~\eqref{GaugeTrafoVektor}) and we want to identify its first and second component as the gauge transformation of $\Phi$ and $A$, respectively. Right now $\mathfrak{a}$ is just fixed by Prop.~\ref{prop:TangentSpaceOfSpaceOfFields} such that it is very arbitrary; as in the standard setting of gauge theory, we want that it is parametrised, which will be by $\varepsilon$, too.

\subsection{Infinitesimal gauge transformation of the field of gauge bosons}
%
%By Prop.~\ref{prop:VariationVonSkalarZeugsEasyPeasy} we have a candidate of an operator on the functional spaces $\mathcal{F}^k_E$ for describing infinitesimal gauge transformations which satisfies the Leibniz rule and which is induced by a connection on some vector bundle, the latter results into that this operator is closed, \textit{i.e.}~a map $\mathcal{F}^k_E \to \mathcal{F}^k_E$. We fixed that operator by fixing a certain vector field $\Psi_\varepsilon$ of $\mathfrak{M}_E(M; N)$ whose components shall describe the infinitesimal gauge transformations of the fields; we already shaped one of the components which will give us the transformation of $\Phi$. For this recall Eq.~\eqref{GaugeTrafoVektor}.
%
%We now want to specify the variations as defined in Def.~\ref{def:VariationsInVariationalCalc}, in order to formulate the infinitesimal gauge transformations. Since we want to study what happens when we apply the gauge transformation twice, we need to be a bit careful about how we formulate this: In the standard theory infinitesimal gauge transformations are parametrised by $\epsilon \in C^\infty(M; \mathfrak{g})$ for a Lie algebra $\mathfrak{g}$, which can be seen as an element of $\Gamma(M \times \mathfrak{g})$. For example, in the abelian case the infinitesimal gauge transformation of a field of gauge bosons is defined by $\delta_\epsilon A = - \mathrm{d}^\nabla \epsilon$, where $\nabla$ is the canonical flat connection of $M \times\mathfrak{g}$ and $A \in \Omega^1(M; \mathfrak{g})$. As we already discussed, $M \times \mathfrak{g}$ is now replaced by $\Phi^*E$ for a given $\Phi$. For this recall that we (are going to) identify the standard theory with a setting where $E = N \times \mathfrak{g}$ is an action Lie algebroid. Then any (constant) global frame $\mleft( e_a \mright)_a$ of $E$ trivializes $\Phi^*E=M\times\mathfrak{g}$, too, using $\mleft( \Phi^*e_a \mright)_a$ as trivialization. Therefore it sounds natural to define infinitesimal gauge transformations by using $\epsilon \in \Gamma(\Phi^*E)$ now. But $\Phi$ will be a field which is in general affected by the gauge transformation, such that $\Phi^*E$ is not fixed with respect to the underlying calculus of variations, hence, we will view the parameter of the infinitesimal gauge transformations itself as a functional in $\Phi$ and $A$ in the following definition, \textit{i.e.}~we are going to use elements of $\mathcal{F}^0_E(M;{}^*E)$ instead for the parametrisation. 
Recall Prop.~\ref{prop:VerticalBundleOfFracM} and its discussion, the tangent vector along the "$A$-direction" is only in the same space as $A$ if the first component is zero, which is $\delta_\varepsilon \Phi$ because we want to think of $\delta_\varepsilon A$ as the second component of $\Psi_\varepsilon$. We cannot expect this to be zero in general, not even in the standard setting because a Lie algebra representation will not act trivially on $\Phi$, as we already discussed after Prop.~\ref{prop:VerticalBundleOfFracM}. However, as in the standard formulation, we want to formulate the gauge transformation of $A$ in such a way that it is somewhat in the same space; we will achieve this by fixing a connection on $E$ as we already did for functionals when defining $\delta_{\Psi_\varepsilon}$. Since $A$ has values in $\Phi^*E$, its image is also now affected by the gauge transformation of $\Phi$, this is why we can do something similar as for functionals; also recall Remark \ref{rem:BosonsAsFunctionalies}. 

One may argue that an involved horizontal projection in the definition for $\delta_\varepsilon A$ may lead to lost information about that object, especially important when one may want to integrate this theory, while we will not need the "full formula" for $\delta_\varepsilon A$ for the infinitesimal gauge transformation of the Lagrangian as we already argued earlier. However, since $A$ has values in $\Phi^*E$, one expects that $\delta_\varepsilon A$ encodes partially what $\delta_\varepsilon \Phi$ already encodes. Prop.~\ref{prop:TangentSpaceOfSpaceOfFields} shows us that $\delta_\varepsilon A$ is still somewhat vertical, because it is a form with values in the vector bundle $\mathrm{T}E \to \mathrm{T}N$, whose linear structure is essentially given by the vertical (prolonged) structure; $\delta_\varepsilon A$ is just shifted "horizontally" by $\delta_\varepsilon \Phi$ due to Eq.~\eqref{HorizontalCompOfDeltaA} and Prop.~\ref{prop:VerticalBundleOfFracM}. Henceforth, our idea is to shape the horizontal projection in such a way that we only "loose" the information we already know by $\delta_\varepsilon \Phi$; making use of Prop.~\ref{prop:TangentSpaceOfSpaceOfFields}.

Let us make it precise: Let us first look at a local trivialization of the Lie algebroid $E \stackrel{\pi}{\to} N$ is trivial. That is let us have base coordinates $\mleft( x^i \mright)_i$ of $N$, lifted to $E$ by $\pi^*x^i$, but we will omit all the given pullbacks in the notation now in the following rough discussion for simplicity; also let $\mleft( y^j \mright)_j$ be fibre coordinates. By Prop.~\ref{prop:TangentSpaceOfSpaceOfFields}, $\delta_\varepsilon A$ should be, for a given $(\Phi, A) \in \mathfrak{M}_E$, a form on $M$ with values in $\mathrm{T}E$ (along some function; but again, we omit the pullbacks and point evaluations for simplicity now). Hence, we expect
\bas
\delta_\varepsilon A
&=
\mleft(\delta_\varepsilon A\mright)^i ~ \frac{\partial}{\partial x^i}
	+ \mleft(\delta_\varepsilon A\mright)^j ~ \frac{\partial}{\partial y^j},
\eas
and $\delta_\varepsilon A$ is the second component of $\Psi_\varepsilon = (\delta_\varepsilon \Phi, \delta_\varepsilon A)$, which we used to define $\delta_{\Psi_\varepsilon}$. Again by Prop.~\ref{prop:TangentSpaceOfSpaceOfFields}, also recall Remark \ref{RemarkAboutThatWeStillHaveLinearStructureinDeltaA}, we know that
\bas
\delta_\varepsilon \Phi
&=
\mathrm{D}\pi \bigl((\delta_\varepsilon A)(Y)\bigr)
=
\mleft(\delta_\varepsilon A\mright)^i(Y) ~ \frac{\partial}{\partial x^i}
\eas
for all $Y \in \mathfrak{X}(M)$, where we used that $\partial/\partial y^j$ are vertical vector fields. Given that trivialization, $\partial/\partial x^i$ defines a canonical horizontal distribution. Hence, using that distribution for a horizontal projection, one could define the infinitesimal gauge transformation of $A$ in that trivialization just with $\mleft(\delta_\varepsilon A\mright)^j ~ \frac{\partial}{\partial y^j}$ which can be identified with a form with values in $E$ since $\partial/\partial y^j$ are vertical. While the components we "loose" because of the horizontal projection is something already encoded by $\delta_\varepsilon \Phi$, such that those are easy to reconstruct if one needs the "full formula" of $\delta_\varepsilon A$.

Globally that means we want to define $\delta_\varepsilon A$ as a form with values in $E$ using a Lie algebroid connection on $E$ as we did in Prop.~\ref{prop:VariationVonSkalarZeugsEasyPeasy} in such a way that $\Psi_\varepsilon$ is uniquely given. In order to do that we need to view $A$ as a functional, which is just $\varpi_2$ of Ex.~\ref{ex:ProjectionOntoGaugeBosonies}. So, we impose a formula for $\delta_\varepsilon \varpi_2$ in such a way that it uniquely defines $\Psi_\varepsilon$, and that we can derive the infinitesimal gauge invariance of the Lagrangian as usual.

But how does one fix the infinitesimal gauge transformation of $A$ normally when integrability is not used? One of the arguments in the standard formulation is given by looking at the transformation of the minimal coupling; we will do the same. Let us recall what that argument was: Again, let $N =W$ be a vector space, and $E = N \times \mathfrak{g}$ an action Lie algebroid associated to a Lie algebra $\mathfrak{g}$ whose Lie algebra action is induced by a Lie algebra representation $\psi: \mathfrak{g} \to \mathrm{End}(W)$. Then, for an $\epsilon \in C^\infty(M; \mathfrak{g})$, we have the infinitesimal gauge transformation $\delta_\epsilon \Phi = \psi(\epsilon)(\Phi)$ for all $\Phi \in C^\infty(M;W)$. The minimal coupling is then defined by $\mathfrak{D}^A \Phi = \mathrm{D}\Phi + \psi(A)(\Phi)$, where $A \in \Omega^1(M; \mathfrak{g})$; recall Def.~\ref{def:ClassicMinimalCoupling}. The (infinitesimal) gauge transformation of $A$ is then chosen in such a way that it is an element of $\Omega^1(M; \mathfrak{g})$, and such that one gets for the infinitesimal gauge transformation of the minimal coupling
\ba\label{StandardArgumenFuerDieMinimaleKopplungImBabyFall}
\mleft(\delta_\epsilon \mathfrak{D} \mright)(\Phi, A)
= 
\psi(\epsilon) \mleft( \mathfrak{D}^A \Phi \mright)
\ea
among the category of gauge theories, where $\delta_\epsilon$ denotes again the classical formulation of the infinitesimal gauge transformation as introduced in Chapter \ref{ClassicGaugeTheory}.

In order to provide a similar argument and since the minimal coupling $\mathfrak{D}$ is an element of $\mathcal{F}^1_E(M; {}^*\mathrm{T}N)$, we need to fix a connection on $\mathrm{T}N$ in order to use Prop.~\ref{prop:VariationVonSkalarZeugsEasyPeasy}. We want to use the basic connection, for this recall that for a given connection $\nabla$ on a Lie algebroid $E\to N$ we have the canonical basic connection $\nabla^{\mathrm{bas}}$, Def.~\ref{def:CanonicalBasicConnection},
\bas
\nabla^{\mathrm{bas}}_\mu \nu
&=
\mleft[ \mu, \nu \mright]_E + \nabla_{\rho(\nu)} \mu, \\
\nabla^{\mathrm{bas}}_\mu X
&=
\mleft[ \rho(\mu), X \mright] + \rho \mleft( \nabla_{X} \mu \mright)
\eas
for all $\mu, \nu \in \Gamma(E)$ and $X\in \mathfrak{X}(N)$. The reason why we want to use the basic connection is the following corollary about the recovery of Eq.~\eqref{StandardArgumenFuerDieMinimaleKopplungImBabyFall}.

\begin{corollaries}{Gauge transformation of the minimal coupling in the standard framework}{EichtrafovonDAPHIinClassicIstBabyEinfach}
Let $N=W$ be a vector space, $E = N \times \mathfrak{g}$ be an action Lie algebroid of a Lie algebra $\mathfrak{g}$ whose action is induced by a Lie algebra representation $\psi: \mathfrak{g} \to \mathrm{End}(W)$, $E$ is also equipped with its canonical flat connection $\nabla$. Also let $\Psi_\varepsilon \in \mathfrak{X}^E(\mathfrak{M}_E(M; N))$ for an $\varepsilon \in \mathcal{F}^0_E(M; {}^*E)$ and for the functional space $\mathcal{F}^\bullet_E(M; {}^*\mathrm{T}N)$ let $\delta_{\Psi_\varepsilon}$ be the unique operator of Prop.~\ref{prop:VariationVonSkalarZeugsEasyPeasy}, using $\nabla^{\mathrm{bas}}$ as $E$-connection on $\mathrm{T}N$. Then we have
\ba\label{EichtrafovonDAPHIinClassicIstBabyEinfachDieAequivalenz}
\bigl(\delta_{\Psi_\varepsilon} \mathfrak{D}\bigr)(\Phi, A)
&=
0
&\Leftrightarrow&&
\bigl(\delta_{\Psi_\varepsilon} \mathfrak{D}^\alpha \bigr)(\Phi, A)
&=
\mleft( \psi\bigl(\varepsilon(\Phi, A)\bigr) \mleft( \mathfrak{D}^A \Phi \mright) \mright)^\alpha
\ea
for all $(\Phi, A) \in \mathfrak{M}_E(M; N)$ and $\alpha \in \{1, \dotsc, \mathrm{dim}(W)\}$,
where the components are with respect to global coordinate vector fields $\mleft( \partial_\alpha \mright)_\alpha$ on $W$, and where we used the canonical trivializations $\mathrm{T}W \cong W\times W$ and $\Phi^*\mathrm{T}W \cong M \times W$ such that $\mathfrak{D}^A \Phi$ can be viewed as an element of $\Omega^1(M; W)$.
\end{corollaries}

\begin{proof}
\leavevmode\newline
Let $\mleft( e_a \mright)_a$ be a global and constant frame of $E$ and $\partial_\alpha$ coordinate vector fields on $N$, then we can write $\mathfrak{D} = \mathfrak{D}^\alpha \otimes {}^*\partial_\alpha$, and, thus, by the Leibniz rule and with $\epsilon \coloneqq \varepsilon(\Phi, A)$
\ba
\bigl(\delta_{\Psi_\varepsilon} \mathfrak{D}^\alpha\bigr)(\Phi, A)
	- \bigl( \underbrace{\mleft( \delta_{\Psi_\varepsilon} \mathfrak{D}\mright)}
	_{ \mathclap{ = \delta_{\Psi_\varepsilon} \mleft(\mathfrak{D}^\alpha\mright) \otimes {}^*\partial_\alpha
		+ \mathfrak{D}^\alpha \otimes \delta_{\Psi_\varepsilon} \mleft({}^*\partial_\alpha\mright) } }
	(\Phi, A) \bigr)^\alpha
&=
	- \biggl( \mleft( \mathfrak{D}^A \Phi \mright)^\beta \otimes \underbrace{\mleft(\delta_{\Psi_\varepsilon} \mleft( {}^* \partial_\beta \mright)\mright) (\Phi, A)}_{\mathclap{\stackrel{\text{Prop.~\ref{prop:VariationVonSkalarZeugsEasyPeasy}}}{=} - \Phi^* \mleft( \nabla^{\mathrm{bas}}_\epsilon \partial_\beta\mright)}} \biggr)^\alpha
\nonumber \\
&=
\epsilon^a ~ \Phi^*\mleft( 
	- \partial_\beta\rho_a^\alpha
	+ \rho^\alpha\mleft( \nabla_{\partial_\beta} e_a \mright) 
\mright) ~ \mleft( \mathfrak{D}^A \Phi \mright)^\beta \label{CompsVonDMinimalAlsErstes}
\ea
for all $\alpha$.
Let us write $\partial_\alpha = \partial/\partial w^\alpha$ for some coordinates $\mleft( w^\alpha \mright)_\alpha$ on $W$. Then by Prop.~\ref{prop:LieRepAndLieAct},
\ba\label{eqAbleitungVomAnkerGibtRepraesentierung}
- \partial_\beta\bigl[ w \mapsto \rho_a^\alpha(w) \bigr]
&=
- \partial_\beta\bigl[ w \mapsto \gamma_a^\alpha(w) \bigr]
=
\partial_\beta\mleft[ w \mapsto \bigl(\psi(e_a)(w)\bigr)^\alpha \mright]
=
\bigl( \psi(e_a) \bigr)^\alpha_\beta
\ea
for $w \in W$, because the differential is then just the differential of a matrix vector-product $W \ni w \mapsto \psi(e_a)(w)$. Since $\nabla$ is the canonical flat connection, constant sections are parallel, thus, we get in total
\bas
\mleft(\delta_{\Psi_\varepsilon} \mathfrak{D}^\alpha\mright)(\Phi, A)
	- \bigl( \mleft( \delta_{\Psi_\varepsilon} \mathfrak{D}\mright) (\Phi, A) \bigr)^\alpha
&=
\epsilon^a ~ \Phi^*\underbrace{\bigl( \psi(e_a) \bigr)^\alpha_\beta}_{\mathclap{\text{const.}}} ~ \mleft( \mathfrak{D}^A \Phi \mright)^\beta
=
\mleft( \psi(\epsilon) \mleft( \mathfrak{D}^A \Phi \mright) \mright)^\alpha
\eas
for all $\alpha$, having $\epsilon \in C^\infty(M; \mathfrak{g})$ and $\mathfrak{D}^A \Phi \in \Omega^1(M;W)$. That shows that we have 
\bas
\mleft(\delta_{\Psi_\varepsilon} \mathfrak{D}^\alpha\mright)(\Phi, A)
&=
\mleft( \psi(\epsilon) \mleft( \mathfrak{D}^A \Phi \mright) \mright)^\alpha
\eas
if and only if 
\bas
\delta_{\Psi_\varepsilon} \mathfrak{D}
&=
0.
\eas
\end{proof}

The right equation in the Equivalence \eqref{EichtrafovonDAPHIinClassicIstBabyEinfachDieAequivalenz} describes precisely the components of the expected infinitesimal gauge transformation of the minimal coupling in the standard formulation of gauge theory, and it is no coincidence that this is equivalent to $\delta_{\Psi_\varepsilon} \mathfrak{D} = 0$ when using the basic connection. 
%For the gauge invariance of the Lagrangian we just need to know the infinitesimal gauge transformation of the components due to the contractions in the definition of the Lagrangian, although the total formula looks different than the classical formula when restricting to the standard formulation.

\begin{lemmata}{Metric compatibilities and their imposed symmetries for gauge theory, \cite{CurvedYMH}}{MetricCompsAdInvUndLieAlgebraRepSymm}
Let $N=W$ be a vector space, $E = N \times \mathfrak{g}$ be an action Lie algebroid of a Lie algebra $\mathfrak{g}$ whose action is induced by a Lie algebra representation $\psi: \mathfrak{g} \to \mathrm{End}(W)$, $E$ is also equipped with its canonical flat connection $\nabla$. Also let $\kappa$ be a fibre metric on $E$ which is a constantly extended scalar product $\widetilde{\kappa}$ of $\mathfrak{g}$; similarly, let $g$ be a fibre metric which is a constant extension of a scalar product $\widetilde{g}$ of $W$.

Then we have
\ba
\nabla^{\mathrm{bas}} \kappa = 0
&\Leftrightarrow
\text{$\widetilde{\kappa}$ is $\mathrm{ad}$-invariant},
\\
\nabla^{\mathrm{bas}} g = 0
&\Leftrightarrow
\text{$\widetilde{g}$ is $\psi$-invariant},
\ea
and $\nabla^{\mathrm{bas}}$ on $E$ and $\mathrm{T}N$ are the adjoint and $\psi$ representation, respectively, when restricted on constant sections, \textit{i.e.}
\ba
\nabla^{\mathrm{bas}}_\mu \nu &= \mleft[ \mu, \nu \mright]_{\mathfrak{g}},
\\
\nabla^{\mathrm{bas}}_\mu Y
&=
\psi(\mu)(Y)
\ea
for all constant $\mu, \nu \in \Gamma(E)$ and constant $Y \in \mathrm{T}N \cong W \times W$.
\end{lemmata}

\begin{remark}
\leavevmode\newline
Here we see that the basic connection $\nabla^{\mathrm{bas}}$ replaces the canonical representations arising in the standard formulation of gauge theory. Moreover, we will later see that we need $R_\nabla^{\mathrm{bas}}=0$ to formulate the gauge theory, that implies that $\nabla^{\mathrm{bas}}$ is flat (both), recall Prop.~\ref{prop:SnablamitREnabla}, such that it makes sense to think about it as a representation in the context of this thesis.
\end{remark}

\begin{proof}
\leavevmode\newline
Let $\mleft( e_a \mright)_a$ be a frame of constant sections. Then $\kappa\mleft(e_a,e_b\mright) = \text{const.}$, and hence
\bas
0
&=
\mathcal{L}_{e_a} \bigl( \kappa(e_b, e_c) \bigr).
\eas
We also have
\bas
\mleft[ e_a, e_b \mright]_{\mathfrak{g}}
&=
\mleft[ e_a, e_b \mright]_{E}
	+ \nabla_{\rho(e_b)} e_a
=
\nabla^{\mathrm{bas}}_{e_a} e_b, 
\eas
because $\nabla$ is the canonical flat connection.
Therefore
\bas
&&
\widetilde{\kappa} &\text{ is $\mathrm{ad}$-invariant}
\\
&\Leftrightarrow&
0
&=
\widetilde{\kappa}\mleft( \mleft[ e_a, e_b \mright]_{\mathfrak{g}}, e_c \mright)
	+ \widetilde{\kappa} \mleft( e_b, \mleft[ e_a, e_c \mright]_{\mathfrak{g}} \mright)
\\
&\Leftrightarrow&
\mathcal{L}_{e_a} \mleft( \kappa(e_b, e_c) \mright)
&=
\kappa\mleft( \mleft[ e_a, e_b \mright]_{\mathfrak{g}}, e_c \mright)
	+ \kappa \mleft( e_b, \mleft[ e_a, e_c \mright]_{\mathfrak{g}} \mright)
\\
&\Leftrightarrow&
\mathcal{L}_{e_a} \mleft( \kappa(e_b, e_c) \mright)
&=
\kappa\mleft( \nabla^{\mathrm{bas}}_{e_a} e_b, e_c \mright)
	+ \kappa \mleft( e_b, \nabla^{\mathrm{bas}}_{e_a} e_c \mright)
\\
&\Leftrightarrow&
\nabla^{\mathrm{bas}}\kappa 
&=
0.
\eas
For $g$ recall Eq.~\eqref{eqAbleitungVomAnkerGibtRepraesentierung}, \textit{i.e.}
\bas
- \partial_\beta\rho_a^\alpha
&=
\bigl( \psi(e_a) \bigr)^\alpha_\beta,
\eas
where we use coordinate vector fields $\mleft( \partial_\alpha \mright)_\alpha$ on $N$ which also describes a constant frame for $\mathrm{T}W \cong W \times W$, and hence also, as before,
\bas
\bigl( \psi(e_a) \bigr)^\alpha_\beta
&=
\mleft[ \rho(e_a), \partial_\beta \mright]
=
\mleft[ \rho(e_a), \partial_\beta \mright]
	+ \rho \mleft( \nabla_{\partial_\beta} e_a \mright)
=
\nabla^{\mathrm{bas}}_{e_a} \partial_\beta,
\eas
and
\bas
0
&=
\mathcal{L}_{e_a} \bigl( g(\partial_\alpha, \partial_\beta) \bigr)
\eas
Thus,
\bas
&&
\widetilde{g} &\text{ is $\psi$-invariant}
\\
&\Leftrightarrow&
0
&=
\widetilde{g}\bigl( \psi(e_a)(\partial_\alpha), \partial_\beta \bigr)
	+ \widetilde{g} \bigl( \partial_\alpha, \psi(e_a) (\partial_\beta) \bigr)
\\
&\Leftrightarrow&
\mathcal{L}_{e_a} \bigl( g(\partial_\alpha, \partial_\beta) \bigr)
&=
g\bigl( \psi(e_a)(\partial_\alpha), \partial_\beta \bigr)
	+ g \bigl( \partial_\alpha, \psi(e_a) \mleft(\partial_\beta\mright) \bigr)
\\
&\Leftrightarrow&
\mathcal{L}_{e_a} \bigl( g(\partial_\alpha, \partial_\beta) \bigr)
&=
g \mleft( \nabla^{\mathrm{bas}}_{e_a} \partial_\beta, \partial_\beta \mright)
	+ g \mleft( \partial_\alpha, \nabla^{\mathrm{bas}}_{e_a} \partial_\beta \mright)
\\
&\Leftrightarrow&
\nabla^{\mathrm{bas}} g  
&=
0.
\eas
\end{proof}

Hence, when using the basic connection, we want that $\delta_{\Psi_\varepsilon} \mathfrak{D} = 0$ such that we can recover the classical formula in sense of Cor.~\ref{cor:EichtrafovonDAPHIinClassicIstBabyEinfach}. To study this and later results we need several auxiliary results, recall also Ex.~\ref{ex:ProjectionOntoGaugeBosonies}, \ref{ex:DAsFunctional} and \ref{ex:AnchorAsFunctional}.

\begin{lemmata}{Several identities related to variations with the basic connection}{VariationsIdentities}
Let $M, N$ be two smooth manifolds, $E \to N$ a Lie algebroid over $N$, $\nabla$ a connection on $E$, and $\Psi_\varepsilon \in \mathfrak{X}^E(\mathfrak{M}_E(M; N))$ for an $\varepsilon \in \mathcal{F}^0_E(M; {}^*E)$. For both functional spaces, $\mathcal{F}^\bullet_E(M; {}^*E)$ and $\mathcal{F}^\bullet_E(M; {}^*\mathrm{T}N)$, let $\delta_{\Psi_\varepsilon}$ be the unique operator of Prop.~\ref{prop:VariationVonSkalarZeugsEasyPeasy}, using $\nabla^{\mathrm{bas}}$ as $E$-connection on $E$ and $\mathrm{T}N$, respectively. Then
\ba
%\delta_\varepsilon \mleft[ \phi \mapsto \phi^*L \mright][\Phi]
%&=
%- \Phi^*\mleft( \nabla^{\mathrm{bas}}_\varepsilon L \mright), \label{PullBackVariation}
%\\
%\delta_\varepsilon \mleft[\phi \mapsto \phi^* \mu \mright][\Phi]
%&=
%- \Phi^* \mleft( {}^E\nabla_\varepsilon \mu \mright), \label{PullBackVariationE}
%\\
%\delta_\varepsilon \mleft[\phi \mapsto \phi^* Y \mright][\Phi]
%&=
%- \Phi^* \mleft( \nabla^{\mathrm{bas}}_\varepsilon Y \mright), \label{PullBackVariationTN}
%\\
%\delta_\varepsilon \mleft[\phi \mapsto \phi^* Y \mright][\Phi]
%&=
%- \Phi^* \mleft( \nabla^{\mathrm{bas}}_\varepsilon Y \mright), \label{PullBackVariationESternchen}
%\\
\delta_{\Psi_\varepsilon} \mathrm{D}
&=
- \mleft( {}^*\rho \mright) \bigl( {}^*\nabla \varepsilon \bigr), \label{DPhiVariation}
\\
\delta_{\Psi_\varepsilon} \mleft({}^*\rho\mright)
&=
0, \label{eqPhiRhoDieGeileSauIstnichtVariiert}
\\
\delta_{\Psi_\varepsilon} \bigl( ({}^*\rho)(\varpi_2) \bigr)
&=
\mleft( {}^* \rho \mright) \bigl( \delta_{\Psi_\varepsilon} \varpi_2 \bigr),
\label{eqRhoAVariation}
\\
\delta_{\Psi_\varepsilon} \mleft( {}^!\mleft(\nabla \mu \mright) \mright)
&=
- \biggl(
	{}^!\mleft(\nabla^{\mathrm{bas}}_\varepsilon \nabla \mu \mright)
	+ {}^*\mleft( \nabla_{({}^*\rho)\mleft( ({}^*\nabla) \varepsilon \mright)} \mu \mright)
\biggr) \label{EqVariationVonFormenBrrrr}
\ea
for all $\mu \in \Gamma(E)$, where we view $\nabla \mu$ as an element of $\Omega^1(N; E)$.
%\bas
%\delta_{\Psi_\varepsilon} \mleft( {}^!\mleft(\nabla \mu \mright) \mright)(\Phi, A)
%&=
%- \biggl(
	%\Phi^!\mleft(\nabla^{\mathrm{bas}}_{\varepsilon(\Phi, A)} \mleft( \nabla \mu \mright)\mright)
	%+ \Phi^*\mleft( \nabla_{(\Phi^*\rho)\mleft( (\Phi^*\nabla) \mleft(\varepsilon(\Phi, A)\mright) \mright)} \mu \mright)
%\biggr).
%\eas
%${}^E\nabla$ is either $\nabla_\rho$ for $j=1$ and $\nabla^{\mathrm{bas}}$ for $j=2$.
\end{lemmata}

\begin{remark}
\leavevmode\newline
We already introduced the notation for Eq.~\eqref{EqVariationVonFormenBrrrr} (also recall Remark \ref{RemarkNotationvonPullbackConnection}), but let us shortly write down what it is for each $(\Phi, A) \in \mathfrak{M}_E(M; N)$,
\bas
\mleft(\delta_{\Psi_\varepsilon} \mleft( {}^!\mleft(\nabla \mu \mright) \mright)\mright)(\Phi, A)
&=
- \biggl(
	\Phi^!\mleft(\nabla^{\mathrm{bas}}_{\epsilon} \mleft( \nabla \mu \mright)\mright)
	+ \Phi^*\mleft( \nabla_{(\Phi^*\rho)\mleft( (\Phi^*\nabla) \epsilon \mright)} \mu \mright)
\biggr)
\eas
where $\epsilon \coloneqq \varepsilon(\Phi, A)$. When $\varepsilon = {}^*\nu$ for a $\nu \in \Gamma(E)$, then $(\Phi^*\nabla ) (\Phi^*\nu) \stackrel{\text{Eq.~\eqref{EqGeilePullBackCommuteFormel}}}{=} \Phi^!(\nabla \nu)$, so, $({}^*\nabla) ({}^*\nu) = {}^!(\nabla \nu)$. Thus, we can then write
\ba\label{EqVariationVonFormenBrrrrVereinfacht}
\delta_{\Psi_{{}^*\nu}} \mleft( {}^!\mleft(\nabla \mu \mright) \mright)
&=
- {}^!\mleft(
	\nabla^{\mathrm{bas}}_\nu \nabla \mu
	+ \nabla_{\rho(\nabla \nu)} \mu
\mright).
\ea
\end{remark}

\begin{proof}[Proof for Lemma \ref{lem:VariationsIdentities}]
\leavevmode\newline
In the following $\mleft( e_a \mright)_a$ denotes a local frame of $E$, and $\partial_\alpha$ are local coordinate vector fields on $N$, and $(\Phi, A) \in \mathfrak{M}_E(M; N)$. Regarding $\varepsilon \in \mathcal{F}^0_E(M; {}^*E)$ we also write $\epsilon \coloneqq \varepsilon(\Phi, A)$.

$\bullet$ For Eq.~\eqref{DPhiVariation} we write locally
\bas
\mathrm{D} \Phi
&=
\mathrm{d} \Phi^\alpha \otimes \Phi^* \partial_\alpha,
\eas
where we view $(\Phi,A) \mapsto \Phi^\alpha$ as an element of $\mathcal{F}^0_E(M)$ (on an open subset of $M$), such that by $\delta_\varepsilon \Phi = - ({}^*\rho) (\varepsilon)$, and by using $\mathrm{d} \delta_{\Psi_\varepsilon} = \delta_{\Psi_\varepsilon} \mathrm{d}$ and $\delta_{\Psi_\varepsilon} = \mathcal{L}_{\Psi_\varepsilon}$ on $\mathcal{F}^0_E(M)$ (recall the discussion around Eq.~\eqref{eqVariationVertauschtMitDifferential}),
\bas
\mleft(\delta_\varepsilon \mathrm{d} \mleft[ (\Phi, A) \mapsto \Phi^\alpha \mright]\mright)(\Phi, A)
&=
\mleft(\mathrm{d} \mathcal{L}_{\Psi_\varepsilon} \mleft[ (\Phi, A) \mapsto \Phi^\alpha \mright]\mright)(\Phi, A)
=
- \mathrm{d} \mleft( \mleft( \rho^\alpha_a \circ \Phi \mright) ~ \epsilon^a \mright)
\eas
then by Eq.~\eqref{PullBackVariation} and the Leibniz rule of $\delta_{\Psi_\varepsilon}$
\bas
\mleft(\delta_{\Psi_\varepsilon} \mathrm{D} \mright)(\Phi, A)
&=
- \mathrm{d} \mleft( \mleft( \rho^\alpha_a \circ \Phi \mright) ~ \epsilon^a \mright) \otimes \Phi^* \partial_\alpha
	- \mathrm{d} \Phi^\alpha \otimes \epsilon^a~ \Phi^* \mleft( \nabla^{\mathrm{bas}}_{e_a}\partial_\alpha \mright) \\
&=
- \Bigl( \underbrace{\mathrm{d} \mleft( \rho^\alpha_a \circ \Phi \mright)}
	_{\mathclap{= ~ \mleft(\partial_\beta \rho^\alpha_a \circ \Phi \mright) ~ \mathrm{d} \Phi^\beta}}
 ~ \epsilon^a
	+ \mleft( \rho^\alpha_a \circ \Phi \mright) ~ \mathrm{d}\epsilon^a \Bigr) \otimes \Phi^* \partial_\alpha \\
&\hspace{1cm}
	- \mathrm{d} \Phi^\alpha \otimes \epsilon^a~ \Phi^* \mleft( 
	- \partial_\alpha \rho^\beta_a ~ \partial_\beta
	+ \rho \mleft( \nabla_{\partial_\alpha} e_a \mright)
	  \mright) \\
&=
- \mleft( \rho^\alpha_a \circ \Phi \mright) ~ \mathrm{d}\epsilon^a \otimes \Phi^* \partial_\alpha
	- \mathrm{d} \Phi^\beta \otimes \epsilon^b~ \mleft( \rho^\alpha_a \circ \Phi \mright) ~ \mleft(\omega_{b\beta}^a \circ \Phi \mright)
	~\Phi^*\partial_\alpha \\
&=
- \mleft( \rho^\alpha_a \circ \Phi \mright) \mleft(
\mathrm{d}\epsilon^a
	+ \epsilon^b~ \mleft(\omega_{b\beta}^a \circ \Phi \mright) ~\mathrm{d} \Phi^\beta
\mright) \otimes \Phi^*\partial_\alpha \\
&=
- \mleft( \Phi^*\rho \mright)\bigl( \mleft( \Phi^* \nabla\mright) \epsilon \bigr).
\eas

$\bullet$ By Eq.~\ref{PullBackVariation},
\bas
\delta_{\Psi_\varepsilon} \mleft( {}^*\rho \mright)
&=
- {}^*\mleft( \nabla^{\mathrm{bas}}_\varepsilon \rho \mright),
\eas
and by $\rho \circ \nabla^{\mathrm{bas}} = \nabla^{\mathrm{bas}} \circ \rho$ we get
\bas
\mleft( \nabla^{\mathrm{bas}} \rho \mright)(\mu)
&=
\nabla^{\mathrm{bas}}\mleft( \rho(\mu) \mright)
	- \rho\mleft( \nabla^{\mathrm{bas}}\mu \mright)
=
0
\eas
for all $\mu \in \Gamma(E)$. Hence,
\bas
\delta_{\Psi_\varepsilon} \mleft( {}^*\rho \mright)
&=
0.
\eas

$\bullet$ By the Leibniz rule and the previous result we also have
\bas
\delta_{\Psi_\varepsilon} \bigl( ({}^*\rho)(\varpi_2) \bigr)
&=
\mleft( {}^* \rho \mright) \bigl( \delta_{\Psi_\varepsilon} \varpi_2 \bigr).
%\delta_{\Psi_\varepsilon} \bigl[ (\phi, w) \mapsto \mleft( \phi^* \rho \mright)(w) \bigr](\Phi, A)
%&=
%\mleft(\delta_\varepsilon \mleft[ \phi \mapsto \phi^*\rho \mright][\Phi] \mright)(A)
	%+ \mleft(\Phi^*\rho\mright)\mleft( \delta_\varepsilon A \mright)
%\\
%&\stackrel{\mathclap{\text{Eq.~\eqref{eqPhiRhoDieGeileSauIstnichtVariiert}}}}{=}~~~~
%\mleft(\Phi^*\rho\mright)\mleft( \delta_\varepsilon A \mright)
%\\
%&\stackrel{\mathclap{\text{Def.~\ref{def:GaugeTrafoOfA}}}}{=}~~~~
%- \mleft(\Phi^*\rho\mright)\bigl( \mleft( \Phi^*\nabla \mright) \varepsilon \bigr).
\eas

$\bullet$ We view terms like $\nabla \mu$ as elements of $\Omega^1(N; E)$ for all $\mu \in \Gamma(E)$, $\mathfrak{X}(N) \ni Y \mapsto (\nabla \mu)(X) = \nabla_X \mu$, and therefore we can use the Leibniz rule on ${}^!(\nabla \mu) = \bigl({}^*(\nabla \mu)\bigr)(\mathrm{D}) = {}^*\mleft(\nabla_{\mathrm{D}} \mu\mright)$, \textit{i.e.}~due to
\bas
\Phi^!(\nabla \mu) &= \bigl(\Phi^*(\nabla \mu)\bigr)(\mathrm{D}\Phi)
\eas
we can view ${}^!(\nabla \mu)$ as a contraction of the functionals ${}^*(\nabla \mu)$ and $\mathrm{D}$. Hence,
\bas
\delta_{\Psi_\varepsilon} \mleft( {}^!\mleft(\nabla \mu \mright) \mright)
&=
\bigl( \delta_{\Psi_\varepsilon} ({}^*(\nabla \mu))\bigr)(\mathrm{D})
	+ {}^*\mleft(\nabla_{\delta_{\Psi_\varepsilon} \mathrm{D}} \mu\mright)
\\
&\stackrel{\mathclap{\text{Eq.~\eqref{PullBackVariation}}}}{=}~~~~
-\mleft({}^*\mleft(\nabla^{\mathrm{bas}}_\varepsilon \nabla \mu \mright)\mright)(\mathrm{D})
	+ {}^*\mleft(\nabla_{\delta_{\Psi_\varepsilon} \mathrm{D}} \mu\mright)
\\
&\stackrel{\mathclap{\text{Eq.~\eqref{DPhiVariation}}}}{=}~~~~
- \biggl(
	{}^!\mleft(\nabla^{\mathrm{bas}}_\varepsilon \nabla \mu \mright)
	+ {}^*\mleft( \nabla_{({}^*\rho)\mleft( ({}^*\nabla) \varepsilon \mright)} \mu \mright)
\biggr).
\eas
\end{proof}

Let us now fix the gauge transformation of $A$ using these results. Recall that we write $\Psi = \Psi_\varepsilon$ for a $\Psi \in \mathfrak{X}^E(\mathfrak{M}_E(M; N))$, where $\varepsilon \in \mathcal{F}^0_E(M;{}^*E)$ such that we can write (recall Eq.~\eqref{GaugeTrafoVektor})
\bas
\Psi_\varepsilon
&=
\mleft( -({}^*\rho_B )(\varepsilon), \mathfrak{a} \mright)
\eas
where $\mathfrak{a}$ is a map on $\mathfrak{M}_E(M; N)$ such that $\Psi|_{(\Phi,A)}$ is a tangent vector for all $(\Phi, A) \in \mathfrak{M}_E(M; N)$, \textit{i.e.}~satisfying the diagram of Prop.~\ref{prop:TangentSpaceOfSpaceOfFields} for all $(\Phi, A)$. For a given $\varepsilon$ such a $\Psi_\varepsilon$ is in general not unique. Recall that for a local frame $\mleft( e_a \mright)_a$ of $E$ and local coordinate functions $\mleft(\partial_\alpha\mright)_\alpha$ on $N$ we have
\bas
\mleft[ e_b, e_c \mright]_E
&=
C^a_{bc} e_a, &
\nabla e_b
&=
\omega^a_b \otimes e_a, &
\nabla_{\partial_\alpha} e_b
&=
\omega^a_{b\alpha} ~ e_a.
\eas

\begin{propositions}{Gauge transformation of the field of gauge bosons}{VariationOfA}
Let $M, N$ be two smooth manifolds, $E \to N$ a Lie algebroid over $N$, $\nabla$ a connection on $E$, $\varepsilon \in \mathcal{F}^0_E(M; {}^*E)$, and for the functional space $\mathcal{F}^\bullet_E(M; {}^*E)$ let $\delta_{\Psi_\varepsilon}$ be the unique operator of Prop.~\ref{prop:VariationVonSkalarZeugsEasyPeasy}, using $\nabla^{\mathrm{bas}}$ as $E$-connection on $E$ and any $\Psi_\varepsilon \in \mathfrak{X}^E\bigl( \mathfrak{M}_E(M;N) \bigr)$. Then there is a unique $\gls{1YPsiEpsilon} \in \mathfrak{X}^E\bigl(\mathfrak{M}_E(M; N)\bigr)$ such that
\ba\label{EichtrafoVonANochmal}
\delta_{\Psi_\varepsilon} \varpi_2
&=
- ({}^*\nabla) \varepsilon.
\ea
Locally with respect to a given frame $\mleft( e_a \mright)_a$
\ba
\mleft(\delta_{\Psi_\varepsilon} \varpi_2^a\mright)(\Phi, A)
&=
\mleft( C^a_{bc} \circ \Phi \mright) ~\epsilon^b A^c
	+ \mleft(\omega^a_{b\alpha} \circ \Phi \mright) ~ \mleft( \rho^\alpha_c \circ \Phi \mright)~\epsilon^b A^c
	- \mathrm{d}\epsilon^a - \epsilon^b ~ \Phi^!\mleft(\omega^a_{b} \mright)
\nonumber \\ \label{eqGaugeTrafoOfAacomps}
&=
\mleft( \epsilon^b A^c \otimes \Phi^*\mleft( \nabla^{\mathrm{bas}}_{e_b} e_c\mright)
	- \mleft(\Phi^*\nabla\mright)\epsilon \mright)^a
\ea
for all $(\Phi, A) \in \mathfrak{M}_E(M; N)$, where $\epsilon \coloneqq \varepsilon(\Phi, A)$.

Moreover, if we also have $\alpha, \beta \in \mathbb{R}$ and $\vartheta \in \mathcal{F}^0_E(M; {}^*E)$, then
\ba\label{LinearityOfPsiEpsilon}
\Psi_{\alpha \varepsilon + \beta \vartheta}
=
\alpha \Psi_\varepsilon + \beta \Psi_\vartheta,
\ea
where the vector fields are the ones uniquely given by Eq.~\eqref{EichtrafoVonANochmal}.
%\ba
%\mleft(\delta_{\Psi_\varepsilon} \mathrm{D}\mright)(\Phi, A)
%&=
%- \mleft( \Phi^* \rho \mright) \bigl( \mleft( \Phi^*\nabla \mright) \varepsilon \bigr), \label{DPhiVariation}
%\\
%\bigl(\delta_\varepsilon \mleft( \mathrm{D} - \mathfrak{D} \mright)\bigr)(\Phi, A)
%&=
%\mleft( \Phi^* \rho \mright) \bigl( \delta_{\Psi_\varepsilon} \varpi_2 \bigr)
%\label{eqRhoAVariation}
%\ea
\end{propositions}

\begin{proof}[Proof of Prop.~\ref{prop:VariationOfA}]
\leavevmode\newline
Since it is about a vector field on $\mathfrak{M}_E(M; N)$, we will classify $\Psi_\varepsilon$ by its flow, using Cor.~\ref{cor:ReasonWhyVectorFieldAlongAnBPath}: We denote its flow through a fixed point $(\Phi_0, A_0) \in \mathfrak{M}_E(M; N)$ by $\gamma: I \to \mathfrak{M}_E(M; N)$, $t \mapsto \gamma(t) \eqqcolon (\Phi_t, A_t) \in \mathfrak{M}_E(M; N)$, where $I$ is an open interval of $\mathbb{R}$ containing 0, and we write $\Psi|_{\gamma(t)} = \mleft( - (\Phi_t^*\rho)(\epsilon_t), \mathcal{a}_t \mright) \in \mathrm{T}^E_{(\Phi_t, A_t)}\mathfrak{M}_E(M; N)$, where $\epsilon_t \coloneqq \varepsilon(\Phi_t, A_t)\in \Gamma(\Phi^*_tE)$, and $\mathcal{a}_t$ is a morphism $\mathrm{T}M \to \mathrm{T}E$ satisfying the diagram in Prop.~\ref{prop:TangentSpaceOfSpaceOfFields}. So, we have a curve $\gamma$ with 
\bas
\gamma(0) &= (\Phi_0, A_0), \\
\frac{\mathrm{d}}{\mathrm{d}t} \gamma
&=
\Psi|_{\gamma(t)}
=
\mleft( - (\Phi_t^*\rho)(\epsilon_t), \mathcal{a}_t \mright).
\eas
$(\Phi_0, A_0)$ and $- (\Phi_t^*\rho)(\epsilon_t)$ are fixed, and we show that Eq.~\eqref{EichtrafoVonANochmal} will fix $\mathcal{a}_t$. Without loss of generality let us assume that everything is small and local enough such that we have frames and coordinates, like a frame $\mleft(e_a\mright)_a$ of $E$.\footnote{One could even fix a point $p \in M$ because we just need an interval for $t$ for $\mathrm{d}/\mathrm{d}t$.}
%, because then $t \mapsto \Phi_t(p)$ is a curve in $N$ over which we want to choose a trivialisation of $E$ for example; we are going to omit the notation of $p$ in the following. 
Making use of Prop.~\ref{prop:VariationVonSkalarZeugsEasyPeasy}, we get
\bas
\mleft(\delta_{\Psi_\varepsilon} \varpi_2 \mright) (\Phi_t, A_t)
&=
%\underbrace{
\mleft.\mathcal{L}_{\Psi_\varepsilon} \mleft(\varpi^a_2\mright)\mright|_{(\Phi_t, A_t)} \otimes \Phi^*_t e_a
%}_{\mathclap{= \mleft.\frac{\mathrm{d}}{\mathrm{d}t}\mright|_{t=0} \mleft(\varpi^a_2\circ\gamma\mright) }} \otimes ~ \Phi^*_0 e_a
	- A_t^a \otimes \Phi^*_t\mleft( \nabla^{\mathrm{bas}}_{\epsilon_t} e_a \mright).
%\\
%&=
%\mleft.\frac{\mathrm{d}}{\mathrm{d}t}\mright|_{t=0} \mleft[ t \mapsto A^a_t \mright] \otimes \Phi^*_0 e_a
	%- A_0^a \otimes \Phi^*_0\mleft( \nabla^{\mathrm{bas}}_\epsilon e_a \mright)
\eas
Let us first assume Eq.~\eqref{EichtrafoVonANochmal} does hold. Then
\bas
&\mleft.\mathcal{L}_{\Psi_\varepsilon} \mleft(\varpi^a_2\mright)\mright|_{(\Phi_t, A_t)} \otimes \Phi^*_t e_a
\\
&=
\epsilon_t^b A_t^c \otimes \Phi^*_t\mleft( \nabla^{\mathrm{bas}}_{e_b} e_c \mright)
	- \mleft(\Phi_t^* \nabla\mright) \epsilon_t
\\
&=
\mleft(
\mleft( C^a_{bc} \circ \Phi_t \mright) ~\epsilon_t^b A_t^c
	+ \mleft(\omega^a_{b\alpha} \circ \Phi_t \mright) ~ \mleft( \rho^\alpha_c \circ \Phi_t \mright)~\epsilon_t^b A_t^c
	- \mathrm{d}\epsilon_t^a - \epsilon_t^b ~ \Phi_t^!\mleft(\omega^a_{b} \mright)
\mright) \otimes \Phi^*_t e_a
\eas
which proves Eq.~\eqref{eqGaugeTrafoOfAacomps} (insert $t=0$). By the definition of $\gamma$ and the Lie derivative we also get
\bas
\mleft.\mathcal{L}_{\Psi_\varepsilon} \mleft(\varpi^a_2\mright)\mright|_{(\Phi_t, A_t)} 
&=
\frac{\mathrm{d}}{\mathrm{d}t} \mleft(\varpi^a_2\circ\gamma\mright)
=
\frac{\mathrm{d}}{\mathrm{d}t} \mleft[ t \mapsto A^a_t \mright],
\eas
and, thus,
\ba\label{DiffEqFuerAComp}
\frac{\mathrm{d}}{\mathrm{d}t} \mleft[ t \mapsto A^a_t \mright]
&=
\mleft( C^a_{bc} \circ \Phi_t \mright) ~\epsilon^b_t A_t^c
	+ \mleft(\omega^a_{b\alpha} \circ \Phi_t \mright) ~ \mleft( \rho^\alpha_c \circ \Phi_t \mright)~\epsilon^b_t A_0^c
	- \mathrm{d}\epsilon^a - \epsilon^b_t ~ \Phi_t^!\mleft(\omega^a_{b} \mright).
\ea
So, Eq.~\eqref{EichtrafoVonANochmal} is equivalent to a set of coupled differential equations: We have a curve $\gamma(t) = (\Phi_t, A_t)$, with $\Phi_{t=0} = \Phi_0$ and 
\bas
\frac{\mathrm{d}}{\mathrm{d}t} [t \mapsto \Phi_t]
&=
- (\Phi_t^*\rho) (\epsilon_t),
\eas
and $A_{t=0} = A_0$, while
\bas
\mathcal{a}_t
&=
\frac{\mathrm{d}}{\mathrm{d}t} \mleft[ t \mapsto A_t \mright]
=
\frac{\mathrm{d}}{\mathrm{d}t} \mleft[ t \mapsto A_t^a \otimes \Phi^*_t e_a \mright].
%\\
%&=
%\mleft.\frac{\mathrm{d}}{\mathrm{d}t}\mright|_{t=0} \mleft[ t \mapsto A^a_t \mright] \otimes \Phi^*_0 e_a
	%+ A_0^a \otimes \bigl(\Phi^*_0\mleft(\mathrm{D}e_a\mright)\bigr)\mleft( \mleft.\frac{\mathrm{d}}{\mathrm{d}t}\mright|_{t=0} \mleft[t\mapsto \Phi_t\mright] \mright)
%\\
%&=
%\mleft.\mathcal{L}_{\Psi_\varepsilon} \mleft(\varpi^a_2\mright)\mright|_{(\Phi_0, A_0)} \otimes \Phi^*_0 e_a
	%- A_0^a \otimes \bigl(\Phi^*_0\mleft(\mathrm{D}e_a\mright)\bigr)\bigl( (\Phi^*_0\rho)(\epsilon_0)\bigr)
\eas
$t \mapsto \Phi_t$ and $t \mapsto A^a_t$ are uniquely given by this system and the differential equation \eqref{DiffEqFuerAComp}, and, so, $t \mapsto A_t = A^a_t \otimes \Phi^*_t e_a$ is uniquely given, too. Hence, $\mathcal{a}_t$ is unique, and, thus, $\Psi_\varepsilon$. Alternatively, the differential equations for $\mathrm{d}/\mathrm{d}t ~ \Phi$ and $\mathrm{d}/\mathrm{d}t ~ A^a$ are the action of the vector field $\Psi_\varepsilon$ on the coordinates of $\mathfrak{M}_E$, and therefore defining $\Psi_\varepsilon$.

The linearity of $\psi_\varepsilon$ in $\varepsilon$ over $\mathbb{R}$ simply follows by the linearity given in the differential equations above: Define $\Theta \coloneqq \alpha \Psi_\varepsilon + \beta \Psi_\vartheta$ for $\alpha, \beta \in \mathbb{R}$ and $\vartheta \in \mathcal{F}^0_E(M; {}^*E)$, where $\Psi_\varepsilon$ and $\Psi_\vartheta$ are the unique vector fields as given above, \textit{i.e.}~$\delta_{\Psi_\varepsilon} \varpi_2 = - ({}^*\nabla) \varepsilon$ and $\delta_{\Psi_\vartheta} \varpi_2 = - ({}^*\nabla) \vartheta$, respectively. Observe that $\Theta \in \mathfrak{X}^E\bigl(\mathfrak{M}_E(M; N)\bigr)$, where the component along the "$\Phi$-direction" is by definition given by
\bas
-\alpha ~ ({}^*\rho)(\varepsilon) - \beta ~ ({}^*\rho)(\vartheta)
&=
- ({}^*\rho)(\alpha \varepsilon + \beta \vartheta),
\eas
then, using the linearity of Eq.~\eqref{DiffEqFuerAComp} in $\varepsilon$,
\bas
\delta_\Theta \varpi_2
&=
\mathcal{L}_\Theta (\varpi_2^a) \otimes {}^*e_a
	- \varpi_2^a \otimes {}^*\mleft(\nabla^{\mathrm{bas}}_{\alpha \varepsilon + \beta \vartheta} e_a\mright)
\\
&=
\mleft( \alpha \mathcal{L}_{\Psi_{\varepsilon}}  + \beta \mathcal{L}_{\Psi_\vartheta} \mright) (\varpi_2^a) \otimes {}^*e_a
	- \varpi_2^a \otimes {}^*\mleft(\nabla^{\mathrm{bas}}_{\alpha \varepsilon + \beta \vartheta} e_a\mright)
	\\
&\stackrel{\mathclap{ \text{ Eq.~\eqref{DiffEqFuerAComp}} }}{=}~~~
\mathcal{L}_{\Psi_{\alpha \varepsilon + \beta \vartheta}} (\varpi_2^a) \otimes {}^*e_a
	- \varpi_2^a \otimes {}^*\mleft(\nabla^{\mathrm{bas}}_{\alpha \varepsilon + \beta \vartheta} e_a\mright)
\\
&=
\delta_{\Psi_{\alpha \varepsilon + \beta \vartheta}} \varpi_2.
\eas
By the shown uniqueness of vector fields like $\Psi_{\alpha \varepsilon + \beta \vartheta}$, we get
\bas
\Theta
&=
\Psi_{\alpha \varepsilon + \beta \vartheta}.
\eas
%viewing $\Phi^*_t e_a$ canonically as a section along $\Phi_t$, hence, $\Phi^*_t e_a = e_a \circ \Phi_t: M \to E$, such that for all $p \in M$
%\bas
%\mleft.\frac{\mathrm{d}}{\mathrm{d}t}\mright|_{t=0} \mleft[ t \mapsto \mleft(\Phi^*_t(p)\mright) e_a \mright]
%&=
%\mathrm{D}_{\Phi_0(p)}e_a \mleft( \mleft.\frac{\mathrm{d}}{\mathrm{d}t}\mright|_{t=0} \mleft[t\mapsto \Phi_t(p)\mright] \mright)
%=
%\mleft.\bigl(\Phi^*_0\mleft(\mathrm{D}e_a\mright)\bigr)\mleft( \mleft.\frac{\mathrm{d}}{\mathrm{d}t}\mright|_{t=0} \mleft[t\mapsto \Phi_t\mright] \mright)\mright|_p
%\eas
%and using $\mathrm{D} e_a \in \Omega^1\mleft(N; e_a^*\mathrm{T}E \mright)$ (as we did for $\mathrm{D}\Phi$ before)\footnote{Rigorously, this form is just defined on an open subset of $N$.} such that $\Phi^*_0\mleft(\mathrm{D}e_a\mright) \in \Gamma\mleft( \Phi^*_0(\mathrm{T}^*N) \otimes (\Phi^*_0e_a)^*\mathrm{T}E \mright)$
\end{proof}

\begin{remark}\label{RemDifferentVersionsOfGaugeTrafos}
\leavevmode\newline
Eq.~\eqref{eqGaugeTrafoOfAacomps} is also \textit{e.g.}~defined in \cite[Eq.~(10); opposite sign of $\varepsilon$]{CurvedYMH}, but in this reference it was not known how a coordinate-free version can look like. This equation recovers the standard formula of the infinitesimal gauge transformation of $A$.
%As argued in the last statement of Remark \ref{RemLeibnizeRegelaufProdukteWeshalbEConnectionNichtWichtigIst} one does not necessarily need to know that, for the variation of the Lagrangian we only need to know $\delta_\varepsilon A^a$. 
In order to see why this restricts to the standard formula, let us look again at the standard setting: When $E= N \times \mathfrak{g}$ is an action Lie algebroid with Lie algebra $\mathfrak{g}$, equipped with its canonical flat connection $\nabla$, then we get the classical formula of gauge transformation by using a constant frame $\mleft( e_a \mright)_a$ for $E$, \textit{i.e.}
\bas
\mleft(\delta_{\Psi_\varepsilon} \varpi_2^a\mright)(\Phi, A)
&=
\Phi^*C^a_{bc} ~ \epsilon^b A^c - \mathrm{d}\epsilon^a 
=
\mleft(
	\mleft[ \epsilon \stackrel{\wedge}{,} A \mright]_{\mathfrak{g}}
	- \mathrm{d}^{\Phi^*\nabla} \epsilon 
\mright)^a
\eas
for all $(\Phi, A) \in \mathfrak{M}_E(M; N)$,
because $\omega^a_b = 0$ and $\Phi^*C^a_{bc} = C^a_{bc} = \text{const.}$, the structure constants of $\mathfrak{g}$. We can understand $\epsilon$ as an element of $C^\infty(M; \mathfrak{g})$ as usual in the standard setting. That is precisely the typical formula of the classical setting, because $\Phi^*\nabla$ is the canonical flat connection of $\Phi^*E \cong M \times \mathfrak{g}$. Moreover, we get in that situation
\bas
%\Phi^*\mleft(
	%\nabla_{\rho(\varepsilon)} e_a 
%\mright)
%&= 
%0, 
%\\
\Phi^*\mleft(
	\nabla^{\mathrm{bas}}_{\epsilon} e_a 
\mright)
&=
\epsilon^b ~ \Phi^*\mleft(\mleft[ e_b, e_a \mright]_E\mright),
\eas
which is the main reason why the transformations of the components recover the classical formula although the total formula, Eq.~\eqref{EichtrafoVonANochmal}, just carries the differential (as we saw in the proof). As already discussed, only the transformation of the components need the "correct form" when it is about the gauge invariance of the Lagrangian.
%
%Hence, this proposition provides two ways how to formulate the infinitesimal gauge transformation of $A$ given in \cite[Eq.~(10)]{CurvedYMH} in a coordinate-free way, that is given by Eq.~\eqref{EqVariationsBestimmungFuerAAaaaaah} and \eqref{EqVariationsBestimmungFuerADoeoeoeo}.
\end{remark}

Using such a $\Psi_\varepsilon$ results into an infinitesimal gauge transformation of the minimal coupling as in Cor.~\ref{cor:EichtrafovonDAPHIinClassicIstBabyEinfach}.

\begin{propositions}{Infinitesimal gauge transformation of the minimal Coupling}{InfinitesimalGaugeTrafoOfMinimalCoupleSmiley}
Let $M, N$ be two smooth manifolds, $E \to N$ a Lie algebroid over $N$, $\nabla$ a connection on $E$, and $\varepsilon \in \mathcal{F}^0_E(M; {}^*E)$ together with the unique $\Psi_\varepsilon \in \mathfrak{X}^E(\mathfrak{M}_E(M; N))$ as given in Prop.~\ref{prop:VariationOfA}. For both functional spaces, $\mathcal{F}^\bullet_E(M; {}^*E)$ and $\mathcal{F}^\bullet_E(M; {}^*\mathrm{T}N)$, let $\delta_{\Psi_\varepsilon}$ be the unique operator of Prop.~\ref{prop:VariationVonSkalarZeugsEasyPeasy}, using $\nabla^{\mathrm{bas}}$ as $E$-connection on $E$ and $\mathrm{T}N$, respectively.

Then we have
\ba
\delta_{\Psi_\varepsilon} \mathfrak{D}
&=
0.
\ea
\end{propositions}

\begin{remark}
\leavevmode\newline
We already have derived the variation of the components of $\mathfrak{D}$, for this recall the general calculation for Eq.~\eqref{CompsVonDMinimalAlsErstes}:
Let $\mleft( e_a \mright)_a$ be a local frame of $E$ and $\partial_\alpha$ coordinate vector fields on $N$, then we can write $\mathfrak{D} = \mathfrak{D}^\alpha \otimes {}^*\partial^\alpha$, and, thus, with $\epsilon \coloneqq \varepsilon(\Phi, A)$, 
\ba
\bigl(\delta_{\Psi_\varepsilon} \mathfrak{D}^\alpha\bigr)(\Phi, A)
&=
\epsilon^a ~ \Phi^*\mleft( 
	- \partial_\beta\rho_a^\alpha
	+ \rho^\alpha\mleft( \nabla_{\partial_\beta} e_a \mright) 
\mright) ~ \mleft( \mathfrak{D}^A \Phi \mright)^\beta.
\ea
That is precisely the same formula as given in \cite[Eq.~(12), different sign for $\epsilon$ there]{CurvedYMH}, but there only the formula for the components was known.
\end{remark}

\begin{proof}[Proof of Prop.~\ref{prop:InfinitesimalGaugeTrafoOfMinimalCoupleSmiley}]
\leavevmode\newline
This quickly follows by Lemma \ref{lem:VariationsIdentities}, especially Eq.~\eqref{DPhiVariation} and \eqref{eqRhoAVariation},
\bas
\delta_{\Psi_\varepsilon} \mathfrak{D}
&=
\delta_{\Psi_\varepsilon} \bigl( \mathrm{D} - ({}^*\rho)(\varpi_2) \bigr)
=
-({}^*\rho)({}^*\nabla\varepsilon)
	- \mleft( {}^* \rho \mright) \bigl( \delta_{\Psi_\varepsilon} \varpi_2 \bigr)
\stackrel{\text{Prop.~\ref{prop:VariationOfA}}}{=}
0.
\eas
\end{proof}

\begin{remark}
\leavevmode\newline
Following the proof of Prop.~\ref{prop:InfinitesimalGaugeTrafoOfMinimalCoupleSmiley} and using the uniqueness of Prop.~\ref{prop:VariationOfA} one could argue that $\Psi_\varepsilon$ is the unique element of $\mathfrak{X}^E(\mathfrak{M}_E(M; N))$ with $\delta_{\Psi_\varepsilon} \mathfrak{D} = 0$ for a given $\varepsilon$ in the category of Lie algebroids, because this must then \textit{e.g.}~hold for the tangent bundle $E = \mathrm{T}N$ as Lie algebroid, too, whose anchor is the identity.
\end{remark}

By this result and Cor.~\ref{cor:EichtrafovonDAPHIinClassicIstBabyEinfach} we define the following.

\begin{definitions}{Infinitesimal gauge transformation of gauge bosons}{GaugeTrafoOfA}
Let $M, N$ be two smooth manifolds, $E \to N$ a Lie algebroid over $N$, $\nabla$ a connection on $E$, and $\varepsilon \in \mathcal{F}^0_E(M; {}^*E)$ together with the unique $\Psi_\varepsilon \in \mathfrak{X}^E\bigl(\mathfrak{M}_E(M; N)\bigr)$ as given in Prop.~\ref{prop:VariationOfA}. For the functional space $\mathcal{F}^\bullet_E(M; {}^*E)$ let $\delta_{\Psi_\varepsilon}$ be the unique operator of Prop.~\ref{prop:VariationVonSkalarZeugsEasyPeasy}, using $\nabla^{\mathrm{bas}}$ as $E$-connection on $E$.

For a $(\Phi,A) \in \mathfrak{M}_E(M; N)$ we define the \textbf{infinitesimal gauge transformation $\delta_{\varepsilon(\Phi, A)} A$ of $A$} as an element of $\Omega^1(M; \Phi^*E)$ by
\ba
\delta_{\varepsilon(\Phi, A)} A
&\coloneqq
\mleft( \delta_{\Psi_\varepsilon} \varpi_2 \mright)(\Phi, A)
=
- (\Phi^*\nabla) \bigl( \varepsilon(\Phi, A) \bigr),
\ea
shortly denoted as $\delta_\varepsilon A \coloneqq \delta_{\Psi_\varepsilon} \varpi_2 = -({}^*\nabla) \varepsilon$. Given a local frame $\mleft( e_a \mright)_a$ of $E$, we also similarly define $\delta_\varepsilon A^a \coloneqq \delta \varpi_2^a$.
\end{definitions}

\begin{remark}\label{WhyNablaBasPartOne}
\leavevmode\newline
As discussed in Remark \ref{RemDifferentVersionsOfGaugeTrafos} we have seen that $\delta_\varepsilon A^a$ (using a frame $\mleft( e_a \mright)_a$ of $E$) recovers the classical formula of the infinitesimal gauge transformation. However, the total formula, $\delta_\varepsilon A$, does not recover it which is no problem due to that the variation of the Lagrangian just depends on the variation of the components; for this also recall that arising differentials of $A$ commute with $\delta_\varepsilon$, Eq.~\eqref{eqVariationVertauschtMitDifferential}, which is needed for the variation of the field strength. Later we will see this explicitly when showing the gauge invariance of the Lagrangian. 

Alternatively, one could use $\nabla_\rho$ as $E$-connection on $E$ instead of $\nabla^{\mathrm{bas}}$ for the definition of $\delta_{\Psi_\varepsilon}$; especially because of results like Thm.~\ref{thm:NewFormulaRecoversOldGaugeTrafoYay} and Thm.~\ref{thm:RecoverOfClassicInfgGaugeTrafo}, which imply that one recovers classical formulas when $\nabla$ is additionally flat.\footnote{A flat connection is locally canonically flat with respect to the trivialization given by a parallel frame; later we will also see that then $E$ is locally an action algebroid and $\nabla$ its canonical flat connection, if $\nabla$ is flat and has vanishing basic curvature.} When using $\nabla_\rho$, the same $\Psi_\varepsilon$ leads to
\ba
\delta_{\Psi_\varepsilon} \varpi_2
&=
-({}^*t_{\nabla_\rho})(\varepsilon, \varpi_2) - ({}^*\nabla)\varepsilon,
\ea
where $t_{\nabla_\rho}$ is the torsion of $\nabla_\rho$. As we have seen before, $\nabla$ will be the canonical flat connection in the standard setting such that then $\delta_{\Psi_\varepsilon} A^a = \mleft(\delta_{\Psi_\varepsilon} A \mright)^a$ by flatness and Thm.~\ref{thm:NewFormulaRecoversOldGaugeTrafoYay}. With similar calculations as before one also shows that the variation of the components, $\delta_{\Psi_\varepsilon} \varpi_2^a$, recovers the classical formula of the infinitesimal gauge transformation of the field of gauge bosons, thus, $\delta_{\Psi_\varepsilon} \varpi_2$ would restrict to the classical formula in the standard setting, too. Hence, $\nabla_\rho$ would look like the canonical choice, not $\nabla^{\mathrm{bas}}$. But we will later see that $\nabla_\rho$ is in general not flat, while $\nabla^{\mathrm{bas}}$ will be flat, such that only for the latter the infinitesimal gauge transformations in form of the operator $\delta_{\Psi_\varepsilon}$ will give rise to a Lie algebra in general. Moreover, we are not going to fix any separate connection on $\mathrm{T}N$ which would be identified with a canonical flat connection in the standard situation, such that the only canonical connection there is the basic connection; using the basic connections also for $E$-valued tensors is then in alignment to $\mathrm{T}N$-valued tensors.
\end{remark}

Hence, we finally arrived at defining the infinitesimal gauge transformation of functionals.

\begin{definitions}{Infinitesimal gauge transformation of functionals}{TotalInfGaugeTrafoYayy}
Let $M, N$ be two smooth manifolds, $E \to N$ a Lie algebroid over $N$, $V \to N$ a vector bundle, $\nabla$ a connection on $E$, ${}^E\nabla$ an $E$-connection on $V$, and $\varepsilon \in \mathcal{F}^0_E(M; {}^*E)$ together with the unique $\Psi_\varepsilon \in \mathfrak{X}^E(\mathfrak{M}_E(M; N))$ as given uniquely in Prop.~\ref{prop:VariationOfA}. For the functional space $\mathcal{F}^\bullet_E(M; {}^*V)$ let $\delta_{\Psi_\varepsilon}$ be the unique operator as in Prop.~\ref{prop:VariationVonSkalarZeugsEasyPeasy}, using ${}^E\nabla$ as $E$-connection on $V$.

Then we define the \textbf{infinitesimal gauge transformation $\gls{1delta0epsilon} L$ of $L \in \mathcal{F}^\bullet_E(M; {}^*V)$} as an element of $\mathcal{F}^\bullet_E(M; {}^*V)$ by
\ba
\delta_\varepsilon L
&\coloneqq
\delta_{\Psi_\varepsilon} L.
\ea

For $V=E$ or $V= \mathrm{T}N$ we take ${}^E\nabla = \nabla^{\mathrm{bas}}$ on $E$ and $\mathrm{T}N$, respectively; for all further tensor spaces constructed of $E$ and $\mathrm{T}N$, like their duals, we take the canonical extensions of the basic connection.
\end{definitions}

\begin{remark}
\leavevmode\newline
In the following we will have just one connection $\nabla$ on $E$ and ${}^E\nabla$ on $V$ given. Without mentioning it further, we always use these connections for the definition of $\delta_\varepsilon$ because it should be clear by context.
\end{remark}

We can quickly list two properties about $\delta_\varepsilon$.

\begin{corollaries}{Linearity in $\varepsilon$}{DeltaEpsilonIstLinearInEpsilon}
Let us assume the same as for Def.~\ref{def:TotalInfGaugeTrafoYayy}. Then
\ba
\delta_{\alpha \varepsilon + \beta \vartheta}
&=
\alpha \delta_\varepsilon
	+ \beta \delta_\vartheta
\ea
for all $\alpha, \beta \in \mathbb{R}$ and $\varepsilon, \vartheta \in \mathcal{F}^0_E(M; {}^*E)$.
\end{corollaries}

\begin{proof}
\leavevmode\newline
Let $k \in \mathbb{N}_0$, $L \in \mathcal{F}^k_E(M; {}^*V)$ and $\mleft( e_a \mright)_a$ a local frame of $V$. Then, using Eq.~\eqref{LinearityOfPsiEpsilon} and the Leibniz rule,
\bas
\delta_{\alpha \varepsilon + \beta \vartheta} L
&=
\underbrace{\mathcal{L}_{\Psi_{\alpha \varepsilon + \beta \vartheta}} L^a}
_{\mathclap{ \stackrel{Eq.~\eqref{LinearityOfPsiEpsilon}}{=} \mathcal{L}_{\alpha \Psi_\varepsilon + \beta \Psi_\vartheta} } }
 \otimes ~ {}^*e_a
	- L^a \otimes {}^*\mleft( {}^E\nabla_{\alpha \varepsilon + \beta \vartheta} e_a \mright)
\\
&=
\alpha ~ \mleft(
	\mathcal{L}_{\Psi_\varepsilon} L^a \otimes {}^*e_a
	- L^a \otimes {}^*\mleft( {}^E\nabla_\varepsilon e_a \mright)
\mright)
	+ \beta ~ \mleft(
	\mathcal{L}_{\Psi_\vartheta} L^a \otimes {}^*e_a
	- L^a \otimes {}^*\mleft( {}^E\nabla_\vartheta e_a \mright)
\mright)
\\
&=
\mleft(\alpha \delta_\varepsilon
	+ \beta \delta_\vartheta\mright) L,
\eas
where vector fields like $\Psi_\varepsilon$ are given by Def.~\ref{def:TotalInfGaugeTrafoYayy}.
\end{proof}

\begin{corollaries}{Independence of $\nabla$}{WennVonAUnabhaengigDannAuchVonNabla}
Let us assume the same as for Def.~\ref{def:TotalInfGaugeTrafoYayy}, and let $L \in \mathcal{F}^k_E(M; {}^*V)$ ($k \in \mathbb{N}_0$) be independent of $A$, \textit{i.e.}~$L(\Phi,A) = L(\Phi, A^\prime)$ for all $(\Phi,A), (\Phi, A^\prime) \in \mathfrak{M}_E(M; N)$.

Then the definition of $\delta_\varepsilon L$ is independent of $\nabla$.\footnote{But not of ${}^E\nabla$, so, if ${}^E\nabla = \nabla^{\mathrm{bas}}$, then there is still the dependency on $\nabla$ in the role of ${}^E\nabla$.}
\end{corollaries}

\begin{remark}
\leavevmode\newline
The independence mentioned in Remark \ref{RemLeibnizeRegelaufProdukteWeshalbEConnectionNichtWichtigIst} is about ${}^E\nabla$, not $\nabla$. Eq.~\eqref{eqGaugeTrafoOfAacomps} shows clearly that $\nabla$ contributes to $\delta_\varepsilon$ in general, that is, the definition of $\Psi_\varepsilon$ is certainly dependent on $\nabla$, where $\Psi_\varepsilon$ is given by Def.~\ref{def:TotalInfGaugeTrafoYayy}.
\end{remark}

\begin{proof}
\leavevmode\newline
Let $\mleft( e_a \mright)_a$ be a local frame of $V$, and write $L = L^a \otimes {}^*e_a$, then, using that $\delta_\varepsilon = \mathcal{L}_{\Psi_\varepsilon}$ on $\mathcal{F}^k_E(M)$ (recall Remark \ref{RemLeibnizeRegelaufProdukteWeshalbEConnectionNichtWichtigIst}, and $\Psi_\varepsilon$ is given by Def.~\ref{def:TotalInfGaugeTrafoYayy}),
\bas
\delta_\varepsilon L
&=
\mathcal{L}_{\Psi_\varepsilon} L^a \otimes {}^*e_a
	- L^a \otimes {}^*\mleft({}^E\nabla_\varepsilon e_a \mright).
\eas
The second summand is already independent of $\nabla$, so, let us look at the first summand. Recall that $\Psi_\varepsilon$ contains two components, the first is the differentiation along the "$\Phi$-direction", given by $-({}^*\rho)(\varepsilon)$, and the second for the "$A$-direction", fixed by Prop.~\ref{prop:VariationOfA} using $\nabla$. Due to the independence of $L$ with respect to $A$ we can conclude that $L^a$ must be independent of $A$ since ${}^* e_a$ is already independent of $A$, thus,
\bas
\mathcal{L}_{\Psi} L^a
&=
\mathcal{L}_{\Psi^\prime} L^a
\eas
for all $\Psi, \Psi^\prime \in \mathfrak{X}(\mathfrak{M}_E(M; N))$ whose first component, the derivative along "$\Phi$"-coordinates, coincide. Hence, regardless which connection $\nabla$ we choose to fix the second component of $\Psi_\varepsilon$ the definition of $\delta_\varepsilon L$ will be unaffected by this choice.
\end{proof}
 %
%Let us now introduce the infinitesimal gauge transformation of the gauge bosons, we will present two different ones. One will be the covariantization of the classical formula while we will motivate that the other one should be used; recall Remark \ref{RemLeibnizeRegelaufProdukteWeshalbEConnectionNichtWichtigIst}: For the infinitesimal gauge invariance of the Lagrangian we only need to know the gauge transformation of $A^a$ which we will now fix.
%
%\begin{propositions}{Infinitesimal gauge transformation of gauge bosons}{GaugeTrafoOfAacomps}
%Let $E \to N$ be a Lie algebroid, $M, N$ smooth manifolds, $\nabla$ a connection on $E$, $\Phi \in C^\infty(M; N)$, $A \in \Omega^1(M;\Phi^*E)$, and $\varepsilon \in \mathcal{F}^0_E(M;{}^*E)$. Furthermore, let $\Psi_\varepsilon$ be a canonical $E$-path for $\varepsilon$ and $\widetilde{\Phi}_\varepsilon$ its base path. Then there is an $\widetilde{A}_{\varepsilon(\Phi, A)} \in \Omega^1\mleft(M; \widetilde{\Phi}_{\varepsilon(\Phi, A)}^*E\mright), I \ni t \mapsto \widetilde{A}_{\varepsilon(\Phi, A), t} \in \Omega^1\mleft( M; \mleft( \widetilde{\Phi}_{\varepsilon(\Phi, A), t} \mright)^*E \mright),$ such that 
%\ba
%\widetilde{A}_{\varepsilon(\Phi, A), t=0} 
%&= 
%A,
%\\
%\mleft.\frac{\mathrm{D}_{\mleft( \Psi_{\varepsilon(\Phi, A)}, \nabla_\rho \mright)}}{\mathrm{d}t}\mright|_{t=0}
%\mleft[ t \mapsto \widetilde{A}_{\varepsilon(\Phi, A), t} \mright]
%&=
%\mleft( \Phi^*t_{\nabla^{\mathrm{bas}}} \mright)\mleft( \varepsilon(\Phi, A), A \mright)
	%- \mathrm{d}^{\Phi^*\nabla} \bigl( \varepsilon(\Phi, A) \bigr),\label{EqVariationsBestimmungFuerAAaaaaah}
%\\
%\mleft.\frac{\mathrm{D}_{\mleft( \Psi_{\varepsilon(\Phi, A)}, \nabla^{\mathrm{bas}} \mright)}}{\mathrm{d}t}\mright|_{t=0}
%\mleft[ t \mapsto \widetilde{A}_{\varepsilon(\Phi, A), t} \mright]
%&=
	%- \mathrm{d}^{\Phi^*\nabla} \bigl( \varepsilon(\Phi, A) \bigr),\label{EqVariationsBestimmungFuerADoeoeoeo}
%\ea
%where $I \subset \mathbb{R}$ is an open interval with $0 \in I$ and $t_{\nabla^{\mathrm{bas}}}$ is the torsion of $\nabla^{\mathrm{bas}}$. With respect to a local frame $\mleft( e_a \mright)_a$ of $E$ and coordinate vector fields $\mleft(\partial_\alpha\mright)_\alpha$ of $N$, $\widetilde{A}_{\varepsilon(\Phi, A)}$ is (not uniquely) characterized by
%\ba
%\widetilde{A}_{\varepsilon(\Phi, A), t=0}^a
%&=
%A^a,
%\\
%\mleft.\frac{\mathrm{d}}{\mathrm{d}t}\mright|_{t=0} \mleft[t\mapsto\widetilde{A}_{\varepsilon(\Phi, A), t}^a\mright]
%&=
%\mleft( C^a_{bc} \circ \Phi \mright) ~\varepsilon^b A^c
	%+ \mleft(\omega^a_{b\alpha} \circ \Phi \mright) ~ \mleft( \rho^\alpha_c \circ \Phi \mright)~\varepsilon^b A^c
	%- \mathrm{d}\varepsilon^a - \varepsilon^b ~ \Phi^!\mleft(\omega^a_{b} \mright)
%\nonumber \\ 
%&=
%\mleft( \varepsilon^b A^c \otimes \Phi^*\mleft( \nabla^{\mathrm{bas}}_{e_b} e_c\mright)
	%- \mathrm{d}^{\Phi^*\nabla}\varepsilon \mright)^a,
%\ea
%where we write $A = A^a \otimes \Phi^*e_a$, $\widetilde{A}_{\varepsilon(\Phi, A), t}=\widetilde{A}_{\varepsilon(\Phi, A), t}^a \otimes \widetilde{\Phi}^*_{\varepsilon(\Phi, A), t} e_a$, and on the right hand side we abbreviated the notation, writing $\varepsilon$ for $\varepsilon(\Phi, A)$.
%
%We will denote the collection of $\widetilde{A}_{\varepsilon(\Phi, A), t}$ among all pairs $(\Phi,A)$ by $\widetilde{A}_{\varepsilon}$.
%\end{propositions}
%
%\begin{proof}[Proof of Prop.~\ref{prop:GaugeTrafoOfAacomps}]
%\leavevmode\newline
%The conditions about how to define $\widetilde{A}_\varepsilon$ are just "initial" conditions at $t=0$ for "position" and "velocity", the existence of such a curve is then given once we have shown that Eq.~\eqref{EqVariationsBestimmungFuerAAaaaaah} and \eqref{EqVariationsBestimmungFuerADoeoeoeo} do not exclude each other; we will show this by showing that both equations are the same in local coordinates, precisely given by Eq.~\eqref{eqGaugeTrafoOfAacomps}. For this recall that terms of the form $\mathrm{D}/\mathrm{d}t \mleft[t\mapsto\widetilde{\Phi}^*_{\varepsilon(\Phi, A), t} e_a\mright]$ are given by Prop.~\ref{prop:DerivationAlongEPath} due to the fact that $\widetilde{\Phi}_{\varepsilon(\Phi, A)}(p)$ is the base path of $\Psi_{\varepsilon(\Phi, A)}(p)$ for all $p \in M$. This implies that Eq.~\eqref{EqVariationsBestimmungFuerAAaaaaah} and \eqref{EqVariationsBestimmungFuerADoeoeoeo}, viewed as conditions on $\widetilde{A}_\varepsilon$, are locally conditions on $\widetilde{A}_{\varepsilon(\Phi, A), t}^a$ given by the Leibniz rule
%\bas
%\mleft(\mleft.\frac{\mathrm{d}}{\mathrm{d}t}\mright|_{t=0} \mleft[t\mapsto\widetilde{A}_{\varepsilon(\Phi, A), t}^a\mright]\mright) \otimes \Phi^*e_a
%&=
%\mleft.\frac{\mathrm{D}_{\mleft( \Psi_{\varepsilon(\Phi, A)}, {}^E\nabla \mright)}}{\mathrm{d}t}\mright|_{t=0}
%\mleft[ t \mapsto \widetilde{A}_{\varepsilon(\Phi, A), t} \mright]
	%+ A^a \otimes \Phi^*\mleft( {}^E\nabla_{\varepsilon(\Phi, A)} e_a \mright)
%\eas
%where we used $\widetilde{A}_{\varepsilon, t=0} = A$, $\Psi_{\varepsilon, t=0} = - \varepsilon$, $\widetilde{\Phi}_{\varepsilon,t=0}=\Phi$, and ${}^E\nabla$ is either $\nabla_\rho$ or $\nabla^{\mathrm{bas}}$.
%Let us calculate the right hand side. For this we will omit the notation of $(\Phi, A)$ for simplicity, especially just writing $\varepsilon$ although we mean $\varepsilon(\Phi, A)$. We have
%\bas
%- \mathrm{d}^{\Phi^*\nabla} \underbrace{\varepsilon}_{\mathclap{= ~ \varepsilon^a ~\Phi^*e_a}}
%&=
%- \mleft(
%\mathrm{d} \varepsilon^a \otimes \Phi^*e_a + \varepsilon^b \wedge \vphantom{\mleft( \Phi^* \nabla \mright)\mleft( \Phi^*e_b \mright)} \smash{\underbrace{\mleft( \Phi^* \nabla \mright)\mleft( \Phi^*e_b \mright)}_{=~ \Phi^!(\nabla e_b)}}
%\mright)
%=
%-\mleft(
%\mathrm{d}\varepsilon^a + \varepsilon^b ~ \Phi^!\mleft(\omega^a_{b} \mright)
%\mright) \otimes  \Phi^*e_a,
%\eas
%and recall that $t_{\nabla^{\mathrm{bas}}} = - t_{\nabla_\rho}$, such that
%\bas
%\mleft( \Phi^*t_{\nabla^{\mathrm{bas}}} \mright)\mleft( \varepsilon, A \mright)
%&=
%\mleft(
%\mleft( C^a_{bc} \circ \Phi \mright) ~\varepsilon^b A^c
	%- \mleft(\omega^a_{b\alpha} \circ \Phi \mright) ~ \mleft( \rho^\alpha_c \circ \Phi \mright)~\varepsilon^c A^b
	%+ \mleft(\omega^a_{b\alpha} \circ \Phi \mright) ~ \mleft( \rho^\alpha_c \circ \Phi \mright)~\varepsilon^b A^c
%\mright) ~ \Phi^*e_a.
%\eas
%We also have
%\bas
%\Phi^*\mleft(
	%\nabla_{\rho(\varepsilon)} e_b
%\mright)
%&=
%\mleft( \omega_{b\alpha}^a \circ \Phi \mright) ~ \mleft( \rho^\alpha_c \circ \Phi \mright) ~\varepsilon^c ~ \Phi^*e_a
%\eas
%for all $p \in M$, and similarly
%\bas
%\Phi^*\mleft(
	%\nabla^{\mathrm{bas}}_{\varepsilon} e_b
%\mright)
%&=
%\bigl( 
%\mleft( C^a_{cb} \circ \Phi \mright) ~\varepsilon^c
	%+ \mleft(\omega^a_{c\alpha} \circ \Phi \mright) ~ \mleft( \rho^\alpha_b \circ \Phi \mright)~\varepsilon^c 
%\bigr) ~ \Phi^*e_a
%\eas
%Then in total
%\bas
%\mleft(\mleft.\frac{\mathrm{d}}{\mathrm{d}t}\mright|_{t=0} \mleft[t\mapsto\widetilde{A}_{\varepsilon, t}^a\mright]\mright) \otimes \Phi^*e_a
%&=
%\mleft.\frac{\mathrm{D}_{\mleft( \Psi_{\varepsilon}, \nabla_\rho \mright)}}{\mathrm{d}t}\mright|_{t=0}
%\mleft[ t \mapsto \widetilde{A}_{\varepsilon, t} \mright]
	%+ A^a \otimes \Phi^*\mleft( \nabla_{\rho(\varepsilon)} e_a \mright)
%\\
%&=
%\mleft(
%\mleft( C^a_{bc} \circ \Phi \mright) ~\varepsilon^b A^c
	%+ \mleft(\omega^a_{b\alpha} \circ \Phi \mright) ~ \mleft( \rho^\alpha_c \circ \Phi \mright)~\varepsilon^b A^c
%-\mathrm{d}\varepsilon^a - \varepsilon^b ~ \Phi^!\mleft(\omega^a_{b} \mright)
%\mright) \otimes  \Phi^*e_a
%\\
%&=
%\varepsilon^b A^c \otimes \Phi^*\mleft( \nabla^{\mathrm{bas}}_{e_b} e_c\mright)
	%- \mathrm{d}^{\Phi^*\nabla}\varepsilon,
%\eas
%exactly the same for $\nabla^{\mathrm{bas}}$.
%\end{proof}
%
%\begin{definitions}{Infinitesimal gauge transformation as an operator on functionals}{GaugeTrafoOfA}
%Let $E \to N$ be a Lie algebroid, $M, N$ smooth manifolds, $\nabla$ a connection on $E$, and $\varepsilon \in \mathcal{F}^0_E(M; {}^*E)$; also let $V \to N$ a vector bundle equipped with an $E$-connection ${}^E\nabla$ on $V$. Furthermore, let $\Psi_\varepsilon$ be a canonical $E$-path for $\varepsilon$ and $\widetilde{\Phi}_\varepsilon$ its base path, and let $\widetilde{A}_\varepsilon$ as in Prop.~\ref{prop:GaugeTrafoOfAacomps}, where we have an open interval $I \subset \mathbb{R}$ containing zero. 
%\begin{itemize}
	%\item Let $L \in \mathcal{F}^k_E(M;{}^*V)$ ($k \in \mathbb{N}_0$). Then we define the \textbf{infinitesimal gauge transformation $\delta_\varepsilon L$ along $\varepsilon$ as an element of $\mathcal{F}^k_E(M; {}^*V)$} by
	%\ba
	%\mleft( \gls{1delta0epsilon} L \mright)(\Phi, A)
	%&\coloneqq
	%\mleft( \delta_{\Psi_{\varepsilon(\Phi, A)}, \widetilde{A}_{\varepsilon(\Phi, A)}} L \mright) (\Phi, A)
	%\nonumber \\
	%&=
	%\mleft.\frac{\mathrm{D}_{\mleft( \Psi_{\varepsilon(\Phi, A)}, {}^E\nabla \mright)}}{\mathrm{d}t}\mright|_{t=0}
	%\mleft[ t \mapsto 
		%L\mleft[ \widetilde{\Phi}_{\varepsilon(\Phi, A), t} \mright]\mleft[ \widetilde{A}_{\varepsilon(\Phi, A), t} \mright]
	%\mright]
	%\ea
	%\item In the special case of $V=E$ let us look at $\mathds{1}_{\Omega^1(M; {}^*E)}$ as an element of $\mathcal{F}^1_E(M; {}^*E)$ defined by 
%\bas
%\mathds{1}_{\Omega^1(M; {}^*E)}(\Phi, A)&\coloneqq A
%\eas
%for all $\Phi \in C^\infty(M;N)$ and $A \in \Omega^1(M; \Phi^*E)$. For a fixed $\Phi\in C^\infty(M;N)$ and $A \in \Omega^1(M; \Phi^*E)$ we then define the \textbf{infinitesimal gauge transformations $\delta_{\varepsilon(\Phi, A)} A$ for $A$ along $\varepsilon(\Phi, A)$} as the gauge transformation of $\mathds{1}_{\Omega^1(M; {}^*E)}$ at $(\Phi, A)$ using $\nabla^{\mathrm{bas}}$ as connection, \textit{i.e.}
%\ba
%\delta_{\varepsilon(\Phi, A)} A
%&\coloneqq
%\mleft( \delta_\varepsilon \mathds{1}_{\Omega^1(M; {}^*E)} \mright) (\Phi, A)
%\nonumber\\
%&=
%\mleft.\frac{\mathrm{D}_{\mleft(\Psi_{\varepsilon(\Phi, A)}, \nabla^{\mathrm{bas}}\mright)}}{\mathrm{d}t}\mright|_{t=0}
%\mleft[ t \mapsto
	%\widetilde{A}_{\varepsilon(\Phi, A), t}
%\mright]
%\nonumber\\
%&=
%- \mathrm{d}^{\Phi^*\nabla}\bigl( \varepsilon (\Phi, A) \bigr).
%\ea
%We often shortly write $\delta_\varepsilon A$ for $\delta_\varepsilon \mathds{1}_{\Omega^1(M; {}^*E)}$.
%\end{itemize}
%
%%Then we define $\delta_\varepsilon \mathds{1}_{\Omega^1(M; {}^*E)}$ ($j\in\{1,2\}$) as elements of $\mathcal{F}^1_E(M;{}^*E)$ by
%%\ba
%%\mleft( \gls{1delta1varepsilon} \mathds{1}_{\Omega^1(M; {}^*E)} \mright) (\Phi, A)
%%&\coloneqq
%%\mleft.\frac{\mathrm{D}_{\mleft( \Psi_{\varepsilon(\Phi, A)}, \nabla_\rho \mright)}}{\mathrm{d}t}\mright|_{t=0}
%%\mleft[ t \mapsto \widetilde{A}_{\varepsilon(\Phi, A), t} \mright]
%%\nonumber\\ \label{DefGaugeTrafoAErsteVariation}
%%&=
%%\mleft( \Phi^*t_{\nabla^{\mathrm{bas}}} \mright)\mleft( \varepsilon(\Phi, A), A \mright)
	%%- \mathrm{d}^{\Phi^*\nabla} \bigl( \varepsilon(\Phi, A) \bigr),
%%\\ \label{DefGaugeTrafoAZweiteVariation}
%%\mleft(\gls{1delta2varepsilon} \mathds{1}_{\Omega^1(M; {}^*E)} \mright)(\Phi, A)
%%&\coloneqq
%%\mleft.\frac{\mathrm{D}_{\mleft( \Psi_{\varepsilon(\Phi, A)}, \nabla^{\mathrm{bas}} \mright)}}{\mathrm{d}t}\mright|_{t=0}
%%\mleft[ t \mapsto \widetilde{A}_{\varepsilon(\Phi, A), t} \mright]
%%=
	%%- \mathrm{d}^{\Phi^*\nabla} \bigl( \varepsilon(\Phi, A) \bigr)
%%\ea
%%for all $\Phi\in C^\infty(M;N)$ and $A \in \Omega^1(M; {}^*E)$. We will denote both shortly by $\delta_\varepsilon A \in \mathcal{F}^1_E(M; {}^*E)$, $(\Phi,A)\mapsto \delta_{\varepsilon(\Phi, A)}^{(j)} A$, and these are called as the .
%\end{definitions}
%
%\begin{remark}
%\leavevmode\newline
%\indent $\bullet$ Observe that we have
%\bas
%\frac{\mathrm{D}_{\mleft( \dot{\widetilde{\Phi}}_{\varepsilon}, \nabla \mright)}}{\mathrm{d}t}
%&= 
%\frac{\mathrm{D}_{\mleft( \Psi_{\varepsilon}(p), \nabla_\rho \mright)}}{\mathrm{d}t},
%\eas
%which is the usual derivative along curves using standard connections, where
%\bas
%\dot{\widetilde{\Phi}}_{\varepsilon}(p)
%&\coloneqq
%\frac{\mathrm{d}}{\mathrm{d}t}\mleft[ t \mapsto \widetilde{\Phi}_{\varepsilon, t}(p) \mright]
%\eas
%for all $p\in M$. That is due to 
%\bas
%\mleft.\frac{\mathrm{D}_{\mleft( \dot{\widetilde{\Phi}}_{\varepsilon}(p), \nabla \mright)}}{\mathrm{d}t}\mright|_t \mleft( \mleft(\widetilde{\Phi}_\varepsilon (p)\mright)^* v \mright)
%&\stackrel{\text{Prop.~\ref{prop:DerivationAlongEPath}}}{=}
%\mleft. \mleft(\widetilde{\Phi}_\varepsilon (p)\mright)^*\mleft(\nabla_{\dot{\widetilde{\Phi}}_{\varepsilon}(p)} v \mright)\mright|_{t} \\
%&=
%\mleft. \mleft(\widetilde{\Phi}_\varepsilon (p)\mright)^*\mleft(\nabla_{\rho_{\Phi(p)}\mleft( \Psi_{\varepsilon}(p) \mright)} v \mright)\mright|_{t} \\
%&=
%\mleft.\frac{\mathrm{D}_{\mleft( \Psi_{\varepsilon}(p), \nabla_\rho \mright)}}{\mathrm{d}t}\mright|_t \mleft( \mleft(\widetilde{\Phi}_\varepsilon (p)\mright)^* v \mright)
%\eas
%for all $v \in \Gamma(E)$, $p\in M$ and $t\in I$, and due to the uniqueness stated in Prop.~\ref{prop:DerivationAlongEPath}. We make use of that in the following, without explicitly mentioning it; for example when $V = N \times \mathbb{R}$, then we take the standard flat connection $\nabla^0$, and with $\nabla^0_\rho$ as an $E-$connection on $V$ we recover the standard parameter derivative. So, in such situations where a very canonical parameter derivative is given we can still recover these canonical derivatives.
%
%%$\bullet$ The taken canonical $E$-path $\Psi_\varepsilon$ can be arbitrary because we always evaluate at $t=0$ which implies that we only need to know $\Psi_{\varepsilon, t=0}$ to calculate the $E$-derivatives along $\Psi_\varepsilon$ as usual for derivatives.
%
%$\bullet$ The variations $\delta_\varepsilon A^a$ of the components $A^a$ with respect to a frame $\mleft( e_a \mright)_a$ of $E$ are defined as an element of $\mathcal{F}^1_E(M)$ by 
%\ba
%\mleft(\delta_\varepsilon A^a \otimes {}^*e_a \mright)(\Phi, A)
%&\coloneqq
%\delta_{\varepsilon(\Phi, A)} A
	%- A^a \otimes \delta_{\varepsilon(\Phi, A)} ({}^*e_a)
%\stackrel{\text{Cor.~\ref{cor:VariationsOfPullBackStuff}}}{=}
%\delta_{\varepsilon(\Phi, A)} A
	%+ A^a \otimes \Phi^*\mleft( \nabla^{\mathrm{bas}}_{\varepsilon(\Phi, A)} e_a \mright).
%\ea
%That is precisely what we have calculated in Prop.~\ref{prop:GaugeTrafoOfAacomps}, recall Eq.~\eqref{eqGaugeTrafoOfAacomps}, which recovers the standard formula of infinitesimal gauge transformations. As we have seen in Remark \ref{RemLeibnizeRegelaufProdukteWeshalbEConnectionNichtWichtigIst}, for the variation of the Lagrangian we just need to know $\delta_\varepsilon A^a$, the total $\delta_\varepsilon A$ is not needed; and, indeed, we will later show that the Lagrangian is gauge invariant. In Remark \ref{RemDifferentVersionsOfGaugeTrafos} we have seen that even the total formula given in Eq.~\eqref{EqVariationsBestimmungFuerAAaaaaah} recovers the standard infinitesimal gauge transformation, while Eq.~\eqref{EqVariationsBestimmungFuerADoeoeoeo} doesn't. We now decided to use Eq.~\eqref{EqVariationsBestimmungFuerADoeoeoeo} as a definition for $\delta_\varepsilon$, although Eq.~\eqref{EqVariationsBestimmungFuerAAaaaaah} is the direct generalization and covariantization of the standard formula. Since their induced variations of the components $A^a$ is the same, we can freely choose which formula we take. Moreover, we will later motivate that our choice has the advantage of that the infinitesimal gauge transformations form then a Lie algebra, while Eq.~\eqref{EqVariationsBestimmungFuerAAaaaaah} would mainly imply this when $\nabla_\rho$ is flat which we do not want to assume in general.
%%
%%$\bullet$ $\delta^{(1)}_\varepsilon A$'s advantage is that it is a direct covariantization of the standard formula. Assume again that $E = N \times \mathfrak{g}$ is an action Lie algebroid, equipped with its canonical flat connection $\nabla$, and we choose $\mleft( e_a \mright)_a$ as a frame of constant sections. Then 
%%\bas
%%&&
%%\delta^{(1)}_\varepsilon \mleft[ \phi \mapsto \phi^*e_a \mright][\Phi]
%%&\stackrel{\text{Cor.~\ref{cor:VariationsOfPullBackStuff}}}{=}
%%- \Phi^*\mleft( \nabla_{(\Phi^*\rho)(\varepsilon)} e_a \mright)
%%=
%%0
%%\\
%%&\Rightarrow&
%%\mleft( \delta^{(1)}_\varepsilon A \mright)^a 
%%&= 
%%\delta^{(1)}_\varepsilon A^a,
%%\eas
%%and by the previous remark we see that $\delta^{(1)}_\varepsilon A$ covariantizes the standard formula, \textit{i.e.}
%%\bas
%%\delta^{(1)}_\varepsilon A
%%&=
%%\mleft[ \varepsilon, A \mright]_{\mathfrak{g}} - \mathrm{d}\varepsilon,
%%\eas
%%where $\mathrm{d} = \mathrm{d}^\nabla$, but that is omitted in the standard formula because there it is canonical to express everything with respect to the global parallel frame, and where we view $\mleft[ \cdot, \cdot\mright]_{\mathfrak{g}}$ fibre-wise extended as if it would be an LAB. Finally, since $\nabla$ is flat, the $\mathrm{D}/\mathrm{d}t$ related to $\delta^{(1)}$ is just like a typical $\mathrm{d}/\mathrm{d}t$, which emphasizes that this is a direct generalization of the standard formula.
%%%is clearly the strong resemblance with the standard formulation of the infinitesimal gauge transformation; the differential is just replaced with an exterior covariant derivative and the Lie bracket is replaced with the torsion to make it tensorial. 
%%%Moreover, when $\rho = 0$, $\Phi$ does not get varied by Def.~\eqref{EqVariationOfHiggsField} such that one does naively not expect a variation of pull-backs of sections. Hence, in that case it looks more natural because then there is no variation of pull-backs of sections due to $\dot{\widetilde{\Phi}}_\varepsilon = - (\Phi^*\rho)(\varepsilon) = 0$ and Eq.~\eqref{DefErsteVariation}. Also here, when $\nabla$ is flat, then one immediately gets the back to the standard formula of infinitesimal gauge transformation, and that makes sense since in the classical situation $E \cong N \times \mathfrak{g}$ and one uses then of course the canonical flat connection to formulate $\mathrm{D}/\mathrm{d}t$. Therefore we can conclude that $\delta^{(1)}$ is the direct generalization of the standard gauge transformation.
%%
%%$\bullet$ $\delta_\varepsilon A$ looks wrong on the first sight because the Lie bracket term seems to be missing, even in the non-abelian situation. But that term hides now in the definition of $\nabla^{\mathrm{bas}}$ and this proposition shows that it still gives the same variation of $A^a$ as the other approach where the structure functions are contained, and therefore this will lead to a correct gauge invariance as we are going to see later, and as already pointed out in Remark \ref{RemLeibnizeRegelaufProdukteWeshalbEConnectionNichtWichtigIst}. When we have again an action Lie algebroid with its canonical connection and looking at a global constant frame, then
%%\bas
%%\delta_\varepsilon \mleft[ \phi \mapsto \phi^*e_a \mright][\Phi]
%%&=
%%- \Phi^*\mleft( \nabla^{\mathrm{bas}}_\varepsilon e_a \mright)
%%=
%%\varepsilon^b ~ \mleft[ e_a, e_b \mright]_{\mathfrak{g}}
%%=
%%\mleft[ e_a, \varepsilon \mright]_{\mathfrak{g}},
%%\eas
%%viewing $\mleft[ \cdot, \cdot\mright]_{\mathfrak{g}}$ again as the canonical LAB bracket. Thus, the missing Lie bracket gets reproduced here because even in the flat situation one still needs to subtract this term to get $\delta_\varepsilon A^a$.
%%%Even when $\rho = 0$, one still has variations of pull-backs of sections, which is important because even in that situation we still need the "missing" Lie bracket: We will later see that $\rho=0$ describes the Yang-Mills-Higgs situation of massless gauge bosons, especially for the gluons we need a Lie bracket in the infinitesimal gauge transformation.
%%So, although $\delta_\varepsilon A$ gives a correct formula and results, it looks less natural, but we will see that it can simplify calculations, and it will recover the typical formula "$[\delta_\varepsilon, \delta_\vartheta] A = - \delta_{[\varepsilon, \vartheta]} A$"\footnote{The sign comes from the sign chosen in the definition of the infinitesimal gauge transformations.} while $\delta^{(1)}$ does only do so when $\nabla_\rho$ is flat; we will discuss these later.
%\end{remark}
%
%%Let us now finally define the total formulation of our gauge transformation.
%%
%%\begin{definitions}{Total infinitesimal gauge transformation}{TotalInfinGaugeTransform}
%%Let $M, N$ be two smooth manifolds, $V \to N$ a vector bundle, $E \to N$ Lie algebroids, ${}^E\nabla$ an $E$-connection on $V$ and $F$ a map, $C^\infty(M;N) \ni \phi \mapsto F[\phi]$, where $F[\phi]$ is another map, $\Omega^1(M; \phi^*E) \ni \omega \mapsto F[\phi][\omega] \in \Omega^k(M; \phi^*V)$ ($k \in \mathbb{N}_0$).
%%
%%Let $j \in \{1,2\}$, then for a $\Phi \in C^\infty(M;N)$, $A \in \Omega^1(M; \Phi^*E)$, and $\varepsilon \in \Gamma(\Phi^*E)$ we define the \textbf{infinitesimal gauge transformation $\gls{1delta1varepsilon} F (\Phi, A)$ and $\gls{1delta2varepsilon} F (\Phi, A)$ of $F$ along $\varepsilon$ (at $(\Phi, A)$)} as an element of $\Omega^k(M; \phi^*V)$ by
%%\ba
%%\mleft( \delta_\varepsilon F (\Phi, A) \mright)_p(Y_1, \dotsc, Y_k)
%%&=
%%\mleft.\frac{\mathrm{D}_{\mleft(\Psi_\varepsilon(p), {}^E\nabla\mright)}}{\mathrm{d}t}\mright|_{t=0}
%%\mleft(
%%\mleft( F\mleft[ \widetilde{\Phi}_{\varepsilon, t} \mright] \mleft[\widetilde{A}_{\varepsilon, t}  \mright] \mright)_p (Y_1, \dotsc, Y_k)
%%\mright)
%%\ea
%%for all $p \in M$ and $Y_1, \dotsc, Y_k \in \mathrm{T}_p M$, where $\widetilde{A}_{\varepsilon} \in \Omega^1\mleft(M; \mleft( \widetilde{\Phi}_\varepsilon \mright)^*E\mright)$, $I \ni t \mapsto \widetilde{A}_{\varepsilon,t} \in \Omega^1\mleft(M; \mleft( \widetilde{\Phi}_{\varepsilon,t} \mright)^*E\mright)$, with 
%%\bas
%%\widetilde{A}_{\varepsilon,t=0} &= A, \\
%%\mleft.\frac{\mathrm{D}^j}{\mathrm{d}t}\mright|_{t=0} \mleft[ t \mapsto \mleft(\widetilde{A}_{\varepsilon,t}\mright)_p(Y) \mright]
%%&= \mleft(\delta_\varepsilon A\mright)_p(Y)
%%\eas
%%for all $p \in M$ and $Y \in \mathrm{T}_pM$, where $\mathrm{D}^1/\mathrm{d}t$ and $\mathrm{D}^2/\mathrm{d}t$ are defined like the differential operators in Eq.~\eqref{DefErsteVariation} and \eqref{DefZweiteVariation}, respectively, and ${}^E\nabla$ is chosen as defined in Def.~\ref{def:VariationenOfAundPhi} for $V = \mathrm{T}N$ and $V = E$.
%%\end{definitions}

\subsection{Curvature of gauge transformations}\label{CurvatureOfGaugePart1}

We want to calculate
\bas
\delta_\vartheta \delta_\varepsilon - \delta_\varepsilon \delta_\vartheta
\eas
for all $\varepsilon, \vartheta \in \mathcal{F}^0_E(M; {}^*E)$, and we want a behaviour similar to representations. For $\Phi \in C^\infty(M;N)$, $\Phi^*E$ is in general not a Lie algebroid, see \cite[\S 3.2ff.]{meinrenkensplitting} or \cite[\S 7.4; page 42ff.]{meinrenkenlie} about conditions on $\Phi$ which imply a natural Lie algebroid structure on $\Phi^*E$. Therefore we cannot expect to have a Lie bracket on sections of $\Phi^*E$. The essential problem is that we do not have an anchor on $\Phi^*E\to M$ in general such that one cannot try to construct first a bracket on pullbacks of sections and then to canonically extend such a bracket (similar to previous constructions), and this problem extends to $\mathcal{F}^0_E(M; {}^*E)$. But there is a better object measuring a "bracket-like" behaviour on this functional space; we will see at the end that this will be actually a Lie bracket.
%
%To understand the following definitions, beware that we need to think about that for calculating terms like $\delta_\vartheta \delta_\varepsilon A$ the section $\varepsilon$ is varied to because it has values in $\Phi^*E$. Hence, it will at least depend on $\Phi$, but we will assume the most difficult situation that $\varepsilon$ even depends on $A$, without assuming any explicit behaviour. That means that we view all sections $\varepsilon \in \Gamma(\Phi^*E)$ also as
%\bas
%(\phi, \omega)
%&\mapsto
%\varepsilon[\phi][\omega] \in \Gamma(\phi^*E)
%\eas
%for all $\phi \in C^\infty(M;N)$ and $\omega \in \Omega^1(M; \phi^*E)$; locally, with respect to a frame $\mleft( e_a \mright)_a$ of $E$,
%\bas
%(\phi, \omega)
%&\mapsto
%\varepsilon^a[\phi][\omega] ~ \phi^*e_a
%\eas
%for $\varepsilon^a[\phi][\omega] \in C^\infty(M)$; hence, $\varepsilon = \varepsilon [\Phi] [A]$ we also view smooth functions on $M$ as $\phi$- and $\omega$-dependent, in general all objects defined on $M$ can now depend on these two "coordinates", and for all of these we use a similar notation. Objects solely defined on $N$ will not depend on these new coordinates, because the idea is that the data over $N$ describes the fixed data of the variation. We will not give any explicit formula of that mapping because we will see that the essential results do not really depend on that. The only real explicit dependency needed is
%\bas
%C^\infty(M;N) \ni \phi \mapsto \phi^*\mu \in \Gamma(\phi^*E)
%\eas
%for all $\mu \in \Gamma(E)$, as already done before as an interpretation for terms like $\Phi^*\mu$ (similar for frames of other vector bundles over $N$). We denote variations now with $\delta (\Phi^* \mu) [\Phi]$ \textit{etc.}~instead of $\delta \mleft[C^\infty(M;N) \ni \phi \mapsto \phi^* \mu\mright] [\Phi]$ as before.

%\begin{definitions}{Sections as functionals}{SectionsAsFunctionals}
%Let $M, N$ be smooth manifolds, $E \to N$ a Lie algebroid, $V\to N$ a vector bundle (real, with finite rank), and $\Phi \in C^\infty(M;N)$. Then we call an $\mathbb{R}$-linear map $\iota: \Omega^k(M; \Phi^*V) \to \mathcal{F}_E^k(M;V)$ ($k \in \mathbb{N}_0$) with
%\ba\label{EqThatIsAllWeNeedForVariationsOfSections}
%\iota\mleft(\Phi^*\omega\mright)
%=
%\mleft[ C^\infty(M;N) \ni \phi \mapsto \phi^*\mu \mright],
%\ea
%for all $\omega \in \Omega^k(N;V)$, a \textbf{functional embedding}. 
%\end{definitions}
%
%\begin{remark}
%\leavevmode\newline
%On the subspace of Pull-backs of sections this definitions is clearly independent of $\Phi$, \textit{i.e.}~let $\Phi^\prime$ be another smooth map $M \to N$, and $\iota^\prime: \Omega^k(M; \Phi^*V) \to \mathcal{F}_E^k(M;V)$ another functional embedding, then clearly
%\bas
%\iota^\prime\mleft(\mleft(\Phi^\prime\mright)^*\omega\mright)
%&=
%\iota(\Phi^*\omega)
%\eas
%for all $\omega \in \Omega^k(N; V)$. We cannot show something similar for arbitrary sections, \textit{e.g.}~let $N=M=\mathbb{R}$, $V=E= N \times \mathbb{R}$, $k=0$, and $\mu_0$ a non-zero section of $E$. We define $\iota$ then by
%\bas
%\iota(f ~ \Phi^*\mu_0)
%\coloneqq
%\mleft[ C^\infty(M;N) \ni \phi \mapsto f ~ \phi^*\mu_0 \mright]
%\eas
%for all $f \in C^\infty(\mathbb{R})$, while we define $\iota^\prime$ by
%\bas
%\iota^\prime\mleft(f ~ \mleft( \Phi^\prime \mright)^*\mu_0\mright)
%&=
%\mleft[ C^\infty(M;N) \ni \phi \mapsto  \phi^*\mleft(f ~ \mu_0\mright) \mright].
%\eas
%Then in the case of $\Phi=\Phi^\prime$ clearly $\iota \neq \iota^\prime$ in general (except for very trivial cases like $\Phi = \mathds{1}_{\mathbb{R}}$). Therefore it may make sense to write $\iota_\Phi$ instead of $\iota$, but we will see that the results about the curvature of the infinitesimal gauge transformation will not be affected by that ambiguity due to the fact that we will just need Eq.~\eqref{EqThatIsAllWeNeedForVariationsOfSections}. $\Phi$ will be clear by the context.
%\end{remark}
%
%\begin{corollaries}{Existence of functional embedding}{ExistenceOfFunctionalEmbedding}
%Let $M, N$ be smooth manifolds, $E \to N$ a Lie algebroid, $V \to N$ a vector bundle, and $\Phi \in C^\infty(M;N)$. Then there exists a functional embedding $\iota: \Omega^k(M; \Phi^*V) \to \mathcal{F}_E^k(M;V)$ ($k \in \mathbb{N}_0$).
%\end{corollaries}
%
%\begin{proof}
%\leavevmode\newline
%We will define such a functional embedding by fixing a local frame $\mleft( e_a \mright)_a$ of $V$, \textit{i.e.}~that means that we will define a frame-dependent functional embedding. Given such a local frame, we define $\iota$ on that neighbourhood (denoted by $U \subset N$) by
%\bas
%\iota\underbrace{(\omega)}_{\mathclap{= \omega^a \otimes \Phi^*e_a}}
%&\coloneqq
%\mleft[ C^\infty(M;N) \ni \phi \mapsto \omega^a \otimes \phi^*e_a \mright]
%\eas
%for all $\omega \in \Omega^k(M; \Phi^*V)$
%\end{proof}
%
%\begin{definitions}{Infinitesimal gauge transformation of $\Gamma(\Phi^*E)$}{GaugeTrafosOfPullbacksections}
%Let $M, N$ be smooth manifolds, $E \to N$ a Lie algebroid, $\Phi \in C^\infty(M;N)$, $A \in \Omega^1(M; \Phi^*E)$, $\nabla$ a connection on $E$, and $\iota: \Gamma(\Phi^*E) \to \mathcal{F}_E^0(M;E)$ a functional embedding. Then we define the \textbf{infinitesimal gauge transformations $\delta_\varepsilon$ for sections of $\Phi^*E$ along $\varepsilon \in \Gamma(\Phi^*E)$ (at $(\Phi, A)$)} by
%\ba
%\delta_\varepsilon \vartheta (\Phi, A)
%\coloneqq
%\mleft(\delta_\varepsilon \iota(\vartheta)\mright)(\Phi, A)
%\ea
%for all $\vartheta \in \Gamma(\Phi^*E)$.
%\end{definitions}

\begin{definitions}{Pre-bracket on $\mathcal{F}^0_E(M; {}^*E)$}{PrebracketonPullbackLiealgebroid}
Let $M, N$ be smooth manifolds, $E \to N$ a Lie algebroid, and $\nabla$ a connection on $E$.

Then we define the \textbf{pre-bracket $\gls{1Delta}: \mathcal{F}^0_E(M; {}^*E) \times \mathcal{F}^0_E(M; {}^*E) \to \mathcal{F}^0_E(M; {}^*E)$} by
\ba
\Delta(\vartheta, \varepsilon)
&\coloneqq
\delta_\varepsilon \vartheta - \delta_\vartheta \varepsilon - \bigl( {}^*t_{\nabla^{\mathrm{bas}}} \bigr)\mleft( \vartheta, \varepsilon \mright)
%\mleft.\frac{\mathrm{D}_{\mleft(\Psi_{\varepsilon}(p), {}^E\nabla\mright)}}{\mathrm{d}t}\mright|_{t=0}
		%\biggl[ t \mapsto \vartheta_p\mleft[ \widetilde{\Phi}_{\varepsilon, t} \mright] \mleft[ \widetilde{A}_{\varepsilon, t} \mright] \biggr] \nonumber \\
%%%%%%%%%%%%%%%%%%%%%%%%%%%
%&\hspace{1cm}
	%- \mleft.\frac{\mathrm{D}_{\mleft(\Psi_{\vartheta}(p), {}^E\nabla\mright)}}{\mathrm{d}t}\mright|_{t=0}
	%\biggl[ t \mapsto \varepsilon_p\mleft[ \widetilde{\Phi}_{\vartheta, t} \mright] \mleft[ \widetilde{A}_{\vartheta, t} \mright] \biggr] \nonumber \\
%&\hspace{1cm}
	%- \mleft. \bigl( \Phi^*t_{{}^E\nabla} \bigr)\mleft( \vartheta, \varepsilon \mright)\mright|_p
\ea
for all $\varepsilon, \vartheta \in \mathcal{F}^0_E(M; {}^*E)$.
%When we want to emphasize which $E$-connection was used in the definition, then we write $\Delta^{(1)}$ when using $\nabla_\rho$ and $\Delta$ in the case of $\nabla^{\mathrm{bas}}$, thence,
%\bas
%\Delta(\vartheta, \varepsilon)
%&=
%\delta_\varepsilon \vartheta - \delta_\vartheta \varepsilon - \bigl( \Phi^*t_{{}^E\nabla} \bigr)\mleft( \vartheta, \varepsilon \mright).
%\eas
\end{definitions}

\begin{remark}\label{IdeaOfPrebracket}
\leavevmode\newline
Given an $E$-connection ${}^E\nabla$ on $E$, Lie brackets can be expressed as
\bas
\mleft[ \mu, \nu \mright]_E
&=
{}^E\nabla_\mu \nu
	- {}^E\nabla_\nu \mu
	- t_{{}^E\nabla}(\mu, \nu)
\eas
for all $\mu, \nu \in \Gamma(E)$. Recall that $\delta$ is strongly related to a certain pullback of $\nabla^{\mathrm{bas}}$; then the idea of the pre-bracket is to use the right-hand side as a definition. Since we know under which conditions and how to make pullbacks of $E$-connections and tensors, we circumvent the problem of defining a Lie bracket and anchor on a pullback bundle.
\end{remark}

Let us study this bracket.

\begin{propositions}{Properties of the pre-bracket}{PropertiesOfThePreBracket}
Let $M, N$ be smooth manifolds, $E \to N$ a Lie algebroid, and $\nabla$ a connection on $E$.

Then we have
\ba\label{DeltaIstZumGlueckANtisymm}
\Delta &\textit{ is antisymmetric}, \\
\Delta &\textit{ is $\mathbb{R}$-bilinear}, \\
%\Delta(\vartheta, f \varepsilon)_p
%&=
%f(p) ~ \Delta(\vartheta, \varepsilon)_p
	%+ \mleft.\frac{\mathrm{d}}{\mathrm{d}t} \mright|_{t=0} \biggl[ t \mapsto f \mleft[ \widetilde{\Phi}_{\vartheta, t} \mright] \mleft[ \widetilde{A}_{\vartheta, t} \mright](p) \biggr] ~ \varepsilon, \label{eqLeibnizRuleofPreBracket} \\
\Delta\mleft( {}^*\mu, {}^*\nu \mright)
&=
{}^*\bigl( \mleft[ \mu, \nu \mright]_E \bigr) \label{EqLieKlammerAufPullBackSections}
\ea
for all $\varepsilon, \vartheta \in \mathcal{F}^0_E(M; {}^*E)$, $f \in \mathcal{F}^0_E(M)$, $\mu, \nu \in \Gamma(E)$, and, when expressing everything with respect to a pull-back of a local frame $\mleft( e_a \mright)_a$ of $E$, we get
\ba\label{EqDeltaInFrameKoord}
\Delta\mleft( \vartheta, \varepsilon \mright)
&=
\delta_{\varepsilon} \vartheta^a ~ {}^*e_a
	- \delta_{\vartheta} \varepsilon^a ~ {}^*e_a
	+ \vartheta^a \varepsilon^b ~  {}^*\bigl( 
	\mleft[ e_a, e_b \mright]_E
	\bigr)
\ea
for all $\vartheta, \varepsilon \in \mathcal{F}^0_E(M; {}^*E)$.

Moreover, $\Delta(\vartheta, \varepsilon)$ is independent of the chosen connection $\nabla$ when both, $\varepsilon$ and $\vartheta$, are independent of $A$, that is, $\varepsilon(\Phi, A) = \varepsilon (\Phi, A^\prime)$ for all $(\Phi, A), (\Phi, A^\prime) \in \mathfrak{M}_E(M;N)$; similar for $\vartheta$.
\end{propositions}
%
%\begin{remark}
%\leavevmode\newline
%It is well-defined to speak about sections of $\Gamma(\Phi^*E)$ independent of $A$ as sections whose components are locally independent of $A$, because a change of the frame $\mleft(e_a\mright)_a$, $f_a = M_a^b e_b$ ($M_a^b$ an invertible matrix), would just contribute to $\Phi$-dependencies due to
%\bas
%\Phi^*\mleft( f_a \mright)
%&=
%\Phi^*\mleft( M_a^b e_b \mright)
%=
%\mleft( M_a^b \circ \Phi \mright) ~ \Phi^*e_b.
%\eas
%\end{remark}

\begin{remark}\label{ClassicalCommutatorRemark}
\leavevmode\newline
Eq.~\eqref{EqLieKlammerAufPullBackSections} and \eqref{EqDeltaInFrameKoord} emphasize that we have a suitable candidate in $\Delta$ as bracket. The latter actually proves that $\Delta$ is independent of the choice about whether or not one uses the basic connection to define $\delta_\varepsilon$ because the infinitesimal gauge transformation of scalar-valued functionals is just a Lie derivative, see also Remark \ref{RemarkBracketIsVeryIndependent}. Similar to how one can express $\mleft[ \cdot, \cdot \mright]_E$ using Lie algebroid connections as in Remark \ref{IdeaOfPrebracket}, but $\mleft[ \cdot, \cdot \mright]_E$ is of course independent of any choice of Lie algebroid connection.

Let $E = N \times \mathfrak{g}$ be an action Lie algebroid, the usual relationship in classical gauge theory is for $\varepsilon, \vartheta \in C^\infty(M;\mathfrak{g})$ that
\bas
\mleft[ \delta^{\mathrm{clas}}_\varepsilon, \delta^{\mathrm{clas}}_\vartheta \mright]A
&=
- \delta^{\mathrm{clas}}_{\mleft[\varepsilon, \vartheta\mright]_{\mathfrak{g}}}A,
\eas
where $\delta^{\mathrm{clas}}_\varepsilon$ is given by Def.~\ref{def:ClassFunctionalGaugeTrafoBlag}, and the negative sign on the right hand side is due to our choice of sign with respect to $\varepsilon$, which we prove later in full generality. As we discussed, we apply the "bookkeeping trick" to formulate infinitesimal gauge transformations, also recall Def.~\ref{def:InfinitesimalGaugeTrafoClassicAsConnection} and Thm.~\ref{thm:RecoverOfClassicInfgGaugeTrafo}. That is, for a constant frame $\mleft( e_a \mright)_a$ of $E$, we have the "bookkeeping trick" $\iota(\varepsilon)$ given by
\bas
\iota(\varepsilon)
&=
\varepsilon^a ~ {}^*e_a,
\eas
hence, the bookeeping trick is essentially a frame-dependent embedding of the functionals given in the classical gauge theory into $\mathcal{F}^\bullet_E$. $\varepsilon^a$ are in this case only functions depending on $M$, but not on $\mathfrak{M}_E(M;N)$, especially, $\delta^{\mathrm{clas}}_\vartheta \varepsilon^a = 0$. By Eq.~\eqref{EqDeltaInFrameKoord} we then have
\bas
\Delta\bigl( \iota(\vartheta), \iota(\varepsilon) \bigr)
&=
\vartheta^a \varepsilon^b ~ {}^*\bigl( 
	\mleft[ e_a, e_b \mright]_{\mathfrak{g}}
	\bigr)
=
\iota\mleft(\mleft[ \vartheta, \varepsilon \mright]_{\mathfrak{g}}\mright),
\eas
which is precisely what we want and expect of a generalized bracket.
\end{remark}

\begin{proof}[Proof of Prop.~\ref{prop:PropertiesOfThePreBracket}]
\leavevmode\newline
%The Leibniz rule \eqref{eqLeibnizRuleofPreBracket} is trivial and follows directly by the definition of $\mathrm{D}/\mathrm{d}t$, Prop.~\ref{prop:DerivationAlongEPath}.
The antisymmetry is clear, and the bilinearity follows by the linearity of $\delta_\varepsilon$ for all $\varepsilon \in \mathcal{F}^0_E(M; {}^*E)$, recall Cor.~\ref{cor:DeltaEpsilonIstLinearInEpsilon}. We have
\bas
\bigl( {}^*t_{\nabla^{\mathrm{bas}}} \bigr)\mleft( {}^*\mu, {}^*\nu \mright)
&=
{}^*\mleft( \bigl( t_{\nabla^{\mathrm{bas}}} \bigr)\mleft( \mu, \nu \mright) \mright)
=
{}^*\mleft( 
	\nabla^{\mathrm{bas}}_\mu \nu 
	- \nabla^{\mathrm{bas}}_\nu \mu 
	- \mleft[ \mu, \nu \mright]_E 
\mright)
\eas
for all $\mu, \nu \in \Gamma(E)$, and
\bas
\delta_{{}^*\nu} \mleft( {}^*\mu \mright)
&=
- {}^*\mleft( \nabla^{\mathrm{bas}}_\nu \mu \mright),
\eas
therefore
\bas
\Delta\mleft( {}^*\mu, {}^*\nu \mright)
&=
{}^*\mleft( \nabla^{\mathrm{bas}}_\mu \nu \mright)
	- {}^*\mleft( \nabla^{\mathrm{bas}}_\nu \mu \mright)
	- {}^*\mleft( 
	\nabla^{\mathrm{bas}}_\mu \nu 
	- \nabla^{\mathrm{bas}}_\nu \mu 
	- \mleft[ \mu, \nu \mright]_E 
\mright)
=
{}^*\bigl( \mleft[ \mu, \nu \mright]_E \bigr),
\eas
which proves Eq.~\eqref{EqLieKlammerAufPullBackSections}.
%We define the following notation
%\ba\label{DefKurzeNotationFuerDenGanzenScheiss}
%\mleft.\frac{\mathrm{d}}{\mathrm{d}t} \mright|_{t=0} \biggl[ t \mapsto f \mleft[ \widetilde{\Phi}_{-\vartheta, t} \mright] \mleft[ \widetilde{A}_{-\vartheta, t} \mright](p) \biggr]
%&\eqqcolon
%\dot{f}_{(\vartheta_p)}(0)
%\ea
%for all $f \in C^\infty(M)$, $\vartheta \in \Gamma(\Phi^*E)$ and $p \in M$. 
For $\varepsilon, \vartheta \in \mathcal{F}^0_E(M;{}^*E)$ we have, with respect to a frame $\mleft( e_a \mright)_a$ of $E$,
\bas
%\mleft.\frac{\mathrm{D}_{\mleft(\Psi_{-\vartheta}(p), {}^E\nabla\mright)}}{\mathrm{d}t}\mright|_{t=0}
	%\biggl[ t \mapsto \varepsilon_p\mleft[ \widetilde{\Phi}_{-\vartheta, t} \mright] \mleft[ \widetilde{A}_{-\vartheta, t} \mright] \biggr]
\delta_\vartheta \varepsilon
&=
\delta_{\vartheta} \varepsilon^a ~ {}^*e_a
	- \varepsilon^a \vartheta^b ~ {}^*\mleft(\nabla^{\mathrm{bas}}_{e_b} e_a\mright),
\eas
and so
\bas
\Delta(\vartheta, \varepsilon)
&=
\delta_{\varepsilon} \vartheta^a ~ {}^*e_a
	- \vartheta^a \varepsilon^b ~ {}^*\mleft(\nabla^{\mathrm{bas}}_{e_b} e_a\mright)
%\\
%&\hspace{1cm}
	- \delta_{\vartheta} \varepsilon^a ~ {}^*e_a
	+ \varepsilon^a \vartheta^b ~ {}^*\mleft(\nabla^{\mathrm{bas}}_{e_b} e_a\mright)
\\
&\hspace{1cm}
	- \vartheta^a \varepsilon^b ~ {}^*\mleft( 
	\nabla^{\mathrm{bas}}_{e_a} e_b 
	- \nabla^{\mathrm{bas}}_{e_b} e_a 
	- \mleft[ e_a, e_b \mright]_E
	\mright) \\
&=
\delta_{\varepsilon} \vartheta^a ~ {}^*e_a
	- \delta_{\vartheta} \varepsilon^a ~ {}^*e_a
	+ \vartheta^a \varepsilon^b ~ 
	{}^*\bigl( 
		\mleft[ e_a, e_b \mright]_E
	\bigr).
\eas
This expression for $\Delta(\vartheta, \varepsilon)$ shows that its value is independent of the chosen $\nabla$, when the functionals $\varepsilon = \varepsilon^a \otimes {}^*e_a$ and $\vartheta= \vartheta^a \otimes {}^*e_a$ are independent of $A$, since then also their components with respect to $\mleft({}^*e_a\mright)_a$ are independent of $A$ because ${}^*e_a$ is already independent of $A$. Then apply Cor.~\ref{cor:WennVonAUnabhaengigDannAuchVonNabla}.
\end{proof}

\begin{corollaries}{$\Delta$ a Lie bracket on the pull-backs of $\Gamma(E)$}{DeltaIstEineLieklammerAufPullbACkSections}
Let $M, N$ be smooth manifolds, $E \to N$ a Lie algebroid, and $\nabla$ a connection on $E$.

Then the restriction of $\Delta$ on pullback functionals is a Lie bracket.
\end{corollaries}

\begin{proof}
\leavevmode\newline
The antisymmetry, the bilinearity over $\mathbb{R}$ and the closedness follow by Prop.~\ref{prop:PropertiesOfThePreBracket}, the same also for the Jacobi identity by observing
\bas
\Delta \mleft( {}^*\mu, \Delta \mleft( {}^*\nu, {}^*\eta \mright) \mright)
&\stackrel{\text{Eq.~\eqref{EqLieKlammerAufPullBackSections}}}{=}
\Delta \mleft( {}^*\mu, {}^*\mleft(\mleft[ \nu,\eta \mright]_E\mright) \mright)
\stackrel{\text{Eq.~\eqref{EqLieKlammerAufPullBackSections}}}{=}
{}^* \mleft( \mleft[ \mu, \mleft[ \nu, \eta \mright]_E \mright]_E\mright)
\eas
for all $\mu, \nu, \eta \in \Gamma(E)$. Hence, the Jacobiator of the restriction of $\Delta$ on pullback functionals is given by the pullback of the Jacobiator of $\mleft[ \cdot, \cdot \mright]_E$, the latter is of course zero.
\end{proof}

We will see that $\Delta$ is actually always a Lie bracket, but for proving this we do not want to show the Jacobi identity directly, due to how we constructed it we rather are going to use the equivalence with Bianchi identities of curvatures; recall the proof of Thm.~\ref{thm:1stBianchi}. Hence, let us define the curvature we are interested into.

\begin{definitions}{Curvature of infinitesimal gauge transformations}{ErsteKruemmungsFormelFuerEichtrafos}
Let $M, N$ be smooth manifolds, $E \to N$ a Lie algebroid, $V \to N$ a vector bundle, $\nabla$ a connection on $E$, and ${}^E\nabla$ an $E$-connection on $V$.

Then we define the \textbf{curvature $\gls{Rdelta}$} by
\ba
\mathcal{F}^0_E(M;{}^*E) \times \mathcal{F}^0_E(M;{}^*E) \times \mathcal{F}^k_E(M;{}^*V) &\to \mathcal{F}^k_E(M;{}^*V)
\nonumber \\
(\vartheta, \varepsilon, L) &\mapsto R_{\delta}(\vartheta, \varepsilon)L
\coloneqq
	\delta_\vartheta \delta_\varepsilon L 
	- \delta_\varepsilon \delta_\vartheta L 
	+ \delta_{\Delta(\vartheta, \varepsilon)} L
\ea
for all $\vartheta, \varepsilon \in \mathcal{F}^0_E(M; {}^*E)$ and $L \in \mathcal{F}^k_E(M;{}^*V)$ ($k \in \mathbb{N}_0$ arbitrary).

In alignment to Def.~\ref{def:GaugeTrafoOfA} we denote $R_\delta(\cdot, \cdot) A \coloneqq R_\delta (\cdot, \cdot) \varpi_2$, and $R_\delta(\cdot, \cdot)A^a \coloneqq R_\delta(\cdot, \cdot)\varpi_2^a$ with respect to a frame $\mleft(e_a\mright)_a$ of $E$.
\end{definitions}

\begin{remark}
\leavevmode\newline
The sign in front of the third term depends on which sign one takes in the definition of $\delta_\varepsilon$. Changing the sign $\varepsilon$ in the definitions of the gauge tranformations would lead to a minus sign in front of the third summand.
\end{remark}

Using a frame of $E$ we can apply the Leibniz rule.

\begin{corollaries}{Relationships between curvatures}{RelationShipsOfCurvatures}
Let $M, N$ be smooth manifolds, $E \to N$ a Lie algebroid, $V \to N$ a vector bundle, $\nabla$ a connection on $E$, and ${}^E\nabla$ an $E$-connection on $V$. Then locally
\ba
R_\delta(\cdot, \cdot)L
&=
R_\delta(\cdot, \cdot)L^a \otimes {}^*e_a
	+ L^a \otimes {}^*\bigl( R_{{}^E\nabla}(\cdot, \cdot)e_a \bigr) 
\ea
for all $L \in \mathcal{F}^k_E(M; {}^*V)$ ($k \in \mathbb{N}_0$), where $\mleft( e_a \mright)_a$ is a local frame of $E$ and viewing $R_{{}^E\nabla}(\cdot, \cdot)e_a$ as an element of $\Omega^2(E;E)$.
\end{corollaries}

\begin{proof}
\leavevmode\newline
Let us first study terms like $R_{\delta}(\vartheta, \varepsilon)\mleft({}^*h\mright)$ for $\varepsilon, \vartheta \in \mathcal{F}^0_E(M;{}^*E)$ and $h \in \Gamma(V)$, using a local frame $\mleft( e_a \mright)_a$ of $E$,
\bas
\delta_\vartheta \delta_\varepsilon ({}^*h)
&=
- \delta_\vartheta \mleft(
	\varepsilon^a ~ {}^*\mleft({}^E\nabla_{e_a} h\mright)
\mright)
=
- \delta_\vartheta \varepsilon^a ~ {}^*\mleft({}^E\nabla_{e_a} h\mright)
	+ \varepsilon^a \vartheta^b ~ {}^*\mleft({}^E\nabla_{e_b} {}^E\nabla_{e_a} h\mright),
\eas
and
\bas
\delta_{\Delta(\vartheta, \varepsilon)} ({}^*h)
~~~~&\stackrel{\mathclap{\text{Eq.~\eqref{EqDeltaInFrameKoord}}}}{=}~~~~
- \mleft(
	\delta_{\varepsilon} \vartheta^a
	- \delta_{\vartheta} \varepsilon^a
	+ \vartheta^b ~ \varepsilon^c ~  \mleft({}^*\bigl( 
	\mleft[ e_b, e_c \mright]_E
	\bigr)\mright)^a
\mright) ~ {}^*\mleft({}^E\nabla_{e_a} h\mright)
\\
&=
\delta_\vartheta \varepsilon^a ~ {}^*\mleft({}^E\nabla_{e_a} h\mright)
	- \delta_\varepsilon \vartheta^a ~ {}^*\mleft({}^E\nabla_{e_a} h\mright)
	- \varepsilon^a \vartheta^b ~ {}^*\mleft( {}^E\nabla_{\mleft[ e_b, e_a \mright]_E} h \mright),
\eas
in total
\bas
R_{\delta}(\vartheta, \varepsilon)\mleft({}^*h\mright)
&=
\varepsilon^a \vartheta^b ~ {}^* \underbrace{\mleft(
	{}^E\nabla_{e_b} {}^E\nabla_{e_a} h
	- {}^E\nabla_{e_a} {}^E\nabla_{e_b} h
	- {}^E\nabla_{\mleft[ e_a, e_b \mright]_E} h
\mright)}_{R_{{}^E\nabla}(e_b,e_a)h}
=
\mleft(
	{}^*\bigl( R_{{}^E\nabla}(\cdot,\cdot)h \bigr)
\mright)(\vartheta, \varepsilon).
\eas
Therefore we arrive at
\bas
R_\delta(\vartheta, \varepsilon)\mleft(L^a \otimes {}^*e_a\mright)
&=
\delta_\vartheta \delta_\varepsilon L^a \otimes {}^*e_a
	+ \delta_\varepsilon L^a \otimes \delta_\vartheta\mleft({}^*e_a\mright)
	+ \delta_\vartheta L^a \otimes \delta_\varepsilon\mleft({}^*e_a\mright)
	+ L^a \otimes \delta_\vartheta \delta_\varepsilon \mleft({}^*e_a\mright)
\\
&\hspace{1cm}
	- (\vartheta \leftrightarrow \varepsilon)
\\
&\hspace{1cm}
	+ \delta_{\Delta(\vartheta, \varepsilon)} L^a \otimes {}^*e_a
	+ L^a \otimes \delta_{\Delta(\vartheta, \varepsilon)} {}^*e_a
\\
&=
R_\delta(\vartheta, \varepsilon)L^a \otimes {}^*e_a
	+ L^a \otimes R_\delta(\vartheta, \varepsilon) ({}^*e_a)
\\
&=
R_\delta(\cdot, \cdot)L^a \otimes {}^*e_a
	+ L^a \otimes \mleft({}^*\bigl( R_{{}^E\nabla}(\cdot, \cdot)e_a \bigr) \mright)(\vartheta, \varepsilon)
\eas
for all $L \in \mathcal{F}^k_E(M; {}^*V)$.
\end{proof}

Keep in mind that $R_\delta$ is not a typical curvature, for example $\delta_\varepsilon$ is not $C^\infty$-linear with respect to $\varepsilon$, such that it is not immediately clear whether this curvature is a tensor in all arguments, so, we need to prove this if we want to simplify calculations. We are first focusing on $R_\delta(\cdot, \cdot) A$. 

\begin{propositions}{$R_{\delta}$ is a tensor}{WirHabenEinenTensorBeiderTrafoKruemmung}
Let $M, N$ be smooth manifolds, $E \to N$ a Lie algebroid, and $\nabla$ a connection on $E$.

Then $R_{\delta}(\cdot, \cdot)A$ is an anti-symmetric tensor, \textit{i.e.}~anti-symmetric and $\mathcal{F}^0_E(M)$-bilinear, and we have
\ba\label{SplittingVonDerEichtrafo}
%R_{\delta}({}^*\mu, {}^*\nu)A
%&=
%\bigl(R_{\delta}({}^*\mu, {}^*\nu)A^a\bigr)
	%\otimes {}^*e_a
	%+ \mleft({}^*\bigl( R_{\nabla^{\mathrm{bas}}}(\mu, \nu)\bigr) \mright) A 
%\ea
%for all $\mu, \nu \in \Gamma(E)$, where $\mleft( e_a \mright)_a$ is a local frame of $E$, and
%%such that we write $A^a \otimes {}^*e_a$, and 
%we view $R_{\nabla^{\mathrm{bas}}}(\mu,\nu)$ as an element of $\Omega^1(E;E)$. If we introduce the notation $R_{\nabla^{\mathrm{bas}}}(\mu, \nu, X) \coloneqq R_{\nabla^{\mathrm{bas}}}(\mu, \nu) X$ for all $\mu, \nu \in \Gamma(E)$ and $X \in \mathfrak{X}(N)$, then we have a more general formula given by 
%\ba\label{SplittingVonDerEichtrafoGeneral}
R_{\delta}(\varepsilon, \vartheta)A
&=
R_{\delta}(\varepsilon, \vartheta)A^a
	\otimes {}^*e_a
	+ \mleft({}^* R_{\nabla^{\mathrm{bas}}} \mright)( \varepsilon, \vartheta) A
\ea
for all $\varepsilon, \vartheta \in \mathcal{F}^0_E(M; {}^*E)$.
\end{propositions}

\begin{proof}
\leavevmode\newline
%For simplicity we will omit to add $(\Phi, A)$ at the gauge transformations.
$\bullet$ The antisymmetry is clear by Prop.~\ref{prop:PropertiesOfThePreBracket}. Fix a local frame $\mleft( e_a \mright)_a$ of $E$, then we have
\bas
\delta_\vartheta \delta_{f\varepsilon}A
~~~~&\stackrel{\mathclap{\text{Def.~\ref{def:GaugeTrafoOfA}}}}{=}~~~~
- \delta_\vartheta \bigl( \mleft({}^*\nabla \mright) (f \varepsilon) \bigr) \\
&=
- \delta_\vartheta \bigl( \mathrm{d}f \otimes \varepsilon 
+ f ~ \mleft({}^*\nabla \mright) \varepsilon \bigr) \\
&=
- \delta_\vartheta \mathrm{d} f \otimes \varepsilon
	- \mathrm{d} f \otimes \delta_\vartheta \varepsilon
%- \delta_\vartheta \bigl( \mathrm{d}f ~ \varepsilon^a \bigr) \otimes {}^*e_a 
	%+ \mathrm{d}f ~ \varepsilon^a  \otimes {}^*\mleft(\nabla^{\mathrm{bas}}_\vartheta e_a \mright)
	- \delta_\vartheta f ~ \mleft({}^*\nabla \mright) \varepsilon 
	- f \delta_\vartheta \bigl(\mleft({}^*\nabla \mright) \varepsilon \bigr)
\\
&=
- \delta_\vartheta \mathrm{d} f \otimes \varepsilon
	- \mathrm{d} f \otimes \delta_\vartheta \varepsilon^a ~ {}^*e_a
	+ \mathrm{d} f \otimes \varepsilon^a \vartheta^b ~ {}^*\mleft(\nabla^{\mathrm{bas}}_{e_b} e_a \mright)
	- \delta_\vartheta f ~ \mleft({}^*\nabla \mright) \varepsilon 
	+ f \delta_\vartheta \delta_\varepsilon A
\eas
for all $\vartheta, \varepsilon \in \mathcal{F}^0_E(M; {}^*E)$ and $f \in \mathcal{F}^0_E(M)$, and
\bas
-\delta_{f\varepsilon} \delta_\vartheta A
&=
\delta_{f\varepsilon} \bigl( \mleft( {}^*\nabla \mright) \vartheta \bigr) \\
&=
\delta_{f\varepsilon} \mleft( \mathrm{d}\vartheta^a \otimes {}^*e_a + \vartheta^b ~ {}^! \mleft( \nabla e_b \mright) \mright) \\
&\stackrel{\mathclap{\text{Eq.~\eqref{EqVariationVonFormenBrrrr}}}}{=}~~~~
\delta_{f\varepsilon} \mathrm{d}\vartheta^a \otimes {}^*e_a 
	- \mathrm{d}\vartheta^a \otimes f \varepsilon^b~ {}^*\mleft( \nabla^{\mathrm{bas}}_{e_b} e_a \mright) \\
&\hspace{1cm}
	+ \delta_{f\varepsilon} \vartheta^b ~ {}^!\mleft( \nabla e_b \mright) 
	- f \vartheta^b ~ {}^!\mleft( \nabla^{\mathrm{bas}}_{\varepsilon} (\nabla e_b) \mright)
	- \vartheta^b ~\underbrace{{}^*\mleft( \nabla_{({}^*\rho)\mleft( ({}^*\nabla) (f \varepsilon) \mright)} e_b \mright)}
	_{\mathclap{= \mathrm{d}f \otimes {}^*\mleft( \nabla_{\mleft({}^*\rho\mright)(\varepsilon)} e_b \mright)
		+ f \cdot \underbrace{(\dotsc)}_{\mathclap{\text{indep. of }f}}
		%{}^*\mleft( \nabla_{({}^*\rho)\mleft( ({}^*\nabla) \varepsilon \mright)} e_b \mright)
		}}
\\
&=
\delta_{f\varepsilon} \mathrm{d}\vartheta^a \otimes {}^*e_a 
	+ \delta_{f\varepsilon} \vartheta^b ~ {}^! \mleft( \nabla e_b \mright) 
	- \vartheta^b \varepsilon^a\mathrm{d}f \otimes {}^*\mleft( \nabla_{\rho(e_a)} e_b \mright)
	+ f \cdot \underbrace{(\dotsc)}_{\mathclap{\text{independent of }f}}.
\eas
Since we want to check the tensorial property, we can ignore the terms proportional to $f$; 
we also have
\bas
\delta_{\Delta(\vartheta, f \varepsilon)} A
&=
\mleft( {}^*\nabla \mright)  \mleft( \Delta(f \varepsilon, \vartheta) \mright) \\
%&=
%\mleft( {}^*\nabla \mright) 
%\bigl(
		%\delta_\vartheta (f \varepsilon)
		%- \delta_{f \varepsilon} \vartheta
		%- \bigl( {}^* t_{\nabla^{\mathrm{bas}}} \bigr)(\vartheta, f\varepsilon)
	%%\delta_{f\varepsilon} \vartheta^a(\Phi, A) ~ {}^*e_a
	%%- \delta_{\vartheta} \mleft(f \varepsilon^a\mright)(\Phi, A) ~ {}^*e_a
	%%+ \vartheta^b ~ f \varepsilon^a ~ {}^*\bigl( 
	%%\mleft[ e_b, e_a \mright]_E
	%%\bigr)
%\bigr) \\
%&=
%\mleft( {}^*\nabla \mright) 
%\mleft(
		%\delta_\vartheta f ~ \varepsilon
		%+ f \delta_\vartheta \varepsilon^a ~ {}^*e_a
		%\vphantom{{}^*\mleft( \nabla^{\mathrm{bas}}_{e_b} e_a \mright)} 
		%- \delta_{f \varepsilon} \vartheta^b ~ {}^*e_b \mright.
%\\
%&\hspace{2cm}\mleft.
		%+ f \varepsilon^a  \vartheta^b ~ {}^*\mleft( \nabla^{\mathrm{bas}}_{e_b} e_a \mright)
		%- f \varepsilon^a \vartheta^b ~ {}^*\mleft( \nabla^{\mathrm{bas}}_{e_a} e_b \mright)
		%- f \varepsilon^a \vartheta^b ~ \bigl( {}^* t_{\nabla^{\mathrm{bas}}} \bigr)({}^*e_a, {}^*e_b)
%\mright)
%\\
&\stackrel{\mathclap{\text{Eq.~\eqref{EqDeltaInFrameKoord}}}}{=}~~~~
\mleft( {}^*\nabla \mright)
\mleft(
		\delta_\vartheta f ~ \varepsilon
		+ f \delta_\vartheta \varepsilon^a ~ {}^*e_a 
		- \delta_{f \varepsilon} \vartheta^b ~ {}^*e_b
		+ f \varepsilon^a \vartheta^b~ {}^*\mleft( \mleft[ e_a, e_b \mright]_E \mright)
\mright)
\\
&\stackrel{\mathclap{\text{Eq.~\eqref{eqVariationVertauschtMitDifferential}}}}{=}~~~~
\delta_\vartheta \mathrm{d} f \otimes \varepsilon
	+ \delta_\vartheta f ~ ({}^*\nabla)\varepsilon
	+ \mathrm{d} f \otimes \delta_\vartheta \varepsilon^a ~ {}^*e_a 
	- \delta_{f \varepsilon}\mathrm{d} \vartheta^b \otimes {}^*e_b
	- \delta_{f \varepsilon} \vartheta^b ~ {}^!\mleft( \nabla e_b \mright)
\\
&\hspace{1cm}~~~~
	+ \varepsilon^a \vartheta^b~\mathrm{d}f \otimes {}^*\mleft( \mleft[ e_a, e_b \mright]_E \mright)
	+ f \cdot \underbrace{(\dotsc)}_{\mathclap{\text{independent of }f}}.
\eas
Hence, we get in total
\bas
R_{\delta}(\vartheta, f \varepsilon)A
&=
\varepsilon^a \vartheta^b \mathrm{d} f \otimes {}^*\underbrace{\mleft( 
	\nabla^{\mathrm{bas}}_{e_b} e_a
	- \nabla_{\rho(e_a)} e_b
	+ \mleft[ e_a, e_b \mright]_E
\mright)}_{= \nabla^{\mathrm{bas}}_{e_b} e_a - \nabla^{\mathrm{bas}}_{e_b} e_a = 0}
	+ f \cdot \underbrace{(\dotsc)}_{\mathclap{\text{independent of }f}}
\\
&=
f \cdot \underbrace{(\dotsc)}_{\mathclap{\text{independent of }f}}
\eas
for all $\vartheta, \varepsilon \in \mathcal{F}^0_E(M; {}^*E)$ and $f \in \mathcal{F}^0_E(M)$. Using the antisymmetry proves that $R_{\delta}(\cdot, \cdot)A$ is a tensor because the shown equation also holds for $f \equiv 1$ such that the remaining terms in the $f$-independent bracket are precisely giving rise to $R_{\delta}(\vartheta, \varepsilon)A$.

$\bullet$ Eq.~\eqref{SplittingVonDerEichtrafo} just follows by Cor.~\ref{cor:RelationShipsOfCurvatures}.
%Now we write $A = A^a \otimes {}^*e_a$, then
%\bas
%\delta_{\Delta(\varepsilon, \vartheta)} A
%&=
%\delta_{\Delta(\varepsilon, \vartheta)} A^a \otimes {}^*e_a
	%- \varepsilon^a \vartheta^b A^a \otimes {}^*\mleft( \nabla^{\mathrm{bas}}_{\mleft[ e_a, e_b \mright]_E} e_a \mright),
%\eas
%and
%\bas
%\delta_{{}^*\mu} \delta_{{}^*\nu} A
%&=
%\delta_{{}^*\mu} \mleft(  
	%\delta_{{}^*\nu} A^a \otimes {}^*e_a
	%- A^a \otimes {}^*\mleft( \nabla^{\mathrm{bas}}_\nu e_a \mright)
%\mright)
%\\
%&=
%\delta_{{}^*\mu} \delta_{{}^*\nu} A^a \otimes {}^*e_a
	%- \delta_{{}^*\nu} A^a \otimes {}^*\mleft( \nabla^{\mathrm{bas}}_\mu e_a \mright)
%\\
%&\hspace{1cm}
	%- \delta_{{}^*\mu} A^a \otimes {}^*\mleft( \nabla^{\mathrm{bas}}_\nu e_a \mright)
	%+ A^a \otimes {}^*\mleft( \nabla^{\mathrm{bas}}_\mu \nabla^{\mathrm{bas}}_\nu e_a \mright).
%\eas
%In total the mixed terms cancel, and we get
%\bas
%R_{\delta}({}^*\mu, {}^*\nu)A
%&=
%\mleft(
%\delta_{{}^*\mu} \delta_{{}^*\nu} A^a
	%- \delta_{{}^*\nu}\delta_{{}^*\mu} A^a
	%+ \delta_{{}^*\mleft( \mleft[ \mu, \nu \mright]_E \mright)} A^a
	%\mright) \otimes {}^*e_a 
	%+ A^a \otimes {}^*\bigl( R_{\nabla^{\mathrm{bas}}}(\mu, \nu) e_a \bigr).
	%%_{= \mleft( {}^*\mleft( R_{\nabla^{\mathrm{bas}}}(\mu, \nu) \mright) \mright) A}.
%\eas
%In the same fashion one shows Eq.~\eqref{SplittingVonDerEichtrafoGeneral}, making use of expressions like $\varepsilon = \varepsilon^a \otimes {}^*e_a$ to derive the second summand.
\end{proof}

Due to the tensorial behaviour, we can study $R_{\delta}(\cdot, \cdot)A$ just with respect to pullback functionals, such that the notations and calculations can be simplified.

\begin{theorems}{Curvature of the infinitesimal gauge transformation measured by the basic curvature}{CurvatureOfBasicStuffIsEquivalentForGaugeTrafoCurvature}
Let $M, N$ be smooth manifolds, $E \to N$ a Lie algebroid, and $\nabla$ a connection on $E$. Then
\ba
R_{\delta}({}^*\mu, {}^*\nu)A
&=
- {}^!\mleft( R^{\mathrm{bas}}_\nabla(\mu, \nu) \mright)
\ea
for all $\mu, \nu \in \Gamma(E)$, viewing $R^{\mathrm{bas}}_\nabla(\mu, \nu)$ as an element of $\Omega^1(N;E)$.
\end{theorems}

\begin{remark}
\leavevmode\newline
\indent $\bullet$ One can then derive with Eq.~\eqref{EqPullBackFormelFuerVerschiedeneDefinitionen} that
\bas
{}^!\mleft( R^{\mathrm{bas}}_\nabla(\mu, \nu) \mright)
&=
\mleft({}^*\mleft( R^{\mathrm{bas}}_\nabla(\mu, \nu) \mright)\mright) \mathrm{D}
=
\mleft({}^* R^{\mathrm{bas}}_\nabla\mright) ({}^*\mu, {}^*\nu) \mathrm{D},
\eas
viewing $\mathrm{D}$ as an element of $\mathcal{F}^1_E(M; {}^*\mathrm{T}N)$; recall Ex.~\ref{ex:DAsFunctional}.
Using that $R_\delta(\cdot, \cdot) A$ is tensorial and that pullbacks are generators as usual, we get
\bas
R_{\delta}(\varepsilon, \vartheta)A
&=
- \mleft({}^* R^{\mathrm{bas}}_\nabla\mright) (\varepsilon, \vartheta) \mathrm{D}
\eas
for all $\varepsilon, \vartheta \in \mathcal{F}^0_E(M; {}^*E)$.

$\bullet$ One could also view this theorem as a physical interpretation of the basic curvature.
\end{remark}

\begin{proof}[Proof of Thm.~\ref{thm:CurvatureOfBasicStuffIsEquivalentForGaugeTrafoCurvature}]
\leavevmode\newline
We have
\bas
\delta_{{}^*\mu} \mleft( \delta_{{}^*\nu} A \mright)
&=
- \delta_{{}^*\mu} \mleft( {}^!\mleft( \nabla \nu \mright) \mright)
\\
&\stackrel{\mathclap{\text{Eq.~\eqref{EqVariationVonFormenBrrrrVereinfacht}}}}{=}~~~~
	%\mleft({}^*\mleft(\nabla^{\mathrm{bas}}_\mu \mleft( \nabla \nu \mright)\mright)\mright)(\mathrm{D}\Phi)
	%+ {}^*\mleft( \nabla_{\rho\mleft(\nabla_{\mathrm{D}\Phi} \mu \mright)} \nu \mright)
%\\
%&=
{}^! \mleft(
	\nabla^{\mathrm{bas}}_\mu \mleft( \nabla \nu \mright)
	+ \nabla_{\rho\mleft(\nabla \mu \mright)} \nu
\mright),
\eas
and
\bas
\mleft(\nabla^{\mathrm{bas}}_\mu \mleft( \nabla \nu \mright)
	+ \nabla_{\rho\mleft(\nabla \mu \mright)} \nu\mright)(Y)
&=
\nabla^{\mathrm{bas}}_\mu \nabla_Y \nu
	- \nabla_{\nabla^{\mathrm{bas}}_\mu Y} \nu
	+ \nabla_{\rho\mleft(\nabla_Y \mu \mright)} \nu
\\
&=
\mleft[ \mu, \nabla_Y \nu \mright]_E
	+ \nabla_{\rho\mleft( \nabla_Y \nu \mright)} \mu
	- \nabla_{\mleft[ \rho(\mu), Y \mright]} \nu
\eas
for all $Y \in \mathfrak{X}(M)$. In total we would then look at the pull-back of the following form, also using Eq.~\eqref{EqLieKlammerAufPullBackSections},
\bas
&\mleft(
	\nabla^{\mathrm{bas}}_\mu \mleft( \nabla \nu \mright)
	+ \nabla_{\rho\mleft(\nabla \mu \mright)} \nu
	- \nabla^{\mathrm{bas}}_\nu \mleft( \nabla \mu \mright)
	- \nabla_{\rho\mleft(\nabla \nu \mright)} \mu
	- \nabla \mleft( \mleft[ \mu, \nu \mright]_E \mright)
\mright)(Y)
\\
&=
\mleft[ \mu, \nabla_Y \nu \mright]_E
	+ \nabla_{\rho\mleft( \nabla_Y \nu \mright)} \mu
	- \nabla_{\mleft[ \rho(\mu), Y \mright]} \nu
	-	\mleft[ \nu, \nabla_Y \mu \mright]_E
	- \nabla_{\rho\mleft( \nabla_Y \mu \mright)} \nu
	+ \nabla_{\mleft[ \rho(\nu), Y \mright]} \mu
	- \nabla_Y \mleft( \mleft[ \mu, \nu \mright]_E \mright)
\\
&=
- \mleft(
	\nabla_Y\mleft(\mleft[\mu, \nu\mright]_E\mright) 
	- \mleft[ \nabla_Y \mu, \nu \mright]_E 
	- \mleft[ \mu, \nabla_Y \nu \mright]_E 
	- \nabla_{\nabla^{\mathrm{bas}}_\nu Y} \mu 
	+ \nabla_{\nabla^{\mathrm{bas}}_\mu Y} \nu
\mright)
\\
&\stackrel{\mathclap{\text{Def.~\ref{def:basiccurvature}}}}{=}~~~~
- R^{\mathrm{bas}}_\nabla(\mu, \nu)Y.
\eas
Therefore we arrive at
\bas
R_{\delta}({}^*\mu, {}^*\nu)A
&=
- {}^!\mleft( R^{\mathrm{bas}}_\nabla(\mu, \nu)Y \mright).
\eas
\end{proof}

We get immediately the following statement.

\begin{corollaries}{Flat infinitesimal gauge transformation}{FlatnessVonEichtrafos}
Let $M, N$ be smooth manifolds, $E \to N$ a Lie algebroid, and $\nabla$ a connection on $E$ with $R^{\mathrm{bas}}_\nabla=0$. Then
\ba\label{DieKruemmungIstNullVonDenEichtrafosGeeeeil}
R_{\delta}(\cdot, \cdot)A
&=
0.
\ea
%while
%\ba
%R_{\delta^{(1)}}\mleft( {}^*\mu, {}^*\nu \mright)A
%&=
%\mleft( {}^*\bigl( R_{\nabla_{\rho}}(\mu, \nu)  \bigr) \mright) A
%\ea
%for all $\mu, \nu \in \Gamma(E)$.
With respect to a frame $\mleft( e_a \mright)_a$ of $E$ we then also have
\ba\label{CoordFuerEichKruemmungsRegel}
R_\delta (\cdot, \cdot) A^a
&=
0
\ea
for all $\mu, \nu \in \Gamma(E)$.
\end{corollaries}

\begin{remark}\label{RemarkUeberNablaRhoCurvatureForGauegTrafo}
\leavevmode\newline
\indent $\bullet$ This discussion, especially Cor.~\ref{cor:FlatnessVonEichtrafos} and Thm.~\ref{thm:CurvatureOfBasicStuffIsEquivalentForGaugeTrafoCurvature}, are generalizations of statements in \cite[especially Prop.~8 and Thm.~1]{EichtrafoKruemmungUrspruenglich} and \cite[especially Eq.~9, 10 and 11; there the $S$ denotes the basic curvature]{mayerlieAuchEichtrafoStuff}.\footnote{The sign of $\varepsilon$ in the gauge transformations there is the opposite of our sign.} In both of these works a coordinate-free formulation of $\delta_\varepsilon A$ was not known, just $\delta_\varepsilon A^a$. It was known that $\delta_\varepsilon A^a$ is dependent on coordinates, but not how it can be written/defined such that it is again an element of $\Omega^1(M; \Phi^*E)$. \cite{EichtrafoKruemmungUrspruenglich} tries to formulate infinitesimal gauge transformations in a covariant way with a completely different approach by assuming a weaker form of equality, but only for a special situation and only for $\varepsilon$ as an element of $\Phi^*(\Gamma(E))$ (\textit{i.e.}~they only looked at pullback functionals, when we express that in our language). \cite{mayerlieAuchEichtrafoStuff} looks at the set $\Gamma(\Phi^*E)$ for $\varepsilon$ but assumes that $\varepsilon^a$ is independent of $\Phi$ and $A$ which is clearly a coordinate-dependent description, because a change of the pull-back frame would introduce a $\Phi$-dependency of the components $\varepsilon^a$ (in our words, they choose a coordinate-dependent embedding of $\Gamma(\Phi^*E)$ as functionals). In one way or the other, both works arrive at Eq.~\eqref{CoordFuerEichKruemmungsRegel}, but only evaluated at pullback functionals, that is, $R_\delta({}^*\mu, {}^*\nu)A^a=0$ for all $\mu, \nu \in \Gamma(E)$.

What we provide is a coordinate-independent and -free definition of such infinitesimal gauge transformations. Moreover, we have generalized Eq.~\eqref{CoordFuerEichKruemmungsRegel} in form of Eq.~\eqref{DieKruemmungIstNullVonDenEichtrafosGeeeeil}, in sense of not only assuming pullback functionals by defining the pre-bracket $\Delta$.

$\bullet$ Recall Remark \ref{WhyNablaBasPartOne}: One could also take $\nabla_\rho$ to define $\delta_\varepsilon$. It has the advantage that then $\delta_\varepsilon A$ directly restricts to the standard formula when restricting ourselves to the classical setting. When defining and calculating $R_\delta$ in a similar manner,
 %one probably still gets Eq.~\eqref{CoordFuerEichKruemmungsRegel} because that is for scalar-valued functionals $A^a$ whose infinitesimal gauge transformation is independent of whether one uses $\nabla_\rho$ or $\nabla^{\mathrm{bas}}$ on $E$-valued functionals. But 
we also get Eq.~\eqref{SplittingVonDerEichtrafo} where the curvature-term will be replaced with the curvature of $\nabla_\rho$ due to Cor.~\ref{cor:RelationShipsOfCurvatures}. Therefore one needs to impose at least flatness of $\nabla_\rho$ in order to get a similar result like Eq.~\eqref{DieKruemmungIstNullVonDenEichtrafosGeeeeil}; actually, one can check that one still needs a vanishing basic curvature, too.
%, but the condition about the vanishing basic curvature in Cor.~\ref{cor:FlatnessVonEichtrafos} would stay additionally. That is, for Eq.~\eqref{CoordFuerEichKruemmungsRegel} we would need a vanishing basic curvature \textbf{and} a flat $\nabla_\rho$. 
But we will later see that the basic connection will be in general flat, while $\nabla_\rho$ will not; especially we will see that the basic curvature will always vanish for the presented gauge theory. Thence, another reason for our choice to use the basic connection for the definition of $\delta$.
\end{remark}

\begin{proof}[Proof of Cor.~\ref{cor:FlatnessVonEichtrafos}]
\leavevmode\newline
That is a trivial consequence of Thm.~\ref{thm:CurvatureOfBasicStuffIsEquivalentForGaugeTrafoCurvature} and Prop.~\ref{prop:WirHabenEinenTensorBeiderTrafoKruemmung}, using $R_\nabla^{\mathrm{bas}}=0$ (and that then the basic connection is flat by Prop.~\ref{prop:SnablamitREnabla}) and that $R_\delta(\cdot, \cdot)A$ is $\mathcal{F}^0_E(M)$-bilinear such that one just needs to look at pullback functionals.
\end{proof}

%\begin{proof}
%\leavevmode\newline
%$R_{\delta}(\cdot, \cdot)A=0$ immediately follows by Thm.~\ref{thm:CurvatureOfBasicStuffIsEquivalentForGaugeTrafoCurvature} due to the fact that $R_{\delta}(\cdot, \cdot)A$ is tensorial by Prop.~\ref{prop:WirHabenEinenTensorBeiderTrafoKruemmung}, and so we can make use of that pull-backs of $\Gamma(E)$ generate $\Gamma(\Phi^*E)$. Now fix a local frame $\mleft( e_a \mright)_a$ of $E$ and recall that $\nabla^{\mathrm{bas}}$ is flat due to $R^{\mathrm{bas}}_\nabla = 0$ (Prop.~\ref{prop:SnablamitREnabla}), then locally
%\bas
%&&
%0
%&=
%R_{\delta}\mleft(\Phi^*\mu, \Phi^*\nu \mright)A
%\\
%&&&\stackrel{\mathclap{\text{Eq.~\eqref{SplittingVonDerEichtrafo}}}}{=}~~~~
%\mleft(
	%\delta_{\Phi^*\mu} \mleft( \delta_{\Phi^*\nu} A^a \mright) (\Phi, A)
	%- \delta_{\Phi^*\nu} \mleft( \delta_{\Phi^*\mu} A^a \mright) (\Phi, A)
	%+ \delta_{\Phi^*\mleft( \mleft[ \mu, \nu \mright]_E \mright)} A^a
	%\mright) \otimes \Phi^*e_a \nonumber \\
%&&&\hspace{1cm}~~~~
	%+ \underbrace{\mleft( \Phi^*\bigl( R_{\nabla^{\mathrm{bas}}}(\mu, \nu)  \bigr) \mright)}_{=0} A
%\\
%&\mathrel{\mathop{\Rightarrow}^{\text{Prop.~\ref{prop:GaugeTrafoOfAacomps}}}_{\delta^{(1)}A^a = \delta A^a}}&
%0
%&=
%\delta^{(1)}_{\Phi^*\mu} \mleft( \delta^{(1)}_{\Phi^*\nu} A^a \mright) (\Phi, A)
	%- \delta^{(1)}_{\Phi^*\nu} \mleft( \delta^{(1)}_{\Phi^*\mu} A^a \mright) (\Phi, A)
	%+ \delta^{(1)}_{\Phi^*\mleft( \mleft[ \mu, \nu \mright]_E \mright)} A^a
%\eas
%for all $\mu, \nu \in \Gamma(E)$ and $a$; we used Prop.~\ref{prop:GaugeTrafoOfAacomps} to argue  that
%\bas
%\delta_{\Phi^*\mu} \mleft( \delta_{\Phi^*\nu} A^a \mright) (\Phi, A)
%&=
%\delta^{(1)}_{\Phi^*\mu} \mleft( \delta^{(1)}_{\Phi^*\nu} A^a \mright) (\Phi, A),
%\eas
%for this one not only needs $\delta^{(1)}A^a = \delta A^a$, one also needs to observe that this is only works because pull-backs of sections of $E$ are independent of $A$ when varying them. This implies that none of the terms in Eq.~\eqref{eqGaugeTrafoOfAacomps} depends on $j$; for this recall Remark \ref{RemLeibnizeRegelaufProdukteWeshalbEConnectionNichtWichtigIst} about the independency of the chosen $E$-connections when varying things.\footnote{Although it surely depends on $\nabla$ itself due to its connection 1-forms in Eq.~\eqref{eqGaugeTrafoOfAacomps} which is not important here.}
%
%By Eq.~\eqref{SplittingVonDerEichtrafo} we finally get
%\bas
%R_{\delta^{(1)}}\mleft( \Phi^*\mu, \Phi^*\nu \mright)A
%&=
%\mleft( \Phi^*\bigl( R_{\nabla_{\rho}}(\mu, \nu)  \bigr) \mright) A.
%\eas
%\end{proof}

These results motivate even further why we use the basic connection to define the infinitesimal gauge transformation. Moreover, $R^{\mathrm{bas}}_\nabla=0$ is also a condition which we will need for gauge invariance; see later. That we have this condition in the standard formulation of gauge theory is also emphasized in the following theorem:

\begin{theorems}{Relation of the basic curvature and action Lie algebroids, \newline \cite[discussion around Eq.~(9)]{CurvedYMH}, \cite[Prop.~2.12]{basicconn}, and \cite[\S 2.5, Theorem A]{blaomTangentBundleAsLieGroup}}{ActionLieALgebroid}
Let $E \to N$ be a Lie algebroid. Then $E$ is locally an action Lie algebroid if and only if it admits locally a flat connection $\nabla$ with $R_\nabla^{\mathrm{bas}} = 0$. If there is such a local isomorphism, then it can be chosen in such a way that $\nabla$ describes the canonical flat connection.
\end{theorems}

\begin{remark}\label{remSimplyConnectedEqualsGlobal}
\leavevmode\newline
As clarification of the last sentence, under that isomorphism we have (locally) $E = N \times \mathfrak{g}$ for some Lie algebra $\mathfrak{g}$, and a basis of $\mathfrak{g}$, that is, a constant frame of $E$, will be parallel with respect to $\nabla$. Especially, the canonical flat connection of every action Lie algebroid has a vanishing basic curvature. Furthermore, over a simply connected base the isomorphism is global as we will see in the proof (because one can then construct a global parallel frame for $\nabla$; see the proof).
\end{remark}

\begin{proof}
\leavevmode\newline
This basically follows by Eq.~\eqref{eq:compcondfast}, \textit{i.e.}
\bas
R_\nabla^{\mathrm{bas}}(\mu, \nu)Y
&= \mleft( \nabla_Y t_{\nabla^{\mathrm{bas}}} \mright)(\mu, \nu) 
- R_\nabla(\rho(\mu), Y) \nu + R_\nabla(\rho(\nu), Y) \mu
\eas
for all $\mu,\nu \in \Gamma(E)$ and $Y \in \mathfrak{X}(N)$.

\underline{"$\Rightarrow$"}: Assume $E|_U \cong U \times \mathfrak{g}$ is an action Lie algebroid for some open subset $U$ of $N$ for some Lie algebra $\mathfrak{g}$. Over $U$ take the canonical flat connection $\nabla$, and let $\mleft( e_a \mright)_a$ be a frame of constant sections on $U$. Then by Eq.~\eqref{eq:compcondfast}
\bas
R_\nabla^{\mathrm{bas}}(e_a, e_b)
&=
\mleft( \nabla t_{\nabla^{\mathrm{bas}}} \mright)(e_a, e_b) 
=
\nabla \bigl( \underbrace{t_{\nabla^{\mathrm{bas}}}(e_a, e_b)}_{\mathclap{= \mleft[ e_a, e_b \mright]_E}} \bigr)
=
\mathrm{d}C_{ab}^c \otimes e_c,
=
0
\eas
where $C_{ab}^c$ are the structure constants of $\mathfrak{g}$.

\underline{"$\Leftarrow$"}: Assume we have a flat connection $\nabla$ over some open subset $U$ with $R_\nabla^{\mathrm{bas}} = 0$. W.l.o.g. assume there is a parallel frame $\mleft( e_a \mright)_a$ for $\nabla$ on $U$ (otherwise restrict $U$ to a smaller subset). Then again by Eq.~\eqref{eq:compcondfast}
\bas
0
&=
\nabla \bigl( t_{\nabla^\mathrm{bas}}(e_a, e_b) \bigr)
=
\mathrm{d}C_{ab}^c \otimes e_c,
\eas
thus, the structure functions related to the parallel frame are constant. Therefore the parallel frame spans the same Lie algebra $\mathfrak{g}$ at each fibre, so, $E|_U \cong U \times \mathfrak{g}$ as vector bundles. Identifying elements of $\mathfrak{g}$ with constant sections, the anchor $\rho$ defines clearly an action for $\mathfrak{g}$ on $N$, and $\mleft[ \cdot,\cdot \mright]_E$ clearly restrict to $\mleft[ \cdot, \cdot \mright]_{\mathfrak{g}}$ on constant sections. The Lie algebroid is thence of the action type by the uniqueness given in Prop.~\ref{prop:ActionLieoidsAreOids}.
\end{proof}
%
%So, we only choose the basic connection when defining the variations/gauge transformations on $\Phi^*E$-valued objects, in order to know that the commutator of two transformations is again a transformation, because, as we will see, the basic curvature will always vanish. In that way we then have $R_{\delta}(\cdot, \cdot)A= 0$, especially Eq.~\eqref{CoordFuerEichKruemmungsRegel} suggest that we have a sort of representation (in sense of an anti-homomorphism due to our sign convention which could be easily reversed). We will later discuss whether $R_{\delta}(\cdot, \cdot)$ is also zero when acting on other type of tensors (see Thm.~\ref{thm:EichtrafoIstIWkrlichKrassFlach}). But for this we need to introduce more about the physical setting.

We now want to generalize Cor.~\ref{cor:FlatnessVonEichtrafos} by using Cor.~\ref{cor:RelationShipsOfCurvatures}, especially we need to understand the behaviour for scalar-valued functionals. For such functionals the infinitesimal gauge transformation is nothing else than the Lie derivative of some vector field in $\mathfrak{M}_E$, which we denoted by $\Psi_\varepsilon$.
Recall Remark \ref{NotASubalgebraXB}, we do in general not expect that $\Psi_\varepsilon \in \mathfrak{X}^E\bigl( \mathfrak{M}_E(M;N) \bigr)$ builds a subalgebra; however, since we restricted the set of those vector fields by defining $\delta_\varepsilon A$ in Prop.~\ref{prop:VariationOfA}, there may be hope for the structure of a subalgebra; this will be discussed now. 

\begin{theorems}{Bracket of gauge transformations a gauge transformation}{VektorfelderSindZumGlueckGeschlossen}
Let $M, N$ be smooth manifolds, $E \to N$ a Lie algebroid, $\nabla$ a connection on $E$ with $R^{\mathrm{bas}}_\nabla=0$. Furthermore let $\Psi_\varepsilon$ and $\Psi_\vartheta$ for $\varepsilon, \vartheta \in \mathcal{F}^0_E(M; {}^*E)$ be the unique elements of $\mathfrak{X}^E\bigl( \mathfrak{M}_E(M;N) \bigr)$ as given by Prop.~\ref{prop:VariationOfA}.\footnote{Recall that those $\Psi_\varepsilon$ are the vector fields describing the infinitesimal gauge transformation; see Def.~\ref{def:TotalInfGaugeTrafoYayy}.}

Then
\ba
\mleft[ \Psi_\varepsilon, \Psi_\vartheta \mright]
&=
- \Psi_{\Delta(\varepsilon, \vartheta)}
\ea
for all $\varepsilon, \vartheta \in \mathcal{F}^0_E(M; {}^*E)$, where $\Psi_{\Delta(\varepsilon, \vartheta)}$ is also the unique element of $\mathfrak{X}^E\bigl( \mathfrak{M}_E(M;N) \bigr)$ as given by Prop.~\ref{prop:VariationOfA}.
\end{theorems}

\begin{proof}
\leavevmode\newline
First recall that we have by Remark \ref{RemLeibnizeRegelaufProdukteWeshalbEConnectionNichtWichtigIst}
\bas
\delta_\varepsilon \omega
&=
\mathcal{L}_{\Psi_\varepsilon} \omega
\eas
for all $\omega \in \mathcal{F}^\bullet_E(M)$ and $\varepsilon \in \mathcal{F}^0_E(M; {}^*E)$. Therefore we want to use Cor.~\ref{cor:FlatnessVonEichtrafos}. As vector fields of $\mathfrak{M}_E(M;N)$, the action of $\mathcal{L}_{\Psi_\varepsilon}$ is uniquely given by its action on coordinates of $\mathfrak{M}_E(M;N)$, and these are essentially given by the components of the fields $(\Phi, A) \in \mathfrak{M}_E(M;N)$: Let $\mleft( x^i \mright)_i$ be local coordinate functions on $N$ and let $\mleft(e_a\mright)_a$ be a local frame of $E$, then coordinates of $\mathfrak{M}_E(M;N)$ are given by the functionals ${}^*\mleft(x^i\mright)$ and $\varpi_2^a$ because of
\bas
\mleft.{}^*\mleft(x^i\mright)\mright|_{(\Phi,A)}
&=
\Phi^i,
\\
\varpi_2^a(\Phi,A)
&=
A^a
\eas
for all $(\Phi, A) \in \mathfrak{M}_E(M;N)$. Recall the first calculation in the proof of Cor.~\ref{cor:RelationShipsOfCurvatures}, we get similarly
\bas
R_\delta(\varepsilon,\vartheta)\mleft( {}^*\mleft(x^i\mright) \mright) 
&= 
\varepsilon^a \vartheta^b ~ {}^*\underbrace{\mleft(
	\mathcal{L}_{\rho(e_a)} \mathcal{L}_{\rho(e_b)} x^i
	- \mathcal{L}_{\rho(e_b)} \mathcal{L}_{\rho(e_a)} x^i
	- \mathcal{L}_{\rho\mleft(\mleft[ e_a, e_b \mright]_E\mright)} x^i
\mright)}_{= \mleft( \mathcal{L}_{\mleft[ \rho(e_a), \rho(e_b) \mright]} - \mathcal{L}_{\rho\mleft(\mleft[ e_a, e_b \mright]_E\mright)} \mright) x^i = 0}
=
0
\eas
for all $\varepsilon, \vartheta \in \mathcal{F}^0_E(M; {}^*E)$, using that $\rho$ is a homomorphisma and Remark \ref{RemLeibnizeRegelaufProdukteWeshalbEConnectionNichtWichtigIst} such that $\delta_\varepsilon \mleft( {}^*\mleft( x^i \mright) \mright) = - \varepsilon^a ~ {}^*\mleft( \mathcal{L}_{\rho(e_a)} x^i \mright)$. By Cor.~\ref{cor:FlatnessVonEichtrafos} we also get
\bas
R_\delta (\varepsilon, \vartheta) \varpi_2^a
&=
0.
\eas
By $\delta_\varepsilon = \mathcal{L}_{\Psi_\varepsilon}$ on scalar-valued functionals we therefore get
\bas
\mleft(\mleft[ \mathcal{L}_{\Psi_\varepsilon}, \mathcal{L}_{\Psi_\vartheta} \mright]
+ \mathcal{L}_{\Psi_{\Delta(\varepsilon, \vartheta)}}\mright)f
&=
0
\eas
for all $f \in C^\infty\bigl(\mathfrak{M}_E(M;N)\bigr)$,
which finishes the proof.
\end{proof}

\begin{remarks}{Curvature of $\delta$ on $\Phi$}{WasIstMitDemHiggsFeldBeiDerDeltaKruemmung}
Keeping the same situation and notation as in the previous proof, observe that we have
\bas
\delta_{{}^*\nu} \delta_{{}^*\mu} \Phi
&=
- \delta_{{}^*\nu} \bigl( {}^*(\rho(\mu)) \bigr)
=
{}^*\mleft( \nabla^{\mathrm{bas}}_\nu \bigl( \rho(\mu) \bigr) \mright)
=
{}^*\mleft( \rho\mleft( \nabla^{\mathrm{bas}}_\nu \mu \mright) \mright)
\eas
for all $\mu, \nu \in \Gamma(E)$, hence,\footnote{Recall Eq.~\eqref{EqLieKlammerAufPullBackSections}.}
\bas
\delta_{{}^*\nu} \delta_{{}^*\mu} \Phi
	- \delta_{{}^*\mu} \delta_{{}^*\nu} \Phi
	+ \delta_{{}^*\mleft( \mleft[ \nu, \mu \mright]_E \mright)} \Phi
&=
{}^*\mleft( \rho\mleft( 
	\nabla^{\mathrm{bas}}_\nu \mu 
	- \nabla^{\mathrm{bas}}_\mu \nu 
	- \mleft[ \nu, \mu \mright]_E
\mright) \mright)
=
{}^*\Bigl( \rho\bigl( 
	t_{\nabla^{\mathrm{bas}}}(\nu, \mu)
\bigr) \Bigr).
\eas
Therefore, if we want that this is zero, too, we would need that the torsion of the basic connection has values in the kernel of the anchor which is in general not the case. However, it is no harm that we do not have a zero value in general here. That is due to the fact that on one hand $\Phi$ just contributes via pull-backs, as we will also see in the following sections; on the other hand $\Phi$ is not vector-bundle valued and hence will not arise in any other form than as the map for the pullbacks in any Lagrangian or physical quantity. Even in the classical case, recall Prop.~\ref{prop:LieRepAndLieAct}, a Lie algebra representation acting on $\Phi$ is just the evaluation of its induced action at $\Phi$.
\newline\newline
However, as we have seen in the proof, we got $R_\delta(\cdot,\cdot)\mleft( {}^*\mleft(x^i\mright) \mright) = 0$, and $\mleft.{}^*\mleft(x^i\mright)\mright|_{(\Phi,A)} = \Phi^i$ for all $(\Phi, A) \in \mathfrak{M}_E(M;N)$. That is, for the components of the Higgs field we have the desired behaviour, which is all we need.
\end{remarks}

Finally, we can generalize Cor.~\ref{cor:FlatnessVonEichtrafos}.

\begin{theorems}{Curvature of $\delta$ on arbitrary functionals}{AllgemEineGeileFormelFuerDieEichKruemmung}
Let $M, N$ be smooth manifolds, $E \to N$ a Lie algebroid, $\nabla$ a connection on $E$ with $R^{\mathrm{bas}}_\nabla=0$. Furthermore let $V\to N$ be a vector bundle, equipped with an $E$-connection ${}^E\nabla$ on $V$. Then
\ba
R_\delta(\varepsilon, \vartheta) L
&=
\mleft({}^*R_{{}^E\nabla} \mright)(\varepsilon, \vartheta)L
\ea
for all $L \in \mathcal{F}_E^k(M; {}^*V)$ ($k \in \mathbb{N}_0$) and $\varepsilon, \vartheta \in \mathcal{F}^0_E(M; {}^*E)$. In short, $R_\delta = {}^*R_{{}^E\nabla}$.
\end{theorems}

\begin{remark}
\leavevmode\newline
This also shows that $R_\delta$ is a tensor.
Moreover, as expected, for flat ${}^E\nabla$ we would get
\ba
R_\delta(\varepsilon, \vartheta) L
&=
0.
\ea
\end{remark}

\begin{proof}[Proof of Thm.~\ref{thm:AllgemEineGeileFormelFuerDieEichKruemmung}]
\leavevmode\newline
We want to use Cor.~\ref{cor:RelationShipsOfCurvatures}, so, for a given frame $\mleft( e_a \mright)_a$ we have
\bas
R_\delta(\varepsilon, \vartheta)L
&=
R_\delta(\varepsilon, \vartheta)L^a \otimes {}^*e_a
	+ \mleft({}^*R_{{}^E\nabla}\mright)(\varepsilon, \vartheta)L
\eas
for all $L \in \mathcal{F}^k_E(M; {}^*V)$ ($k \in \mathbb{N}_0$) and $\varepsilon, \vartheta \in \mathcal{F}^0_E(M; {}^*E)$. Hence, we just need to show that $R_\delta(\varepsilon, \vartheta)L^a = 0$. Again by Remark \ref{RemLeibnizeRegelaufProdukteWeshalbEConnectionNichtWichtigIst} we have $\delta_\varepsilon = \mathcal{L}_{\Psi_\varepsilon}$ on scalar-valued functionals, where $\Psi_\varepsilon$ still denotes vector fields as uniquely given by Prop.~\ref{prop:VariationOfA}. $\Psi_\varepsilon$ are elements of $\mathfrak{X}\bigl( \mathfrak{M}_E(M;N) \bigr)$, hence, 
\bas
(\underbrace{\delta_\varepsilon L^a}_{\mathclap{= \mathcal{L}_{\Psi_\varepsilon}L^a }})_p(Y_1, \dotsc, Y_k)
&=
\mathcal{L}_{\Psi_\varepsilon} \mleft(L^a_p(Y_1, \dotsc, Y_k)\mright)
\eas
for all $p \in M$ and $Y_1, \dotsc, Y_k \in \mathrm{T}_pM$. We know that $L^a\in\mathcal{F}^k_E(M)$, and therefore $L^a_p(Y_1, \dotsc, Y_k) \in C^\infty\bigl(\mathfrak{M}_E(M;N)\bigr)$, so, we just need to use Thm.~\ref{thm:VektorfelderSindZumGlueckGeschlossen} to get
\bas
\mleft(R_\delta(\varepsilon, \vartheta) L^a \mright)_p(Y_1, \dotsc, Y_k)
&=
\mleft(
	\mleft(\mleft[ \mathcal{L}_{\Psi_\varepsilon}, \mathcal{L}_{\Psi_\vartheta} \mright]
	+ \mathcal{L}_{\Psi_{\Delta(\varepsilon, \vartheta)}}\mright) L^a
\mright)_p(Y_1, \dotsc, Y_k)
\\
&=
\mleft(\mleft[ \mathcal{L}_{\Psi_\varepsilon}, \mathcal{L}_{\Psi_\vartheta} \mright]
	+ \mathcal{L}_{\Psi_{\Delta(\varepsilon, \vartheta)}}\mright)
\mleft(L^a_p(Y_1, \dotsc, Y_k)\mright)
\\
&\stackrel{\mathclap{ \text{Thm.~\ref{thm:VektorfelderSindZumGlueckGeschlossen}} }}{=}\quad~~
0,
\eas
which concludes the proof.
\end{proof} 

Let us conclude this section by showing that this finally implies that $\Delta$ is a Lie bracket.

\begin{theorems}{Pre-bracket a Lie bracket}{PreKlammerEineSuperLieKlammer}
Let $M, N$ be smooth manifolds, $E \to N$ a Lie algebroid, $\nabla$ a connection on $E$ with $R^{\mathrm{bas}}_\nabla=0$. Then $\Delta$ is a Lie bracket.
\end{theorems}

\begin{proof}
\leavevmode\newline
By Prop.~\ref{prop:PropertiesOfThePreBracket} we already know antisymmetry and $\mathbb{R}$-bilinearity. 
%and by Remark \ref{rem:PolynomeSindZumGlueckGeschlossenNachGaugeTrafo} we know that
%\bas
%\Delta(\vartheta, \varepsilon)
%&=
%\delta_\varepsilon \vartheta - \delta_\vartheta \varepsilon - \bigl( {}^*t_{\nabla^{\mathrm{bas}}} \bigr)\mleft( \vartheta, \varepsilon \mright)
%\in \mathcal{H}^0_E(M;{}^*E)
%\eas
%for all $\varepsilon, \vartheta \in \mathcal{H}_E^0(M; {}^*E)$, also using that clearly $\bigl( {}^*t_{\nabla^{\mathrm{bas}}} \bigr)\mleft( \vartheta, \varepsilon \mright) \in \mathcal{H}^0(M; {}^*E)$.
Thus, only the Jacobi identity is left to show, and the calculation is very similar to the calculation of the first Bianchi identity in Thm.~\ref{thm:1stBianchi},
\bas
\Delta\mleft( \eta, \Delta\mleft( \vartheta, \varepsilon \mright) \mright)
&=
\Delta\mleft( \eta, \delta_\varepsilon \vartheta - \delta_\vartheta \varepsilon - \bigl( {}^*t_{\nabla^{\mathrm{bas}}} \bigr)\mleft( \vartheta, \varepsilon \mright) \mright)
\\
&=
\underbrace{\delta_{\delta_\varepsilon \vartheta} \eta
	- \delta_{\delta_\vartheta \varepsilon} \eta
	- \delta_{\bigl( {}^*t_{\nabla^{\mathrm{bas}}} \bigr)\mleft( \vartheta, \varepsilon \mright)} \eta}
	_{\delta_{\Delta(\vartheta, \varepsilon)} \eta}
\\
&\hspace{1cm}
	- \delta_\eta \delta_\varepsilon \vartheta
	+ \delta_\eta \delta_\vartheta \varepsilon
	+ \bigl( {}^*t_{\nabla^{\mathrm{bas}}} \bigr)\mleft( \eta, \mleft( \bigl( {}^*t_{\nabla^{\mathrm{bas}}} \bigr)\mleft( \vartheta, \varepsilon \mright) \mright) \mright)
\\
&\hspace{1cm}
	+ \delta_\eta \mleft( \bigl( {}^*t_{\nabla^{\mathrm{bas}}} \bigr)\mleft( \vartheta, \varepsilon \mright) \mright)
	- \bigl( {}^*t_{\nabla^{\mathrm{bas}}} \bigr)\mleft( \eta, \delta_\varepsilon \vartheta \mright)
	+ \bigl( {}^*t_{\nabla^{\mathrm{bas}}} \bigr)\mleft( \eta, \delta_\vartheta \varepsilon \mright)
\\
&=
\delta_\eta \delta_\vartheta \varepsilon
	- \delta_\eta \delta_\varepsilon \vartheta
	+ \delta_{\Delta(\vartheta, \varepsilon)} \eta
\\
&\hspace{1cm}
	+ \delta_\eta \mleft( \bigl( {}^*t_{\nabla^{\mathrm{bas}}} \bigr)\mleft( \vartheta, \varepsilon \mright) \mright)
	- \bigl( {}^*t_{\nabla^{\mathrm{bas}}} \bigr)\mleft( \eta, \delta_\varepsilon \vartheta \mright)
	+ \underbrace{\bigl( {}^*t_{\nabla^{\mathrm{bas}}} \bigr)\mleft( \eta, \delta_\vartheta \varepsilon \mright)}_{\mathclap{= - \bigl( {}^*t_{\nabla^{\mathrm{bas}}} \bigr)\mleft( \delta_\vartheta \varepsilon, \eta \mright)}}
\\
&\hspace{1cm}
	+ \bigl( {}^*t_{\nabla^{\mathrm{bas}}} \bigr)\mleft( \eta, \mleft( \bigl( {}^*t_{\nabla^{\mathrm{bas}}} \bigr)\mleft( \vartheta, \varepsilon \mright) \mright) \mright)
\eas
for all $\varepsilon, \vartheta, \eta \in \mathcal{F}_E^0(M; {}^*E)$. Taking the cyclic sum, we collect the terms as in the proof of Thm.~\ref{thm:1stBianchi}, and hence we get, using that $\nabla^{\mathrm{bas}}$ is used for the definition of $\delta$ on $E$-valued functionals,
\bas
&\Delta\mleft( \eta, \Delta\mleft( \vartheta, \varepsilon \mright) \mright)
	+ \Delta\mleft( \vartheta, \Delta\mleft( \varepsilon, \eta \mright) \mright)
	+ \Delta\mleft( \varepsilon, \Delta\mleft( \eta, \vartheta \mright) \mright)
\\
&=
\underbrace{R_{\delta}( \eta, \vartheta ) \varepsilon
	+ R_{\delta}( \varepsilon,\eta ) \vartheta
	+ R_{\delta}( \vartheta, \varepsilon) \eta}_{\stackrel{\text{Thm.~\ref{thm:AllgemEineGeileFormelFuerDieEichKruemmung}}}{=}0}
\\
&\hspace{1cm}
	+ \bigl( {}^*t_{\nabla^{\mathrm{bas}}} \bigr)\mleft( \eta, \bigl( {}^*t_{\nabla^{\mathrm{bas}}} \bigr)\mleft( \vartheta, \varepsilon \mright) \mright) 
	+ \bigl( {}^*t_{\nabla^{\mathrm{bas}}} \bigr)\mleft( \varepsilon, \bigl( {}^*t_{\nabla^{\mathrm{bas}}} \bigr)\mleft( \eta, \vartheta \mright) \mright) 
\\
&\hspace{1cm}
	+ \bigl( {}^*t_{\nabla^{\mathrm{bas}}} \bigr)\mleft( \vartheta, \bigl( {}^*t_{\nabla^{\mathrm{bas}}} \bigr)\mleft( \varepsilon, \eta \mright) \mright) 
\\
&\hspace{1cm}
	+ \underbrace{\mleft( \delta_\eta \mleft({}^*t_{\nabla^{\mathrm{bas}}} \mright) \mright)}_{= - {}^*\mleft( \nabla^{\mathrm{bas}}_\eta t_{\nabla^{\mathrm{bas}}} \mright)} (\vartheta, \varepsilon)
	+ \mleft( \delta_\varepsilon \mleft({}^*t_{\nabla^{\mathrm{bas}}} \mright) \mright) (\eta, \vartheta)
	+ \mleft( \delta_\vartheta \mleft({}^*t_{\nabla^{\mathrm{bas}}} \mright) \mright) (\varepsilon, \eta)
\\
&=
- \vartheta^a \varepsilon^b \eta^c ~ {}^*\biggl(
		t_{\nabla^{\mathrm{bas}}} \mleft( t_{\nabla^{\mathrm{bas}}}(e_a, e_b), e_c \mright)
		+ t_{\nabla^{\mathrm{bas}}} \mleft( t_{\nabla^{\mathrm{bas}}}(e_b, e_c), e_a \mright)
		+ t_{\nabla^{\mathrm{bas}}} \mleft( t_{\nabla^{\mathrm{bas}}}(e_c, e_a), e_b \mright)
\\
&\hspace{1cm}\hphantom{- \vartheta^a \varepsilon^b \eta^c ~ {}^*\biggl(}
	+ \mleft( \nabla^{\mathrm{bas}}_{e_c} t_{\nabla^{\mathrm{bas}}} \mright) (e_a, e_b)
	+ \mleft( \nabla^{\mathrm{bas}}_{e_a} t_{\nabla^{\mathrm{bas}}} \mright) (e_b, e_c)
	+ \mleft( \nabla^{\mathrm{bas}}_{e_b} t_{\nabla^{\mathrm{bas}}} \mright) (e_c, e_a)
\biggr)
\\
&\stackrel{\mathclap{\text{Thm.~\ref{thm:1stBianchi}}}}{=}~~~~~
0
\eas
for all $\varepsilon, \vartheta, \eta \in \mathcal{F}^0_E(M; {}^*E)$, where $\mleft( e_a \mright)_a$ is a local frame of $E$, and we also used that $\nabla^{\mathrm{bas}}$ is flat by Prop.~\ref{prop:SnablamitREnabla}; the flatness was applied when we used Thm.~\ref{thm:1stBianchi} and Thm.~\ref{thm:AllgemEineGeileFormelFuerDieEichKruemmung}.\footnote{But flatness is not actually needed here; see also the following remark.} Thence, the Jacobi identity follows.
\end{proof}

\begin{remark}\label{RemarkBracketIsVeryIndependent}
\leavevmode\newline
The proof is essentially based on the first Bianchi identity of curvatures. Hence, taking any other $E$-connection $\nabla^\prime$ on $E$ one could define the bracket $\Delta$ by using the torsion of $\nabla^\prime$ instead of $\nabla^{\mathrm{bas}}$, and then also define the $\delta$ operator with respect to $\nabla^\prime$ on $E$-valued form. By Thm.~\ref{thm:AllgemEineGeileFormelFuerDieEichKruemmung} we could not expect $R_\delta=0$ in general, but $\Delta$ should be nevertheless a Lie bracket due to the fact that the first Bianchi identity always holds and that Thm.~\ref{thm:AllgemEineGeileFormelFuerDieEichKruemmung} provides the needed curvature terms for the Bianchi identity. Furthermore, already Eq.~\eqref{EqDeltaInFrameKoord} points out that the definition of $\Delta$ is independent of the choice of $\nabla^\prime$ because $\delta_\varepsilon$ is just a Lie derivative on scalar-valued functionals, so that it is clear that it is always the same Lie bracket. By the very last statement of Prop.~\ref{prop:PropertiesOfThePreBracket}, we achieve a Lie bracket completely independent of connections, if the parameters are just functionals depending on the Higgs field $\Phi$. Recall Remark \ref{ClassicalCommutatorRemark} (the part about the bookkeeping trick of the parameters) and the first bullet point of Remark \ref{RemarkUeberNablaRhoCurvatureForGauegTrafo} about that it is in general unavoidable to assume that the parameters depend on $\Phi$.
\end{remark}

\section{Infinitesimal gauge invariance}\label{InfInvariance}

Let us now calculate the infinitesimal gauge transformations needed for the Lagrangian.

\begin{propositions}{Infinitesimal gauge transformations of the field strength}{GaugeTrafosOfFieldStrengthAndMinimalCoupling}
Let $M, N$ be smooth manifolds, $E \to N$ a Lie algebroid, and $\nabla$ a connection on $E$. Then we have
\ba
\delta_\varepsilon F
&=
- \mleft(
	\frac{1}{2} ~ \mleft(	{}^* R_{\nabla} \mright)\mleft( \mathfrak{D} \stackrel{\wedge}{,} \mathfrak{D} \mright) \varepsilon
	+ \mleft({}^* R_\nabla^{\mathrm{bas}} \mright) \mleft(\varepsilon \stackrel{\wedge}{,} \varpi_2  \stackrel{\wedge}{,} \mathrm{D} \mright)
\mright) \label{EqVariationForF}
\ea
for all $\varepsilon \in \mathcal{F}^0_E(M; {}^*E)$, where we write $\Gamma(E) \times \Gamma(E) \times \mathfrak{X}(N) \ni (\mu, \nu, Y) \mapsto R_\nabla^{\mathrm{bas}}(\mu, \nu, Y) \coloneqq R_\nabla^{\mathrm{bas}}(\mu, \nu) Y$.
\end{propositions}

\begin{proof}
\leavevmode\newline
%For simplicity we will omit to denote $(\Phi, A)$. 
Let $\mleft( e_a \mright)_a$ be a frame of $E$, then by Eq.~\eqref{eqGaugeTrafoOfAacomps}
\bas
\mathrm{d} \delta_\varepsilon \varpi_2^a \otimes {}^*e_a
&=
\mathrm{d}\mleft( \varepsilon^b \varpi_2^c \otimes {}^*\mleft( \nabla^{\mathrm{bas}}_{e_b} e_c\mright)
	- \mleft({}^*\nabla\mright)\varepsilon \mright)^a \otimes {}^*e_a
\\
&=
\mathrm{d}\varepsilon^b \wedge \varpi_2^c \otimes {}^*\mleft( \nabla^{\mathrm{bas}}_{e_b} e_c\mright)
	+ \varepsilon^b ~ \mathrm{d}\varpi_2^c \otimes {}^*\mleft( \nabla^{\mathrm{bas}}_{e_b} e_c\mright)
\\
&\hspace{1cm}
	- \varepsilon^b \varpi_2^c \wedge \mathrm{d}\mleft( {}^*\mleft( \nabla^{\mathrm{bas}}_{e_b} e_c\mright) \mright)^a \otimes {}^*e_a
	- \mathrm{d} \bigl( \mleft({}^*\nabla\mright)\varepsilon \bigr)^a \otimes {}^*e_a
\eas
also recall Eq.~\eqref{eqVariationVertauschtMitDifferential}, and \eqref{EqVariationVonFormenBrrrr} (and also the calculation for Eq.~\eqref{EqVariationVonFormenBrrrrVereinfacht}),
then, using the previous calculation,
\bas
\delta_\varepsilon \mleft(\mathrm{d}^{{}^*\nabla} \varpi_2 \mright)
&=
\delta_\varepsilon \mleft(
	\mathrm{d} \varpi_2^a \otimes {}^*e_a 
	- \varpi_2^b \wedge {}^!(\nabla e_b)
\mright)
\\
&=
\mathrm{d} \delta_\varepsilon \varpi_2^a \otimes {}^*e_a 
	- \mathrm{d} \varpi_2^a \otimes {}^*\mleft( \nabla^{\mathrm{bas}}_\varepsilon e_a \mright)
	- \delta_\varepsilon \varpi_2^b \wedge {}^!(\nabla e_b)
\\
&\hspace{1cm}
	+ \varpi_2^b \wedge \biggl(
	\underbrace{\mleft({}^*\mleft(\nabla^{\mathrm{bas}}_\varepsilon \mleft( \nabla e_b \mright)\mright)\mright)(\mathrm{D})
	+ {}^*\mleft( \nabla_{({}^*\rho)\mleft( ({}^*\nabla) \varepsilon \mright)} e_b \mright)}
	_{\mathclap{= \varepsilon^a  ~ {}^! \mleft( \nabla^{\mathrm{bas}}_{e_a} \mleft( \nabla e_b \mright) \mright)
		+ \varepsilon^a ~ {}^!\mleft( \nabla_{  \rho\mleft( \nabla e_a \mright)} e_b \mright)
		+ \mathrm{d}\varepsilon^a \otimes {}^*\mleft( \nabla_{  \rho\mleft( e_a \mright) }e_b \mright)}}
\biggr)
\\
&=
\mathrm{d}\varepsilon^a \wedge \varpi_2^b \otimes {}^*\underbrace{\mleft( 
	\nabla^{\mathrm{bas}}_{e_a} e_b
	- \nabla_{  \rho\mleft( e_a \mright) }e_b
\mright)}
_{\mathclap{= t_{\nabla^{\mathrm{bas}}}(e_a, e_b)}}
\\
&\hspace{1cm}
	- \varepsilon^a \varpi_2^c \wedge \mathrm{d}\mleft( {}^*\mleft( \nabla^{\mathrm{bas}}_{e_a} e_c\mright) \mright)^b \otimes {}^*e_b
	- \mathrm{d} \bigl( \mleft({}^*\nabla\mright)\varepsilon \bigr)^b \otimes {}^*e_b
\\
&\hspace{1cm}
	- \mleft( \varepsilon^a \varpi_2^c \otimes {}^*\mleft( \nabla^{\mathrm{bas}}_{e_a} e_c\mright)
	- \mleft({}^*\nabla\mright)\varepsilon \mright)^b \wedge {}^!(\nabla e_b)
\\
&\hspace{1cm}
	+ \varepsilon^a \varpi_2^b \wedge {}^! \biggl(
	  \nabla^{\mathrm{bas}}_{e_a} \mleft( \nabla e_b \mright)
		+ \nabla_{  \rho\mleft( \nabla e_a \mright)} e_b
\biggr)
\\
&=
\mathrm{d}\varepsilon^a \wedge \varpi_2^b \otimes {}^* \mleft(t_{\nabla^{\mathrm{bas}}}(e_a, e_b) \mright)
	- \varepsilon^a \varpi_2^c \wedge \underbrace{\mathrm{d}^{{}^*\nabla} \mleft( {}^*\mleft( \nabla^{\mathrm{bas}}_{e_a} e_c \mright) \mright)}
	_{\mathclap{\stackrel{\text{Eq.~\eqref{EqGeilePullBackCommuteFormel}}}{=} {}^!\mleft( \nabla \mleft( \nabla^{\mathrm{bas}}_{e_a} e_c \mright) \mright)}}
	- \underbrace{\mleft( \mathrm{d}^{{}^*\nabla} \mright)^2 \varepsilon}
	_{\mathclap{= R_{{}^*\nabla}(\cdot, \cdot) \varepsilon}}
\\
&\hspace{1cm}
	+ \varepsilon^a  \varpi_2^b \wedge {}^!  \biggl(
	\nabla^{\mathrm{bas}}_{e_a} \mleft( \nabla e_b \mright)
		+ \nabla_{  \rho\mleft( \nabla e_a \mright)} e_b
\biggr)
\\
&=
\mathrm{d}\varepsilon^a \wedge \varpi_2^b ~ {}^* \mleft(t_{\nabla^{\mathrm{bas}}}(e_a, e_b) \mright)
	- \varepsilon^a ~ {}^!\bigl( R_{\nabla}(\cdot, \cdot) e_a \bigr)
\\
&\hspace{1cm}
	+ \varepsilon^a  \varpi_2^b \wedge {}^!  \biggl(
	\underbrace{
	\nabla^{\mathrm{bas}}_{e_a} \mleft( \nabla e_b \mright)
		- \nabla \mleft( \nabla^{\mathrm{bas}}_{e_a} e_b \mright)
		+ \nabla_{  \rho\mleft( \nabla e_a \mright)} e_b
		}
		_{\mathclap{\mathfrak{X}(N) \ni Y \mapsto \mleft[ e_a, \nabla_Y e_b \mright]_E
	+ \nabla_{\rho\mleft( \nabla_Y e_b \mright)} e_a
	- \nabla_{\mleft[ \rho(e_a), Y \mright]} e_b
	- \nabla_Y\mleft( \mleft[ e_a, e_b \mright]_E \mright)
	- \nabla_Y \nabla_{\rho(e_b)} e_a}}
	\biggr),
%\\
%&=
%\mathrm{d}^{{}^*\nabla} \mleft( \delta_\varepsilon \varpi_2 \mright)
	%- \mathrm{d} \varpi_2^a \otimes {}^*\mleft( \nabla^{\mathrm{bas}}_\varepsilon e_a \mright)
%\\
%&\hspace{1cm}
	%- \varpi_2^b \otimes \biggl(
	%\mleft({}^*\mleft(\nabla^{\mathrm{bas}}_\varepsilon \mleft( \nabla e_b \mright)\mright)\mright)(\mathrm{D})
	%+ {}^*\mleft( \nabla_{({}^*\rho)\mleft( ({}^*\nabla) \varepsilon \mright)} e_b \mright)
%\biggr)
\eas
using the second calculation in the proof of Thm.~\ref{thm:CurvatureOfBasicStuffIsEquivalentForGaugeTrafoCurvature}. Moreover,
\bas
\mleft( \nabla^{\mathrm{bas}}_\eta t_{\nabla^{\mathrm{bas}}} \mright) (\mu, \nu)
&\stackrel{\text{Thm.~\ref{thm:modBianchithm}}}{=}
R_{\nabla_\rho}(\mu, \nu) \eta
%&=
%\nabla^{\mathrm{bas}}_\eta \bigl( t_{\nabla^{\mathrm{bas}}} (\mu, \nu) \bigr)
	%- \underbrace{t_{\nabla^{\mathrm{bas}}}}_{\mathclap{= - t_{\nabla_\rho}}} \mleft(\nabla^{\mathrm{bas}}_\eta \mu,\nu\mright)
	%- t_{\nabla^{\mathrm{bas}}} \mleft( \mu, \nabla^{\mathrm{bas}}_\eta \nu\mright)
%\\
%&=
%\bigl[\eta, t_{\nabla^{\mathrm{bas}}} (\mu, \nu) \bigr]_E
	%+ \nabla_{\rho \bigl( t_{\nabla^{\mathrm{bas}}} (\mu, \nu) \bigr)} \eta
%\\
%&\hspace{1cm}
	%+ \nabla_{\rho\mleft( \nabla^{\mathrm{bas}}_\eta \mu \mright)} \nu
	%- \nabla_{\rho(\nu)} \nabla^{\mathrm{bas}}_\eta \mu
	%- \mleft[ \nabla^{\mathrm{bas}}_\eta \mu, \nu \mright]_E
%\\
%&\hspace{1cm}
	%+ \nabla_{\rho(\mu)} \nabla^{\mathrm{bas}}_\eta \nu
	%- \nabla_{\rho\mleft( \nabla^{\mathrm{bas}}_\eta \nu \mright)} \mu
	%- \mleft[ \mu, \nabla^{\mathrm{bas}}_\eta \nu \mright]_E
\eas
for all $\mu, \nu, \eta \in \Gamma(E)$, such that, also using Eq.~\eqref{PullBackVariation},
\bas
\delta_\varepsilon \mleft( \frac{1}{2} \mleft( {}^* t_{\nabla^{\mathrm{bas}}} \mright)\mleft( \varpi_2 \stackrel{\wedge}{,} \varpi_2 \mright) \mright)
&=
-\frac{1}{2} \biggl(
	\mleft( {}^* \mleft( \nabla^{\mathrm{bas}}_\varepsilon t_{\nabla^{\mathrm{bas}}} \mright) \mright) \mleft( \varpi_2 \stackrel{\wedge}{,} \varpi_2 \mright)
\\
&\hspace{1cm}\hphantom{-\frac{1}{2} \biggl(}
	+ \mleft( {}^* t_{\nabla^{\mathrm{bas}}} \mright)\bigl( \mleft( {}^*\nabla\mright) \varepsilon \stackrel{\wedge}{,} \varpi_2 \bigr)
	+ \underbrace{\mleft( {}^* t_{\nabla^{\mathrm{bas}}} \mright)\bigl( \varpi_2 \stackrel{\wedge}{,} \mleft( {}^*\nabla\mright) \varepsilon \bigr)}
	_{\mathclap{= \mleft( {}^* t_{\nabla^{\mathrm{bas}}} \mright)\bigl( \mleft( {}^*\nabla\mright) \varepsilon \stackrel{\wedge}{,} \varpi_2 \bigr)}}
\biggr)
\\
&=
-\frac{\varepsilon^a}{2} ~
	\bigl( {}^* \mleft( R_{\nabla_\rho}(\cdot, \cdot) e_a \mright) \bigr) \mleft( \varpi_2 \stackrel{\wedge}{,} \varpi_2 \mright)
\\
&\hspace{1cm}
	- \mathrm{d}\varepsilon^a \wedge \varpi_2^b ~ {}^*\bigl( t_{\nabla^{\mathrm{bas}}} \mleft( e_a, e_b \mright)\bigr)
	+ \varepsilon^a ~ \varpi_2^b \wedge {}^! \bigl( t_{\nabla^{\mathrm{bas}}} \mleft( \nabla e_a, e_b \mright) \bigr)
\eas
where we used that the torsion is anti-symmetric such that by Prop.~\ref{prop:GradedExtensionPlusAntiSymm}
\ba\label{IchMussDasDringendVerallgemeinern}
\mleft( {}^* t_{\nabla^{\mathrm{bas}}} \mright)\bigl( \varpi_2 \stackrel{\wedge}{,} \mleft( {}^*\nabla\mright) \varepsilon \bigr) = \mleft( {}^* t_{\nabla^{\mathrm{bas}}} \mright)\bigl( \mleft( {}^*\nabla\mright) \varepsilon \stackrel{\wedge}{,} \varpi_2 \bigr),
\ea
because both arguments are 1-forms. We also have
\bas
&\mleft[ e_a, \nabla_Y e_b \mright]_E
	+ \nabla_{\rho\mleft( \nabla_Y e_b \mright)} e_a
	- \nabla_{\mleft[ \rho(e_a), Y \mright]} e_b
	- \nabla_Y\mleft( \mleft[ e_a, e_b \mright]_E \mright)
	- \nabla_Y \nabla_{\rho(e_b)} e_a
	+ t_{\nabla^{\mathrm{bas}}} \mleft( \nabla_Y e_a, e_b \mright)
\\
&=
\mleft[ e_a, \nabla_Y e_b \mright]_E
	+ \nabla_{\rho\mleft( \nabla_Y e_b \mright)} e_a
	- \nabla_{\mleft[ \rho(e_a), Y \mright]} e_b
	- \nabla_Y\mleft( \mleft[ e_a, e_b \mright]_E \mright)
	- \nabla_Y \nabla_{\rho(e_b)} e_a
\\
&\hspace{1cm}
	+ \mleft[ \nabla_Y e_a, e_b \mright]_E
	- \nabla_{\rho\mleft( \nabla_Y e_a \mright)} e_b
	+ \nabla_{\rho(e_b)} \nabla_Y e_a
\\
&=
- \nabla_Y\mleft( \mleft[ e_a, e_b \mright]_E \mright)
	+ \mleft[ e_a, \nabla_Y e_b \mright]_E
	+ \mleft[ \nabla_Y e_a, e_b \mright]_E
	+ \nabla_{\nabla^{\mathrm{bas}}_{e_b} Y} e_a
	- \nabla_{\nabla^{\mathrm{bas}}_{e_a} Y} e_b
\\
&\hspace{1cm}
	+ \nabla_{\rho(e_b)} \nabla_Y e_a
	- \nabla_Y \nabla_{\rho(e_b)} e_a
	- \nabla_{\mleft[ \rho(e_b), Y \mright]} e_a
\\
&\stackrel{\mathclap{\text{Def.~\ref{def:basiccurvature}}}}{=}~~~~
- R_\nabla^{\mathrm{bas}}(e_a, e_b) Y
	+ R_\nabla \mleft( \rho(e_b), Y \mright) e_a
\eas
for all $Y\in \mathfrak{X}(N)$, and we are going to view $Y \mapsto - R_\nabla^{\mathrm{bas}}(e_a, e_b) Y + R_\nabla \mleft( \rho(e_b), Y \mright) e_a$ as an element of $\Omega^1(N; E)$ (locally). Hence, altogether
\bas
\delta_\varepsilon F
~~~~&\stackrel{\mathclap{\text{Def.~\ref{def:EichbosonenUndFeldstaerke}}}}{=}~~~~
- \varepsilon^a ~ {}^!\bigl( R_{\nabla}(\cdot, \cdot) e_a \bigr)
	-\frac{\varepsilon^a}{2} ~
	\mleft( {}^* \mleft( R_{\nabla_\rho}(\cdot, \cdot) e_a \mright) \mright) \mleft( \varpi_2 \stackrel{\wedge}{,} \varpi_2 \mright)
\\
&\hspace{1cm}~~~~
	+ \varepsilon^a \varpi_2^b \wedge {}^! \mleft( R_\nabla(\rho(e_b), \cdot) e_a - R_\nabla^{\mathrm{bas}}(e_a, e_b) \mright)
\\
&\stackrel{\mathclap{\text{Eq.~\eqref{EqPullBackFormelFuerVerschiedeneDefinitionen}}}}{=}~~~~
-\frac{1}{2} ~ \Bigl(
 \mleft(	{}^* R_{\nabla} \mright)\mleft(\mathrm{D} \stackrel{\wedge}{,} \mathrm{D}\mright) \varepsilon
	+ \underbrace{\mleft( {}^* R_{\nabla_\rho} \mright)\mleft( \varpi_2 \stackrel{\wedge}{,} \varpi_2 \mright) \varepsilon}
	_{\mathclap{= \mleft( {}^* R_{\nabla} \mright)\mleft( \mleft({}^*\rho\mright)(\varpi_2) ~\stackrel{\wedge}{,} ~\mleft({}^*\rho\mright)(\varpi_2) \mright) \varepsilon}}
	\Bigr)
\\
&\hspace{1cm}~~~~
	+ \underbrace{\mleft( {}^* R_\nabla \mright) \bigl(\mleft({}^*\rho\mright)(\varpi_2) \stackrel{\wedge}{,} \mathrm{D}\bigr) \varepsilon}
	_{\mathclap{= \frac{1}{2}\mleft( 
		\mleft( {}^* R_\nabla \mright) \bigl(\mleft({}^*\rho\mright)(\varpi_2) ~\stackrel{\wedge}{,}~ \mathrm{D}\bigr) \varepsilon
		+ \mleft( {}^* R_\nabla \mright) \bigl(\mathrm{D} ~\stackrel{\wedge}{,}~ \mleft({}^*\rho\mright)(\varpi_2) \bigr) \varepsilon
	\mright) }}
	- \mleft({}^* R_\nabla^{\mathrm{bas}} \mright) \mleft(\varepsilon \stackrel{\wedge}{,} \varpi_2  \stackrel{\wedge}{,} \mathrm{D}\mright)
\\
&\stackrel{\mathclap{\text{Def.~\ref{def:MinimalCoupling}}}}{=}~~~~
-\frac{1}{2} \Bigl(
	 \mleft(	{}^* R_{\nabla} \mright)\mleft( \mathfrak{D} \stackrel{\wedge}{,} \mathrm{D}\mright) \varepsilon
	- \mleft( {}^* R_{\nabla} \mright)\mleft( \mathfrak{D} \stackrel{\wedge}{,} \mleft({}^*\rho\mright)(\varpi_2) \mright) \varepsilon
\Bigr)
\\
&\hspace{1cm}~~~~
	- \mleft({}^* R_\nabla^{\mathrm{bas}} \mright) \mleft(\varepsilon \stackrel{\wedge}{,} \varpi_2  \stackrel{\wedge}{,} \mathrm{D}\mright)
\\
&=
- \mleft(
	\frac{1}{2} ~ \mleft(	{}^* R_{\nabla} \mright)\mleft( \mathfrak{D} \stackrel{\wedge}{,} \mathfrak{D} \mright) \varepsilon
	+ \mleft({}^* R_\nabla^{\mathrm{bas}} \mright) \mleft(\varepsilon \stackrel{\wedge}{,} \varpi_2  \stackrel{\wedge}{,} \mathrm{D}\mright)
\mright),
\eas
where we introduced the notation $\Gamma(E) \times \Gamma(E) \times \mathfrak{X}(N) \ni (\mu, \nu, Y) \mapsto R_\nabla^{\mathrm{bas}}(\mu, \nu, Y) = R_\nabla^{\mathrm{bas}}(\mu, \nu) Y$ in order to emphasize the anti-symmetrization when applying the graded extension on $R_\nabla^{\mathrm{bas}}$, and we used the same argument on $\mleft( {}^* R_\nabla \mright) \bigl(\mleft({}^*\rho\mright)(\varpi_2) \stackrel{\wedge}{,} \mathrm{D}\bigr) \varepsilon$ as in Eq.~\eqref{IchMussDasDringendVerallgemeinern}.
\end{proof}

\begin{remark}\label{RemVergleicheVonVariationenVonFUndDAPhi}
\leavevmode\newline
These formulas look different when comparing it with the standard formulas, but that is again related to that we use the basic connection for the variations instead. As introduced, we should look at the variation of the components to see how the variation affects the variation of the Lagrangian.

$\bullet$ In order to define gauge invariance the idea is as in \cite{CurvedYMH}, $\delta_\varepsilon F^a$ should be proportional to $F$ which is not the case here for both terms. Explicitly we need that $\delta_\varepsilon F = 0$; in that case we would have for the components (with respect to a frame $\mleft( e_a \mright)_a$ of $E$)
\ba
\delta_\varepsilon F^a
&=
\underbrace{\mleft(\delta_\varepsilon F\mright)^a}_{= 0}
	- F^b ~ \bigl( \delta_\varepsilon ({}^* e_b) \bigr)^a
=
\mleft({}^*\mleft( \nabla^{\mathrm{bas}}_\varepsilon e_b \mright) \mright)^a  F^b
=
\varepsilon^c ~ \mleft({}^*\mleft( \mleft[ e_c, e_b \mright]_E + \nabla_{\rho(e_b)} e_c \mright) \mright)^a F^b \label{eqVariationVonFKomps}
\ea
such that the variation of the components is proportional to themselves and we can then formulate the symmetry on scalar products as usual as a symmetry under (infinitesimal) "rotations", see also the next theorem.

In the proof we saw that we can also write
\bas
\delta_\varepsilon F
&=
-\frac{1}{2} ~ \Bigl(
 \mleft(	{}^* R_{\nabla} \mright)\mleft(\mathrm{D} \stackrel{\wedge}{,} \mathrm{D} \mright) \varepsilon
	+ \mleft( {}^* R_{\nabla_\rho} \mright)\mleft( \varpi_2 \stackrel{\wedge}{,} \varpi_2 \mright) \varepsilon
	\Bigr)
\\
&\hspace{1cm}~~~~
	+ \mleft( {}^* R_\nabla \mright) \bigl(\mleft({}^*\rho\mright)(\varpi_2) \stackrel{\wedge}{,} \mathrm{D} \bigr) \varepsilon
	- \mleft({}^* R_\nabla^{\mathrm{bas}} \mright) \mleft(\varepsilon \stackrel{\wedge}{,} \varpi_2  \stackrel{\wedge}{,} \mathrm{D} \mright).
\eas
Since $\Phi$ and $A$ are regarded as the fields with respect to which the theory gets varied and $M$, $N$ \textit{etc.}~are completely arbitrary up to this point, so, thinking about the whole category of possible manifolds, $\mathrm{D}$ and $\varpi_2$ can be viewed as (in general) independent functionals while $\varepsilon$ is very arbitrary. Thus, in order to get $\delta_\varepsilon F = 0$ we need $R_\nabla = 0$ and $R_\nabla^{\mathrm{bas}} = 0$ in general. $R_\nabla^{\mathrm{bas}} = 0$ sounds reasonable as we discussed in the previous section, recall the discussion around Cor.~\ref{cor:FlatnessVonEichtrafos}, but the condition that $\nabla$ is flat is not a good condition because this will lead to that we have locally the standard formulation of gauge theory which is not the aim of this new formulation. The problems with flatness we are going to discuss later, instead let us discuss why this formula recovers the standard formula when using again action Lie algebroids with canonical flat connections. 

$\bullet$ As usual we use again Cor.~\ref{cor:StandardTheory}, for this assume that $E = N \times \mathfrak{g}$ is an action Lie algebroid for some Lie algebra $\mathfrak{g}$, equipped with the canonical flat connection $\nabla$; as in the proof of Thm.~\ref{thm:ActionLieALgebroid} the canonical flat connection satisfies $R_\nabla^{\mathrm{bas}}= 0$. Thus, we have then $\delta_\varepsilon F = 0$, and by the previous calculation
\bas
\delta_\varepsilon F^a
&=
\varepsilon^c ~ \underbrace{\Phi^*\mleft( \mleft[ e_c, e_b \mright]_{\mathfrak{g}} \mright)^a}_{\mathclap{\text{const.}}} ~ F^b
=
\mleft( \mleft[ \varepsilon, F \mright]_{\mathfrak{g}} \mright)^a
\eas
for $\mleft( e_a \mright)_a$ a constant frame. This is again precisely the expected formula, recall Prop.~\ref{prop:ClassicGaugeTrafoOfFieldStrengthAndMinimalCoupling}, and this is also shown and argued in \cite[see the second paragraph after Eq.~(11), keep in mind that there is a different sign for $\varepsilon$]{CurvedYMH}, where also the general formula with the curvature got stated, but again only for the components without knowing the full tensor. 
%
%Finally, observe in the general situation that when we would rewrite the formula using $\delta^{(1)}$ instead (recall the discussion around Def.~\ref{def:TotalInfinGaugeTransform}) such that $\delta_\varepsilon^{(1)} F^a = \delta_\varepsilon F^a$, then we get, using the previous calculations,
%\bas
%\delta_\varepsilon^{(1)}F
%&=
%\delta^{(1)}_\varepsilon F^a \otimes \Phi^*e_a
	%+ F^b \otimes \delta^{(1)}_\varepsilon \mleft( \Phi^*e_b \mright)
%\\
%&=
%\varepsilon^c F^b \otimes \Phi^*\mleft( \mleft[ e_c, e_b \mright]_E + \nabla_{\rho(e_b)} e_c \mright)
	%- \varepsilon^c F^b \otimes \Phi^*\mleft( \nabla_{\rho(e_c)} e_b \mright)
%\\
%&\hspace{1cm}
	%- \mleft(
	%\frac{1}{2} ~ \mleft(	\Phi^* R_{\nabla} \mright)\mleft( \mathfrak{D}^A \Phi \stackrel{\wedge}{,} \mathfrak{D}^A \Phi \mright) \varepsilon
	%+ \mleft(\Phi^* R_\nabla^{\mathrm{bas}} \mright) \mleft(\varepsilon \stackrel{\wedge}{,} A  \stackrel{\wedge}{,} \mathrm{D}\Phi\mright)
%\mright)
%\\
%&=
%\mleft( \Phi^* t_{\nabla_\rho} \mright)\mleft( \varepsilon, F \mright) 
	%- \mleft(
	%\frac{1}{2} ~ \mleft(	\Phi^* R_{\nabla} \mright)\mleft( \mathfrak{D}^A \Phi \stackrel{\wedge}{,} \mathfrak{D}^A \Phi \mright) \varepsilon
	%+ \mleft(\Phi^* R_\nabla^{\mathrm{bas}} \mright) \mleft(\varepsilon \stackrel{\wedge}{,} A  \stackrel{\wedge}{,} \mathrm{D}\Phi\mright)
%\mright),
%\eas
%where we used Eq.~\eqref{eqVariationVonFKomps} but now with non-zero $\delta_\varepsilon F$.
\end{remark}

Using this and Remark \ref{RemVergleicheVonVariationenVonFUndDAPhi} we can finally formulate what we need to have a gauge-invariant Lagrangian; for this we need to calculate $\delta_\varepsilon \mathfrak{L}_{\mathrm{YMH}}$ (Def.~\ref{def:CurvedYMHLagrangian}).

\begin{theorems}{The gauge invariance of the Lagrangian, \newline \cite[especially the discussion around Eq.~(16)]{CurvedYMH}}{GaugeInvariantStandardLagrangian}
Let $M$ be a spacetime with a spacetime metric $\eta$, $N$ a smooth manifold, $E \to N$ a Lie algebroid, $\nabla$ a connection on $E$, $\kappa$ and $g$ fibre metrics on $E$ and $\mathrm{T}N$, respectively. Also let $V \in C^\infty(N)$ and assume that the following \textbf{compatibility conditions} hold:
\ba
	R_\nabla &= 0, \\
	R_\nabla^{\mathrm{bas}} &= 0, \\
	\nabla^{\mathrm{bas}} \kappa &= 0, \\
	\nabla^{\mathrm{bas}} g &= 0, \\
	{}^*\mleft(\mathcal{L}_{({}^*\rho)(\varepsilon)} V\mright) &= 0 \label{PotentialCompatibility}
\ea
for all $\varepsilon \in \mathcal{F}^0_E(M; {}^*E)$. Then we have
\ba
\delta_\varepsilon \mathfrak{L}_{\mathrm{YMH}}
&=
0
\ea
for all $\varepsilon \in \mathcal{F}^0_E(M; {}^*E)$.
\end{theorems}

\begin{remark}\label{RemarkUeberPotentialCompatibility}
\leavevmode\newline
Since Lie derivatives describe the canonical flat connection on smooth functions (canonical flatness with respecto to the trivial line bundle over $N$, the notation of Eq.~\eqref{PotentialCompatibility} is the same as introduced in Remark \ref{RemarkNotationvonPullbackConnection} and as in other similar terms, \textit{i.e.}
\bas
\mleft.\mleft({}^*\mleft(\mathcal{L}_{({}^*\rho)(\varepsilon)} V\mright)\mright)(\Phi,A)\mright|_p
&=
\mleft.\Phi^*\mleft(\mathcal{L}_{(\Phi^*\rho)(\epsilon)} V\mright)\mright|_p
=
\mathcal{L}_{\mleft(\rho_{\Phi(p)}\mright)\mleft(\epsilon_p\mright)} V
\eas
for all $(p,\Phi,A) \in M \times \mathfrak{M}_E(M;N)$, where $\epsilon \coloneqq \varepsilon(\Phi,A) \in \Gamma(\Phi^*E)$. It is clear that Eq.~\eqref{PotentialCompatibility} generalizes Eq.~\eqref{ClassicPotential} if $E$ is an action Lie algebroid.
\end{remark}

\begin{proof}
\leavevmode\newline
Observe that $* ({}^*V ) = {}^*V ~ \mathrm{dvol}_\eta$, where $\mathrm{dvol}_\eta$ is the canonical volume form of $\eta$ and the sign might differ depending on the definition of the Hodge star operator. Using that, we only need to look at the variation of ${}^*V$ because $\mathrm{dvol}_\eta$ is clearly not affected by $\delta$, hence,
\bas
\delta_\varepsilon ({}^*V)
&=
- {}^*\mleft(\mathcal{L}_{({}^*\rho)(\varepsilon)} V\mright)
=
0
\eas
for all $\varepsilon \in \mathcal{F}^0_E(M; {}^*E)$, where we used the last condition. Up to a sign we also have\footnote{As also defined in \cite[\S 7.2, Definition 7.2.4; page 408]{hamilton}.}
\bas
\omega \wedge *\psi
&=
\langle \omega, \psi \rangle ~ \mathrm{dvol}_\eta
\eas
for all $\omega, \psi \in \Omega^k(M)$ $(k \in \mathbb{N}_0)$, where $\langle \cdot, \cdot \rangle$ is the standard scalar product defined on $\Omega^k(M)$ using $\eta$, \textit{i.e.}
\bas
\langle \omega, \psi \rangle
&=
\frac{1}{k!} ~ \omega_{\alpha_1, \dotsc, \alpha_k} \psi^{\alpha_1, \dotsc, \alpha_k}
\eas
where we express the forms with respect to coordinate vector fields $\mleft(\partial_\alpha\mright)_\alpha$ on $M$ and raising an index is done by using $\eta$; especially, $\delta_\varepsilon$ satisfies the Leibniz rule on $\langle \cdot, \cdot \rangle$ because $\delta_\varepsilon \eta = 0$. Hence, similar to before,
\bas
\delta_\varepsilon\mleft(\omega \wedge *\psi\mright)
&=
\delta_\varepsilon \bigl( \langle \omega, \psi \rangle ~ \mathrm{dvol}_\eta \bigr)
=
\bigl( 
	\langle \delta_\varepsilon \omega, \psi \rangle
	+ \langle \omega, \delta_\varepsilon \psi \rangle
 \bigr) ~ \mathrm{dvol}_\eta
=
\delta_\varepsilon \omega \wedge *\psi
	+ \omega \wedge *\mleft(\delta_\varepsilon\psi\mright)
\eas
for all $\varepsilon \in \mathcal{F}^0_E(M; {}^*E)$. This clearly extends to Def.~\ref{def:GradingOfProducts} by the Leibniz rule (\textit{e.g.}~this is immediate by the coordinate expression of graded extensions), in the sense of
\bas
\delta_\varepsilon \bigl(
\mleft( {}^*\kappa \mright)\mleft(F \stackrel{\wedge}{,} *F\mright)
\bigr)
&=
\mleft( \delta_\varepsilon \mleft({}^*\kappa \mright) \mright)\mleft(F \stackrel{\wedge}{,} *F\mright)
	+ \mleft( {}^*\kappa \mright)\mleft(\delta_\varepsilon F \stackrel{\wedge}{,} *F\mright)
	+ \mleft( {}^*\kappa \mright)\bigl(F \stackrel{\wedge}{,} *(\delta_\varepsilon F)\bigr)
\eas
for all $\varepsilon \in \mathcal{F}^0_E(M; {}^*E)$, similarly for other all terms of that form.
Observe that we have $\delta_\varepsilon F = 0$ additionally to $\delta_\varepsilon \mathfrak{D} = 0$ by Prop.~\ref{prop:InfinitesimalGaugeTrafoOfMinimalCoupleSmiley} and \ref{prop:GaugeTrafosOfFieldStrengthAndMinimalCoupling} and due to $R_\nabla = 0$ and $R_\nabla^{\mathrm{bas}} = 0$.
So, we get in total, using the result of the variation of the potential $V$,
\bas
\delta_\varepsilon \mathfrak{L}_{\mathrm{YMH}}
&=
\delta_\varepsilon \mleft(
	- \frac{1}{2} \mleft( {}^*\kappa \mright)\mleft(F \stackrel{\wedge}{,} *F\mright)
	+ \mleft( {}^*g \mright)\mleft(\mathfrak{D} \stackrel{\wedge}{,} *\mathfrak{D} \mright)
	- *({}^*V)
\mright)
\\
&=
- \frac{1}{2} \bigl( \delta_\varepsilon \mleft( {}^*\kappa \mright)\bigr)\mleft(F \stackrel{\wedge}{,} *F\mright)
	+ \bigl( \delta_\varepsilon \mleft( {}^*g \mright) \bigr)\mleft(\mathfrak{D} \stackrel{\wedge}{,} *\mathfrak{D} \mright)
\\
&\stackrel{\mathclap{\text{Eq.~\eqref{PullBackVariation}}}}{=}~~~~
\frac{1}{2} \biggl( {}^*\mleft( \nabla^{\mathrm{bas}}_\varepsilon \kappa \mright)\biggr)\mleft(F \stackrel{\wedge}{,} *F\mright)
	- \biggl( {}^*\mleft( \nabla^{\mathrm{bas}}_\varepsilon g \mright) \biggr)\mleft(\mathfrak{D}  \stackrel{\wedge}{,} *\mathfrak{D} \mright)
\\
&=
0
\eas
for all $\varepsilon \in \mathcal{F}^0_E(M; {}^*E)$, using the metric compatibilities in the assumed conditions.
\end{proof}

Lie algebroids equipped with a connection with vanishing basic curvature are also called \textbf{Cartan algebroids} as \textit{e.g.}~defined in \cite[\S 2.3]{blaomTangentBundleAsLieGroup}; hence, this special type of Lie algebroid seems to be the relevant one for gauge theories, as we already have noticed in the discussion about gauge transformations. Let us collect all the results we got along the way in relation to the standard formulation.

\begin{theorems}{Standard formulation of gauge theory is recovered, \cite{CurvedYMH}}{StandardEichtheorieStecktInDenBedingung}
Assume that $N=W$ is a vector space, $E= N \times \mathfrak{g}$ an action Lie algebroid for a Lie algebra $\mathfrak{g}$ whose Lie algebra action $\gamma$ is induced by a Lie algebra representation $\psi: \mathfrak{g} \to \mathrm{End}(W)$, and assume that $\nabla$ is the canonical flat connection of $E$. Moreover, let $\kappa$ be a fibre metric of $E$ which is a canonical extension of an $\mathrm{ad}$-invariant scalar product of $\mathfrak{g}$, similarly $g$ is a metric on $\mathrm{T}W \cong W \times W$ constantly extending an $\psi$-invariant scalar product of $W$. Finally, let $V \in C^\infty(N)$ such that it satisfies Eq.~\eqref{PotentialCompatibility}.

Then the compatibility conditions of Thm.~\ref{thm:GaugeInvariantStandardLagrangian} are satisfied, and we recover the standard theory: The Lagrangian $\mathfrak{L}_{\mathrm{YMH}}$ is as in the standard formulation and gauge-invariant, as does the field strength $F$, the minimal coupling $\mathfrak{D}$, the field of gauge bosons $A$, the field $\Phi$, and its variation $\delta_\varepsilon \Phi$; with respect to a constant frame $\mleft( e_a \mright)_a$ of $E$ and a constant frame $\mleft(\partial_\alpha\mright)_\alpha$ of $\mathrm{T}W$, $\delta_\varepsilon A^a$ coincide with the components of the variation of $A$ of the standard formulation, as does $\delta_\varepsilon F^a$ and $\delta_\varepsilon \mleft( \mathfrak{D} \mright)^\alpha$.
\end{theorems}

\begin{remark}
\leavevmode\newline
As discussed in subsection \ref{InfinitesimalGaugeTransformation}, the infinitesimal gauge transformation of the Lagrangian is just $\delta_\varepsilon \mathfrak{L}_{\mathrm{YMH}} = \mathcal{L}_{\Psi_\varepsilon} \mathfrak{L}_{\mathrm{YMH}}$. Thence, the definition of $\Psi_\varepsilon$ is of importance for the gauge invariance of the Lagrangian, that is, how $\Phi$ and how the components of $A$ transform; recall Prop.~\ref{prop:VariationOfA}. Given that unique $\Psi_\varepsilon$ of Prop.~\ref{prop:VariationOfA} (for a fixed $\nabla$) one can take any other connection on $E$ to formulate $\delta_\varepsilon A$ and $\delta_\varepsilon$ in general, one will always get the gauge invariance of the Lagrangian, and the components of $A$ \textit{etc.}~will also transform the same. Hence, the statement about the transformations of the components could also be formulated as that $\Psi_\varepsilon$ reduces to the same vector field on the space of fields as in the classical situation.

However, as already mentioned before, the definition of $\Psi_\varepsilon$ depends on $\nabla$; but \textbf{given a $\Psi_\varepsilon$} the choice of connections for the definition of $\delta_\varepsilon$ does not affect the gauge invariance of the Lagrangian.

When we would use $\nabla_\rho$ to define the gauge transformations of $E$-valued functionals, then many of the total formulas would also restrict to standard formulas due to the flatness of $\nabla$ in the standard situation, not just their components, recall Thm.~\ref{thm:NewFormulaRecoversOldGaugeTrafoYay}. That is especially due to that $\nabla_\rho$ will be a canonical flat connection, while the basic connection is flat but it may not have a parallel frame due to the kernel of the anchor. If we would use $\nabla_\rho$, we would loose the flatness of the gauge transformations as discussed in Cor.~\ref{cor:FlatnessVonEichtrafos} whenever $\nabla_\rho$ is not flat anymore. However, we have now seen that $\nabla$ needs to be flat for the gauge invariance of the Lagrangian such that this does seemingly not matter; but we will see later that there is the possibility to allow non-flat $\nabla$.
\end{remark}

\begin{proof}[Proof of Thm.~\ref{thm:StandardEichtheorieStecktInDenBedingung}]
\leavevmode\newline
First recall Thm.~\ref{thm:ActionLieALgebroid}, especially, the canonical flat connection satisfies $R_\nabla^{\mathrm{bas}}=0$; the metric compatibilities follow by Lemma \ref{lem:MetricCompsAdInvUndLieAlgebraRepSymm}, hence, all compatibility conditions of Thm.~\ref{thm:GaugeInvariantStandardLagrangian} are satisfied. That the formulas restrict to the standard ones we have discussed in Cor.~\ref{cor:StandardTheory} and \ref{cor:EichtrafovonDAPHIinClassicIstBabyEinfach}, and Remarks \ref{RemUeberVariationVonHiggs}, \ref{RemDifferentVersionsOfGaugeTrafos}, and \ref{RemVergleicheVonVariationenVonFUndDAPhi}.
\end{proof}

But due to the compatibility condition about the flatness we arrive locally now at an action Lie algebroid, regardless of the specific choice of $E$; and as we have seen multiple times, action Lie algebroids recover the classical theory.

\begin{corollaries}{Gauge invariance implies standard theory, \newline \cite[the discussion around Eq.~(9)ff.]{CurvedYMH}}{ManBrauchZetaWahrscheinlich}
Let us have the same conditions as in Thm.~\ref{thm:GaugeInvariantStandardLagrangian}. Then $E$ is locally isomorphic to an action Lie algebroid $N \times \mathfrak{g}$ such that $\nabla$ is its canonical flat connection and $N =W$ is a vector space, also, $\delta_\varepsilon A^a$ are then of the form as in the standard formulation of gauge theory with respect to a constant frame $\mleft( e_a \mright)_a$, as does $\delta_\varepsilon F^a$.
\end{corollaries}

\begin{remark}
\leavevmode\newline
Using Thm.~\ref{thm:StandardEichtheorieStecktInDenBedingung} one can also derive the other classical formulas depending on the conditions about the structure, like a given Lie algebra representation. But those are just technicalities, the important part is to have an action Lie algebroid and its canonical flat connection.
\end{remark}

\begin{proof}[Proof of Cor.~\ref{cor:ManBrauchZetaWahrscheinlich}]
\leavevmode\newline
By Thm.~\ref{thm:ActionLieALgebroid} we immediately know that $E\cong N \times \mathfrak{g}$ is an action Lie algebroid for a Lie algebra $\mathfrak{g}$ with Lie algebra action $\gamma: \mathfrak{g} \to \mathfrak{X}(N)$ on some open neighbourhood around each point, in such a way that $\nabla$ is its canonical flat connection. Restricting the neighbourhood even further results into $N=W$ for some vector space $W$. The remaining proof is exactly as in Thm.~\ref{thm:StandardEichtheorieStecktInDenBedingung}.
\end{proof}

Hence, we arrive locally always at the standard situation; at least at something very similar to it. 
%(up to whether the fibre metrics\footnote{Not speaking about the spacetime metric of course.} are extensions of scalar products, the potential is a polynomial and whether there is a Lie algebra representation behind the action and so on, but those are rather technical than important details). 
The Lie algebra action might not come from a Lie algebra representation and the metrics might look exotic, but these are just technicalities which are not important for us, especially when one recalls that the aim of this theory is that gauge theory is covariantized in order to easily replace $\nabla$ with non-flat connections. However, there is a possibility in allowing non-flat connections, and for this we need to change the field strength to compensate the curvature term in Prop.~\ref{prop:GaugeTrafosOfFieldStrengthAndMinimalCoupling} which is mainly the reason behind the compatibility condition about flatness, as also argued as an ansatz in \cite[second paragraph after Equation (11)]{CurvedYMH}. We want to motivate this change by a field redefinition instead, a transformation which keeps the Lagrangian invariant after a modification, but breaking the condition about flatness.

Before we do this let us shortly summarize an aspect of the classical theory which is now obvious due to this formulation.

\begin{corollaries}{Abelian Lie algebras and zero torsion}{AbelianIffNablaBasIsLeviCivita}
Let $E = N \times \mathfrak{g}$ be an action Lie algebroid over $N$ for a Lie algebra $\mathfrak{g}$, equipped with the canonical flat connection $\nabla$. Then
\ba
t_{\nabla^{\mathrm{bas}}}=0
&\Leftrightarrow
\mathfrak{g} \text{ is abelian}.
\ea
\end{corollaries}

\begin{remark}
\leavevmode\newline\label{remELEVICITAOfBasnbala}
Given a fixed fibre metric $\kappa$ such that $\nabla^{\mathrm{bas}} \kappa = 0$, as in one of the compatibility conditions, we would therefore know that $\nabla^{\mathrm{bas}}$ is an $E$-Levi-Civita connection if and only if $\mathfrak{g}$ is abelian.\footnote{See \textit{e.g.}~\cite[\S 2.5]{ELeviCivita} for a definition of such Levi-Civita connections. However, it is precisely defined as usual.}
\end{remark}

\begin{proof}
\leavevmode\newline
We only need to check under which conditions the tensor of the torsion of the basic connection is zero for constant sections $\mu, \nu$ since these generate all sections, especially we have $\nabla \mu = \nabla \nu = 0$ and $\mleft[ \mu, \nu \mright]_E = \mleft[ \mu,\nu \mright]_{\mathfrak{g}}$:
\bas
&&
0
&=
\underbrace{t_{\nabla^{\mathrm{bas}}}( \mu, \nu)}_{\mathclap{= - t_{\nabla_{\rho}}( \mu, \nu)}}
\\
&\Leftrightarrow&
0
&=
t_{\nabla_{\rho}}( \mu, \nu)
\\
&\Leftrightarrow&
0
&=
\mleft[ \mu, \nu \mright]_{\mathfrak{g}}.
\eas
\end{proof}