\section{Field redefinition}\label{FieldRedefSection}

We want to study a certain transformation which keeps the action invariant; for this recall first Sylvester's determinant theorem (\cite[Appendix B; page 271]{DeterminantenTheorem}), also called Weinstein-Aronszajn identity, which says
\ba\label{SylvestersDeterminante}
\mathrm{det}\mleft( \mathds{1}_n + C B \mright)
&=
\mathrm{det}\mleft( \mathds{1}_m + B C \mright),
\ea
where $n, m \in \mathbb{N}$, $\mathds{1}_n$ and $\mathds{1}_m$ are the identity matrices on $\mathbb{R}^n$ and $\mathbb{R}^m$, respectively, and $C \in \mathbb{R}^{n \times m}$ and $B \in \mathbb{R}^{m \times n}$. 

Abstractly spoken, the typical idea of field redefinitions is the same as for covariantizing physical theories and definitions. One applies a non-constant change of coordinates in such a way that one leaves the "inertial frame" as in classical mechanics, resulting to that one gets extra terms in several formulas like contributions coming from "inertial forces"; but one still has the same physics, because the Lagrangian is actually invariant under that change of coordinates. Usually one reformulates the same theory naturally supporting those extra terms, leading to a theory naturally invariant under the observed changes of coordinates in all definitions, which is often referred to as covariantization by physicists. Up to this point it is just something aesthetic one could say, however, the next step is then to study whether the mentioned extra terms always vanish in some coordinate system. Think \textit{e.g.}~of connection 1-forms of connections and one started with a theory with an underlying flat connection such that the initial coordinate system was also the parallel frame where the 1-forms are zero, and the connection 1-forms then arise as those extra terms in other coordinate systems. Studying whether those connection 1-forms always can vanish in some coordinate system, means, whether or not non-trivial curvatures are possible.

In our case the "coordinates" we speak of is the structural data, especially $A$, a coordinate of $\mathfrak{M}_E$, but also for example $\nabla$, and, so, the extra terms are going to be in the compatibility condition about the curvature of $\nabla$. To keep the same physics, that is, the Lagrangian stays invariant, we need to correct especially the field strength since the field strength is of course directly affected by non-trivial changes of $A$. Since the previously-discussed flatness of $\nabla$ is given by the infinitesimal gauge transformation of the field strength, there is the hope that whatever we need to add to "correct" the field strength will also lead to a gauge invariant theory allowing non-flat connections. As a next step it is then natural to rewrite gauge theory allowing those extra terms, leading to a theory naturally invariant under the chosen change of "coordinates" (as in coordinate-independence), while the classical theory is just the same theory, written with respect to "coordinates" where those extra terms are zero. Finally, one may want to discuss what happens when these extra terms actually never vanish, even after such changes of "coordinates". So, precisely the same as in the previous paragraph, just happening with a different type of "coordinate", which is why we are not going to say covariantization but field redefinition.

Let us start defining that field redefinition.

\begin{definitions}{Field redefinition}{FieldRedefinition}
Let $M, N$ be smooth manifolds, $E \to N$ a Lie algebroid, $\nabla$ a connection on $E$, and $\kappa$ and $g$ fibre metrics on $E$ and $\mathrm{T}N$, respectively. Also let $\gls{1lambda} \in \Omega^1(N; E)$ such that $\gls{1Lambda} \coloneqq \mathds{1}_E - \lambda \circ \rho$ is an element of $\sAut(E)$. We then define the \textbf{field redefinition} by 
\ba\label{EqFieldRedefFuerA}
\gls{1pivarwidetildelambda}
&\coloneqq 
\mleft( {}^* \Lambda \mright) (\varpi_2)+ {}^! \lambda,
\\
\gls{0nabla0widetildelambda}
&\coloneqq
\nabla
	+ \mleft( \Lambda \circ \mathrm{d}^{\nabla^{\mathrm{bas}}} \circ \Lambda^{-1} \mright) \lambda,
\label{FieldTrafoOfNabla} \\
\gls{1kappawidetildelambda}
&\coloneqq
\kappa \circ \mleft( \Lambda^{-1}, \Lambda^{-1} \mright),
\label{FieldTrafoOfKappa} \\
\gls{gwidetildelambda}
&\coloneqq
g \circ \mleft( \widehat{\Lambda}^{-1}, \widehat{\Lambda}^{-1} \mright), \label{FieldTrafoOfG}
\ea
where $\gls{1Lambdatilde} \coloneqq \mathds{1}_{\mathrm{T}N} - \rho \circ \lambda$.
\end{definitions}

\begin{remark}
\leavevmode\newline
\indent $\bullet$ $\widehat{\Lambda}$ and $\Lambda$ are already endomorphisms by definition, and, so, by Eq.~\eqref{SylvestersDeterminante} we know that $\widehat{\Lambda} \in \sAut(\mathrm{T}N)$ if and only if $\Lambda \in \sAut(E)$. Also recall that we view elements of $\Omega^1(N;E)$ also as elements of $\Omega^{1,0}(N,E;E)$, Def.~\ref{def:ExteriorCovariantDerivatives}, therefore $\mleft( \Lambda \circ \mathrm{d}^{\nabla^{\mathrm{bas}}} \circ \Lambda^{-1} \mright) \lambda \in \Omega^{1,1}(N,E;E) \cong \Omega^1(N; \mathrm{End}(E))$. 

$\bullet$ 
We can rewrite $\widetilde{\varpi_2}^\lambda$ to
\ba
\widetilde{\varpi_2}^\lambda
&= 
\mleft( {}^* \Lambda \mright) (\varpi_2)+ {}^! \lambda
\stackrel{\text{ Eq.~\eqref{EqPullBackFormelFuerVerschiedeneDefinitionen} }}{=}
\varpi_2
	- \bigl(\underbrace{{}^*(\lambda \circ \rho)}_{\mathclap{ = ({}^*\lambda) \circ ({}^*\rho) }}\bigr)(\varpi_2)
	+ ({}^*\lambda)(\mathrm{D})
= 
\varpi_2 + \mleft( {}^* \lambda \mright) \mleft( \mathfrak{D} \mright).
\ea
With respect to points $(\Phi, A) \in \mathfrak{M}_E(M;N)$ this implies
\ba\label{EqAlternativeFormelFuerFieldredefofA}
\mleft(\widetilde{\varpi_2}^\lambda\mright)(\Phi,A)
&=
\gls{a0widetildelambda}
= 
\mleft( \Phi^* \Lambda \mright) (A)+ \Phi^! \lambda
= 
A + \mleft( \Phi^* \lambda \mright) \mleft( \mathfrak{D}^A \Phi \mright).
\ea
Viewing $A$ and $\varpi_2$ as coordinates on $\mathfrak{M}_E(M;N)$, the idea of the field redefinition is a change of coordinates, consisting of a translation and a rotation with $\Lambda$ which is basically a first order approximation of the typical rotation given by an exponential. The other formulas of the field redefinition are taken in such a way to keep all compatibility conditions in Thm.~\ref{thm:GaugeInvariantStandardLagrangian} but the one about the curvature of $\nabla$. We will see this in the following.

$\bullet$ If we additionally have $R_\nabla^{\mathrm{bas}} = 0$, then we have
\bas
\mleft( \mathrm{d}^{\nabla^{\mathrm{bas}}} \mright)^2
&=
0
\eas
by Prop.~\ref{prop:SnablamitREnabla}, thus, also
\bas
\mleft( \Lambda \circ \mathrm{d}^{\nabla^{\mathrm{bas}}} \circ \Lambda^{-1} \mright)^2
&=
\Lambda \circ \mleft( \mathrm{d}^{\nabla^{\mathrm{bas}}} \mright)^2 \circ \Lambda^{-1} 
=
0,
\eas
hence, we add then an exact term to $\nabla$.

$\bullet$ Eq.~\eqref{EqFieldRedefFuerA} was suggested by one of my supervisors, Thomas Strobl, and the first task of my PhD was to calculate all the remaining formulas and properties needed for the following discussions. In \cite[the example at the very end, right before the conclusion]{CurvedYMH} some transformation was discussed which is a special and simplified situation of the field redefinition. Thomas Strobl got this special example of the field redefinition after a private dialogue with Edward Witten.
\end{remark}

\begin{remarks}{An important note about notation}{TrickyLeibnizRuleForConnections}
Due to $\lambda \in \Omega^{1,0}(N,E;E)$ one may want to write
\bas
\mleft( \Lambda \circ \mathrm{d}^{\nabla^{\mathrm{bas}}} \circ \Lambda^{-1} \mright) \lambda
&=
\mleft( \Lambda \circ \nabla^{\mathrm{bas}} \circ \Lambda^{-1} \mright) \lambda
=
\mathrm{d}^{\Lambda \circ \nabla^{\mathrm{bas}} \circ \Lambda^{-1}} \lambda,
\eas
but the first equality is \textbf{not} correct with our notation! Keep in mind that we have two degrees in form of the spaces $\Omega^{p,q}(N,E;E)$ ($p,q \in \mathbb{N}_0$), so, there are Leibniz rules involved on the $p$-degree if $p \neq 0$, here $p = 1$. That is, for $Y \in \mathfrak{X}(N)$ and $\nu \in \Gamma(E)$, compare
\bas
\mleft(\mleft( \Lambda \circ \mathrm{d}^{\nabla^{\mathrm{bas}}} \circ \Lambda^{-1} \mright) \lambda\mright)(Y, \nu)
&=
\Lambda\mleft( 
	\mleft(\nabla_\nu^{\mathrm{bas}}\mleft( \Lambda^{-1}\circ \lambda \mright)\mright)(Y)
\mright)
\\
&=
\Lambda\mleft(
	\nabla_\nu^{\mathrm{bas}}\mleft( \mleft(\Lambda^{-1} \circ \lambda\mright)(Y) \mright)
\mright)
	- \lambda \mleft( \nabla_\nu^{\mathrm{bas}} Y \mright)
\\
&=
\mleft( \Lambda \circ \nabla^{\mathrm{bas}}_\nu \circ \Lambda^{-1} \mright)\bigl(\lambda(Y)\bigr)
	- \lambda\mleft( \nabla^{\mathrm{bas}}_\nu Y \mright)
\eas
with
\bas
\mleft(\mathrm{d}^{\Lambda \circ \nabla^{\mathrm{bas}} \circ \Lambda^{-1}} \lambda\mright)(Y, \nu)
&=
\mleft( \Lambda \circ \nabla^{\mathrm{bas}}_\nu \circ \Lambda^{-1} \mright)\bigl(\lambda(Y)\bigr)
	- \lambda\mleft( \mleft(\Lambda \circ \nabla^{\mathrm{bas}}_\nu \circ \Lambda^{-1}\mright)Y \mright).
\eas
Hence, due to the Leibniz rules, a composition of maps with connections is not the same as usual compositions of maps, here with a differential. With $\Lambda \circ \nabla^{\mathrm{bas}} \circ \Lambda^{-1}$ we mean the whole object as a connection, so, acting on $\lambda$, extending $\Lambda \circ \nabla^{\mathrm{bas}} \circ \Lambda^{-1}$ as an $E$-connection to $\Omega^1(N;E)$. While each component in $\Lambda \circ \mathrm{d}^{\nabla^{\mathrm{bas}}} \circ \Lambda^{-1}$ acts separately on forms like $\lambda$, and $\nabla^{\mathrm{bas}}$ is extended as $E$-connection to $\Omega^1(N;E)$ (without the conjugation). Therefore one needs to be very careful about how to use conjugations like $\Lambda \circ \dotsc \circ \Lambda^{-1}$ and how to put square brackets, especially when connections are involved. Thus, also
\ba\label{OneofmanyformulasForTildeNabla}
\mleft(\mleft( \Lambda \circ \mathrm{d}^{\nabla^{\mathrm{bas}}} \circ \Lambda^{-1} \mright) \lambda\mright)(\cdot, \nu)
&=
\Lambda\mleft(
	\nabla^{\mathrm{bas}}_\nu\mleft(
		\Lambda^{-1} \circ \lambda
	\mright)
\mright)
\neq
\mleft( \Lambda \circ \nabla^{\mathrm{bas}}_\nu \circ \Lambda^{-1} \mright)\lambda.
\ea
If one always wants to write $\mathrm{d}^{\nabla^{\mathrm{bas}}} = \nabla^{\mathrm{bas}}$ for elements of $\Omega^{p,0}(N,E;E)$ as at the beginning of this remark, then one needs to introduce a notation for extensions as of $\nabla^{\mathrm{bas}}$ to $\Omega^1(N;E)$ in order to avoid precisely the confusion of notation discussed here.
\end{remarks}

We have actually the following corollary relating both notations/notions.

\begin{corollaries}{Conjugation of differentials}{ConjugationOfDifferentialsAreShitty}
Let $N$ be smooth manifolds, $E \to N$ a Lie algebroid, and $\nabla$ a connection on $E$. Also let $\lambda \in \Omega^1(N; E)$ such that $\Lambda = \mathds{1}_E - \lambda \circ \rho$ is an element of $\sAut(E)$. Then
\ba\label{eqKrassWieDasBeiETileNablaAussieht}
&\left(\mathrm{d}^{ \Lambda \circ \nabla^{\mathrm{bas}} \circ \Lambda^{-1}} \omega\right)\left( X_1, \dotsc, X_p, \nu_0, \dotsc, \nu_q \right) \nonumber \\
&= 
\Biggl( \mleft( \Lambda \circ \mathrm{d}^{\nabla^{\mathrm{bas}}} \circ \Lambda^{-1} \mright)
\mleft( \omega \circ \mleft( \vphantom{\widehat{\Lambda}, \dotsc, \widehat{\Lambda}} \smash{\underbrace{\widehat{\Lambda}, \dotsc, \widehat{\Lambda}}_{\mathclap{p \text{ times}}}}, \vphantom{\mathds{1}_{E}, \dotsc, \mathds{1}_E} \smash{\underbrace{\mathds{1}_{E}, \dotsc, \mathds{1}_E}_{q \text{ times}}} \mright) \mright)
 \Biggr)
\mleft( \widehat{\Lambda}^{-1}(X_1), \dotsc, \widehat{\Lambda}^{-1}(X_p), \nu_0, \dotsc, \nu_q \mright)
\ea
for all $\omega \in \Omega^{p,q}(N,E;E)$ ($p, q \in \mathbb{N}_0$), $X_1, \dotsc, X_p \in \mathfrak{X}(N)$ and $\nu_0, \dotsc, \nu_q \in \Gamma(E)$. Equivalently,
\ba
&\mathrm{d}^{ \Lambda \circ \nabla^{\mathrm{bas}} \circ \Lambda^{-1}} \biggl(
	\Lambda \circ \omega \circ \Bigl( \underbrace{\widehat{\Lambda}^{-1}, \dotsc, \widehat{\Lambda}^{-1}}_{p \text{ times}}, \underbrace{\mathds{1}_{E}, \dotsc, \mathds{1}_E}_{q \text{ times}} \Bigr)
\biggr)
\nonumber\\
&=
\Lambda \circ \mleft(\mathrm{d}^{\nabla^{\mathrm{bas}}} \omega\mright) \circ \Bigl( \underbrace{\widehat{\Lambda}^{-1}, \dotsc, \widehat{\Lambda}^{-1}}_{p \text{ times}}, \underbrace{\mathds{1}_{E}, \dotsc, \mathds{1}_E}_{q+1 \text{ times}} \Bigr).
\ea
\end{corollaries}

\begin{remark}
\leavevmode\newline
The second formulation emphasizes that it is roughly about a commutation relation between the conjugation with $\Lambda$ and the differential with the basic connection.
\end{remark}

\begin{proof}[Proof of Cor.~\ref{cor:ConjugationOfDifferentialsAreShitty}]
\leavevmode\newline
That is a straightforward calculation, writing $ {}^E\widetilde{\nabla} \coloneqq \Lambda \circ \nabla^{\mathrm{bas}} \circ \Lambda^{-1}$,
\bas
&\left(\mathrm{d}^{{}^E\widetilde{\nabla}} \omega\right)\left( X_1, \dots, X_p, \nu_0, \dots, \nu_q \right) \nonumber \\
&= \sum_{i=0}^q (-1)^i \biggl( {}^E\widetilde{\nabla}_{\nu_i} \mleft( \omega\mleft( \mleft(\widehat{\Lambda} \circ \widehat{\Lambda}^{-1}\mright) (X_1), \dots, \mleft(\widehat{\Lambda} \circ \widehat{\Lambda}^{-1}\mright) (X_p), \nu_0, \dots, \widehat{\nu}_i, \dots \nu_q\mright) \mright) \nonumber \\
&\qquad\qquad\qquad - \sum_{j=1}^p \quad \underbrace{\omega}_{\mathclap{\Lambda \circ \Lambda^{-1} \circ \omega}}\mleft( \mleft(\widehat{\Lambda} \circ \widehat{\Lambda}^{-1}\mright) (X_1), \dots, {}^E\widetilde{\nabla}_{\nu_i} X_j, \dots, \mleft(\widehat{\Lambda} \circ \widehat{\Lambda}^{-1}\mright) (X_p), \nu_0, \dots, \widehat{\nu}_i, \dots, \nu_q \mright) \biggr) \nonumber \\
&\quad + \sum_{0 \leq i < j \leq q} (-1)^{i+j} \underbrace{\omega}_{\mathclap{= \Lambda \circ \Lambda^{-1} \circ \omega}}\mleft( \mleft(\widehat{\Lambda} \circ \widehat{\Lambda}^{-1}\mright) (X_1), \dots, \mleft(\widehat{\Lambda} \circ \widehat{\Lambda}^{-1}\mright) (X_p), [\nu_i, \nu_j]_E, \nu_0, \dots, \widehat{\nu}_i, \dots, \widehat{\nu}_j, \dots, \nu_q \mright) \nonumber \\
&= 
\Bigg( \mleft( \Lambda \circ \mathrm{d}^{\nabla^{\mathrm{bas}}} \circ \Lambda^{-1} \mright)
\mleft( \omega \circ \mleft( \vphantom{\widehat{\Lambda}, \dots, \widehat{\Lambda}} \smash{\underbrace{\widehat{\Lambda}, \dots, \widehat{\Lambda}}_{\mathclap{p \text{ times}}}}, \vphantom{\mathds{1}_{E}, \dots, \mathds{1}_E} \smash{\underbrace{\mathds{1}_{E}, \dots, \mathds{1}_E}_{q \text{ times}}} \mright) \mright)
 \Bigg)
\mleft( \widehat{\Lambda}^{-1}(X_1), \dots, \widehat{\Lambda}^{-1}(X_p), \nu_0, \dots, \nu_q \mright)
\eas
for all $\omega \in \Omega^{p,q}(N,E;E)$ ($p, q \in \mathbb{N}_0$), $X_1, \dotsc, X_p \in \mathfrak{X}(N)$ and $\nu_0, \dotsc, \nu_q \in \Gamma(E)$. The second equation is of course just that formula applied to 
\bas
\Lambda \circ \omega \circ \Bigl( \underbrace{\widehat{\Lambda}^{-1}, \dotsc, \widehat{\Lambda}^{-1}}_{p \text{ times}}, \underbrace{\mathds{1}_{E}, \dotsc, \mathds{1}_E}_{q \text{ times}} \Bigr).
\eas
\end{proof}

Before we can study and discuss this field redefinition let us list several useful properties.

\begin{propositions}{Properties of $\Lambda$ and $\widehat{\Lambda}$}{PropsOfBigLambdas}
Let $N$ be a smooth manifold, $E \to N$ a Lie algebroid, $\nabla$ a connection on $E$, and $\kappa$ and $g$ fibre metrics on $E$ and $\mathrm{T}N$, respectively. Also let $\lambda \in \Omega^1(N; E)$ such that $\Lambda = \mathds{1}_E - \lambda \circ \rho$ is an element of $\sAut(E)$. Then we have
\ba
\Lambda^{-1}
&=
\sum_{k=0}^l \mleft( \lambda \circ \rho \mright)^k
	+ \Lambda^{-1} \circ \mleft( \lambda \circ \rho \mright)^{l+1},
&
\widehat{\Lambda}^{-1}
&=
\sum_{k=0}^l \mleft( \rho \circ \lambda \mright)^k
	+ \widehat{\Lambda}^{-1} \circ \mleft( \rho \circ \lambda \mright)^{l+1},
\\
\label{basicconnectionTrafoRefield}
\mleft(\widetilde{\nabla}^{\lambda}\mright)^{\mathrm{bas}}
&=
\Lambda \circ \nabla^{\mathrm{bas}} \circ \Lambda^{-1},
&
\mleft(\widetilde{\nabla}^{\lambda}\mright)^{\mathrm{bas}}
&=
\widehat{\Lambda} \circ \nabla^{\mathrm{bas}} \circ \widehat{\Lambda}^{-1},
\\
\rho \circ \Lambda
&=
\widehat{\Lambda} \circ \rho,
&
\Lambda \circ \lambda
&=
\lambda \circ \widehat{\Lambda},
\label{EqCommutationWithLambda}\\
\rho \circ \Lambda^{-1}
&=
\widehat{\Lambda}^{-1} \circ \rho,
&
\Lambda^{-1} \circ \lambda
&=
\lambda \circ \widehat{\Lambda}^{-1}
\label{EqCommutationWithLambdaInverse}
\ea
for all $l \in \mathbb{N}_0$, where we mean the basic connection on $E$ on the left and the one on $\mathrm{T}N$ on the right in the second line. Moreover, we have several identities for the redefinition of the connection
\ba\label{AndereFormelFuerNablaTrafoBesserFuerDasRechnen}
\widetilde{\nabla}^\lambda
&=
\nabla^\prime
	- \mleft( \mathrm{d}^{\nabla^\prime} \lambda \mright) \circ \mleft( \mathds{1}_{\mathrm{T}N}, \rho \mright)
	+ \Lambda \circ t_{\nabla_\rho} \circ \mleft( \Lambda^{-1} \circ \lambda, \mathds{1}_E \mright),
\ea
where $\nabla^\prime \coloneqq \Lambda\circ\nabla\circ\Lambda^{-1}$, and
\ba\label{dievielBessereFormuelFuersRechnenFragezeichen}
\widetilde{\nabla}^\lambda_Y \mu
&=
\Lambda \mleft( \nabla_{\widehat{\Lambda}^{-1}(Y)} \mu
- \mleft[ \mleft( \Lambda^{-1} \circ \lambda \mright)(Y), \mu \mright]_E \mright)
+ \lambda \big([Y, \rho(\mu)] \big)
\ea
for all $\mu\in \Gamma(E)$ and $Y \in \mathfrak{X}(N)$, finally also
\ba\label{KuerzesteFormelForRedefOfNabla}
\widetilde{\nabla}^\lambda_{\widehat{\Lambda}}
&=
\nabla_{\widehat{\Lambda}}
	+ \mathrm{d}^{\nabla^{\mathrm{bas}}}\lambda.
\ea
\end{propositions}

\begin{remark}
\leavevmode\newline
We especially need the formulas of the inverse for $l=0$, \textit{i.e.}
\bas
\Lambda^{-1}
&=
\mathds{1}_E
	+ \Lambda^{-1} \circ \lambda \circ \rho,
\\
\widehat{\Lambda}^{-1}
&=
\mathds{1}_{\mathrm{T}N}
	+ \widehat{\Lambda}^{-1} \circ \lambda \circ \rho.
\eas
\end{remark}

\begin{proof}
\leavevmode\newline
\indent $\bullet$ The Eq.~\eqref{EqCommutationWithLambda} simply follow by definition, and inverting these with respect to $\Lambda$ and $\widehat{\Lambda}$ gives Eq.~\eqref{EqCommutationWithLambdaInverse}.
Using these, we also have
\bas
\Lambda \circ 
\mleft( 
\sum_{k=0}^l \mleft( \lambda \circ \rho \mright)^k
	+ \Lambda^{-1} \circ \mleft( \lambda \circ \rho \mright)^{l+1} 
\mright)
&=
\sum_{k=0}^l \underbrace{\mleft( \mathds{1}_E - \lambda \circ \rho \mright) \circ \mleft( \lambda \circ \rho \mright)^k}
_{= \mleft( \lambda \circ \rho \mright)^k - \mleft( \lambda \circ \rho \mright)^{k+1}}
	+ ~\mleft( \lambda \circ \rho \mright)^{l+1} 
\\
&\stackrel{\mathclap{\text{telescoping sum}}}{=}\qquad~
\mleft( \lambda \circ \rho \mright)^0
	- \mleft( \lambda \circ \rho \mright)^{l+1}
	+ \mleft( \lambda \circ \rho \mright)^{l+1}
\\
&=
\mathds{1}_E,
\eas
which proves the claim. In the same manner one shows the formula for $\widehat{\Lambda}^{-1}$.

$\bullet$ We have
\bas
\mleft(\mleft( \Lambda \circ \mathrm{d}^{\nabla^{\mathrm{bas}}} \circ \Lambda^{-1} \mright) \lambda \mright)(Y, \mu)
&=
\Lambda\biggl(
\nabla^{\mathrm{bas}}_\mu \mleft( \mleft(\Lambda^{-1} \circ \lambda\mright) (Y) \mright)
	- \mleft(\Lambda^{-1} \circ \lambda\mright)\mleft(\nabla^{\mathrm{bas}}_\mu Y\mright)
\biggr)
\\
&=
\Lambda \biggl(
	- \mleft[ \mleft( \Lambda^{-1} \circ \lambda \mright) (Y), \mu \mright]_E
	+ \nabla_{\widehat{\Lambda}^{-1} \circ \rho \circ \lambda(Y)} \mu
\biggr)
	+ \lambda\mleft( \mleft[ Y, \rho(\mu) \mright] \mright) 
\\
&\hspace{1cm}
\underbrace{- \lambda \circ \rho\mleft( \nabla_Y \mu \mright)
	+ \nabla_Y \mu}_{\Lambda \mleft( \nabla_Y \mu \mright)} - \nabla_Y \mu 
\\
&=
\Lambda \biggl( 
	\nabla_{\widehat{\Lambda}^{-1}(Y)} \mu
	- \mleft[ \mleft(\Lambda^{-1} \circ \lambda\mright)(Y), \mu \mright]_E
\biggr)
	+ \lambda\mleft( \mleft[ Y, \rho(\mu) \mright] \mright)
	- \nabla_Y \mu,
\eas
which proves Eq.~\eqref{dievielBessereFormuelFuersRechnenFragezeichen} by using Def.~\eqref{FieldTrafoOfNabla}.
Let $\nabla^\prime \coloneqq \Lambda \circ \nabla \circ \Lambda^{-1}$, then by Prop.~\ref{prop:PropsOfBigLambdas}
\bas
&\nabla^{\prime}_Y\mu
- \mleft( \mathrm{d}^{\nabla^{\prime}}\lambda \mright)(Y, \rho(\mu))
+ \Lambda \Big( t_{\nabla_\rho}\mleft( \Lambda^{-1} (\lambda(Y)), \mu \mright) \Big) 
\\
&=
\underbrace{\nabla^{\prime}_Y\mu
- \nabla^{\prime}_Y \big( (\lambda \circ \rho)(\mu) \big)}_{= \nabla^\prime_Y \mleft( \Lambda(\mu) \mright)}
+ \nabla^{\prime}_{\rho(\mu)} \big( \lambda(Y) \big)
+ \lambda \big([Y, \rho(\mu)] \big) 
\\
&\hspace{1cm}
	+ \Lambda \bigg( - \mleft[ \mleft( \Lambda^{-1} \circ \lambda \mright)(Y), \mu \mright]_E
	+ \nabla_{\mleft( \rho \circ \Lambda^{-1} \circ \lambda \mright)(Y)} \mu
	- \nabla_{\rho(\mu)} \Big( \mleft( \Lambda^{-1} \circ \lambda \mright) (Y) \Big) \bigg) 
\\
&=
\Lambda \mleft( \nabla_{\widehat{\Lambda}^{-1}(Y)} \mu
- \mleft[ \mleft( \Lambda^{-1} \circ \lambda \mright)(Y), \mu \mright]_E \mright)
+ \lambda \big([Y, \rho(\mu)] \big),
\eas
comparing it with the previous formula, we arrive at
\bas
\widetilde{\nabla}^\lambda
&=
\nabla^\prime
	- \mleft( \mathrm{d}^{\nabla^\prime} \lambda \mright) \circ \mleft( \mathds{1}_{\mathrm{T}N}, \rho \mright)
	+ \Lambda \circ t_{\nabla_\rho} \circ \mleft( \Lambda^{-1} \circ \lambda, \mathds{1}_E \mright)
%\\
%&=
%\mleft[
%\mathfrak{X}(N) \times \Gamma(E) \ni (Y, \mu) \mapsto
%\Lambda \mleft( \nabla_{\widehat{\Lambda}^{-1}(Y)} \mu
%- \mleft[ \mleft( \Lambda^{-1} \circ \lambda \mright)(Y), \mu \mright]_E \mright)
%+ \lambda \big([Y, \rho(\mu)] \big)
%\mright]
.
\eas
For $I \coloneqq \mleft( \Lambda \circ \mathrm{d}^{\nabla^{\mathrm{bas}}} \circ \Lambda^{-1} \mright) \lambda \in \Omega^1(N; \mathrm{End}(E)) \cong \Omega^{1,1}(N,E;E)$ we also have
\bas
I(Y, \nu)
&=
\mleft( \Lambda \circ \nabla^{\mathrm{bas}}_\nu \circ \Lambda^{-1} \circ \lambda - \lambda \circ \nabla^{\mathrm{bas}}_\nu \mright)(Y)
\stackrel{\text{Eq.~\eqref{FieldTrafoOfNabla}}}{=}
\widetilde{\nabla}^\lambda_\nu Y
	- \nabla_\nu Y
\eas
for all $\nu \in \Gamma(E)$ and $Y \in \mathfrak{X}(N)$; especially with $\rho \circ \nabla^{\mathrm{bas}} = \nabla^{\mathrm{bas}} \circ \rho$ we get
\bas
I\mleft(\widehat{\Lambda}(Y), \nu\mright)
&=
\mleft( 
	\Lambda \circ \nabla^{\mathrm{bas}}_\nu \circ \lambda 
	- \lambda \circ \nabla^{\mathrm{bas}}_\nu \circ \widehat{\Lambda} 
\mright)(Y)
\\
&=
\mleft( 
	\nabla^{\mathrm{bas}}_\nu \circ \lambda 
	- \lambda \circ \nabla^{\mathrm{bas}}_\nu
	- \lambda \circ \rho \circ \nabla^{\mathrm{bas}}_\nu \circ \lambda 
	+ \lambda \circ \nabla^{\mathrm{bas}}_\nu \circ \rho \circ\lambda
\mright)(Y)
\\
&=
\mleft( 
	\nabla^{\mathrm{bas}}_\nu \circ \lambda 
	- \lambda \circ \nabla^{\mathrm{bas}}_\nu
\mright)(Y)
\\
&=
\mleft(\mathrm{d}^{\nabla^{\mathrm{bas}}} \lambda \mright)(Y, \nu),
\eas
which proves the last equation. Alternatively, use Cor.~\ref{cor:ConjugationOfDifferentialsAreShitty}.

$\bullet$ Finally, using the things just shown,
\bas
\mleft(\widetilde{\nabla}^{\lambda}\mright)^{\mathrm{bas}}_\mu \nu
&=
\mleft[ \mu, \nu \mright]_E
	+ \widetilde{\nabla}^{\lambda}_{\rho(\nu)} \mu
\\
&=
\mleft[ \mu, \nu \mright]_E
	+ \Lambda \mleft( \nabla_{\mleft(\widehat{\Lambda}^{-1}\circ\rho\mright)(\nu)} \mu 
	- \mleft[ \mleft( \Lambda^{-1} \circ \lambda \circ \rho \mright)(\nu), \mu \mright]_E \mright)
	+ \lambda \big([\rho(\nu), \rho(\mu)] \big)
\\
&=
\underbrace{\mleft[ \mu, \nu \mright]_E
	+ \mleft[ \mu, \mleft( \Lambda^{-1} \circ \lambda \circ \rho \mright)(\nu) \mright]_E}
	_{= \mleft[ \mu, \Lambda^{-1}(\nu) \mright]_E}
	+ \Lambda \mleft( \nabla_{\mleft(\rho \circ \Lambda^{-1}\mright)(\nu)} \mu \mright)
\\
&\hspace{1cm}
	+ \underbrace{(\lambda\circ\rho)\mleft( \mleft[ \mleft( \Lambda^{-1} \circ \lambda \circ \rho \mright)(\nu), \mu \mright]_E \mright)
	+ (\lambda\circ\rho) \bigl([\nu, \mu]_E \bigr)}
	_{= (\lambda \circ \rho)\mleft( \mleft[ \Lambda^{-1}(\nu), \mu \mright]_E \mright)}
\\
&=
\Lambda\mleft(\mleft[ \mu, \Lambda^{-1}(\nu) \mright]_E\mright)
	+ \Lambda \mleft( \nabla_{\mleft(\rho \circ \Lambda^{-1}\mright)(\nu)} \mu \mright)
\\
&=
\Lambda\biggl(
	\nabla^{\mathrm{bas}}_\mu \mleft( \Lambda^{-1} (\nu) \mright)
\biggr)
\eas
for all $\mu,\nu \in \Gamma(E)$. Similarly,
\bas
\mleft(\widetilde{\nabla}^{\lambda}\mright)^{\mathrm{bas}}_\mu Y
&=
[\rho(\mu), Y]
	+ \rho \mleft( \widetilde{\nabla}^\lambda_Y \mu \mright)
\\
&=
[\rho(\mu), Y]
	+ \rho\biggl( 
		\Lambda \mleft( \nabla_{\widehat{\Lambda}^{-1}(Y)} \mu
		- \mleft[ \mleft( \Lambda^{-1} \circ \lambda \mright)(Y), \mu \mright]_E \mright)
		+ \lambda \big([Y, \rho(\mu)] \big)
	\biggr)
\\
&=
\underbrace{[\rho(\mu), Y]
	+ \mleft[ \rho(\mu), \mleft( \widehat{\Lambda}^{-1} \circ\rho \circ \lambda \mright)(Y) \mright]}
	_{= \mleft[ \rho(\mu), \widehat{\Lambda}^{-1}(Y) \mright]}
	+ (\rho \circ \Lambda )\mleft( \nabla_{\widehat{\Lambda}^{-1}(Y)} \mu \mright)
\\
&\hspace{1cm}
	+ \underbrace{(\rho \circ \lambda )\mleft( \mleft[ \mleft( \widehat{\Lambda}^{-1} \circ \rho \circ \lambda \mright)(Y) + Y, \rho(\mu) \mright] \mright)}
	_{- (\rho\circ\lambda) \mleft(\mleft[ \rho(\mu), \widehat{\Lambda}^{-1}(Y) \mright]\mright)}
\\
&=
\widehat{\Lambda} \biggl(
	\nabla^{\mathrm{bas}}_\mu \mleft( \widehat{\Lambda}^{-1} (Y) \mright)
\biggr)
\eas
for all $\mu \in \Gamma(E)$ and $Y\in \mathfrak{X}(N)$.
\end{proof}

We will use these small results all the time, and we will not necessarily mention each equation each time when we use it. Using the formulas of the inverse, we can show the following.

\begin{lemmata}{Invertible field redefinition}{FieldRedefinitionIsInvertible}
Let $M, N$ be smooth manifolds, $E \to N$ a Lie algebroid, $\nabla$ a connection on $E$, and $\kappa$ and $g$ fibre metrics on $E$ and $\mathrm{T}N$, respectively. Also let $\lambda \in \Omega^1(N; E)$ such that $\Lambda = \mathds{1}_E - \lambda \circ \rho$ is an element of $\sAut(E)$. Then
\ba
\widehat{\varpi_2}^{-\lambda}
&=
\varpi_2,
\\
\widehat{\nabla}^{-\lambda}
&=
\nabla,
\\
\widehat{\kappa}^{-\lambda}
&=
\kappa,
\\
\widehat{g}^{-\lambda}
&=
g,
\ea
where we denote 
\bas
\widehat{\varpi_2}^{-\lambda}
&\coloneqq
\widetilde{ \widetilde{\varpi_2}^\lambda }^{- \Lambda^{-1} \circ \lambda}
\eas
and so on.
\end{lemmata}

\begin{remark}
\leavevmode\newline
All following formulas implied by the field redefinition, like a field redefinition of the basic connection, are defined by taking their typical definition and replacing the terms with the field redefinitions given in Def.~\ref{def:FieldRedefinition}. That will imply similar inversion formulas for those terms.
\end{remark}

\begin{proof}
\leavevmode\newline
First observe that, using Prop.~\ref{prop:PropsOfBigLambdas},
\bas
\mathfrak{\Lambda}
&\coloneqq
\mathds{1}_E
	- \mleft( - \Lambda^{-1} \circ \lambda \mright) \circ \rho
=
\mathds{1}_E
	+ \Lambda^{-1} \circ \lambda \circ \rho
=
\Lambda^{-1},
\\
\widehat{\mathfrak{\Lambda}}
&\coloneqq
\mathds{1}_{\mathrm{T}N}
	- \rho \circ \mleft( - \Lambda^{-1} \circ \lambda \mright) 
=
\mathds{1}_{\mathrm{T}N}
	+ \widehat{\Lambda}^{-1} \circ \lambda \circ \rho
=
\widehat{\Lambda}^{-1}.
\eas
Those are invertible, thus, we can apply the field redefinition using $-\Lambda^{-1} \circ \lambda$. Using these formulas, we get trivially,
\bas
\widehat{\kappa}^{-\lambda}
&=
\mleft(\kappa \circ \mleft( \Lambda^{-1}, \Lambda^{-1} \mright)\mright) \circ 
\mleft( \mathfrak{\Lambda}^{-1}, \mathfrak{\Lambda}^{-1} \mright)
=
\kappa,
\eas
similarly for $g$. Moreover,
\bas
\widehat{\varpi_2}^{-\lambda}
&=
\mleft( {}^* \mathfrak{\Lambda} \mright) (\widetilde{\varpi_2}^\lambda)
	- {}^! \mleft( \Lambda^{-1} \circ \lambda \mright)
\\
&=
\mleft( {}^* \mathfrak{\Lambda} \mright) \mleft(\mleft( {}^* \Lambda \mright) (\varpi_2)+ {}^! \lambda\mright)
	- {}^! \mleft( \Lambda^{-1} \circ \lambda \mright)
\\
&=
\varpi_2
	+ {}^! \mleft( \Lambda^{-1} \circ \lambda \mright)
	- {}^! \mleft( \Lambda^{-1} \circ \lambda \mright)
\\
&=
\varpi_2,
\eas
and
\bas
\widehat{\nabla}^{-\lambda}
&=
\widetilde{\nabla}^\lambda
	- \mleft( \mathfrak{\Lambda} \circ \mathrm{d}^{\mleft(\widetilde{\nabla}^\lambda\mright)^{\mathrm{bas}}} \circ \mathfrak{\Lambda}^{-1} \mright) \mleft( \Lambda^{-1} \circ \lambda\mright)
\\
&\stackrel{\mathclap{ \text{Cor.~\ref{cor:ConjugationOfDifferentialsAreShitty}} }}{=}\quad~
\nabla
	+ \mleft( \Lambda \circ \mathrm{d}^{\nabla^{\mathrm{bas}}} \circ \Lambda^{-1} \mright) \lambda
	- \mleft(\mathrm{d}^{\nabla^{\mathrm{bas}}} \lambda \mright) \circ \mleft( \widehat{\Lambda}^{-1}, \mathds{1}_E \mright)
\\
&\stackrel{\mathclap{ \text{Eq.~\eqref{KuerzesteFormelForRedefOfNabla}} }}{=}\quad~~
\nabla
	+ \mleft( \Lambda \circ \mathrm{d}^{\nabla^{\mathrm{bas}}} \circ \Lambda^{-1} \mright) \lambda
	- \mleft( \Lambda \circ \mathrm{d}^{\nabla^{\mathrm{bas}}} \circ \Lambda^{-1} \mright) \lambda
\\
&=
0,
\eas
viewing $\mathrm{d}^{\nabla^{\mathrm{bas}}} \lambda$ as an element of $\Omega^{1,1}(N,E;E)$.
\end{proof}

\section{Redefined gauge theory}\label{NastyCalculationsForTheseFieldRedefsBaeaeaeae}

We now want to calculate what the field redefinition changes, especially with respect to the field strength.

\begin{theorems}{Field redefinition of the field strength}{FieldRedefofstandardFieldStrengthF}
Let $M, N$ be smooth manifolds, $E \to N$ a Lie algebroid, and $\nabla$ a connection on $E$. Also let $\lambda \in \Omega^1(N; E)$ such that $\Lambda = \mathds{1}_E - \lambda \circ \rho$ is an element of $\sAut(E)$. Then we have
\ba
\widetilde{\mathfrak{D}}^\lambda
&=
\mleft( {}^* \widehat{\Lambda} \mright)\mleft(\mathfrak{D}\mright),
\\
\widetilde{F}^\lambda
&=
\mleft( {}^* \Lambda \mright) \mleft(
	F
	- \frac{1}{2} \mleft({}^* \xi \mright) \mleft( \mathfrak{D} \stackrel{\wedge}{,} \mathfrak{D} \mright)
\mright),\label{FieldRedefOfClassicF}
\ea
where
\ba
\widetilde{\mathfrak{D}}^\lambda
&\coloneqq
\mathrm{D}
	- ({}^*\rho)\mleft(\widetilde{\varpi_2}^\lambda\mright),
\\
\widetilde{F}^\lambda
&\coloneqq
\mathrm{d}^{{}^*\widetilde{\nabla}^\lambda} \widetilde{\varpi_2}^\lambda
	- \frac{1}{2} \mleft( {}^* t_{\widetilde{\nabla}^\lambda_\rho} \mright)\mleft( \widetilde{\varpi_2}^\lambda \stackrel{\wedge}{,} \widetilde{\varpi_2}^\lambda \mright),
\\
\xi
&\coloneqq
\Lambda^{-1} \circ \widehat{\zeta}^\lambda \circ \mleft( \widehat{\Lambda}, \widehat{\Lambda} \mright)
\ea
and $\gls{1fZetaTilHat}$ is an element of $\Omega^2(N; E)$ defined by
\ba\label{FormulaForZetaTildeWithZetaEqualzero}
\mleft(-\widehat{\zeta}^\lambda \circ \mleft( \widehat{\Lambda}, \widehat{\Lambda} \mright)\mright) (X,Y)
&\coloneqq
\mleft(\mathrm{d}^{\widetilde{\nabla}^\lambda} \lambda
	- t_{\widetilde{\nabla}^\lambda_\rho} \circ (\lambda, \lambda)\mright)(X,Y)
\nonumber\\
&=
\mleft( \mathrm{d}^\nabla \lambda \mright)(X,Y)
%}_{\mathclap{= \nabla_X \bigl( \lambda(Y) \bigr) - \nabla_Y \bigl( \lambda(X) \bigr) - \lambda\bigl( [X,Y] \bigr)}}
	+ \lambda\Bigl(
		\nabla^{\mathrm{bas}}_{\lambda(X)} Y
		- \nabla^{\mathrm{bas}}_{\lambda(Y)} X
		%}_{= [(\rho\circ\lambda)(Y), X] + \rho\mleft( \nabla_X ( \lambda(Y) ) \mright)}
	\Bigr)
	- \mleft[ \lambda(X), \lambda(Y) \mright]_E
\ea
for all $X, Y \in \mathfrak{X}(N)$.
\end{theorems}

\begin{remark}\label{OtherNotationForZetaTransform}
\leavevmode\newline
When we define the \textbf{formal torsion}\footnote{It is formal because $\nabla^{\mathrm{bas}}_\lambda$ is not a connection due to the fact that $\rho \circ \lambda \neq \mathds{1}_{\mathrm{T}N}$, otherwise $\widehat{\Lambda} = 0$ and, so, $\Lambda$ would not be invertible by Sylvester's determinant theorem. Therefore the Leibniz rule is not as usual. That is, $\nabla^{\mathrm{bas}}_\lambda: \mathrm{T}N \to \mathfrak{D}(E)$ is in general not anchor-preserving.} $t_{\nabla^{\mathrm{bas}}_\lambda}$ of $\nabla^{\mathrm{bas}}_\lambda$, $\mathfrak{X}(N) \times \Gamma(E) \ni (Y, \nu) \mapsto \nabla^{\mathrm{bas}}_{\lambda(Y)} \nu$, as an element of $\Omega^2(N; \mathrm{T}N)$ by
\ba
t_{\nabla^{\mathrm{bas}}_\lambda}(X,Y)
&\coloneqq
\nabla^{\mathrm{bas}}_{\lambda(X)} Y
	- \nabla^{\mathrm{bas}}_{\lambda(Y)} X
	- [X,Y]
\ea
for all $X, Y \in \mathfrak{X}(N)$, then recall Def.~\ref{def:GeneralDefOfCurvMorphisms} for
\bas
R_\lambda(X,Y)
=
\bigl[ \lambda(X), \lambda(Y) \bigr]
	- \lambda \bigl( [X,Y] \bigr),
\eas
hence, we can write
\bas
\mleft(
	\lambda\mleft(t_{\nabla^{\mathrm{bas}}_\lambda}\mright)
	- R_\lambda
\mright)(X,Y)
&=
\lambda\Bigl(
		\nabla^{\mathrm{bas}}_{\lambda(X)} Y
		- \nabla^{\mathrm{bas}}_{\lambda(Y)} X
	\Bigr)
	- \mleft[ \lambda(X), \lambda(Y) \mright]_E,
\eas
in total arriving to
\ba\label{TollsteFormelFuerZetaTrafoFragezeichen}
-\widehat{\zeta}^\lambda \circ \mleft( \widehat{\Lambda}, \widehat{\Lambda} \mright)
&=
\mathrm{d}^\nabla \lambda
	+ \lambda\mleft(t_{\nabla^{\mathrm{bas}}_\lambda}\mright)
	- R_\lambda.
\ea
Observe the (very rough) similarity with the Maurer-Cartan equation; especially for Lie algebra bundles, that is, zero anchor, this will look like a covariantized Maurer-Cartan equation with inhomogeneity. We will see this later.
\end{remark}

\begin{proof}[Proof of Thm.~\ref{thm:FieldRedefofstandardFieldStrengthF}]
\leavevmode\newline
In the following let $(\Phi,A) \in \mathfrak{M}_E(M;N)$.

$\bullet$ The field redefinition of the minimal coupling directly follows by Def.~\eqref{EqFieldRedefFuerA}, so,
\bas
\mathfrak{D}^{\widetilde{A}^\lambda} \Phi
&=
\underbrace{\mathrm{D}\Phi}_{\mathclap{= (\Phi^* \mathds{1}_{\mathrm{T}N}) (\mathrm{D}\Phi)}}
	- (\Phi^*\rho)\bigl( 
		\mleft( \Phi^* \Lambda \mright) (A)
		+ (\Phi^* \lambda)(\mathrm{D}\Phi) 
	\bigr)
\\
&=
\mleft( \Phi^* \widehat{\Lambda} \mright) (\mathrm{D}\Phi)
	- \mleft(\Phi^* \mleft(\widehat{\Lambda} \circ \rho\mright)\mright) (A)
\\
&=
\mleft( \Phi^* \widehat{\Lambda} \mright)\mleft(\mathfrak{D}^{A} \Phi\mright).
\eas

$\bullet$ With respect to a local frame $\mleft( e_a \mright)_a$ of $E$ and viewing terms like $\widetilde{\nabla}^\lambda - \nabla$ as an element of $\Omega^1(N; \mathrm{End}(E))$,
\bas
\mathrm{d}^{\Phi^*\widetilde{\nabla}^\lambda} \bigl( \mleft( \Phi^* \Lambda \mright) (A) \bigr)
&=
\mathrm{d}^{\Phi^*\mleft( \nabla
	+ \widetilde{\nabla}^\lambda - \nabla
	\mright)}
	\bigl(
		(\Phi^*\Lambda)(A)
	\bigr)
\\
%&\stackrel{\mathclap{\text{Eq.~\eqref{eqDifferentialSplit}}}}{=}~~~
%\mathrm{d}^{\Phi^*\nabla}
%\mleft(
		%A^a \otimes \Phi^*\bigl( \Lambda(e_a) \bigr)
%\mright)
	%+ \Phi^!\Biggl( \biggl(\mleft( \Lambda \circ \mathrm{d}^{\nabla^{\mathrm{bas}}} \circ \Lambda^{-1} \mright) \lambda \biggr) \bigl( \Lambda(e_a) \bigr) \Biggr) \wedge  A^a 
%\\
%&=
%\mathrm{d}^{\Phi^*\nabla}
%\mleft(
		%A^a \otimes \Phi^*\bigl( \Lambda(e_a) \bigr)
%\mright)
	%+ \Phi! 
	%\mleft( \mleft( \widetilde{\nabla}^\lambda  - \nabla \mright) (\Lambda(e_a)) \mright)
	%\wedge A^a
%\\
&\stackrel{\mathclap{\text{Eq.~\eqref{eqDifferentialSplit}}}}{=}~~~
\mathrm{d}^{\Phi^*\nabla} \mleft( A^a \otimes \Phi^*\bigl( \Lambda(e_a) \bigr) \mright)
	+ \Phi^!\mleft( \widetilde{\nabla}^\lambda - \nabla \mright) \wedge \mleft( A^a \otimes \Phi^*\bigl( \Lambda(e_a) \bigr)\mright)
\\
&=
\mathrm{d}A^a \otimes \Phi^*\bigl( \Lambda(e_a) \bigr)
	- A^a \wedge \Phi^!\underbrace{\Bigl( \nabla \bigl(\Lambda(e_a)\bigr) \Bigr)}
	_{\mathclap{= (\nabla \Lambda)(e_a) + \Lambda(\nabla e_a)}}
\\
&\hspace{1cm}
	- A^a \wedge \Phi! \mleft(  \widetilde{\nabla}^\lambda (\Lambda(e_a))  - (\nabla \Lambda)(e_a) - \Lambda(\nabla e_a)) \mright)
\\
&=
\mathrm{d}A^a \otimes \Phi^*\bigl( \Lambda(e_a) \bigr)
	- A^a \wedge \underbrace{\Phi^! \mleft( \Lambda(\nabla e_a) \mright)}
		_{\mathclap{= \mleft( \Phi^* \Lambda \mright) \mleft( \Phi^! (\nabla e_a) \mright)}}
	+ \Phi! \mleft(  \widetilde{\nabla}^\lambda (\Lambda(e_a))  - \Lambda(\nabla e_a)) \mright) \wedge A^a
\\
&=
\mleft( \Phi^* \Lambda \mright)\mleft( \mathrm{d}^{\Phi^*\nabla} A \mright)
	+ \mleft( \Phi^! \mleft( \widetilde{\nabla}^\lambda \circ \Lambda - \Lambda \circ \nabla \mright) \mright) (A)
\\
&\stackrel{\mathclap{\text{Eq.~\eqref{EqPullBackFormelFuerVerschiedeneDefinitionen}}}}{=}~~~
\mleft( \Phi^* \Lambda \mright)\mleft( \mathrm{d}^{\Phi^*\nabla} A \mright)
	+ \mleft( \Phi^* \mleft( \widetilde{\nabla}^\lambda \circ \Lambda - \Lambda \circ \nabla \mright) \mright) \mleft( \mathrm{D}\Phi\stackrel{\wedge}{,} A \mright),
\eas
and
\bas
\mathrm{d}^{\Phi^*\widetilde{\nabla}^\lambda} \mleft( \Phi^!\lambda \mright)
&\stackrel{\text{Eq.~\eqref{EqGeilePullBackCommuteFormel}}}{=}
\Phi^!\mleft( \mathrm{d}^{\widetilde{\nabla}^\lambda} \lambda \mright)
\stackrel{\text{Eq.~\eqref{EqPullBackFormelFuerVerschiedeneDefinitionen}}}{=}
\frac{1}{2} \mleft( \Phi^* \mleft( \mathrm{d}^{\widetilde{\nabla}^\lambda} \lambda \mright) \mright) \mleft( \mathrm{D}\Phi \stackrel{\wedge}{,} \mathrm{D} \Phi \mright),
\eas
also
%\bas
 %t_{\widetilde{\nabla}^\lambda_\rho} \mleft( \mu, \nu \mright)
	%- t_{\nabla_\rho}(\mu, \nu)
%~~~~&\stackrel{\mathclap{\text{Def.~\eqref{FieldTrafoOfNabla}}}}{=}~~~~
%\mleft(\mleft( \Lambda \circ \mathrm{d}^{\nabla^{\mathrm{bas}}} \circ \Lambda^{-1} \mright) \lambda\mright) \mleft( \rho(\mu), \nu \mright)
	%+ \mleft(\mleft( \Lambda \circ \mathrm{d}^{\nabla^{\mathrm{bas}}} \circ \Lambda^{-1} \mright) \lambda\mright) \mleft( \rho(\nu), \mu \mright)
%\eas
\bas
\frac{1}{2} \mleft( \Phi^* t_{\widetilde{\nabla}^\lambda_\rho} \mright) \mleft( \widetilde{A}^\lambda \stackrel{\wedge}{,} \widetilde{A}^\lambda \mright)
~~~~~&\stackrel{\mathclap{\text{Prop.~\ref{prop:GradedExtensionPlusAntiSymm}}}}{=}~~~~~
\frac{1}{2} \Biggl(
	\mleft( \Phi^* t_{\widetilde{\nabla}^\lambda_\rho} \mright) \bigl( (\Phi^*\Lambda) (A) \stackrel{\wedge}{,} (\Phi^* \Lambda) (A) \bigr)
	+  \mleft( \Phi^* t_{\widetilde{\nabla}^\lambda_\rho} \mright) \mleft( \Phi^!\lambda \stackrel{\wedge}{,} \Phi^!\lambda \mright) 
\Biggr)
\\
&\hspace{1cm}~~~~~
	+ \mleft( \Phi^* t_{\widetilde{\nabla}^\lambda_\rho} \mright) \mleft( \Phi^!\lambda \stackrel{\wedge}{,} (\Phi^* \Lambda) (A) \mright)
\\
&\stackrel{\mathclap{\text{Eq.~\eqref{EqPullBackFormelFuerVerschiedeneDefinitionen}}}}{=}~~~
\frac{1}{2} \Biggl(\Phi^* \mleft( t_{\widetilde{\nabla}^\lambda_\rho} \circ (\Lambda, \Lambda) \mright) \Biggr) \mleft( A \stackrel{\wedge}{,} A \mright)
	+ \Biggl( \Phi^* \mleft( t_{\widetilde{\nabla}^\lambda_\rho} \circ \mleft( \lambda, \Lambda \mright) \mright) \Biggr) (\mathrm{D}\Phi \stackrel{\wedge}{,} A)
\\
&\hspace{1cm}~~~
	+ \frac{1}{2} \Biggl( \Phi^*\mleft( t_{\widetilde{\nabla}^\lambda_\rho} \circ (\lambda, \lambda) \mright) \Biggr) \mleft( \mathrm{D}\Phi \stackrel{\wedge}{,} \mathrm{D}\Phi \mright).
\eas
So, in total we get, adding the missing term of the torsion in the definition of the field strength,
\bas
\widetilde{F}^\lambda(\Phi, A)
&=
\mleft( \Phi^* \Lambda \mright)\mleft( \mathrm{d}^{\Phi^*\nabla} A \mright)
	- \frac{1}{2} (\Phi^* \Lambda) \mleft( \mleft( \Phi^* t_{\nabla_\rho} \mright) \mleft( A \stackrel{\wedge}{,} A \mright) \mright)
\\
&\hspace{1cm}
	+ \frac{1}{2} (\Phi^* \Lambda) \mleft( \mleft( \Phi^* t_{\nabla_\rho} \mright) \mleft( A \stackrel{\wedge}{,} A \mright) \mright)
	+ \mleft( \Phi^* \mleft( \widetilde{\nabla}^\lambda \circ \Lambda - \Lambda \circ \nabla \mright) \mright) \mleft( \mathrm{D}\Phi\stackrel{\wedge}{,} A \mright)
\\
&\hspace{1cm}
	+ \frac{1}{2} \mleft( \Phi^* \mleft( \mathrm{d}^{\widetilde{\nabla}^\lambda} \lambda \mright) \mright) \mleft( \mathrm{D}\Phi \stackrel{\wedge}{,} \mathrm{D} \Phi \mright)
	- \frac{1}{2} \Biggl(\Phi^* \mleft( t_{\widetilde{\nabla}^\lambda_\rho} \circ (\Lambda, \Lambda) \mright) \Biggr) \mleft( A \stackrel{\wedge}{,} A \mright)
\\
&\hspace{1cm}
	- \Biggl( \Phi^* \mleft( t_{\widetilde{\nabla}^\lambda_\rho} \circ \mleft( \lambda, \Lambda \mright) \mright) \Biggr) (\mathrm{D}\Phi \stackrel{\wedge}{,} A)
	- \frac{1}{2} \Biggl( \Phi^*\mleft( t_{\widetilde{\nabla}^\lambda_\rho} \circ (\lambda, \lambda) \mright) \Biggr) \mleft( \mathrm{D}\Phi \stackrel{\wedge}{,} \mathrm{D}\Phi \mright)
\\
%%%%%%%%%%%%%%%%%%%%%%%%%%%%%%%%%%%%%%%%%%%%%%%%%%%%%%
&=
(\Phi^* \Lambda) (F)
\\
&\hspace{1cm}
+ \Biggl(\Phi^* \mleft(
	\widetilde{\nabla}^\lambda \circ \Lambda - \Lambda \circ \nabla
	- t_{\widetilde{\nabla}^\lambda_\rho} \circ \mleft( \lambda, \Lambda \mright)
\mright) \Biggr) \mleft( \mathrm{D}\Phi \stackrel{\wedge}{,} A \mright)
\\
&\hspace{1cm}
+ \frac{1}{2}\Biggl( \Phi^* \mleft(
	\Lambda \circ t_{\nabla_\rho}
	- t_{\widetilde{\nabla}^\lambda_\rho} \circ (\Lambda, \Lambda)
\mright) \Biggr) \mleft( A \stackrel{\wedge}{,} A \mright)
\\
&\hspace{1cm}
+ \frac{1}{2}\Biggl( \Phi^* \mleft(
	\mathrm{d}^{\widetilde{\nabla}^\lambda} \lambda
	- t_{\widetilde{\nabla}^\lambda_\rho} \circ (\lambda, \lambda)
\mright) \Biggr) \mleft( \mathrm{D}\Phi \stackrel{\wedge}{,} \mathrm{D} \Phi \mright).
\eas
Now we need to insert the definition of $\widetilde{\nabla}^\lambda$,
\bas
\mleft( \mathrm{d}^{\widetilde{\nabla}^\lambda} \lambda \mright)(X, Y)
&=
\mleft( \mathrm{d}^{
		\nabla
	+ \mleft( \Lambda \circ \mathrm{d}^{\nabla^{\mathrm{bas}}} \circ \Lambda^{-1} \mright) \lambda} \lambda \mright)(X, Y)
\\
&\stackrel{\mathclap{\text{Eq.~\eqref{eqDifferentialSplit}}}}{=}~~~
%\Biggl( 
	%\mathrm{d}^{\nabla^\prime} \lambda
	%+ \biggl( 
		%\Lambda \circ t_{\nabla_\rho} \circ \mleft( \Lambda^{-1} \circ \lambda, \mathds{1}_E \mright) 
		%- \mleft( \mathrm{d}^{\nabla^\prime} \lambda \mright) \circ \mleft( \mathds{1}_{\mathrm{T}N}, \rho \mright)
	%\biggr) \wedge \lambda
%\Biggr) (X, Y)
%\\
%%&=
%%\nabla^\prime_X \mleft( \lambda(Y) \mright)
	%%- \nabla^\prime_Y \mleft( \lambda(X) \mright)
	%%- \lambda \mleft( [X,Y] \mright)
%%\\
%%&\hspace{1cm}
	%%+ \biggl(
		%%\Lambda \circ t_{\nabla_\rho} \circ \mleft( \Lambda^{-1} \circ \lambda, \lambda \mright)
		%%+ \Lambda \circ t_{\nabla_\rho} \circ \mleft( \lambda, \Lambda^{-1} \circ \lambda \mright)
	%%\biggr) (X,Y)
%%\\
%%&\hspace{1cm}
	%%- \biggl(
		%%\mleft(\mathrm{d}^{\nabla^\prime} \lambda \mright) \circ \mleft( \mathds{1}_{\mathrm{T}N}, \rho \circ \lambda \mright)
		%%+ \mleft(\mathrm{d}^{\nabla^\prime} \lambda \mright) \circ \mleft( \rho \circ \lambda, \mathds{1}_{\mathrm{T}N} \mright)
	%%\biggr) (X, Y)
%%\\
%&=
%\mleft( \mathrm{d}^{\nabla^\prime} \lambda \mright)(X, Y)
%\\
%&\hspace{1cm}
	%+ \Lambda \biggl(
		%\nabla_{\mleft(\widehat{\Lambda}^{-1} \circ \rho \circ \lambda \mright) (X)} \mleft( \lambda(Y) \mright)
		%- \nabla_{(\rho\circ\lambda)(Y)} \mleft( \mleft(\Lambda^{-1} \circ \lambda \mright) (X) \mright)
%\\
%&\hspace{1cm}\hphantom{+ \Lambda \biggl(}
		%- \nabla_{\mleft(\widehat{\Lambda}^{-1} \circ \rho \circ \lambda \mright) (Y)} \mleft( \lambda(X) \mright)
		%+ \nabla_{(\rho\circ\lambda)(X)} \mleft( \mleft(\Lambda^{-1} \circ \lambda \mright) (Y) \mright)
%\\
%&\hspace{1cm}\hphantom{+ \Lambda \biggl(}
		%- \mleft[ \mleft(\Lambda^{-1} \circ \lambda \mright) (X), \lambda(Y) \mright]_E
		%+ \mleft[ \mleft(\Lambda^{-1} \circ \lambda \mright) (Y), \lambda(X) \mright]_E
	%\biggr)
%\\
%&\hspace{1cm}
	%- \nabla^\prime_X \bigl( (\lambda \circ \rho \circ \lambda) (Y) \bigr)
		%+ \nabla^\prime_{(\rho \circ \lambda)(Y)} \bigl( \lambda(X) \bigr)
		%+ \lambda\bigl( \mleft[ X, (\rho \circ \lambda) (Y) \mright] \bigr)
%\\
%&\hspace{1cm}
	%+ \nabla^\prime_Y \bigl( (\lambda \circ \rho \circ \lambda) (X) \bigr)
		%- \nabla^\prime_{(\rho \circ \lambda)(X)} \bigl( \lambda(Y) \bigr)
		%- \lambda\bigl( [Y, (\rho \circ \lambda) (X)] \bigr)
%\\
%&=
%\mleft( \mathrm{d}^{\nabla^\prime} \lambda \mright)(X, Y)
%\\
%&\hspace{1cm}
	%+ \Lambda \biggl(
		%\nabla_{\mleft(\widehat{\Lambda}^{-1} \circ \rho \circ \lambda \mright) (X)} \mleft( \lambda(Y) \mright)
		%- \nabla_{\mleft(\widehat{\Lambda}^{-1} \circ \rho \circ \lambda \mright) (Y)} \mleft( \lambda(X) \mright)
	%\biggr)
%\\
%&\hspace{1cm}
		%- \mleft[ \mleft(\Lambda^{-1} \circ \lambda \mright) (X), \lambda(Y) \mright]_E
		%+ \mleft[ \mleft(\Lambda^{-1} \circ \lambda \mright) (Y), \lambda(X) \mright]_E
%\\
%&\hspace{1cm}
		%+ \lambda\biggl( \mleft[ \mleft(\widehat{\Lambda}^{-1} \circ\rho \circ \lambda \mright) (X), (\rho\circ\lambda)(Y) \mright] \biggr)
		%- \lambda \biggl( \mleft[ \mleft(\widehat{\Lambda}^{-1} \circ \rho \circ \lambda \mright) (Y), (\rho\circ\lambda)(X) \mright] \biggr)
%\\
%&\hspace{1cm}
	%- \Lambda\mleft( \nabla_{\widehat{\Lambda}^{-1}(X) - \mleft(\widehat{\Lambda}^{-1} \circ \rho \circ \lambda)(X) \mright)} \bigl( (\Lambda^{-1} \circ \lambda \circ \rho \circ \lambda) (Y) \bigr) \mright)
		%+ \lambda\bigl( \mleft[ X, (\rho \circ \lambda) (Y) \mright] \bigr)
%\\
%&\hspace{1cm}
	%+ \Lambda \mleft( \nabla_{\widehat{\Lambda}^{-1}(Y) - \mleft(\widehat{\Lambda}^{-1} \circ \rho \circ \lambda)(Y) \mright)} \bigl( (\Lambda^{-1} \circ \lambda \circ \rho \circ \lambda) (X) \bigr) \mright)
		%- \lambda\bigl( [Y, (\rho \circ \lambda) (X)] \bigr)
%\\
%&=
%\mleft( \mathrm{d}^{\nabla^\prime} \lambda \mright)(X, Y)
%\\
%&\hspace{1cm}
		%- \mleft[ \mleft(\Lambda^{-1} \circ \lambda \mright) (X), \lambda(Y) \mright]_E
		%+ \mleft[ \mleft(\Lambda^{-1} \circ \lambda \mright) (Y), \lambda(X) \mright]_E
%\\
%&\hspace{1cm}
		%+ \lambda\biggl( \mleft[ \widehat{\Lambda}^{-1} (X), (\rho\circ\lambda)(Y) \mright]
		%- \mleft[ \widehat{\Lambda}^{-1} (Y), (\rho\circ\lambda)(X) \mright] \biggr)
%...
%\\
%&=
\mleft( \mathrm{d}^\nabla \lambda \mright)(X, Y)
	+ \Lambda\biggl(
\nabla^{\mathrm{bas}}_{\lambda(Y)} \mleft( \mleft(\Lambda^{-1} \circ \lambda\mright) (X) \mright)
	- \mleft(\Lambda^{-1} \circ \lambda\mright)\mleft(\nabla^{\mathrm{bas}}_{\lambda(Y)} X\mright)
\biggr)
\\
&\hspace{1cm}
	- \Lambda\biggl(
\nabla^{\mathrm{bas}}_{\lambda(X)} \mleft( \mleft(\Lambda^{-1} \circ \lambda\mright) (Y) \mright)
	- \mleft(\Lambda^{-1} \circ \lambda\mright)\mleft(\nabla^{\mathrm{bas}}_{\lambda(X)} Y\mright)
\biggr)
\\
&=
\mleft( \mathrm{d}^\nabla \lambda \mright)(X,Y)
	+ \lambda\mleft(
		\nabla^{\mathrm{bas}}_{\lambda(X)} Y
		- \nabla^{\mathrm{bas}}_{\lambda(Y)} X
	\mright)
\\
&\hspace{1cm}
	+ \Lambda\biggl(
		\nabla^{\mathrm{bas}}_{\lambda(Y)} \mleft( \mleft(\Lambda^{-1} \circ \lambda\mright) (X) \mright)
		- \nabla^{\mathrm{bas}}_{\lambda(X)} \mleft( \mleft(\Lambda^{-1} \circ \lambda\mright) (Y) \mright)
	\biggr)
\\
&=
\Lambda\mleft( \nabla_X \bigl( \lambda(Y) \bigr) \mright)
	- \Lambda\mleft( \nabla_Y \bigl( \lambda(X) \bigr) \mright)
	+ \lambda\mleft(
		\mleft[ \widehat{\Lambda}(Y), X \mright]
		+ \mleft[ (\rho\circ\lambda)(X), Y \mright]
	\mright)
\\
&\hspace{1cm}
	+ \Lambda\biggl(
		\nabla^{\mathrm{bas}}_{\lambda(Y)} \mleft( \mleft(\Lambda^{-1} \circ \lambda\mright) (X) \mright)
		- \nabla^{\mathrm{bas}}_{\lambda(X)} \mleft( \mleft(\Lambda^{-1} \circ \lambda\mright) (Y) \mright)
	\biggr)
\eas
for all $X, Y \in \mathfrak{X}(N)$, and, by using the results about the field redefinition of the basic connection,
\bas
- t_{\widetilde{\nabla}^\lambda_\rho} (\lambda(X), \lambda(Y))
&=
t_{\mleft(\widetilde{\nabla}^\lambda\mright)^{\mathrm{bas}}} (\lambda(X), \lambda(Y))
\\
&=
\Lambda\biggl( \nabla^{\mathrm{bas}}_{\lambda(X)} \mleft( \mleft(\Lambda^{-1} \circ \lambda\mright) (Y) \mright) \biggr)
	- \Lambda\biggl( \nabla^{\mathrm{bas}}_{\lambda(Y)} \mleft( \mleft(\Lambda^{-1} \circ \lambda\mright) (X) \mright) \biggr)
	-\mleft[ \lambda(X),\lambda(Y) \mright]_E.
\eas
Then
\bas
\mleft(-\widehat{\zeta}^\lambda \circ \mleft( \widehat{\Lambda}, \widehat{\Lambda} \mright)\mright) (X,Y)
&\coloneqq
\mleft(\mathrm{d}^{\widetilde{\nabla}^\lambda} \lambda
	- t_{\widetilde{\nabla}^\lambda_\rho} \circ (\lambda, \lambda)\mright)(X,Y)
\\
&=
\mleft( \mathrm{d}^\nabla \lambda \mright)(X,Y)
%}_{\mathclap{= \nabla_X \bigl( \lambda(Y) \bigr) - \nabla_Y \bigl( \lambda(X) \bigr) - \lambda\bigl( [X,Y] \bigr)}}
	+ \lambda\Bigl(
		\nabla^{\mathrm{bas}}_{\lambda(X)} Y
		- \nabla^{\mathrm{bas}}_{\lambda(Y)} X
		%}_{= [(\rho\circ\lambda)(Y), X] + \rho\mleft( \nabla_X ( \lambda(Y) ) \mright)}
	\Bigr)
	- \mleft[ \lambda(X), \lambda(Y) \mright]_E
%\\
%&=
%\Lambda\mleft( \nabla_X \bigl( \lambda(Y) \bigr) \mright)
	%- \Lambda\mleft( \nabla_Y \bigl( \lambda(X) \bigr) \mright)
%\\
%&\hspace{1cm}
	%+ \lambda\mleft(
		%\mleft[ \widehat{\Lambda}(Y), X \mright]
		%+ \mleft[ (\rho\circ\lambda)(X), Y \mright]
	%\mright)
	%- \mleft[ \lambda(X), \lambda(Y) \mright]_E,
\eas
and, using $\rho \circ \nabla^{\mathrm{bas}} = \nabla^{\mathrm{bas}} \circ \rho$ and $t_{\nabla_\rho} = t_{\nabla^{\mathrm{bas}}}$,
\bas
\mleft(\Lambda \circ t_{\nabla_\rho}
	- t_{\widetilde{\nabla}^\lambda_\rho} \circ (\Lambda, \Lambda)\mright)(\mu, \nu)
&=
t_{\mleft(\widetilde{\nabla}^\lambda\mright)^{\mathrm{bas}}} (\Lambda(\mu), \Lambda(\nu))
	- \mleft(\Lambda \circ t_{\nabla^{\mathrm{bas}}}\mright)(\mu,\nu)
\\
&=
\Lambda\mleft( 
	\nabla^{\mathrm{bas}}_{\Lambda(\mu)} \nu
	- \nabla^{\mathrm{bas}}_{\Lambda(\nu)} \mu 
	- \nabla^{\mathrm{bas}}_\mu \nu + \nabla^{\mathrm{bas}}_\nu \mu
\mright)
\\
&\hspace{1cm}
	-\mleft[ \Lambda(\mu),\Lambda(\nu) \mright]_E
	+ \Lambda\mleft( \mleft[ \mu, \nu \mright]_E \mright)
\\
&=
\Lambda
\mleft(
	\nabla^{\mathrm{bas}}_{(\lambda \circ \rho)(\nu)} \mu
	- \nabla^{\mathrm{bas}}_{(\lambda \circ \rho)(\mu)} \nu
\mright)
	-\mleft[ (\lambda \circ \rho)(\mu),(\lambda \circ \rho) (\nu) \mright]_E
\\
&\hspace{1cm}
	- \mleft[ \mu, \nu \mright]_E
	+ \mleft[ (\lambda \circ \rho)(\mu), \nu \mright]_E
	+ \mleft[ \mu, (\lambda \circ \rho)(\nu) \mright]_E
\\
&\hspace{1cm}
	+ \mleft[ \mu, \nu \mright]_E
	- (\lambda\circ\rho)\mleft( \mleft[ \mu, \nu \mright]_E \mright)
\\
&=
\lambda\mleft(
	\nabla^{\mathrm{bas}}_{(\lambda \circ \rho)(\mu)} \bigl( \rho(\nu) \bigr)
	- \nabla^{\mathrm{bas}}_{(\lambda \circ \rho)(\nu)} \bigl( \rho(\mu) \bigr)
\mright)
	-\mleft[ (\lambda \circ \rho)(\mu),(\lambda \circ \rho) (\nu) \mright]_E
\\
&\hspace{1cm}
	+ \mleft[ (\lambda\circ\rho)(\nu), \mu \mright]_E
	+ \nabla_{\rho(\mu)} \mleft( (\lambda\circ\rho)(\nu) \mright)
\\
&\hspace{1cm}
	- \mleft[ (\lambda\circ\rho)(\mu), \nu \mright]_E
	- \nabla_{\rho(\nu)} \mleft( (\lambda\circ\rho)(\mu) \mright)
\\
&\hspace{1cm}
	+ \mleft[ (\lambda \circ \rho)(\mu), \nu \mright]_E
	+ \mleft[ \mu, (\lambda \circ \rho)(\nu) \mright]_E
	- \lambda\mleft( \mleft[ \rho(\mu), \rho(\nu) \mright] \mright)
\\
&=
\mleft( \mathrm{d}^\nabla \lambda \mright)(\rho(\mu),\rho(\nu))
	+ \lambda\mleft(
	\nabla^{\mathrm{bas}}_{(\lambda \circ \rho)(\mu)} \bigl( \rho(\nu) \bigr)
	- \nabla^{\mathrm{bas}}_{(\lambda \circ \rho)(\nu)} \bigl( \rho(\mu) \bigr)
\mright)
\\
&\hspace{1cm}
	-\mleft[ (\lambda \circ \rho)(\mu),(\lambda \circ \rho) (\nu) \mright]_E
\\
&=
\mleft(-\widehat{\zeta}^\lambda \circ \mleft( \widehat{\Lambda}, \widehat{\Lambda} \mright)\mright) (\rho(\mu), \rho(\nu))
\\
&=
\mleft(-\widehat{\zeta}^\lambda \circ \mleft( \widehat{\Lambda}\circ \rho, \widehat{\Lambda}\circ\rho \mright)\mright) (\mu, \nu)
\eas
for all $\mu, \nu \in \Gamma(E)$. In a similar very straightforward fashion,
\bas
&\mleft(
\widetilde{\nabla}^\lambda \circ \Lambda - \Lambda \circ \nabla
	- t_{\widetilde{\nabla}^\lambda_\rho} \circ \mleft( \lambda, \Lambda \mright)
\mright) (Y, \mu)
\\
&=
\biggl(
	\nabla \circ \Lambda - \Lambda \circ \nabla 
	+ t_{\mleft(\widetilde{\nabla}^\lambda\mright)^{\mathrm{bas}}} (\lambda, \Lambda)
%\\
%&\hspace{1cm}\hphantom{\biggl(}
	+ \mleft(\mleft( \Lambda \circ \mathrm{d}^{\nabla^{\mathrm{bas}}} \circ \Lambda^{-1} \mright) \lambda\mright) \circ (\mathds{1}_{\mathrm{T}N}, \Lambda)
\biggr) (Y, \mu)
\\
&=
\nabla_Y \bigl( \Lambda(\mu) \bigr)
	- \Lambda (\nabla_Y \mu)
%\\
%&\hspace{1cm}
	+ \Lambda\mleft( \nabla^{\mathrm{bas}}_{\lambda(Y)} \mu \mright)
	- \mleft(\Lambda \circ \nabla^{\mathrm{bas}}_{\Lambda(\mu)} \circ \Lambda^{-1}\mright) \bigl( \lambda(Y) \bigr)
	-\mleft[ \lambda(Y),\Lambda(\mu) \mright]_E
\\
&\hspace{1cm}
	+ \mleft(\Lambda \circ \nabla^{\mathrm{bas}}_{\Lambda(\mu)} \circ \Lambda^{-1} \mright) \bigl( \lambda(Y)\bigr) 
	- \lambda\mleft(\nabla^{\mathrm{bas}}_{\Lambda(\mu)} Y \mright)
\\
&= \dotsc
\\
&=
	\nabla_Y \Bigl( \bigl(\lambda \circ(-\rho)\bigr)(\mu) \Bigr)
	- \nabla_{-\rho(\mu)} \bigl( \lambda(Y) \bigr)
	- \lambda\bigl( \mleft[ Y, - \rho(\mu) \mright] \bigr)
\\
&\hspace{1cm}
	- \mleft[ \lambda(Y), \bigl(\lambda \circ (-\rho)\bigr)(\mu) \mright]_E
	+ \lambda 
	\mleft(
		\nabla^{\mathrm{bas}}_{\lambda(Y)}\bigl(- \rho(\mu)\bigr) 
		- \nabla^{\mathrm{bas}}_{\mleft(\lambda \circ (-\rho) \mright)(\mu)} Y
	\mright)
\\
&=
\mleft( \mathrm{d}^\nabla \lambda \mright) \bigl( Y, - \rho(\mu) \bigr)
	+ \lambda 
	\mleft(
		\nabla^{\mathrm{bas}}_{\lambda(Y)}\bigl(- \rho(\mu)\bigr) 
		- \nabla^{\mathrm{bas}}_{\mleft(\lambda \circ (-\rho) \mright)(\mu)} Y
	\mright)
	- \mleft[ \lambda(Y), \bigl(\lambda \circ (-\rho)\bigr)(\mu) \mright]_E
\\
&=
\mleft(-\widehat{\zeta}^\lambda \circ \mleft( \widehat{\Lambda}, \widehat{\Lambda} \mright)\mright) \bigl(Y, -\rho(\mu) \bigr)
\\
&=
\mleft(-\widehat{\zeta}^\lambda \circ \mleft( \widehat{\Lambda}, \widehat{\Lambda} \circ (-\rho) \mright)\mright) \bigl(Y, \mu \bigr)
\eas
for all $\mu \in \Gamma(E)$ and $Y \in \mathfrak{X}(N)$. Finally, we can therefore conclude, by using that $-\widehat{\zeta}^\lambda \circ \mleft( \widehat{\Lambda}, \widehat{\Lambda} \mright)$ is clearly an antisymmetric tensor by definition,
\bas
\widetilde{F}^\lambda(\Phi, A)
&=
(\Phi^* \Lambda) (F)
+ \underbrace{\Biggl(\Phi^* \biggl(
	-\widehat{\zeta}^\lambda \circ \mleft( \widehat{\Lambda}, \widehat{\Lambda} \circ (-\rho) \mright)
\biggr) \Biggr) \mleft( \mathrm{D}\Phi \stackrel{\wedge}{,} A \mright)}
_{\mathclap{\stackrel{\text{Prop.~\ref{prop:GradedExtensionPlusAntiSymm}}}{=}  \frac{1}{2}\mleft(  
	\mleft(\Phi^* \mleft( -\widehat{\zeta}^\lambda \circ \mleft( \widehat{\Lambda}, \widehat{\Lambda} \mright) \mright)\mright)\mleft(\mathrm{D}\Phi \stackrel{\wedge}{,} -(\Phi^*\rho)(A) \mright)
	+ \mleft(\Phi^* \mleft( -\widehat{\zeta}^\lambda \circ \mleft( \widehat{\Lambda}, \widehat{\Lambda} \mright) \mright)\mright)\mleft( -(\Phi^*\rho)(A) \stackrel{\wedge}{,} \mathrm{D}\Phi \mright)
	\mright)}}
\\
&\hspace{1cm}
+ \frac{1}{2}\underbrace{\Biggl( \Phi^* \biggl(
	-\widehat{\zeta}^\lambda \circ \mleft( \widehat{\Lambda}\circ \rho, \widehat{\Lambda}\circ\rho \mright)
\biggr) \Biggr) \mleft( A \stackrel{\wedge}{,} A \mright)}
_{= \mleft( \Phi^* \mleft(
	-\widehat{\zeta}^\lambda \circ \mleft( \widehat{\Lambda}, \widehat{\Lambda} \mright)
\mright) \mright) \mleft( -(\Phi^*\rho)(A) \stackrel{\wedge}{,} - (\Phi^*\rho)(A) \mright)}
+ \frac{1}{2}\Biggl( \Phi^* \biggl(
	-\widehat{\zeta}^\lambda \circ \mleft( \widehat{\Lambda}, \widehat{\Lambda} \mright)
\biggr) \Biggr) \mleft( \mathrm{D}\Phi \stackrel{\wedge}{,} \mathrm{D} \Phi \mright)
\\
&=
(\Phi^*\Lambda)(F)
	+ \frac{1}{2} \Biggl(
		\Phi^* \biggl(
			-\widehat{\zeta}^\lambda \circ \mleft( \widehat{\Lambda}, \widehat{\Lambda} \mright)
		\biggr)
	\Biggr) \mleft( \mathfrak{D}^A\Phi \stackrel{\wedge}{,} \mathfrak{D}^A \Phi \mright)
\\
&=
\mleft( \Phi^* \Lambda \mright) \mleft(
	F
	- \frac{1}{2} \Biggl(\Phi^*\biggl(\Lambda^{-1} \circ \widehat{\zeta}^\lambda \circ \mleft( \widehat{\Lambda}, \widehat{\Lambda} \mright)\biggr)\Biggr) \mleft( \mathfrak{D}^A\Phi \stackrel{\wedge}{,} \mathfrak{D}^A \Phi \mright)
\mright)
\\
&=
\mleft( \Phi^* \Lambda \mright) \mleft(
	F
	- \frac{1}{2} \mleft(\Phi^*\xi\mright) \mleft( \mathfrak{D}^A\Phi \stackrel{\wedge}{,} \mathfrak{D}^A \Phi \mright)
\mright).
\eas
\end{proof}

Let us now look at the compatibility conditions of Thm.~\ref{thm:GaugeInvariantStandardLagrangian} and how they change under the field redefinition. For this we need the following auxiliary results.

\begin{propositions}{Change of (basic) curvature under a change of the connection}{ChangeofCurvaturesUnderCHangesOfConnections}
Let $E\to N$ be a Lie algebroid, equipped with a vector bundle connection $\nabla$. For any other connection $\nabla^\prime$ write $\nabla^\prime = \nabla + I$ where $I \in \Omega^1(N; \mathrm{End}(E))$. Then we have
\ba
R^{\mathrm{bas}}_{\nabla^\prime}
&=
R^{\mathrm{bas}}_\nabla
	- \mathrm{d}^{\nabla^{\mathrm{bas}}} I
	- I \wedge (\rho \circ I).
\ea
For the curvatures of the connections we get
\ba
R_{\nabla^\prime}
&=
R_\nabla
	+ \mathrm{d}^\nabla I
	+ I \wedge I.
\ea
\end{propositions}

\begin{remark}\label{Wedgies}
\leavevmode\newline
$I \wedge (\rho \circ I)$ is similarly defined to Def.~\eqref{DefVonWedgedemitEnd} although $\rho \circ I$ has values in $\mathrm{T}N$, the first factor $I$ simply acts on the $\mathrm{T}N$ part then, \textit{i.e.}~$I \wedge (\rho \circ I)$ is an element of $\Omega^{1,2}(N, E;E)$ defined by
\bas
\bigl(I \wedge (\rho \circ I)\bigr)(Y, \mu, \nu)
&=
I \bigl( (\rho\circ I)\bigl(Y, \nu \bigr), \mu\bigr)
	- I\bigl((\rho\circ I)\bigl(Y, \mu \bigr), \nu \bigr)
\eas
for all $\mu, \nu \in \Gamma(E)$ and $Y \in \mathfrak{X}(N)$.

$I \wedge I \in \Omega^2(N; \mathrm{End}(E))$ makes direct use of Def.~\eqref{DefVonWedgedemitEnd}, but the second factor is directly contracted with a section of $E$, that is
\bas
\mleft( I \wedge I \mright)(X, Y, \nu )
&=
I\bigl( X, I(Y, \nu) \bigr)
	- I\bigl( Y, I(X, \nu) \bigr)
\eas
for all $\nu \in \Gamma(E)$ and $X, Y \in \mathfrak{X}(N)$. Using the definition of derivations $\mathcal{D}(V)$ of vector bundles $V$ one could also write
\bas
(I \wedge I)(X,Y, \cdot)
&=
\mleft[ I(X, \cdot), I(Y, \cdot) \mright]_{\mathcal{D}(E)}
\eas
for all $X, Y \in \mathfrak{X}(N)$.
\end{remark}

\begin{proof}[Proof of Prop.~\ref{prop:ChangeofCurvaturesUnderCHangesOfConnections}]
\leavevmode\newline
We have
\bas
\mleft(\nabla^\prime\mright)^{\mathrm{bas}}_\nu Y
&=
\mleft[ \rho(\nu), Y \mright]
	+ \rho\mleft( \nabla^\prime_Y \nu \mright)
=
\nabla^{\mathrm{bas}}_\nu Y
	+ \rho\bigl( I(Y, \nu) \bigr),
%\\
%\mleft(\nabla^\prime\mright)^{\mathrm{bas}}_\nu \mu
%&=
%\mleft[ \nu, \mu \mright]_E
	%+ \nabla^\prime_{\rho(\mu)} \nu
%=
%\nabla^{\mathrm{bas}}_\nu \mu
	%+ I\bigl( \rho(\mu), \nu \bigr)
\eas
for all $\mu, \nu \in \Gamma(E)$ and $Y \in \mathfrak{X}(N)$. Using these identities we get
\bas
R_{\nabla^\prime}^\mathrm{bas}(\mu, \nu) Y
&=
\nabla^\prime_Y\mleft(\mleft[\mu, \nu\mright]_E\mright) 
	- \mleft[ \nabla^\prime_Y \mu, \nu \mright]_E 
	- \mleft[ \mu, \nabla^\prime_Y \nu \mright]_E 
	- \nabla^\prime_{\mleft(\nabla^\prime\mright)^{\mathrm{bas}}_\nu Y} \mu 
	+ \nabla^\prime_{\mleft(\nabla^\prime\mright)^{\mathrm{bas}}_\mu Y} \nu
\\
&=
\underbrace{\nabla_Y\mleft(\mleft[\mu, \nu\mright]_E\mright) 
	- \mleft[ \nabla_Y \mu, \nu \mright]_E 
	- \mleft[ \mu, \nabla_Y \nu \mright]_E
	- \nabla_{\nabla^{\mathrm{bas}}_\nu Y} \mu 
	+ \nabla_{\nabla^{\mathrm{bas}}_\mu Y} \nu}
	_{= R_\nabla^{\mathrm{bas}}(\mu, \nu)Y}
\\
&\hspace{1cm}
	- \mleft[ I(Y, \mu), \nu \mright]_E 
	- \mleft[ \mu, I(Y, \nu) \mright]_E
	+ I\mleft(Y, \mleft[\mu, \nu\mright]_E\mright)
\\
&\hspace{1cm}
	- \nabla_{\mleft(\rho \circ I \mright)(Y, \nu)} \mu
	+ \nabla_{\mleft(\rho \circ I \mright)(Y, \mu)} \nu
\\
&\hspace{1cm}
	- I \mleft( \nabla_\nu^{\mathrm{bas}} Y, \mu\mright)
	+ I\mleft(\nabla^{\mathrm{bas}}_\mu Y, \nu \mright)
	- I \Bigl( (\rho\circ I)\bigl(Y, \nu \bigr), \mu\Bigr)
	+ I\Bigl((\rho\circ I)\bigl(Y, \mu \bigr), \nu \Bigr)
\\
&=
R_\nabla^{\mathrm{bas}}(\mu, \nu)Y
\\
&\hspace{1cm}
	+ \nabla^{\mathrm{bas}}_\nu \mleft(
		I(Y, \mu)
	\mright)
	- I \mleft( \nabla_\nu^{\mathrm{bas}} Y, \mu\mright)
\\
&\hspace{1cm}
	- \nabla^{\mathrm{bas}}_\mu \mleft(
		I(Y, \nu)
	\mright)
	+ I\mleft(\nabla^{\mathrm{bas}}_\mu Y, \nu \mright)
\\
&\hspace{1cm}
	+ I\mleft(Y, \mleft[\mu, \nu\mright]_E\mright)
	- I \Bigl( (\rho\circ I)\bigl(Y, \nu \bigr), \mu\Bigr)
	+ I\Bigl((\rho\circ I)\bigl(Y, \mu \bigr), \nu \Bigr)
\\
&=
\mleft(
R^{\mathrm{bas}}_\nabla
	- \mathrm{d}^{\nabla^{\mathrm{bas}}} I
	- I \wedge (\rho \circ I)
\mright) (Y, \mu, \nu)
\eas
for all $\mu, \nu \in \Gamma(E)$ and $Y \in \mathfrak{X}(N)$.
 For the curvatures we get
\bas
R_{\nabla^\prime}(\cdot, \cdot) \nu
&=
\mathrm{d}^{\nabla^\prime} \mleft( \nabla^\prime \nu \mright)
%\\
%&=
%\mathrm{d}^{\nabla^\prime} (\nabla\nu)
	%+ \mathrm{d}^{\nabla^\prime} \bigl(I(\cdot, \nu) \bigr)
\\
&\stackrel{\mathclap{\text{Eq.~\eqref{eqDifferentialSplit}}}}{=}~~~
\mathrm{d}^{\nabla} \mleft( \nabla^\prime \nu \mright) + I \wedge \nabla^\prime \nu
\\
&=
	R_\nabla(\cdot, \cdot)\nu
	+ \underbrace{\mathrm{d}^\nabla \bigl( I(\cdot, \nu) \bigr)}
	_{\mathclap{\stackrel{\text{Eq.~\eqref{TypischerSplitdesDifferentialsaufdasWedgeProdukt}}}{=} \mleft( \mathrm{d}^\nabla I \mright)(\nu)- I \wedge \nabla \nu}}
	+ ~I \wedge \nabla \nu
	+ I \wedge I(\cdot, \nu)
\\
&=
\mleft(
R_\nabla
	+ \mathrm{d}^\nabla I
	+ I \wedge I
\mright) (\nu)
\eas
for all $\nu \in \Gamma(E)$, where we used that $T \wedge \nu = T(\nu) \in \Omega^\bullet(N; E)$ for all $T \in \Omega^\bullet(N;\mathrm{End}(E))$.
%By Eq.~\eqref{eqDifferentialSplit} we can write $\mathrm{d}^{\nabla^\prime} I - I \wedge I = \mathrm{d}^\nabla I + I \wedge I$. 
%\bas
%\mleft(
	%\mathrm{d}^\nabla \bigl( I(\cdot, \nu) \bigr)
	%+ I \wedge \nabla \nu
%\mright)(X,Y)
%&=
%\nabla_X \bigl( I(Y, \nu) \bigr)
	%- \nabla_Y \bigl( I(X, \nu) \bigr)
	%- I\bigl( [X,Y], \nu \bigr)
%\\
%&\hspace{1cm}	
	%+ I\mleft( X, \nabla_Y \nu \mright)
	%- I\mleft( Y, \nabla_X \nu \mright)
%\eas
%for all $X, Y \in \mathfrak{X}(N)$ and $\nu \in \Gamma(E)$.
\end{proof}

Let us first look at the compatibility conditions besides the curvature of $\nabla$; we want that these are preserved with the field redefinition.

\begin{theorems}{Field redefinition of the compatibility conditions except curvature}{FieldRedefDerEinfacherenCompatibilities}
Let $N$ be smooth manifolds, $E \to N$ a Lie algebroid, $\nabla$ a connection on $E$, and $\kappa$ and $g$ fibre metrics on $E$ and $\mathrm{T}N$, respectively. Assume that the compatibility conditions of Thm.~\ref{thm:GaugeInvariantStandardLagrangian} are satisfied, but $\nabla$ is allowed to be non-flat. Also let $\lambda \in \Omega^1(N; E)$ such that $\Lambda = \mathds{1}_E - \lambda \circ \rho$ is an element of $\sAut(E)$. Then we have
\ba
\mleft(\widetilde{\nabla}^{\lambda}\mright)^{\mathrm{bas}} \widetilde{\kappa}^\lambda
&=
0,
\\
\mleft(\widetilde{\nabla}^{\lambda}\mright)^{\mathrm{bas}} \widetilde{g}^\lambda
&=
0,
\\
R_{\widetilde{\nabla}^\lambda}^\mathrm{bas}
&=
0.
%\\
%R_{\widetilde{\nabla}^\lambda}
%&=
%-\mathrm{d}^{\mleft(\widetilde{\nabla}^\lambda\mright)^{\mathrm{bas}}} \widehat{\zeta}^\lambda,
\ea
%where $\widehat{\zeta}^\lambda$ is the element of $\Omega^2(N;E)$ as defined in Thm.~\ref{thm:FieldRedefofstandardFieldStrengthF}.
\end{theorems}

\begin{proof}
\leavevmode\newline
For the compatibilities with the metrics use Eq.~\eqref{FieldTrafoOfKappa}, \eqref{FieldTrafoOfG} and \eqref{basicconnectionTrafoRefield}, so,
\bas
&\mleft(\mleft(\widetilde{\nabla}^{\lambda}\mright)^{\mathrm{bas}} \widetilde{g}^\lambda\mright)
\mleft( \widehat{\Lambda}(X), \widehat{\Lambda}(Y) \mright)
\\
&=
\mathrm{d}\mleft(  
	\widetilde{g}^\lambda \mleft( \widehat{\Lambda}(X), \widehat{\Lambda}(Y) \mright)
\mright)
	- \widetilde{g}^\lambda \mleft( \mleft(\widetilde{\nabla}^{\lambda}\mright)^{\mathrm{bas}} \mleft( \widehat{\Lambda}(X) \mright), \widehat{\Lambda}(Y) \mright)
	- \widetilde{g}^\lambda \mleft( \widehat{\Lambda}(X), \mleft(\widetilde{\nabla}^{\lambda}\mright)^{\mathrm{bas}} \mleft( \widehat{\Lambda}(Y) \mright) \mright)
\\
&=
\mathrm{d}\mleft(  
	g\mleft( X, Y \mright)
\mright)
	- g \mleft( \nabla^{\mathrm{bas}} X, Y \mright)
	- g \mleft( X, \nabla^{\mathrm{bas}} Y \mright)
\\
&=
\mleft(\nabla^{\mathrm{bas}} g\mright)(X, Y)
\\
&=
0,
\eas
for all $X, Y \in \mathfrak{X}(N)$,
similarly for $\kappa$. For $I \coloneqq \mleft( \Lambda \circ \mathrm{d}^{\nabla^{\mathrm{bas}}} \circ \Lambda^{-1} \mright) \lambda \in \Omega^1(N; \mathrm{End}(E)) \cong \Omega^{1,1}(N,E;E)$ we also have
\bas
I(Y, \nu)
%&=
%\mleft( \Lambda \circ \nabla^{\mathrm{bas}}_\nu \circ \Lambda^{-1} \circ \lambda - \lambda \circ \nabla^{\mathrm{bas}}_\nu \mright)(Y)
&\stackrel{\text{Eq.~\eqref{FieldTrafoOfNabla}}}{=}
\widetilde{\nabla}^\lambda_\nu Y
	- \nabla_\nu Y
\eas
for all $\nu \in \Gamma(E)$ and $Y \in \mathfrak{X}(N)$, and
%and by Eq.~\eqref{KuerzesteFormelForRedefOfNabla}
%\bas
%I\mleft(\widehat{\Lambda}(Y), \nu\mright)
%&=
%\mleft(\mathrm{d}^{\nabla^{\mathrm{bas}}} \lambda \mright)(Y, \nu).
%\eas
%We also have
\bas
%\mleft(\widetilde{\nabla}^\lambda\mright)^{\mathrm{bas}}_\nu Y
%&=
%\mleft[ \rho(\nu), Y \mright]
	%+ \rho\mleft( \widetilde{\nabla}^\lambda_Y \nu \mright)
%=
%\nabla^{\mathrm{bas}}_\nu Y
	%+ \rho\bigl( I(Y, \nu) \bigr),
%\\
\underbrace{\nabla^{\mathrm{bas}}_\nu \mu}
_{\mathclap{ = \mleft[ \nu, \mu \mright]_E + \nabla_{\rho(\mu)} \nu }}
	+ I\bigl( \rho(\mu), \nu \bigr)
&=
\mleft[ \nu, \mu \mright]_E
	+ \widetilde{\nabla}^\lambda_{\rho(\mu)} \nu
=
\mleft(\widetilde{\nabla}^\lambda\mright)^{\mathrm{bas}}_\nu \mu
\stackrel{\text{Eq.~\eqref{basicconnectionTrafoRefield}}}{=}
\mleft( \Lambda \circ \nabla^{\mathrm{bas}}_\nu \circ \Lambda^{-1}\mright) \mu
\eas
for all $\mu, \nu \in \Gamma(E)$.
Using these identities and $R_\nabla^\mathrm{bas} = 0$, we can show
\bas
\mleft(
\mathrm{d}^{\nabla^{\mathrm{bas}}} I
	+ I \wedge (\rho \circ I) 
\mright)(Y, \mu,\nu)
&=
\nabla^{\mathrm{bas}}_\mu \bigl( I(Y, \nu) \bigr)
	- I \mleft( \nabla^{\mathrm{bas}}_\mu Y, \nu \mright)
\\
&\hspace{1cm}
	- \nabla^{\mathrm{bas}}_\nu \bigl( I(Y, \mu) \bigr)
	+ I \mleft( \nabla^{\mathrm{bas}}_\nu Y, \mu \mright)
\\
&\hspace{1cm}
	- I \mleft( Y, \mleft[ \mu, \nu \mright]_E \mright)
	+ I\mleft( (\rho \circ I)(Y, \nu), \mu \mright)
	- I\mleft( (\rho \circ I)(Y, \mu), \nu \mright)
\\
&=
\mleft( \Lambda \circ \nabla^{\mathrm{bas}}_\mu \circ \Lambda^{-1}\mright) \bigl( I(Y, \nu) \bigr)
	- I \mleft( \nabla^{\mathrm{bas}}_\mu Y, \nu \mright)
\\
&\hspace{1cm}
	- \mleft( \Lambda \circ \nabla^{\mathrm{bas}}_\nu \circ \Lambda^{-1}\mright) \bigl( I(Y, \mu) \bigr)
	+ I \mleft( \nabla^{\mathrm{bas}}_\nu Y, \mu \mright)
\\
&\hspace{1cm}
	- I \mleft( Y, \mleft[ \mu, \nu \mright]_E \mright)
\\
&=
\mleft(\mleft( \Lambda \circ \mathrm{d}^{\nabla^{\mathrm{bas}}} \circ \Lambda^{-1} \mright) I \mright) (Y, \nu ,\mu)
\\
&=
\mleft(\mleft( \Lambda \circ \mathrm{d}^{\nabla^{\mathrm{bas}}} \circ \Lambda^{-1} \mright)^2 \lambda \mright) (Y, \nu ,\mu)
\\
&=
\Biggl(\biggl( \Lambda \circ \underbrace{\mleft(\mathrm{d}^{\nabla^{\mathrm{bas}}}\mright)^2}
_{\mathclap{\stackrel{\text{Prop.~\ref{prop:SnablamitREnabla}}}{=}~ 0}} 
\circ ~\Lambda^{-1} \biggr) \lambda \Biggr) (Y, \nu ,\mu)
\\
&=
0.
\eas
for all $\mu, \nu \in \Gamma(E)$ and $Y \in \mathfrak{X}(N)$.
Using this and $R_\nabla^\mathrm{bas} = 0$, we get
\bas
R_{\widetilde{\nabla}^\lambda}^\mathrm{bas}
&\stackrel{\text{Prop.~\ref{prop:ChangeofCurvaturesUnderCHangesOfConnections}}}{=}
R_\nabla^{\mathrm{bas}}
	- \mathrm{d}^{\nabla^{\mathrm{bas}}} I
	- I \wedge (\rho \circ I)
=
0.
\eas
%%%%%%%%%%%%%%%%%%%%%%%%%%%%%%%%%%%%%%%%%%%%%%%%%%%%%%%%%%%%%%%%%%%%%%%%%%%%%%%%%%%%%%%%%%%%%%%%%%%%%%%%%%%%%%%%%%%%%%%%%%%%%%%%%%%%%%%%%%%%%%%%%%%%%%%%%%%%%%%%%%%%%%%%%%%%%%%%%%%%%%%%%%%%%%%%%%%%%%
%We now turn to the curvature, using $R_\nabla = 0$ and Prop.~\ref{prop:ChangeofCurvaturesUnderCHangesOfConnections},
%\bas
%R_{\widetilde{\nabla}^\lambda}
%&=
%\mathrm{d}^\nabla I
	%+ I \wedge I.
%\eas
%In order to calculate this, it is useful to think of $\Lambda(\omega)$ as $\Lambda \wedge \omega$ for all $\omega\in\Omega^{p,q}(N,E;E)$ ($p,q\in\mathbb{N}_0$), where we view $\Lambda$ as an element of $\Omega^0(N;\mathrm{End}(E))$, such that we can apply Eq.~\eqref{TypischerSplitdesDifferentialsaufdasWedgeProdukt} (similarly for other similar terms). Moreover, due to $R_\nabla=0$ and $R_\nabla^{\mathrm{bas}}=0$ we have by Prop.~\ref{prop:commutationrelation}
%\bas
%\mathrm{d}^\nabla
%\mathrm{d}^{\nabla^{\mathrm{bas}}}
%&=
%\mathrm{d}^{\nabla^{\mathrm{bas}}}
%\mathrm{d}^\nabla,
%\eas
%hence,
%\bas
%\mathrm{d}^\nabla I
%&=
%\mathrm{d}^\nabla \mleft( \Lambda \wedge \mleft( \mathrm{d}^{\nabla^{\mathrm{bas}}} \circ \Lambda^{-1} \mright) (\lambda) \mright)
%\\
%&=
%\mathrm{d}^\nabla \Lambda \wedge \mleft(\mleft( \mathrm{d}^{\nabla^{\mathrm{bas}}} \circ \Lambda^{-1} \mright) (\lambda)\mright)
	%+ \Lambda \wedge \mleft( \mathrm{d}^\nabla \mathrm{d}^{\nabla^{\mathrm{bas}}} \circ \Lambda^{-1} \mright) (\lambda)
%\\
%&=
%\mathrm{d}^\nabla \Lambda \wedge \mleft(\mleft( \mathrm{d}^{\nabla^{\mathrm{bas}}} \circ \Lambda^{-1} \mright) (\lambda)\mright)
	%+ \Lambda \wedge \mleft( \mathrm{d}^{\nabla^{\mathrm{bas}}} \mathrm{d}^\nabla \mleft( \Lambda^{-1} \wedge \lambda \mright) \mright)
%\\
%&=
%\mathrm{d}^\nabla \Lambda \wedge \mleft(\mleft( \mathrm{d}^{\nabla^{\mathrm{bas}}} \circ \Lambda^{-1} \mright) (\lambda)\mright)
	%+ \underbrace{\Lambda \wedge \mathrm{d}^{\nabla^{\mathrm{bas}}} \mleft( \mathrm{d}^\nabla \Lambda^{-1} \wedge \lambda \mright)}
	%_{= \mleft( \Lambda \circ \mathrm{d}^{\nabla^{\mathrm{bas}}} \mright) \mleft( \mathrm{d}^\nabla \Lambda^{-1} \wedge \lambda \mright)}
	%+ \underbrace{\Lambda \wedge \mathrm{d}^{\nabla^{\mathrm{bas}}} \mleft( \Lambda^{-1} \wedge \mathrm{d}^\nabla \lambda \mright)}
	%_{= \mleft( \Lambda \circ \mathrm{d}^{\nabla^{\mathrm{bas}}} \circ \Lambda^{-1} \mright) \mathrm{d}^\nabla \lambda}
%\\
%&=
%\mleft(\mathrm{d}^\nabla \mathrm{d}^{\nabla^{\mathrm{bas}}} \circ \Lambda^{-1} \mright)\lambda
	%- \mleft(\mathrm{d}^\nabla \circ \lambda \circ \rho \circ \mathrm{d}^{\nabla^{\mathrm{bas}}} \circ \Lambda^{-1} \mright)\lambda
%\eas
%where
%\bas
%&\mleft(\mleft( \Lambda \circ \mathrm{d}^{\nabla^{\mathrm{bas}}} \circ \Lambda^{-1} \mright)\mleft(
	%\lambda\mleft( t_{\nabla^{\mathrm{bas}}_\lambda} \mright)
	%- R_\lambda
%\mright)\mright) \mleft(X, Y, \nu\mright)
%\\
%&=
%\mleft(\Lambda \circ \nabla^{\mathrm{bas}}_\nu \circ \Lambda^{-1}\mright)\mleft(
	%\lambda\Bigl(
		%\nabla^{\mathrm{bas}}_{\lambda(X)} Y
		%- \nabla^{\mathrm{bas}}_{\lambda(Y)} X
	%\Bigr)
	%- \mleft[ \lambda(X), \lambda(Y) \mright]_E
%\mright)
%\\
%&\hspace{1cm}
	%- \lambda\mleft(
		%\nabla^{\mathrm{bas}}_{\lambda\mleft(\nabla^{\mathrm{bas}}_\nu X\mright)} Y
		%- \nabla^{\mathrm{bas}}_{\lambda(Y)} \nabla^{\mathrm{bas}}_\nu X
		%%}_{= [(\rho\circ\lambda)(Y), X] + \rho\mleft( \nabla_X ( \lambda(Y) ) \mright)}
	%\mright)
	%+ \mleft[ \lambda\mleft(\nabla^{\mathrm{bas}}_\nu X\mright), \lambda(Y) \mright]_E
%\\
%&\hspace{1cm}
	%- \lambda\mleft(
		%\nabla^{\mathrm{bas}}_{\lambda(X)} \nabla^{\mathrm{bas}}_\nu Y
		%- \nabla^{\mathrm{bas}}_{\lambda\mleft(\nabla^{\mathrm{bas}}_\nu Y\mright)} X
		%%}_{= [(\rho\circ\lambda)(Y), X] + \rho\mleft( \nabla_X ( \lambda(Y) ) \mright)}
	%\mright)
	%+ \mleft[ \lambda(X), \lambda\mleft(\nabla^{\mathrm{bas}}_\nu Y\mright) \mright]_E
%\eas
%where
%\bas
%&\mleft( 
	%\mleft(\mathrm{d}^{\nabla^{\mathrm{bas}}} \mathrm{d}^\nabla \circ \Lambda^{-1} \mright)\lambda \mright)\mleft(\widehat{\Lambda}(X),\widehat{\Lambda}(Y), \nu\mright)
%\\
%&=
%\nabla^{\mathrm{bas}}_\nu \mleft(
	%\mleft(\nabla_{\widehat{\Lambda}(X)} \circ \lambda\mright)(Y) 
	%- \mleft(\nabla_{\widehat{\Lambda}(Y)} \circ \lambda\mright)(X) 
	%- \lambda \mleft( \mleft[ \widehat{\Lambda}(X), \widehat{\Lambda}(Y) \mright] \mright)
%\mright)
%\\
%&\hspace{1cm}
	%- \mleft( \nabla_{\nabla^{\mathrm{bas}}_\nu\mleft(\widehat{\Lambda}(X)\mright)} \circ \lambda \mright)(Y)
	%+ \mleft(\nabla_{\widehat{\Lambda}(Y)} \circ \lambda\circ\widehat{\Lambda}^{-1} \circ \nabla^{\mathrm{bas}}_\nu \circ \widehat{\Lambda}\mright)(X) 
	%+ \lambda \mleft( \mleft[ \nabla^{\mathrm{bas}}_\nu\mleft(\widehat{\Lambda}(X)\mright), \widehat{\Lambda}(Y) \mright] \mright)
%\\
%&\hspace{1cm}
	%- \mleft( \nabla_{\widehat{\Lambda}(X)} \circ\lambda\circ\widehat{\Lambda}^{-1} \circ \nabla^{\mathrm{bas}}_\nu \circ \widehat{\Lambda}\mright)(Y) 
	%+ \mleft(\nabla_{\nabla^{\mathrm{bas}}_\nu\mleft(\widehat{\Lambda}(Y)\mright)} \circ \lambda\mright)(X) 
	%+ \lambda \mleft( \mleft[ \widehat{\Lambda}(X), \nabla^{\mathrm{bas}}_\nu\mleft(\widehat{\Lambda}(Y)\mright) \mright] \mright)
%\eas
%and
%\bas
%&-\mleft( \mleft(\mathrm{d}^\nabla \circ \lambda \circ \rho \circ \mathrm{d}^{\nabla^{\mathrm{bas}}} \circ \Lambda^{-1} \mright)\lambda\mright) \mleft(\widehat{\Lambda}(X),\widehat{\Lambda}(Y), \nu\mright)
%\\
%&=
%\nabla_{\widehat{\Lambda}(Y)}\mleft( \mleft( 
	%\lambda \circ \rho \circ \nabla^{\mathrm{bas}}_\nu \circ \lambda 
	%- \lambda \circ \rho \circ \lambda \circ \widehat{\Lambda}^{-1} \circ \nabla^{\mathrm{bas}}_\nu \circ \widehat{\Lambda} 
%\mright)(X) \mright)
%\\
%&\hspace{1cm}
	%- \mleft( \lambda \circ \rho \circ \nabla^{\mathrm{bas}}_{\nabla_{\widehat{\Lambda}(Y)}\nu} \circ \lambda 
		%- \lambda \circ \rho \circ \lambda \circ \widehat{\Lambda}^{-1} \circ \nabla^{\mathrm{bas}}_{\nabla_{\widehat{\Lambda}(Y)}\nu} \circ \widehat{\Lambda} 
	%\mright)(X)
%\\
%&\hspace{1cm}
	%- \nabla_{\widehat{\Lambda}(X)}\mleft(\mleft( 
		%\lambda \circ \rho \circ \nabla^{\mathrm{bas}}_\nu \circ \lambda 
		%- \lambda \circ \rho \circ \lambda \circ \widehat{\Lambda}^{-1} \circ \nabla^{\mathrm{bas}}_\nu \circ \widehat{\Lambda} 
%\mright)(Y)\mright)
%\\
%&\hspace{1cm}
	%+ \mleft( \lambda \circ \rho \circ \nabla^{\mathrm{bas}}_{\nabla_{\widehat{\Lambda}(X)}\nu} \circ \lambda 
		%- \lambda \circ \rho \circ \lambda \circ \widehat{\Lambda}^{-1} \circ \nabla^{\mathrm{bas}}_{\nabla_{\widehat{\Lambda}(X)}\nu} \circ \widehat{\Lambda} 
	%\mright)(Y)
%\\
%&\hspace{1cm}
	%+ \mleft( 
	%\lambda \circ \rho \circ \nabla^{\mathrm{bas}}_\nu \circ \lambda \circ \widehat{\Lambda}^{-1}
	%- \lambda \circ \rho \circ \lambda \circ \widehat{\Lambda}^{-1} \circ \nabla^{\mathrm{bas}}_\nu
%\mright)\mleft(\mleft[ \widehat{\Lambda}(X), \widehat{\Lambda}(Y) \mright]\mright)
%\eas
%%and
%%\bas
%%\mleft( \mathrm{d}^\nabla \Lambda \mright) \mleft(\widehat{\Lambda}(Y), \nu\mright)
%%&=
%%\nabla_{\widehat{\Lambda}(Y)} \bigl( \Lambda(\nu) \bigr)
	%%- \Lambda \mleft( \nabla_{\widehat{\Lambda}(Y)} \nu \mright)
%%=
%%(\lambda \circ \rho) \mleft( \nabla_{\widehat{\Lambda}(Y)} \nu \mright)
	%%- \nabla_{\widehat{\Lambda}(Y)} \bigl( (\lambda\circ\rho)(\nu) \bigr)
%%\eas
%for all $\nu \in \Gamma(E)$ and $Y \in \mathfrak{X}(N)$, and
%\bas
%(I \wedge I)\mleft(\widehat{\Lambda}(X),\widehat{\Lambda}(Y), \nu\mright)
%&=
%I\mleft(\widehat{\Lambda}(X), I\mleft(\widehat{\Lambda}(Y), \nu\mright)\mright)
	%- I\mleft(\widehat{\Lambda}(Y), I\mleft(\widehat{\Lambda}(X), \nu\mright)\mright)
%\\
%&=
%\mleft( \nabla^{\mathrm{bas}}_{\mleft(\nabla^{\mathrm{bas}}_\nu \circ \lambda - \lambda \circ \nabla^{\mathrm{bas}}_\nu \mright)(Y)} \circ \lambda 
	%- \lambda \circ \nabla^{\mathrm{bas}}_{\mleft(\nabla^{\mathrm{bas}}_\nu \circ \lambda - \lambda \circ \nabla^{\mathrm{bas}}_\nu \mright)(Y)} \mright)(X)
%\\
%&\hspace{1cm}
	%- \mleft( \nabla^{\mathrm{bas}}_{\mleft(\nabla^{\mathrm{bas}}_\nu \circ \lambda - \lambda \circ \nabla^{\mathrm{bas}}_\nu \mright)(X)} \circ \lambda 
	%- \lambda \circ \nabla^{\mathrm{bas}}_{\mleft(\nabla^{\mathrm{bas}}_\nu \circ \lambda - \lambda \circ \nabla^{\mathrm{bas}}_\nu \mright)(X)} \mright)(Y)
%\eas
%in total, using again $\rho \circ \nabla^{\mathrm{bas}} = \nabla^{\mathrm{bas}} \circ \rho$ and Prop.~\ref{prop:PropsOfBigLambdas},
%\bas
%R_{\widetilde{\nabla}^\lambda}\mleft(\widehat{\Lambda}(X),\widehat{\Lambda}(Y), \nu\mright)
%&=
%\nabla^{\mathrm{bas}}_\nu \mleft(
	%- \mleft(\nabla_{(\rho\circ\lambda)(X)} \circ \lambda\mright)(Y) 
	%+ \mleft(\nabla_{(\rho\circ\lambda)(Y)} \circ \lambda\mright)(X) \mright)
%\\
%&\hspace{1cm}
	%+ \mleft(\nabla^{\mathrm{bas}}_\nu \circ \lambda \mright) \Bigl(
	%\mleft[ (\rho\circ\lambda)(X), Y \mright] 
	%+ \mleft[ X, (\rho\circ\lambda)(Y) \mright]
	%- \mleft[ (\rho\circ\lambda)(X), (\rho\circ\lambda)(Y) \mright]
%\Bigr)
%\\
%&\hspace{1cm}
	%+ \mleft(\nabla^{\mathrm{bas}}_{\lambda(Y)} \circ \nabla^{\mathrm{bas}}_\nu\circ\lambda\mright)(X)
%\\
%&\hspace{1cm}
	%- \lambda \mleft( \mleft[ \nabla^{\mathrm{bas}}_\nu X, (\rho \circ \lambda) (Y) \mright] \mright)
	%+ \lambda \mleft( \mleft[ \mleft(\nabla^{\mathrm{bas}}_\nu \circ \rho \circ \lambda\mright) (X), (\rho\circ\lambda)(Y) \mright] \mright)
%\\
%&\hspace{1cm}
	%+ \mleft(\nabla_{\mleft(\nabla^{\mathrm{bas}}_\nu\circ\rho\circ\lambda \mright)(Y)} \circ \lambda\mright)(X) 
%\\
%&\hspace{1cm}
		%+ (\lambda\circ \rho)\mleft( \nabla_{\nabla^{\mathrm{bas}}_\nu Y} \bigl( \lambda(X) \bigr) 
	%\mright)
%\\
%&\hspace{1cm}
	%- \mleft( \lambda \circ \widehat{\Lambda}^{-1} \circ \nabla^{\mathrm{bas}}_{\nabla_{\widehat{\Lambda}(Y)}\nu} \circ \rho \circ \lambda 
		%- \lambda \circ \rho \circ \lambda \circ \widehat{\Lambda}^{-1} \circ \nabla^{\mathrm{bas}}_{\nabla_{\widehat{\Lambda}(Y)}\nu}
	%\mright)(X)
%\\
%&\hspace{1cm}
	%+ \mleft( \lambda \circ \widehat{\Lambda}^{-1} \circ \nabla^{\mathrm{bas}}_{\nabla_{\widehat{\Lambda}(X)}\nu} \circ \rho \circ \lambda 
		%- \lambda \circ \rho \circ \lambda \circ \widehat{\Lambda}^{-1} \circ \nabla^{\mathrm{bas}}_{\nabla_{\widehat{\Lambda}(X)}\nu}
	%\mright)(Y)
%\\
%&\hspace{1cm}
	%+ \mleft( 
	%\lambda \circ \rho \circ \nabla^{\mathrm{bas}}_\nu \circ \Lambda^{-1} \circ \lambda
	%+ \lambda \circ \nabla^{\mathrm{bas}}_\nu
	%- \lambda \circ \widehat{\Lambda}^{-1} \circ \nabla^{\mathrm{bas}}_\nu
%\mright)\mleft(\mleft[ \widehat{\Lambda}(X), \widehat{\Lambda}(Y) \mright]\mright)
%\\
%&\hspace{1cm}
	%+ \Lambda\mleft( \mleft[ \mleft(\nabla^{\mathrm{bas}}_\nu \circ \lambda \mright)(Y), \lambda(X) \mright]_E\mright)
	%+ \mleft( \nabla_{(\rho \circ \lambda)(X)} \circ \nabla^{\mathrm{bas}}_\nu \circ \lambda \mright) (Y)
%\\
%&\hspace{1cm}
	%- \mleft( \lambda \circ \rho \circ \nabla_X \circ \nabla^{\mathrm{bas}}_\nu \circ \lambda \mright)(Y)
%\\
%&\hspace{1cm}
	%- \mleft( \nabla_{(\rho \circ \lambda)(Y)} \circ \nabla^{\mathrm{bas}}_\nu \circ \lambda \mright) (X)
%\\
%&\hspace{1cm}
	%+ \mleft(\lambda \circ\rho \circ\nabla_Y \circ \nabla^{\mathrm{bas}}_\nu \circ\lambda \mright)(X)
%\\
%&Do not look here
%\\
%%%%%%%%%%%%%%%%%%%%%%%%%%%%%%%%%%%%%%%%%%%%%%%%%%%%%%%%%%%%%%%%%%%%%%%%%%%%%%%%%%%%%%%%%%%%%%%%%%%%%%%%%%%%%%%%%%%%%%%%%%%%%%%%%%%%%%%%%%%%%%%%%%%%%%%%%%%%%%%%%%%%%%%%%%%%%%%%%%%%%%%%%%%%%%%%%%%%%%%%%%%%%%%%%%%%%%%%%%%%%%%%%%%%%%%%%%%%%%%%%%%%%%%%%%%%%%%%%%%%%%%%%%%%%%%%%%%%%%%%%%%%%%%%%%%%%%%%%%
%&\hspace{1cm}
%- \mleft(\lambda \circ \nabla^{\mathrm{bas}}_{\mleft(\lambda \circ \nabla^{\mathrm{bas}}_\nu \mright)(X)} \mright)(Y)
	%+ \mleft(\lambda \circ \nabla^{\mathrm{bas}}_{\mleft(\lambda \circ \nabla^{\mathrm{bas}}_\nu \mright)(Y)} \mright)(X)
	%- \lambda \mleft( \nabla^{\mathrm{bas}}_{\lambda(X)} \nabla^{\mathrm{bas}}_\nu Y\mright)
%\\
%&\hspace{1cm}
	%+ \mleft[ \mleft(\lambda \circ \nabla^{\mathrm{bas}}_\nu\mright)(X), \lambda(Y) \mright]_E
	%+ \mleft[ \lambda(X), \mleft(\lambda \circ \nabla^{\mathrm{bas}}_\nu\mright)(Y) \mright]_E
%\\
%&\hspace{1cm}
	%+ \nabla^{\mathrm{bas}}_\nu \underbrace{\mleft(
	%\mleft(\nabla_{X} \circ \lambda\mright)(Y) 
	%- \mleft(\nabla_{Y} \circ \lambda\mright)(X) 
	%- \lambda \mleft( \mleft[ X, Y \mright] \mright)
%\mright)}_{= \mleft(\mathrm{d^\nabla}\lambda\mright)(X,Y)}
%\\
%&\hspace{1cm}
	%\underbrace{- \mleft( \nabla_{\nabla^{\mathrm{bas}}_\nu (X)} \circ \lambda \mright)(Y)
	%+ \mleft(\nabla_{Y} \circ \lambda \circ \nabla^{\mathrm{bas}}_\nu \mright)(X) 
	%+ \lambda \mleft( \mleft[ \nabla^{\mathrm{bas}}_\nu X, Y \mright] \mright)}
	%_{= - \mleft( \mathrm{d}^\nabla \lambda \mright)\mleft( \nabla^{\mathrm{bas}}_\mu X, Y \mright)}
%\\
%&\hspace{1cm}
	%\underbrace{- \mleft( \nabla_{X} \circ\lambda \circ \nabla^{\mathrm{bas}}_\nu \mright)(Y) 
	%+ \mleft(\nabla_{\nabla^{\mathrm{bas}}_\nu Y} \circ \lambda\mright)(X) 
	%+ \lambda \mleft( \mleft[ X, \nabla^{\mathrm{bas}}_\nu Y \mright] \mright)}
	%_{= - \mleft( \mathrm{d}^\nabla \lambda \mright)\mleft( X, \nabla^{\mathrm{bas}}_\mu Y \mright)}
%\eas
%and
%\bas
%&\mleft(
%\mathrm{d}^\nabla \Lambda \wedge \mleft(\mleft( \mathrm{d}^{\nabla^{\mathrm{bas}}} \circ \Lambda^{-1} \mright) (\lambda)\mright)
%\mright) \mleft(\widehat{\Lambda}(X),\widehat{\Lambda}(Y), \nu\mright)
%\\
%&=
	%(\lambda \circ \rho) \mleft( \nabla_{\widehat{\Lambda}(X)} \mleft(  
		%\nabla^{\mathrm{bas}}_\nu \bigl( \lambda(Y) \bigr)
		%- \mleft( \Lambda^{-1} \circ \lambda \circ \nabla^{\mathrm{bas}}_\nu \circ \widehat{\Lambda} \mright)(Y)
	%\mright) \mright)
%\\
%&\hspace{1cm}
	%- \nabla_{\widehat{\Lambda}(X)} \mleft( (\lambda\circ\rho)\mleft(  
		%\nabla^{\mathrm{bas}}_\nu \bigl( \lambda(Y) \bigr)
		%- \mleft( \Lambda^{-1} \circ \lambda \circ \nabla^{\mathrm{bas}}_\nu \circ \widehat{\Lambda} \mright)(Y)
	%\mright)\mright)
%\\
%&\hspace{1cm}
	%- (\lambda \circ \rho) \mleft( \nabla_{\widehat{\Lambda}(Y)} \mleft(  
		%\nabla^{\mathrm{bas}}_\nu \bigl( \lambda(X) \bigr)
		%- \mleft( \Lambda^{-1} \circ \lambda \circ \nabla^{\mathrm{bas}}_\nu \circ \widehat{\Lambda} \mright)(X)
	%\mright) \mright)
%\\
%&\hspace{1cm}
	%+ \nabla_{\widehat{\Lambda}(Y)} \bigl( (\lambda\circ\rho)\mleft(  
		%\nabla^{\mathrm{bas}}_\nu \bigl( \lambda(X) \bigr)
		%- \mleft( \Lambda^{-1} \circ \lambda \circ \nabla^{\mathrm{bas}}_\nu \circ \widehat{\Lambda} \mright)(X)
	%\mright)\bigr)
%\eas
%and
%\bas
%&\mleft( \Lambda \circ \mathrm{d}^{\nabla^{\mathrm{bas}}} \mright) \mleft( \mathrm{d}^\nabla \Lambda^{-1} \wedge \lambda \mright) \mleft(\widehat{\Lambda}(X),\widehat{\Lambda}(Y), \nu\mright)
%\\
%&=
%\Lambda \circ \biggl(
	%\nabla^{\mathrm{bas}}_\nu\mleft(
		%\mleft(\nabla_{\widehat{\Lambda}(X)} \circ \lambda 
		%- \Lambda^{-1} \circ \lambda \circ \nabla_{\widehat{\Lambda}(X)} \circ \widehat{\Lambda} \mright)(Y) 
		%- \mleft(\nabla_{\widehat{\Lambda}(Y)} \circ \lambda 
		%- \Lambda^{-1} \circ \lambda \circ \nabla_{\widehat{\Lambda}(Y)} \circ \widehat{\Lambda} \mright)(X) 
	%\mright)
%\\
%&\hspace{1cm}
		%- \mleft(
			%\nabla_{\nabla^{\mathrm{bas}}_\nu\mleft(\widehat{\Lambda}(X)\mright)} \circ \lambda 
			%- \Lambda^{-1} \circ \lambda \circ \nabla_{\nabla^{\mathrm{bas}}_\nu\mleft(\widehat{\Lambda}(X)\mright)} \circ \widehat{\Lambda} \mright)(Y) 
%\\
%&\hspace{1cm}
		%+ \mleft(
			%\nabla_{\widehat{\Lambda}(Y)} \circ \lambda \circ \widehat{\Lambda}^{-1} \circ \nabla^{\mathrm{bas}}_\nu \circ \widehat{\Lambda}
			%- \Lambda^{-1} \circ \lambda \circ \nabla_{\widehat{\Lambda}(Y)} \circ \nabla^{\mathrm{bas}}_\nu \circ \widehat{\Lambda} \mright)(X)
%\\
%&\hspace{1cm}
	%- \mleft(\mathrm{d}^\nabla \Lambda^{-1} \wedge \lambda \mright)\mleft( \widehat{\Lambda}(X), \nabla^{\mathrm{bas}}_\nu\mleft(\widehat{\Lambda}(Y) \mright)\mright)
%\biggr)
%\eas
%%using $\mathrm{ad}_\Lambda \mleft( \mathrm{d}^{\nabla^{\mathrm{bas}}} \mright) \coloneqq \Lambda \circ \mathrm{d}^{\nabla^{\mathrm{bas}}} \circ \Lambda^{-1}$,
%\bas
%\mleft(\mathrm{d}^\nabla I\mright)(X,Y, \mu)
%&=
%\nabla_X \bigl( I(Y,\mu) \bigr)
	%- I \mleft( Y, \nabla_X \mu \mright)
	%- \nabla_Y \bigl( I(X,\mu) \bigr)
	%+ I \mleft( X, \nabla_Y \mu \mright)
	%- I\bigl( [X,Y], \mu \bigr)
%\\
%&=
%\nabla_X \mleft(
	%\mleft( \Lambda \circ \nabla^{\mathrm{bas}}_\mu \circ \Lambda^{-1} \circ \lambda - \lambda \circ \nabla^{\mathrm{bas}}_\mu \mright)(Y)
%\mright)
%\eas
%for all $X,Y \in\mathfrak{X}(N)$ and $\mu \in \Gamma(E)$.
\end{proof}

Let us now look at what happens with the curvature of $\nabla$. 

\begin{theorems}{Flatness breaking}{BrokenFlatness}
Let $N$ be smooth manifolds, $E \to N$ a Lie algebroid, and $\nabla$ a connection on $E$ with vanishing basic curvature. Also let $\lambda \in \Omega^1(N; E)$ such that $\Lambda = \mathds{1}_E - \lambda \circ \rho$ is an element of $\sAut(E)$. Then
\ba
R_{\widetilde{\nabla}^\lambda}
&=
\Lambda \circ R_\nabla \circ \mleft( \widehat{\Lambda}^{-1}, \widehat{\Lambda}^{-1} \mright)
	- \mathrm{d}^{\mleft(\widetilde{\nabla}^\lambda\mright)^{\mathrm{bas}}} \widehat{\zeta}^\lambda,
\ea
where $\widehat{\zeta}^\lambda$ is defined as in Thm.~\ref{thm:FieldRedefofstandardFieldStrengthF} and viewing the curvatures as elements of $\Omega^2(N; \mathrm{End}(E))$.
\end{theorems}

\begin{proof}[Sketch of the proof]
\leavevmode\newline
\indent $\bullet$ The proof of this theorem is extremely tedious and long, but very straightforward. Essentially, just insert all the formulas of the field redefinition on both sides, then compare both sides, making use of the vanishing of the basic curvature. However, you may want to use certain tricks to make the calculation less tedious (but it is still extremely tedious with tricks). Hence, we show the first steps until one "just" needs to insert all definitions.

First let us observe that we can rewrite $\mathrm{d}^{\mleft(\widetilde{\nabla}^\lambda\mright)^{\mathrm{bas}}} \widehat{\zeta}^\lambda$ using Cor.~\ref{cor:ConjugationOfDifferentialsAreShitty}, also recall Remark \ref{OtherNotationForZetaTransform},
\bas
- \mleft(\mathrm{d}^{\mleft(\widetilde{\nabla}^\lambda\mright)^{\mathrm{bas}}} \widehat{\zeta}^\lambda\mright)\mleft( \widehat{\Lambda}(X), \widehat{\Lambda}(Y), \nu \mright)
&=
- \Biggl( \mleft( \Lambda \circ \mathrm{d}^{\nabla^{\mathrm{bas}}} \circ \Lambda^{-1} \mright)
\mleft( \widehat{\zeta}^\lambda \circ \mleft( \widehat{\Lambda}, \widehat{\Lambda} \mright) \mright)
 \Biggr) (X, Y, \nu)
\\
&=
\Biggl( \mleft( \Lambda \circ \mathrm{d}^{\nabla^{\mathrm{bas}}} \circ \Lambda^{-1} \mright)
	\mleft(\mathrm{d}^\nabla \lambda
	+ \lambda\mleft(t_{\nabla^{\mathrm{bas}}_\lambda}\mright)
	- R_\lambda\mright)
 \Biggr) (X, Y, \nu)
\eas
for all $X, Y \in \mathfrak{X}(N)$ and $\nu \in \Gamma(E)$, where $-\widehat{\zeta}^\lambda \circ \mleft( \widehat{\Lambda}, \widehat{\Lambda} \mright)$ is given by Eq.~\eqref{FormulaForZetaTildeWithZetaEqualzero}, also recall Eq.~\eqref{TollsteFormelFuerZetaTrafoFragezeichen}. We also have
\bas
&\mleft(\mleft( \Lambda \circ \mathrm{d}^{\nabla^{\mathrm{bas}}} \circ \Lambda^{-1} \mright)\mleft(
	\lambda\mleft( t_{\nabla^{\mathrm{bas}}_\lambda} \mright)
	- R_\lambda
\mright)\mright) \mleft(X, Y, \nu\mright)
\\
&=
\mleft(\Lambda \circ \nabla^{\mathrm{bas}}_\nu \circ \Lambda^{-1}\mright)\mleft(
	\lambda\Bigl(
		\nabla^{\mathrm{bas}}_{\lambda(X)} Y
		- \nabla^{\mathrm{bas}}_{\lambda(Y)} X
	\Bigr)
	- \mleft[ \lambda(X), \lambda(Y) \mright]_E
\mright)
\\
&\hspace{1cm}
	- \lambda\mleft(
		\nabla^{\mathrm{bas}}_{\lambda\mleft(\nabla^{\mathrm{bas}}_\nu X\mright)} Y
		- \nabla^{\mathrm{bas}}_{\lambda(Y)} \nabla^{\mathrm{bas}}_\nu X
		%}_{= [(\rho\circ\lambda)(Y), X] + \rho\mleft( \nabla_X ( \lambda(Y) ) \mright)}
	\mright)
	+ \mleft[ \lambda\mleft(\nabla^{\mathrm{bas}}_\nu X\mright), \lambda(Y) \mright]_E
\\
&\hspace{1cm}
	- \lambda\mleft(
		\nabla^{\mathrm{bas}}_{\lambda(X)} \nabla^{\mathrm{bas}}_\nu Y
		- \nabla^{\mathrm{bas}}_{\lambda\mleft(\nabla^{\mathrm{bas}}_\nu Y\mright)} X
		%}_{= [(\rho\circ\lambda)(Y), X] + \rho\mleft( \nabla_X ( \lambda(Y) ) \mright)}
	\mright)
	+ \mleft[ \lambda(X), \lambda\mleft(\nabla^{\mathrm{bas}}_\nu Y\mright) \mright]_E.
\eas
%
%For the curvature $R_{\widetilde{\nabla}^\lambda}$ we instead calculate $R_{\widetilde{\nabla}^\lambda}\mleft( \widehat{\Lambda}(X), \widehat{\Lambda}(Y) \mright) \nu$, because then we can then use Eq.~\eqref{KuerzesteFormelForRedefOfNabla}

Now let us start to calculate the left hand side given by $R_{\widetilde{\nabla}^\lambda}$, using the second equation in Prop.~\ref{prop:ChangeofCurvaturesUnderCHangesOfConnections}, especially we need to calculate
\bas
\mathrm{d}^\nabla\mleft( \mleft(\Lambda \circ \mathrm{d}^{\nabla^{\mathrm{bas}}} \circ \Lambda^{-1} \mright) \lambda \mright),
\eas
and for this we want to use Cor.~\ref{cor:commutationS=0}. Using the commutator of operators, we see
\bas
\mleft[ \mathrm{d}^\nabla, \Lambda \circ \mathrm{d}^{\nabla^{\mathrm{bas}}} \circ \Lambda^{-1} \mright]
&=
\mleft[ \mathrm{d}^\nabla, \Lambda \mright] \circ \mathrm{d}^{\nabla^{\mathrm{bas}}} \circ \Lambda^{-1}
	+ \Lambda \circ \mleft[ \mathrm{d}^\nabla, \mathrm{d}^{\nabla^{\mathrm{bas}}} \mright] \circ \Lambda^{-1}
	+ \Lambda \circ \mathrm{d}^{\nabla^{\mathrm{bas}}} \circ \mleft[ \mathrm{d}^\nabla, \Lambda^{-1} \mright],
\eas
with that we can write
\bas
\mathrm{d}^\nabla\mleft( \mleft(\Lambda \circ \mathrm{d}^{\nabla^{\mathrm{bas}}} \circ \Lambda^{-1} \mright) \lambda \mright)
&=
\mleft[ \mathrm{d}^\nabla, \Lambda \circ \mathrm{d}^{\nabla^{\mathrm{bas}}} \circ \Lambda^{-1} \mright](\lambda)
	+ \mleft( \Lambda \circ \mathrm{d}^{\nabla^{\mathrm{bas}}} \circ \Lambda^{-1} \mright)
	\mleft(\mathrm{d}^\nabla \lambda\mright).
\eas
One needs to calculate the first summand, the summand in the middle in the formula of $\mleft[ \mathrm{d}^\nabla, \Lambda \circ \mathrm{d}^{\nabla^{\mathrm{bas}}} \circ \Lambda^{-1} \mright]$ is given by Cor.~\ref{cor:commutationS=0} due to the vanishing basic curvature of $\nabla$, so,
\bas
\mleft[ \mathrm{d}^\nabla, \mathrm{d}^{\nabla^{\mathrm{bas}}} \mright]\mleft( \Lambda^{-1} \circ \lambda \mright)(X,Y, \nu)
&=
R_\nabla\mleft(X, \mleft(\rho \circ \Lambda^{-1} \circ \lambda\mright)(Y)\mright)\nu
	- R_\nabla\mleft(Y, \mleft(\rho \circ \Lambda^{-1} \circ \lambda\mright)(X)\mright)\nu
\\
&\hspace{1cm}
	- \mleft(\Lambda^{-1} \circ \lambda \circ \rho \mright)\bigl( R_\nabla(X,Y)\nu \bigr)
\eas
for all $X, Y \in \mathfrak{X}(N)$ and $\nu \in \Gamma(E)$,
and
\bas
\mleft[ \mathrm{d}^\nabla, \Lambda \mright]
&=
\mleft[ \mathrm{d}^\nabla, \mathds{1}_E - \lambda \circ \rho \mright]
=
- \mleft[ \mathrm{d}^\nabla, \lambda \circ \rho \mright],
\eas
and for the last summand in the second equation of Prop.~\ref{prop:ChangeofCurvaturesUnderCHangesOfConnections} we have, also recall Remark \ref{Wedgies} and Eq.~\eqref{KuerzesteFormelForRedefOfNabla},
\bas
\mleft[ I\mleft(\widehat{\Lambda}(X), \cdot\mright), I\mleft(\widehat{\Lambda}(Y), \cdot\mright) \mright]_{\mathcal{D}(E)}(\nu)
&=
\nabla^{\mathrm{bas}}_{\nabla^{\mathrm{bas}}_{\nu}(\lambda(Y)) - \lambda\mleft( \nabla^{\mathrm{bas}}_\nu Y \mright)}\bigl(\lambda(X)\bigr)
%\\
%&\hspace{1cm}
	-\lambda\mleft(
		\nabla^{\mathrm{bas}}_{\nabla^{\mathrm{bas}}_{\nu}(\lambda(Y)) - \lambda\mleft( \nabla^{\mathrm{bas}}_\nu Y \mright)} X
	\mright)
\\
&\hspace{1cm}
	- (Y \leftrightarrow X \text{ of all previous lines}).
\eas
Now the purely tedious but straightforward part comes. Insert $X, Y, \nu$ everywhere\footnote{In general use $\widehat{\Lambda}(X)$ instead of $X$, similar for $Y$, as we did at the beginning and at the end, then it will be easier to compare the terms since a lot of $\Lambda$ will get canceled.} and the definition of the basic connection on both sides of the desired equation; although you may already recognize some similar terms of the calculation of the right hand side at the beginning, for those terms one does not need to insert the definition of the basic connection. Also make heavily use of Prop.~\ref{prop:PropsOfBigLambdas}, and also directly use the vanishing of the basic curvature on the right hand side (which implies flatness of the basic connection). We already got three curvature terms, and there is one additional by Prop.~\ref{prop:ChangeofCurvaturesUnderCHangesOfConnections}; there is actually one missing, but that term will be produced by the other remaining terms, for example by some of the form "$\nabla^{\mathrm{bas}}_{\nabla^{\mathrm{bas}}}$".

$\bullet$ As a proof of concept, you can also look at \cite[proof of Theorem 3.6, the first equation for the transformed curvature there]{My1stpaper} where I have calculated this for Lie algebra bundles; the structure of the calculation there is, abstractly-spoken, the same, but extremely shorter and less tedious due to a vanishing anchor. However, we will actually not need this theorem for the gauge invariance of the transformed Lagrangian as we are going to see, and we will argue later why the gauge invariance of the Lagrangian in general proves this theorem, too, avoiding the tedious calculation.
\end{proof}

%and observe that for all $f \in \Omega^1(E;E)$ that
%\bas
%\mleft[ \mathrm{d}^\nabla, f \mright]
%&=
%\mathrm{d}^\nabla f \wedge,
%\eas
%that is for all $\omega \in \Omega^{1, q}(N,E;E)$ ($q \in \mathbb{N}_0$)
%\bas
%\mleft(\mathrm{d}^\nabla f \wedge \omega \mright)(X, Y, \nu_1, \dotsc, \nu_q)
%&=
%\mleft(\mathrm{d}^{\nabla}f \mright) \bigl(X, \omega(Y, \nu_1, \dotsc, \nu_q)\bigr)
	%- \mleft(\mathrm{d}^{\nabla}f \mright) \bigl(Y, \omega(X, \nu_1, \dotsc, \nu_q)\bigr)
%\\
%&=
%\nabla_X\bigl( (f \circ \omega)(Y, \nu_1, \dotsc, \nu_q) \bigr)
	%- f\Bigl( \nabla_X \bigl( \omega(Y, \nu_1, \dotsc, \nu_q) \bigr) \Bigr)
%\eas
%for all $X, Y \in \mathfrak{X}(N)$ and $\nu_1, \dotsc, \nu_q \in \Gamma(E)$.

%\bas
%\widetilde{\nabla}^\lambda_{\widehat{\Lambda}(X)} \widetilde{\nabla}^\lambda_{\widehat{\Lambda}(X)} \nu
%&=
%\nabla_{\widehat{\Lambda}(X)} \nabla_{\widehat{\Lambda}(Y)} \nu
	%+ \nabla_{\widehat{\Lambda}(X)} \mleft( \mleft(\mathrm{d}^{\nabla^{\mathrm{bas}}} \lambda \mright)\mleft(Y, \nu\mright) \mright)
%\\
%&\hspace{1cm}
	%+ \mleft(\mathrm{d}^{\nabla^{\mathrm{bas}}} \lambda \mright)
		%\mleft( X , \nabla_{\widehat{\Lambda}(Y)} \nu + \mleft(\mathrm{d}^{\nabla^{\mathrm{bas}}} \lambda \mright)\mleft(Y, \nu\mright) \mright)
%\eas
Therefore we see that the curvature is not necessarily flat after a field redefinition. We have seen that the other remaining compatibility conditions are still satisfied, but what about infinitesimal gauge invariance when flatness is gone? Eq.~\eqref{FieldRedefOfClassicF} shows us that we get an offset in the field strength, which one may want to correct for preserving gauge invariance and the Lagrangian itself, and Thm.~\ref{thm:BrokenFlatness} motivates that the derivative of this offset using a basic connection has something to do with the curvature of $\nabla$ such that there is hope that the offset compensates the curvature, leading to a gauge invariant theory with a non-flat connection! Let us prove this.

\begin{theorems}{Infinitesimal gauge transformation after field redefinition}{FieldRedefOfGaugeTrafo}
Let $M, N$ be smooth manifolds, $E \to N$ a Lie algebroid, and $\nabla$ a connection on $E$. Also let $\lambda \in \Omega^1(N; E)$ such that $\Lambda = \mathds{1}_E - \lambda \circ \rho$ is an element of $\sAut(E)$. Then
\ba
\widetilde{\delta}_\varepsilon^\lambda
&=
{}^*\Lambda \circ \delta_\varepsilon \circ {}^*\mleft(\Lambda^{-1}\mright)
\ea
on $E$ and
\ba
\widetilde{\delta}_\varepsilon^\lambda
&=
{}^*\widehat{\Lambda} \circ \delta_\varepsilon \circ {}^*\mleft(\widehat{\Lambda}^{-1}\mright)
\ea
on $\mathrm{T}N$
for all $\varepsilon \in \mathcal{F}^0_E(M; {}^*E)$, where $\widetilde{\delta}_\varepsilon^\lambda$ is similarly defined to $\delta_\varepsilon$ but using $\widetilde{\nabla}^\lambda$ instead of $\nabla$ and $\widetilde{\varpi_2}^\lambda$ instead of $\varpi_2$ in Def.~\ref{def:TotalInfGaugeTrafoYayy}.\footnote{$\varpi_2$ was needed for fixing the vector fields like $\Psi_\varepsilon \in \mathfrak{X}^E\bigl( \mathfrak{M}_E(M;N) \bigr)$ by Prop.~\ref{prop:VariationOfA}.} Moreover, on scalar-valued functionals we have
\ba
\widetilde{\delta}_\varepsilon^\lambda
&=
\mathcal{L}_{\Psi_\varepsilon}
=
\delta_\varepsilon,
\ea
where $\Psi_\varepsilon \in \mathfrak{X}^E(\mathfrak{M}_E(M; N))$ is the vector field behind the definition of $\delta_\varepsilon$, recall Def.~\ref{def:TotalInfGaugeTrafoYayy}.
\end{theorems}

\begin{remark}
\leavevmode\newline
Observe how $\Psi_\varepsilon$ is unaffected by the field redefinition although $\varpi_2$ and $\nabla$ transform by the field redefinition, both of which were essential in the construction of infinitesimal gauge transformations.
\end{remark}

\begin{proof}[Proof of Thm.~\ref{thm:FieldRedefOfGaugeTrafo}]
\leavevmode\newline
We will prove this by using the uniqueness behind the construction of operators like $\delta_\varepsilon$, especially recall Prop.~\ref{prop:VariationVonSkalarZeugsEasyPeasy} and \ref{prop:VariationOfA}. We write 
\bas
\delta_\varepsilon^\prime
&\coloneqq
{}^*\Lambda \circ \delta_\varepsilon \circ {}^*\mleft(\Lambda^{-1}\mright)
\eas
and first observe that
\bas
\delta_\varepsilon^\prime ({}^*\nu)
&=
{}^*\Lambda \biggl(
	\delta_\varepsilon\mleft( {}^*\mleft( \Lambda^{-1}(\nu) \mright) \mright)
\biggr)
=
- {}^*\Lambda \mleft(
	{}^*\mleft( \nabla^{\mathrm{bas}}_\varepsilon \mleft( \Lambda^{-1} (\nu) \mright) \mright)
\mright)
\stackrel{ \text{Eq.~\eqref{basicconnectionTrafoRefield}} }{=}
- {}^*\mleft(
	\mleft(\widetilde{\nabla}^{\lambda}\mright)^{\mathrm{bas}}_\varepsilon \nu
\mright)
\eas
for all $\nu \in \Gamma(E)$. Hence, it shares this property with $\widetilde{\delta}_\varepsilon^\lambda$, $\delta_\varepsilon^\prime$ is also clearly $\mathbb{R}$-linear and satisfies Eq.~\eqref{VertauschungMitVerjuengungVonEichtrafo}. In order to use the uniqueness of Prop.~\ref{prop:VariationVonSkalarZeugsEasyPeasy} we need to check the Leibniz rule \eqref{LeibnizForGauging}. $\delta_\varepsilon^\prime$ certainly satisfies the Leibniz rule by
\bas
\delta_\varepsilon^\prime (f ~ L)
&=
{}^*\Lambda\mleft(
	\delta_\varepsilon \mleft(
		f ~ \mleft({}^*\Lambda^{-1}\mright)(L)
	\mright)
\mright)
\\
&=
{}^*\Lambda\mleft(
	f ~ \delta_\varepsilon \biggl(
		\mleft({}^*\mleft(\Lambda^{-1}\mright)\mright)(L)
	\biggr)
	+ \mathcal{L}_{\Psi_\varepsilon}(f) ~ \mleft({}^*\Lambda^{-1}\mright)(L)
\mright)
\\
&=
f ~ \delta_\varepsilon^\prime L
	+ \mathcal{L}_{\Psi_\varepsilon} (f) ~ L
\eas
for all $L\in \mathcal{F}^\bullet_E(M; {}^*E)$ and $f \in C^\infty\bigl( M \times \mathfrak{M}_E(M;N) \bigr)$. Therefore $\delta_\varepsilon^\prime$ is of the type of operator as in Prop.~\ref{prop:VariationVonSkalarZeugsEasyPeasy}, it even uses precisely the same vector field $\Psi_\varepsilon$. So, we only need to check whether $\Psi_\varepsilon$ is the same vector field as the one behind the definition of $\widetilde{\delta}^\lambda_\varepsilon$.

For this let us use the uniqueness given in the Prop.~\ref{prop:VariationOfA}, there it was about the uniqueness of vector fields like $\Psi_\varepsilon \in \mathfrak{X}^E(\mathfrak{M}_E(M; N))$ behind the Leibniz rule. The component along the direction of the Higgs field is of course always $({}^*\rho)(\varepsilon)$ by definition. Hence, we only need to check the second component fixed by Eq.~\eqref{EichtrafoVonANochmal}. So, using Prop.~\ref{prop:VariationOfA} for $\delta_\varepsilon$,
\bas
\delta_\varepsilon^\prime \widetilde{\varpi_2}^\lambda\quad~
&\stackrel{\mathclap{ \text{Def.~\eqref{EqFieldRedefFuerA}} }}{=}\quad~
\delta_\varepsilon^\prime \mleft(
	\mleft( {}^* \Lambda \mright) (\varpi_2)+ {}^! \lambda
\mright)
\\
&\stackrel{\mathclap{ \text{Eq.~\eqref{EqPullBackFormelFuerVerschiedeneDefinitionen}} }}{=}\quad
\mleft( {}^*\Lambda \circ \delta_\varepsilon \circ {}^*\mleft(\Lambda^{-1}\mright) \mright) \bigl(
	\mleft( {}^* \Lambda \mright) (\varpi_2)+ ({}^* \lambda)(\mathrm{D})
\bigr)
\\
&=
{}^*\Lambda\biggl(
	\delta_\varepsilon \varpi_2
	+ \delta_\varepsilon \biggl( \mleft( {}^* \mleft( \Lambda^{-1} \circ \lambda \mright) \mright) (\mathrm{D}) \biggr)
\biggr)
\\
&\stackrel{\mathclap{ \text{Eq.~\eqref{DPhiVariation}} }}{=}\quad~
{}^*\Lambda\Biggl(
	- ({}^*\nabla) \varepsilon
	- \mleft( {}^* \biggl( \nabla^{\mathrm{bas}}_\varepsilon \mleft(\Lambda^{-1} \circ \lambda \mright) \biggr) \mright) (\mathrm{D})
	- \mleft( {}^* \mleft( \Lambda^{-1} \circ \lambda \mright) \mright) \bigl( ({}^*\rho)\bigl(({}^*\nabla) \varepsilon\bigr) \bigr)
\Biggr)
\\
&=
\underbrace{- \mleft({}^*\Lambda\mright)\bigl(({}^*\nabla)\varepsilon\bigr)
	- \bigl( {}^*(\lambda \circ \rho) \bigr)\bigl( ({}^*\nabla) \varepsilon \bigr)}
	_{= - ({}^*\nabla) \varepsilon}
	- \underbrace{\mleft( {}^*\biggl( \Lambda\mleft( \nabla^{\mathrm{bas}}_\varepsilon \mleft(\Lambda^{-1} \circ \lambda \mright) \mright)\biggr)\mright) (\mathrm{D})}
	_{\mathclap{ \stackrel{\text{Eq.~\eqref{EqPullBackFormelFuerVerschiedeneDefinitionen} }}{=} {}^! \mleft(\Lambda\mleft( \nabla^{\mathrm{bas}}_\varepsilon \mleft(\Lambda^{-1} \circ \lambda \mright) \mright) \mright) }}
\\
&\stackrel{\mathclap{ \text{Eq.~\eqref{OneofmanyformulasForTildeNabla}} }}{=}\quad~
- \mleft({}^*\mleft(
	\widetilde{\nabla}^\lambda 
\mright)\mright) \varepsilon
\eas
using that ${}^*\mleft( \nabla^\prime \mright) = {}^*\nabla + {}^!I$ for all other connections $\nabla^\prime = \nabla + I$, where $I \in \Omega^1(N; \mathrm{End}(E))$; this just follows by the definition of pullbacks of vector bundle connections. Hence, the vector field behind $\widetilde{\delta}^\lambda_\varepsilon$ is precisely the one of $\delta^\prime_\varepsilon$, that is, $\Psi_\varepsilon$, using the uniqueness of Prop.~\ref{prop:VariationOfA}.

Finally, we have shown everything what we need to use the uniqueness of Prop.~\ref{prop:VariationVonSkalarZeugsEasyPeasy}, hence,
\bas
\widetilde{\delta}^\lambda_\varepsilon
&=
\delta_\varepsilon^\prime.
\eas
Similarly one shows this for the one on $\mathrm{T}N$, and that $\widetilde{\delta}_\varepsilon^\lambda = \mathcal{L}_{\Psi_\varepsilon}$ on scalar-valued functionals we have already shown by observing that $\Psi_\varepsilon$ is behind the definition of $\widetilde{\delta}_\varepsilon^\lambda$; also recall Remark \ref{RemLeibnizeRegelaufProdukteWeshalbEConnectionNichtWichtigIst}.
\end{proof}

That leads to the following important statement.

\begin{theorems}{Still a gauge theory after field redefinition}{WeHaveGladlyStillAGaugeTheoryAfterTheFieldRedefinition}
Let $M$ be a spacetime with a spacetime metric $\eta$, $N$ a smooth manifold, $E \to N$ a Lie algebroid, $\nabla$ a connection on $E$, $\kappa$ and $g$ fibre metrics on $E$ and $\mathrm{T}N$, respectively. Also let $V \in C^\infty(N)$, assume that the compatibility conditions of Thm.~\ref{thm:GaugeInvariantStandardLagrangian} hold, and let $\lambda \in \Omega^1(N; E)$ such that $\Lambda = \mathds{1}_E - \lambda \circ \rho$ is an element of $\sAut(E)$. Then we have
\ba
	R_{\widetilde{\nabla}^\lambda}
&=
	- \mathrm{d}^{\mleft(\widetilde{\nabla}^\lambda\mright)^{\mathrm{bas}}} \widehat{\zeta}^\lambda, \\
	R_{\widetilde{\nabla}^\lambda}^{\mathrm{bas}} &= 0, \\
	\mleft(\widetilde{\nabla}^\lambda \mright)^{\mathrm{bas}} \widetilde{\kappa}^\lambda 
	&= 0, \\
	\mleft(\widetilde{\nabla}^\lambda \mright)^{\mathrm{bas}} \widetilde{g}^\lambda  
	&= 0, \\
	{}^*\mleft(\mathcal{L}_{({}^*\rho)(\varepsilon)} V\mright) &= 0
\ea
for all $\varepsilon \in \mathcal{F}^0_E(M; {}^*E)$. Then we have
\ba
\widetilde{\mathfrak{L}}^\lambda_{\mathrm{YMH}}
&=
\mathfrak{L}_{\mathrm{YMH}},
\ea
and
\ba
\widetilde{\delta}^\lambda_\varepsilon \widetilde{\mathfrak{L}}^\lambda_{\mathrm{YMH}}
&=
0
\ea
for all $\varepsilon \in \mathcal{F}^0_E(M; {}^*E)$, where
\ba
\widetilde{\mathfrak{L}}^\lambda_{\mathrm{YMH}}
&\coloneqq
- \frac{1}{2} \biggl( {}^*\mleft(\widetilde{\kappa}^\lambda\mright) \biggr)\mleft(\widetilde{G}^\lambda
\stackrel{\wedge}{,} *\mleft(\widetilde{G}^\lambda
 \mright)\mright)
	+ \biggl( {}^*\mleft(\widetilde{g}^\lambda\mright) \biggr)\mleft(\widetilde{\mathfrak{D}}^\lambda \stackrel{\wedge}{,} *\mleft( \widetilde{\mathfrak{D}}^\lambda \mright) \mright)
	- *({}^*V),
\ea
with
\ba\label{MaybeANewFieldStrength}
\widetilde{G}^\lambda
&\coloneqq
\widetilde{F}^\lambda 
	+ \frac{1}{2} \biggl({}^* \mleft(\widehat{\zeta}^\lambda\mright) \biggr) \mleft( \widetilde{\mathfrak{D}}^\lambda  \stackrel{\wedge}{,} \widetilde{\mathfrak{D}}^\lambda\mright)
\ea
and $\widetilde{F}^\lambda$, $\widehat{\zeta}^\lambda$ and $\widetilde{\mathfrak{D}}^\lambda$ are defined in Thm.~\ref{thm:FieldRedefofstandardFieldStrengthF}.
\end{theorems}

\begin{remark}
\leavevmode\newline
Recall our discussion about Cor.~\ref{cor:FlatnessVonEichtrafos}, where we mentioned that the vanishing basic curvature is essential.
\end{remark}

\begin{proof}[Proof of Thm.~\ref{thm:WeHaveGladlyStillAGaugeTheoryAfterTheFieldRedefinition}]
\leavevmode\newline
The first four equations we have proven by Thm.~\ref{thm:FieldRedefDerEinfacherenCompatibilities} and \ref{thm:BrokenFlatness}, for the first equation recall that the first compatibility condition in Thm.~\ref{thm:GaugeInvariantStandardLagrangian} imposes that $\nabla$ is flat, and the fifth equation is just the same compatibility condition as of Thm.~\ref{thm:GaugeInvariantStandardLagrangian}.

Using Thm.~\ref{thm:FieldRedefofstandardFieldStrengthF},
 %and \ref{thm:FieldRedefOfGaugeTrafo},
%\bas
%\widetilde{\delta}_\varepsilon^\lambda \widetilde{\mathfrak{D}}^\lambda
%&=
%\mleft({}^*\widehat{\Lambda} \circ \delta_\varepsilon \circ {}^*\mleft(\widehat{\Lambda}^{-1}\mright) \mright)
	%\mleft( \mleft( {}^* \widehat{\Lambda} \mright)\mleft(\mathfrak{D}\mright) \mright)
%=
%\mleft({}^*\widehat{\Lambda} \mright)\mleft(\delta_\varepsilon \mathfrak{D} \mright)
%\stackrel{\text{Prop.~\ref{prop:InfinitesimalGaugeTrafoOfMinimalCoupleSmiley}} }{=}
%0
%\eas
%and, additionally using the compatibility conditions, Prop.~\ref{prop:GaugeTrafosOfFieldStrengthAndMinimalCoupling} and Thm.~\ref{thm:BrokenFlatness},
%\bas
%\widetilde{\delta}_\varepsilon^\lambda \widetilde{F}^\lambda
%&=
%\mleft({}^*\Lambda \circ \delta_\varepsilon \circ {}^*\mleft(\Lambda^{-1}\mright) \mright)
%\mleft(
		%\mleft( {}^* \Lambda \mright) \mleft(
		%F
		%- \frac{1}{2} \mleft({}^* \xi \mright) \mleft( \mathfrak{D} \stackrel{\wedge}{,} \mathfrak{D} \mright)
	%\mright)
%\mright)
%\\
%&=
%\mleft( {}^*\Lambda \mright) \mleft(  
	%\frac{1}{2} \mleft({}^* \mleft( \nabla^{\mathrm{bas}}_\varepsilon \xi \mright) \mright) \mleft( \mathfrak{D} \stackrel{\wedge}{,} \mathfrak{D} \mright)
%\mright)
%\\
%&\stackrel{\mathclap{ \text{Cor.~\ref{cor:ConjugationOfDifferentialsAreShitty}} }}{=}\quad~
%\frac{1}{2} \mleft({}^* \mleft( \mleft(\widetilde{\nabla}^\lambda\mright)^{\mathrm{bas}}_\varepsilon \widehat{\zeta}^\lambda \mright) \mright) \mleft( \mathfrak{D} \stackrel{\wedge}{,} \mathfrak{D} \mright)
%\eas
%and observe
\ba\label{FieldRedefOfGWithZeroZeta}
\widetilde{G}^\lambda
&=
\widetilde{F}^\lambda 
	+ \frac{1}{2} \mleft({}^* \widehat{\zeta}^\lambda \mright) \mleft( \widetilde{\mathfrak{D}}^\lambda  \stackrel{\wedge}{,} \widetilde{\mathfrak{D}}^\lambda\mright)
=
\mleft( {}^* \Lambda \mright) \mleft(
		F
		- \frac{1}{2} \mleft({}^* \xi \mright) \mleft( \mathfrak{D} \stackrel{\wedge}{,} \mathfrak{D} 	\mright)
\mright)
	+ \frac{1}{2} \mleft({}^* \widehat{\zeta}^\lambda \mright) \mleft( \widetilde{\mathfrak{D}}^\lambda  \stackrel{\wedge}{,} \widetilde{\mathfrak{D}}^\lambda\mright)
=
\mleft( {}^* \Lambda \mright) (F),
\ea
where $\xi = \Lambda^{-1} \circ \widehat{\zeta}^\lambda \circ \mleft( \widehat{\Lambda}, \widehat{\Lambda} \mright)$. 
%Therefore, as for the minimal coupling, additionally using the compatibility conditions and Prop.~\ref{prop:GaugeTrafosOfFieldStrengthAndMinimalCoupling},
%\bas
%\widetilde{\delta}_\varepsilon^\lambda \widetilde{G}^\lambda
%=
%\mleft( {}^*\Lambda \mright)(\delta_\varepsilon F)
%=
%0.
%\eas
Thence, we immediately have by Def.~\ref{def:FieldRedefinition} and Thm.~\ref{thm:FieldRedefofstandardFieldStrengthF}
\bas
\widetilde{\mathfrak{L}}^\lambda_{\mathrm{YMH}}
&=
\mathfrak{L}_{\mathrm{YMH}},
\eas
and finally, by Thm.~\ref{thm:FieldRedefOfGaugeTrafo}, 
\bas
\widetilde{\delta}^\lambda_\varepsilon
&=
\delta_\varepsilon,
\eas
such that by Thm.~\ref{thm:GaugeInvariantStandardLagrangian}
\bas
\widetilde{\delta}^\lambda_\varepsilon \widetilde{\mathfrak{L}}^\lambda_{\mathrm{YMH}}
&=
\delta_\varepsilon \mathfrak{L}_{\mathrm{YMH}}
=
0.
\eas
\end{proof}

That theorem is a good starting point of formulating a new version of gauge theory allowing non-flat connections, especially because the physics stay the same due to the invariance of the Lagrangian under the field redefinition. Indeed, using theorems like Thm.~\ref{thm:FieldRedefOfGaugeTrafo} and \ref{thm:BrokenFlatness} we could have shown the gauge invariance of the adjusted and transformed Lagrangian similarly to Thm.~\ref{thm:GaugeInvariantStandardLagrangian}.

Let us now redefine gauge theory, using these results.

\section{Curved Yang-Mills-Higgs gauge theory}\label{SectionAboutCYMHGTs}

Let us first redefine the field strength adding the correction term in Eq.~\eqref{MaybeANewFieldStrength}.

\begin{definitions}{New field strength, \cite[Equation (14)]{CurvedYMH}}{FinallyIAmAtTheNewFieldStrength}
Let $M, N$ be smooth manifolds, $E \to N$ a Lie algebroid equipped with a connection $\nabla$ on $E$, and $\gls{1fZeta}\in \Omega^2(N;E)$, the \textbf{primitive of $\nabla$}. We define the \textbf{(generalized) field strength $\gls{G}$} as an element of $\mathcal{F}_E^2(M; {}^*E)$ by
\ba
G
&\coloneqq
F
	+ \frac{1}{2} ({}^*\zeta)\mleft( \mathfrak{D} \stackrel{\wedge}{,} \mathfrak{D} \mright).
\ea
\end{definitions}

Let us quickly state its infinitesimal gauge transformation.

\begin{corollaries}{Infinitesimal gauge transformation of the new field strength}{NewGaugeTrafoOfFieldStrengthG}
Let $M, N$ be smooth manifolds, $E \to N$ a Lie algebroid equipped with a connection $\nabla$ on $E$, and $\zeta \in \Omega^2(N;E)$. Then
\ba
\delta_\varepsilon G
&=
- \Biggl(
	\frac{1}{2} ~\biggl( 
		\mleft(	{}^* R_{\nabla} \mright)\mleft( \mathfrak{D} \stackrel{\wedge}{,} \mathfrak{D} \mright) \varepsilon
		+ \mleft({}^* \mleft( \nabla^{\mathrm{bas}}_\varepsilon\zeta \mright) \mright)\mleft( \mathfrak{D} \stackrel{\wedge}{,} \mathfrak{D} \mright)
	\biggr)
	+ \mleft({}^* R_\nabla^{\mathrm{bas}} \mright) \mleft(\varepsilon \stackrel{\wedge}{,} \varpi_2  \stackrel{\wedge}{,} \mathrm{D} \mright)
\Biggr)
\ea
for all $\varepsilon \in \mathcal{F}^0_E(M; {}^*E)$.
\end{corollaries}

\begin{remark}
\leavevmode\newline
That is a generalized version of \cite[Equation (15)]{CurvedYMH}.
\end{remark}

\begin{proof}
\leavevmode\newline
Observe, using Prop.~\ref{prop:InfinitesimalGaugeTrafoOfMinimalCoupleSmiley} and \ref{prop:VariationVonSkalarZeugsEasyPeasy},
\bas
\delta_\varepsilon\bigl(
	({}^*\zeta)\mleft( \mathfrak{D} \stackrel{\wedge}{,} \mathfrak{D} \mright)
\bigr)
&=
- \mleft({}^*\mleft( \nabla^{\mathrm{bas}}_\varepsilon \zeta\mright) \mright)\mleft( \mathfrak{D} \stackrel{\wedge}{,} \mathfrak{D} \mright),
\eas
such that the statement follows by Prop.~\ref{prop:GaugeTrafosOfFieldStrengthAndMinimalCoupling}.
\end{proof}

Now towards the Lagrangian.

\begin{definitions}{Curved Yang-Mills-Higgs Lagrangian, \newline \cite[Eq.~(2) and (16)]{CurvedYMH}}{NowReallyTheFinalLagrangian}
Let $M$ be a spacetime with a spacetime metric $\eta$, $N$ a smooth manifold, $E \to N$ a Lie algebroid, $\nabla$ a connection on $E$, $\zeta \in \Omega^2(N;E)$, and let $\kappa$ and $g$ be fibre metrics on $E$ and $\mathrm{T}N$, respectively. Also let $V \in C^\infty(N)$, which we still call the \textbf{potential of the Higgs field}. Then we define the \textbf{curved Yang-Mills-Higgs Lagrangian $\gls{LZYMH}$} as an element of $\mathcal{F}_E^{\mathrm{dim}(M)}(M)$ by
\ba
\mathfrak{L}_{\mathrm{CYMH}}
&\coloneqq
- \frac{1}{2} \mleft( {}^*\kappa \mright)\mleft(G \stackrel{\wedge}{,} *G\mright)
	+ \mleft( {}^*g \mright)\mleft(\mathfrak{D} \stackrel{\wedge}{,} *\mathfrak{D} \mright)
	- *({}^*V),
\ea
where $*$ is the Hodge star operator with respect to $\eta$.
\end{definitions}

The gauge invariance is immediate by the previous results.

\begin{theorems}{Infinitesimal gauge invariance of the curved Yang-Mills-Higgs Lagrangian}{FinallyTheGaugeInvarianceWeWant}
Let $M$ be a spacetime with a spacetime metric $\eta$, $N$ a smooth manifold, $E \to N$ a Lie algebroid, $\nabla$ a connection on $E$, $\zeta \in \Omega^2(N;E)$, $\kappa$ and $g$ fibre metrics on $E$ and $\mathrm{T}N$, respectively. Also let $V \in C^\infty(N)$ and assume that the following \textbf{compatibility conditions} hold:
\ba
	R_\nabla &= - \mathrm{d}^{\nabla^{\mathrm{bas}}} \zeta,\label{EqMyFormulationOfZetaCondition} \\
	R_\nabla^{\mathrm{bas}} &= 0, \label{VanishingBasicCurvComp} \\
	\nabla^{\mathrm{bas}} \kappa &= 0, \\
	\nabla^{\mathrm{bas}} g &= 0, \\
	{}^*\mleft(\mathcal{L}_{({}^*\rho)(\varepsilon)} V\mright) &= 0
\ea
for all $\varepsilon \in \mathcal{F}^0_E(M; {}^*E)$. Then we have
\ba
\delta_\varepsilon \mathfrak{L}_{\mathrm{CYMH}}
&=
0
\ea
for all $\varepsilon \in \mathcal{F}^0_E(M; {}^*E)$.
\end{theorems}

\begin{remarks}{}{CYMH}
We call a setup like this a \textbf{curved Yang-Mills-Higgs gauge theory}, short as \gls{CYMH}, or also \textbf{CYMH GT} for emphasizing the part with gauge theory.

We speak of that we have found a CYMH GT structure, if we were able to define $\nabla$, $\kappa$ and $g$ for $E \to N$ satisfying the first four compatibility conditions. The spacetime and the potential are not our focus and thoroughly discussed elsewhere, so, we always assume that these exist in a suitable way.
\end{remarks}

\begin{remark}
\leavevmode\newline
This is basically the essential statement of \cite[especially the discussion around Equation (16)]{CurvedYMH}, but Eq.~\eqref{EqMyFormulationOfZetaCondition} has there a different form, see \cite[Equation (13)]{CurvedYMH}. We have reformulated that equation, and this equation and the other compatibility conditions naturally arise if using the basic connection in the definition of the infinitesimal gauge transformation.

Eq.~\eqref{EqMyFormulationOfZetaCondition} means
\bas
R_\nabla(\cdot, \cdot)\nu
&=
- \nabla^{\mathrm{bas}}_\nu \zeta
\eas
for all $\nu \in \Gamma(E)$.
\end{remark}

\begin{proof}[Proof of Thm.~\ref{thm:FinallyTheGaugeInvarianceWeWant}]
\leavevmode\newline
By Eq.~\eqref{EqMyFormulationOfZetaCondition}, the vanishing of the basic curvature and Cor.~\ref{cor:NewGaugeTrafoOfFieldStrengthG} we immediately get
\bas
\delta_\varepsilon G
&=
0
\eas
for all $\varepsilon \in \mathcal{F}^0_E(M; {}^*E)$. Therefore the remaining part of the proof is precisely as in Thm.~\ref{thm:GaugeInvariantStandardLagrangian}.
\end{proof}

Finally, we now arrived at a covariantized formulation of gauge theory allowing non-flat $\nabla$. We can still apply Thm.~\ref{thm:ActionLieALgebroid}, so, a flat connection locally still applies the structure of an action Lie algebroid such that one may argue that flatness already implies a classical theory. However, $\zeta$ is not necessarily zero, it is then just constant with respect to the basic connection by compatibility condition \eqref{EqMyFormulationOfZetaCondition}; we will actually see some examples for this later. Hence, one cannot expect that the field strength looks as in the classical formulation if $\nabla$ is flat, and, so, we can only apply Thm.~\ref{thm:StandardEichtheorieStecktInDenBedingung} if both, $R_\nabla$ and $\zeta$ vanish. This motivates the following definitions.

\begin{definitions}{Classical gauge theory}{ClassicalGT}
Let us assume the same structure as in Thm.~\ref{thm:FinallyTheGaugeInvarianceWeWant}.
Then we say that we have a \textbf{pre-classical gauge theory}, if $\nabla$ is flat.

If we have additionally $\zeta = 0$, then we say that we have a \textbf{classical gauge theory}.
\end{definitions}

\begin{remark}
\leavevmode\newline
If we have a classical CYMH GT, then also a pre-classical one by compatibility condition \ref{EqMyFormulationOfZetaCondition}.
\end{remark}

However, we motivated $\zeta$ by the field redefinition; there might be of course a field redefinition making $\nabla$ flat and/or $\zeta$ zero. This is what we mainly study in the remaining part of this thesis. We have seen that we needed to add the part with $\zeta$ to the classical field strength $F$ after the field redefinition in order to get the same Lagrangian. That can be seen as that the "actual field redefinition" of $F$ was not just given by the field redefinition of $\varpi_2$ and $\nabla$; or, in other words, that means we need a field redefinition of $\zeta$, too, while $\zeta$ was zero in Thm.~\ref{thm:WeHaveGladlyStillAGaugeTheoryAfterTheFieldRedefinition} and $\widehat{\zeta}^\lambda$ was the field redefinition of $\zeta \equiv 0$.

\begin{definitions}{Field redefinition of the primitive}{FieldRedefinitionOfThePrimitive}
Let $E \to N$ a Lie algebroid over a smooth manifold $N$, $\nabla$ a connection on $E$, $\zeta \in \Omega^2(N;E)$, and $\lambda \in \Omega^1(N;E)$ such that $\Lambda = \mathds{1} - \lambda \circ \rho \in \sAut(E)$. Then we define the \textbf{field redefinition $\gls{1fZetaTilde}$ of $\zeta$} by
\ba
\widetilde{\zeta}^\lambda
&\coloneqq
\Lambda \circ \zeta \circ \mleft( \widehat{\Lambda}^{-1}, \widehat{\Lambda}^{-1} \mright)
	+ \widehat{\zeta}^\lambda,
\ea
where $\widehat{\zeta}^\lambda$ is given as in Thm.~\ref{thm:FieldRedefofstandardFieldStrengthF}, that is,
\bas
\widetilde{\zeta}^\lambda\mleft(\widehat{\Lambda}(X),\widehat{\Lambda}(Y)\mright)
&=
\Lambda\bigl(
	\zeta\mleft( X, Y \mright)
\bigr)
	- \mleft(\mathrm{d}^{\widetilde{\nabla}^\lambda} \lambda\mright)(X,Y)
	+ t_{\widetilde{\nabla}^\lambda_\rho}(\lambda(X), \lambda(Y))
\\
&=
\Lambda\bigl(
	\zeta\mleft( X, Y \mright)
\bigr)
	- \mleft( \mathrm{d}^\nabla \lambda \mright)(X,Y)
%}_{\mathclap{= \nabla_X \bigl( \lambda(Y) \bigr) - \nabla_Y \bigl( \lambda(X) \bigr) - \lambda\bigl( [X,Y] \bigr)}}
	- \lambda\Bigl(
		\nabla^{\mathrm{bas}}_{\lambda(X)} Y
		- \nabla^{\mathrm{bas}}_{\lambda(Y)} X
		%}_{= [(\rho\circ\lambda)(Y), X] + \rho\mleft( \nabla_X ( \lambda(Y) ) \mright)}
	\Bigr)
	+ \mleft[ \lambda(X), \lambda(Y) \mright]_E
\eas
for all $X, Y \in \mathfrak{X}(N)$.
\end{definitions}

\begin{remarks}{Field redefinition of CYMH GTs}{FieldRedefOfFullCYMHGT}
The field redefinition is therefore given by using Def.~\ref{def:FieldRedefinition} and \ref{def:FieldRedefinitionOfThePrimitive} altogether, so, when we speak of the field redefinition of anything else besides the quantities in these definitions, then it is just canonically given; for example the field redefinition of something depending on $\zeta$ is then the same definition but replacing $\zeta$ with $\widetilde{\zeta}^\lambda$; similarly for dependencies on $\nabla$, $\varpi_2$ and the metrics $\kappa$ on $E$ and $g$ on $\mathrm{T}N$ as we already did before. We call this procedure the \textbf{field redefinition of a CYMH GT} on a given spacetime $M$, a smooth manifold $N$ and Lie algebroid $E \to N$. We are going to show that the Lagrangian stays invariant under the field redefinition and that this describes an equivalence relation of CYMH GTs on given $M, N$ and $E$.
\end{remarks}

For the invariance of the Lagrangian we do not need to prove everything again, we just need to check the field redefinition of the field strength $G$ and whether compatibility condition \eqref{EqMyFormulationOfZetaCondition} stays form-invariant.

\begin{lemmata}{Field redefinition of the new field strength and compatibility condition}{FinallyNiceTrafoOfEverything}
Let $M, N$ be smooth manifolds, $E \to N$ a Lie algebroid, $\nabla$ a connection on $E$, $\zeta \in \Omega^2(N;E)$, and $\lambda \in \Omega^1(N;E)$ such that $\Lambda = \mathds{1} - \lambda \circ \rho \in \sAut(E)$. Then we have
\ba
\widetilde{G}^\lambda
&=
\mleft( {}^*\Lambda \mright)(G),
\ea
where
\ba
\widetilde{G}^\lambda
&\coloneqq
\widetilde{F}^\lambda
	+ \frac{1}{2} \biggl({}^*\mleft(\widetilde{\zeta}^\lambda\mright)\biggr)\mleft( \widetilde{\mathfrak{D}}^\lambda \stackrel{\wedge}{,} \widetilde{\mathfrak{D}}^\lambda \mright),
\ea
for which $\widetilde{F}^\lambda$ and $\widetilde{\mathfrak{D}}^\lambda$ are given by Thm.~\ref{thm:FieldRedefofstandardFieldStrengthF}.

If the basic curvature of $\nabla$ vanishes additionally and satisfies $R_\nabla = - \mathrm{d}^{\nabla^{\mathrm{bas}}} \zeta$, then we have
\ba
R_{\widetilde{\nabla}^\lambda}
&= 
- \mathrm{d}^{\mleft(\widetilde{\nabla}^\lambda\mright)^{\mathrm{bas}}} \widetilde{\zeta}^\lambda.
\ea
\end{lemmata}

\begin{proof}
\leavevmode\newline
Those results are an immediate consequence of our calculations in the previous section, that is,
\bas
\widetilde{G}^\lambda
&=
\underbrace{\widetilde{F}^\lambda
	+ \frac{1}{2} \biggl({}^*\mleft(\widehat{\zeta}^\lambda\mright)\biggr)\mleft( \widetilde{\mathfrak{D}}^\lambda \stackrel{\wedge}{,} \widetilde{\mathfrak{D}}^\lambda \mright)}
	_{\stackrel{ \text{Eq.~\eqref{FieldRedefOfGWithZeroZeta}} }{=} ({}^*\Lambda)(F)}
	+ \frac{1}{2} \biggl({}^*\mleft( \Lambda \circ \zeta \circ \mleft( \widehat{\Lambda}^{-1}, \widehat{\Lambda}^{-1} \mright) \mright)\biggr)\mleft( \widetilde{\mathfrak{D}}^\lambda \stackrel{\wedge}{,} \widetilde{\mathfrak{D}}^\lambda \mright)
\\
&\stackrel{\mathclap{ \text{Thm.~\ref{thm:FieldRedefofstandardFieldStrengthF}} }}{=}\quad~
({}^*\Lambda)(F)
	+ \frac{1}{2} \bigl(
		{}^*\mleft( \Lambda \circ \zeta \mright)
	\bigr)\mleft( \mathfrak{D} \stackrel{\wedge}{,} \mathfrak{D} \mright)
\\
&=
({}^*\Lambda)\mleft(
	F
	+ \frac{1}{2} \mleft( {}^*\zeta \mright)\mleft( \mathfrak{D} \stackrel{\wedge}{,} \mathfrak{D} \mright)
\mright)
\\
&=
({}^*\Lambda)(G),
\eas
and, by Thm.~\ref{thm:BrokenFlatness} (for which we need the vanishing of the basic curvature), Prop.~\ref{prop:PropsOfBigLambdas} and compatibility condition \eqref{EqMyFormulationOfZetaCondition},
\bas
R_{\widetilde{\nabla}^\lambda}
&=
\Lambda \circ R_\nabla \circ \mleft( \widehat{\Lambda}^{-1}, \widehat{\Lambda}^{-1} \mright)
	- \mathrm{d}^{\mleft(\widetilde{\nabla}^\lambda\mright)^{\mathrm{bas}}} \widehat{\zeta}^\lambda
\\
&=
- \Lambda \circ \mathrm{d}^{\nabla^{\mathrm{bas}}} \zeta \circ \mleft( \widehat{\Lambda}^{-1}, \widehat{\Lambda}^{-1} \mright)
	- \mathrm{d}^{\mleft(\widetilde{\nabla}^\lambda\mright)^{\mathrm{bas}}} \widehat{\zeta}^\lambda
\\
&\stackrel{\mathclap{ \text{Cor.~\ref{cor:ConjugationOfDifferentialsAreShitty}} }}{=}\quad~
- \mathrm{d}^{\mleft( \widetilde{\nabla}^\lambda \mright)^{\mathrm{bas}}} \mleft(
	\Lambda \circ \zeta \circ \mleft( \widehat{\Lambda}^{-1}, \widehat{\Lambda}^{-1} \mright)
\mright)
	- \mathrm{d}^{\mleft(\widetilde{\nabla}^\lambda\mright)^{\mathrm{bas}}} \widehat{\zeta}^\lambda
\\
&=
- \mathrm{d}^{\mleft( \widetilde{\nabla}^\lambda \mright)^{\mathrm{bas}}} \mleft(
	\widetilde{\zeta}^\lambda
\mright).
\eas
\end{proof}

Hence, we immediately get:

\begin{theorems}{Gauge theory invariant under the field redefinition}{InvarianceUnderTheFieldRedefinition}
Let $M$ be a spacetime with a spacetime metric $\eta$, $N$ a smooth manifold, $E \to N$ a Lie algebroid, $\nabla$ a connection on $E$, $\zeta \in \Omega^2(N;E)$, $\kappa$ and $g$ fibre metrics on $E$ and $\mathrm{T}N$, respectively. Also let $V \in C^\infty(N)$, assume that the compatibility conditions of Thm.~\ref{thm:FinallyTheGaugeInvarianceWeWant} hold, and let $\lambda \in \Omega^1(N; E)$ such that $\Lambda = \mathds{1}_E - \lambda \circ \rho$ is an element of $\sAut(E)$. Then we have
\ba
	R_{\widetilde{\nabla}^\lambda}
&=
	- \mathrm{d}^{\mleft(\widetilde{\nabla}^\lambda\mright)^{\mathrm{bas}}} \widetilde{\zeta}^\lambda, \\
	R_{\widetilde{\nabla}^\lambda}^{\mathrm{bas}} &= 0, \\
	\mleft(\widetilde{\nabla}^\lambda \mright)^{\mathrm{bas}} \widetilde{\kappa}^\lambda 
	&= 0, \\
	\mleft(\widetilde{\nabla}^\lambda \mright)^{\mathrm{bas}} \widetilde{g}^\lambda  
	&= 0, \\
	{}^*\mleft(\mathcal{L}_{({}^*\rho)(\varepsilon)} V\mright) &= 0
\ea
for all $\varepsilon \in \mathcal{F}^0_E(M; {}^*E)$. Then we have
\ba
\widetilde{\mathfrak{L}}^\lambda_{\mathrm{CYMH}}
&=
\mathfrak{L}_{\mathrm{CYMH}},
\ea
and
\ba
\widetilde{\delta}^\lambda_\varepsilon \widetilde{\mathfrak{L}}^\lambda_{\mathrm{CYMH}}
&=
0
\ea
for all $\varepsilon \in \mathcal{F}^0_E(M; {}^*E)$, where
\ba
\widetilde{\mathfrak{L}}^\lambda_{\mathrm{CYMH}}
&\coloneqq
- \frac{1}{2} \biggl( {}^*\mleft(\widetilde{\kappa}^\lambda\mright) \biggr)\mleft(\widetilde{G}^\lambda
\stackrel{\wedge}{,} *\mleft(\widetilde{G}^\lambda
 \mright)\mright)
	+ \biggl( {}^*\mleft(\widetilde{g}^\lambda\mright) \biggr)\mleft(\widetilde{\mathfrak{D}}^\lambda \stackrel{\wedge}{,} *\mleft( \widetilde{\mathfrak{D}}^\lambda \mright) \mright)
	- *({}^*V),
\ea
and where $\widetilde{G}^\lambda$ is given as in Lemma \ref{lem:FinallyNiceTrafoOfEverything}, $\widetilde{\mathfrak{D}}^\lambda$ is defined as in Thm.~\ref{thm:FieldRedefofstandardFieldStrengthF} and $\widetilde{\delta}^\lambda_\varepsilon$ as in Thm~\ref{thm:FieldRedefOfGaugeTrafo}.
\end{theorems}

\begin{remark}
\leavevmode\newline
It is important to note for future proofs that the field redefinition already preserves the vanishing of the basic curvature if $\nabla$ has vanishing basic curvature, so, this is independent to whether or not the other compatibility conditions are satisfied. Similar for the metric compatibilities. However, for the invariance of compatibility condition \eqref{EqMyFormulationOfZetaCondition} one not only needs the condition itself but also additionally the vanishing of the basic curvature as stated in Lemma \ref{lem:FinallyNiceTrafoOfEverything}. We sometimes make use of this information when speaking about compatibility conditions in the context of the field redefinition. However, we will not necessarily mention it again; recall the previous calculations and proofs.
\end{remark}

\begin{proof}
\leavevmode\newline
This is precisely the same proof as in Thm.~\ref{thm:WeHaveGladlyStillAGaugeTheoryAfterTheFieldRedefinition}, using Lemma \ref{lem:FinallyNiceTrafoOfEverything} and $\widetilde{\zeta}^\lambda$ instead of just $\widehat{\zeta}^\lambda$.
\end{proof}

\begin{remarks}{Avoidance of the calculation in the proof of Thm.~\ref{thm:BrokenFlatness}}{HaesslicherBeweisUnwichtig}
As we have seen in the proofs for Thm.~\ref{thm:InvarianceUnderTheFieldRedefinition} and \ref{thm:WeHaveGladlyStillAGaugeTheoryAfterTheFieldRedefinition} we only needed Thm.~\ref{thm:BrokenFlatness} for the proof about the relationship of $R_\nabla$ with $\zeta$ after the field redefinition, everything else follows independent of Thm.~\ref{thm:BrokenFlatness}, especially the other compatibility conditions and the gauge invariance of the Lagrangian. Hence, one may want to argue, given the gauge invariance of the Lagrangian and the other compatibility conditions after the field redefinition, that the gauge transformation of the transformed field strength has to vanish, using similar calculations. By Cor.~\ref{cor:NewGaugeTrafoOfFieldStrengthG} one may then be able to argue in general that the compatibility condition of $\zeta$ has to be preserved by the field redefinition. However, for this one needs to discuss certain edge cases and that the contraction with $\kappa$ can be ignored (to avoid an argument about orthogonality). If one is able to argue like this, then one can avoid the tedious calculation behind the proof of Thm.~\ref{thm:BrokenFlatness}.
\end{remarks}

Therefore the field redefinition is now a transformation of the curved Yang-Mills-Higgs (infinitesimal) gauge theory which keeps the Lagrangian invariant. Furthermore, the field redefinition is an equivalence of CYMH GTs, which we now prove. We start with something similar to Lemma \ref{lem:FieldRedefinitionIsInvertible} but for the primitive.

\begin{lemmata}{Invertible behaviour of the field redefinition of the primitive}{InverseOfZetaLambda}
Let $E \to N$ a Lie algebroid over a smooth manifold $N$, $\nabla$ a connection on $E$, $\zeta \in \Omega^2(N;E)$, and $\lambda \in \Omega^1(N;E)$ such that $\Lambda = \mathds{1}_E - \lambda \circ \rho \in \sAut(E)$. Then
\ba
\overline{\zeta}^{-\lambda}
&=
\zeta,
\ea
where
\bas
\overline{\zeta}^{-\lambda}
&\coloneqq
\widetilde{ \widetilde{\zeta}^\lambda }^{- \Lambda^{-1} \circ \lambda}.
\eas
\end{lemmata}

\begin{proof}
\leavevmode\newline
That is similar to the proof of Lemma \ref{lem:FieldRedefinitionIsInvertible}, hence, let us summarize what we have derived there,
\bas
\mathfrak{\Lambda}
&\coloneqq
\mathds{1}_E
	- \mleft( - \Lambda^{-1} \circ \lambda \mright) \circ \rho
=
\Lambda^{-1},
\\
\widehat{\mathfrak{\Lambda}}
&\coloneqq
\mathds{1}_{\mathrm{T}N}
	- \rho \circ \mleft( - \Lambda^{-1} \circ \lambda \mright)
=
\widehat{\Lambda}^{-1},
\eas
those are invertible, thus, we can apply the field redefinition using $-\Lambda^{-1} \circ \lambda$. Then by Def.~\ref{def:FieldRedefinitionOfThePrimitive}, especially also recall Def.~\eqref{FormulaForZetaTildeWithZetaEqualzero},
\bas
\overline{\zeta}^{-\lambda}
&=
\mathfrak{\Lambda} \circ \widetilde{\zeta}^\lambda \circ \mleft( \widehat{\mathfrak{\Lambda}}^{-1}, \widehat{\mathfrak{\Lambda}}^{-1} \mright)
	+ \widehat{\widetilde{\zeta}^\lambda}^{-\Lambda^{-1} \circ \lambda},
\eas
where, recalling Eq.~\eqref{AndereFormelFuerNablaTrafoBesserFuerDasRechnen},
\bas
\mathfrak{\Lambda} \circ \widetilde{\zeta}^\lambda \circ \mleft( \widehat{\mathfrak{\Lambda}}^{-1}, \widehat{\mathfrak{\Lambda}}^{-1} \mright)
&=
\zeta
	+ \Lambda^{-1} \circ \widehat{\zeta}^\lambda \circ \mleft( \widehat{\Lambda}, \widehat{\Lambda} \mright)
\\
&=
\zeta
	- \Lambda^{-1} \circ \mleft(
		\mathrm{d}^{\widetilde{\nabla}^\lambda} \lambda
		- t_{\widetilde{\nabla}^\lambda_\rho} \circ (\lambda, \lambda)
		\mright)
\\
&\stackrel{\mathclap{ \eqref{eqDifferentialSplit} }}{=}~ 
\zeta
	- \Lambda^{-1} \circ \Bigl(
		\underbrace{\mathrm{d}^{\Lambda\circ\nabla\circ \Lambda^{-1}} \lambda}
		_{\mathclap{ = \mleft( \Lambda \circ \mathrm{d}^\nabla \circ \Lambda^{-1} \mright) \lambda }}
		+	D \wedge \lambda
		- t_{\widetilde{\nabla}^\lambda_\rho} \circ (\lambda, \lambda)
	\Bigr)
\\
&=
\zeta
	- \mathrm{d}^\nabla \mleft( \Lambda^{-1} \circ \lambda \mright)
\\
&\hspace{1cm}
	+ \mathrm{d}^\nabla \mleft( \Lambda^{-1} \circ \lambda \mright) \circ \mleft( \mathds{1}_{\mathrm{T}N}, \rho \circ \lambda \mright)
	+ \mathrm{d}^\nabla \mleft( \Lambda^{-1} \circ \lambda \mright) \circ \mleft( \rho \circ \lambda, \mathds{1}_{\mathrm{T}N} \mright)
\\
&\hspace{1cm}
	- t_{\nabla_\rho} \circ \mleft( \Lambda^{-1} \circ \lambda, \lambda \mright)
	- t_{\nabla_\rho} \circ \mleft( \lambda, \Lambda^{-1} \circ \lambda \mright)
	+ \Lambda^{-1} \circ t_{\widetilde{\nabla}^\lambda_\rho} \circ (\lambda, \lambda)
\eas
viewing $D \coloneqq - \mleft( \mathrm{d}^{\Lambda\circ\nabla\circ \Lambda^{-1}} \lambda \mright) \circ \mleft( \mathds{1}_{\mathrm{T}N}, \rho \mright) + \Lambda \circ t_{\nabla_\rho} \circ \mleft( \Lambda^{-1} \circ \lambda, \mathds{1}_E \mright)$ as an element of $\Omega^1(N;\mathrm{End}(E))$,
and, using Prop.~\ref{prop:PropsOfBigLambdas},
\bas
&\Bigl(- t_{\nabla_\rho} \circ \mleft( \Lambda^{-1} \circ \lambda, \lambda \mright)
	- t_{\nabla_\rho} \circ \mleft( \lambda, \Lambda^{-1} \circ \lambda \mright)
	+ \Lambda^{-1} \circ \underbrace{t_{\widetilde{\nabla}^\lambda_\rho} \circ (\lambda, \lambda)}
	_{\mathclap{ = - t_{\mleft(\widetilde{\nabla}^\lambda\mright)^{\mathrm{bas}}} \circ (\lambda, \lambda) }}
\Bigr)(X, Y)
\\
&\hspace{1cm}=
- \nabla_{(\rho \circ \lambda)(X)} \mleft( \mleft(\Lambda^{-1} \circ \lambda\mright)(Y) \mright)
	+ \nabla_{\mleft(\rho \circ \Lambda^{-1} \circ \lambda\mright)(Y)} \bigl( \lambda(X) \bigr)
	+ \mleft[ \lambda(X), \mleft(\Lambda^{-1} \circ \lambda\mright)(Y) \mright]_E
\\
&\hspace{2cm}
	+ \nabla_{(\rho \circ \lambda)(Y)} \mleft( \mleft(\Lambda^{-1} \circ \lambda\mright)(X) \mright)
	- \nabla_{\mleft(\rho \circ \Lambda^{-1} \circ \lambda\mright)(X)} \bigl( \lambda(Y) \bigr)
	+ \mleft[ \mleft(\Lambda^{-1} \circ \lambda\mright)(X), \lambda(Y) \mright]_E
\\
&\hspace{2cm}
	- \nabla^{\mathrm{bas}}_{\lambda(X)} \mleft( \mleft( \Lambda^{-1} \circ \lambda \mright) (Y) \mright)
	+ \nabla^{\mathrm{bas}}_{\lambda(Y)} \mleft( \mleft( \Lambda^{-1} \circ \lambda \mright) (Y) \mright)
	+ \Lambda^{-1} \mleft( \mleft[ \lambda(X), \lambda(Y) \mright]_E \mright)
\\
&\hspace{1cm}=
- \nabla_{(\rho \circ \lambda)(X)} \mleft( \mleft(\Lambda^{-1} \circ \lambda\mright)(Y) \mright)
	+ \nabla_{\mleft(\rho \circ \Lambda^{-1} \circ \lambda\mright)(Y)} \bigl( \lambda(X) \bigr)
	+ \mleft[ \lambda(X), \mleft(\Lambda^{-1} \circ \lambda\mright)(Y) \mright]_E
\\
&\hspace{2cm}
	+ \nabla_{(\rho \circ \lambda)(Y)} \mleft( \mleft(\Lambda^{-1} \circ \lambda\mright)(X) \mright)
	- \nabla_{\mleft(\rho \circ \Lambda^{-1} \circ \lambda\mright)(X)} \bigl( \lambda(Y) \bigr)
	+ \mleft[ \mleft(\Lambda^{-1} \circ \lambda\mright)(X), \lambda(Y) \mright]_E
\\
&\hspace{2cm}
	- \mleft[ \lambda(X), \mleft( \Lambda^{-1} \circ \lambda \mright) (Y) \mright]_E
	- \nabla_{\mleft(\rho \circ \Lambda^{-1} \circ \lambda \mright) (Y)} \bigl( \lambda(X) \bigr)
\\
&\hspace{2cm}
	+ \mleft[ \lambda(Y), \mleft( \Lambda^{-1} \circ \lambda \mright) (X) \mright]_E
	+ \nabla_{\mleft(\rho \circ \Lambda^{-1} \circ \lambda \mright) (X)} \bigl( \lambda(Y) \bigr)
\\
&\hspace{2cm}
	+ \Lambda^{-1} \mleft( \mleft[ \lambda(X), \lambda(Y) \mright]_E \mright)
\\
&\hspace{1cm}=
- \nabla_{(\rho \circ \lambda)(X)} \mleft( \mleft(\Lambda^{-1} \circ \lambda\mright)(Y) \mright)
	+ \nabla_{(\rho \circ \lambda)(Y)} \mleft( \mleft(\Lambda^{-1} \circ \lambda\mright)(X) \mright)
	+ \Lambda^{-1} \mleft( \mleft[ \lambda(X), \lambda(Y) \mright]_E \mright)
\eas
for all $X, Y \in \mathfrak{X}(N)$,
and, using additionally Lemma \ref{lem:FieldRedefinitionIsInvertible},
\bas
\widehat{\widetilde{\zeta}^\lambda}^{-\Lambda^{-1} \circ \lambda}
&\coloneqq
\mleft(\mathrm{d}^{\widehat{\nabla}^{-\lambda}} \mleft( \Lambda^{-1}\circ \lambda\mright)
	+ t_{\widehat{\nabla}^{-\lambda}_\rho} \circ (\Lambda^{-1}\circ \lambda, \Lambda^{-1}\circ \lambda)\mright)
	\circ \mleft(\widehat{\mathfrak{\Lambda}}^{-1}, \widehat{\mathfrak{\Lambda}}^{-1}\mright)
\\
&=
\mleft(\mathrm{d}^{\nabla} \mleft( \Lambda^{-1}\circ \lambda\mright)
	+ t_{\nabla_\rho} \circ \mleft(\Lambda^{-1}\circ \lambda, \Lambda^{-1}\circ \lambda\mright)\mright)
	\circ \mleft(\widehat{\Lambda}, \widehat{\Lambda}\mright)
\\
&=
\mathrm{d}^{\nabla} \mleft( \Lambda^{-1}\circ \lambda\mright) \circ \mleft(\widehat{\Lambda}, \widehat{\Lambda}\mright)
	+ t_{\nabla_\rho} \circ \mleft(\lambda, \lambda\mright)
\eas
Therefore altogether, using $\widehat{\Lambda} = \mathds{1}_{\mathrm{T}N} - \rho \circ \lambda$ and again Prop.~\ref{prop:PropsOfBigLambdas},
\bas
\overline{\zeta}^{-\lambda}(X, Y)
&=
\zeta(X, Y)
	+ \mathrm{d}^{\nabla} \mleft( \Lambda^{-1}\circ \lambda\mright) \bigl((\rho \circ \lambda)(X), (\rho\circ\lambda)(Y)\bigr)
	+ t_{\nabla_\rho} \bigl(\lambda(X), \lambda(Y)\bigr)
\\
&\hspace{1cm}
	- \nabla_{(\rho \circ \lambda)(X)} \mleft( \mleft(\Lambda^{-1} \circ \lambda\mright)(Y) \mright)
	+ \nabla_{(\rho \circ \lambda)(Y)} \mleft( \mleft(\Lambda^{-1} \circ \lambda\mright)(X) \mright)
	+ \Lambda^{-1} \mleft( \mleft[ \lambda(X), \lambda(Y) \mright]_E \mright)
\\
&=
\zeta(X, Y)
	+ \mathrm{d}^{\nabla} \mleft( \Lambda^{-1}\circ \lambda\mright) \bigl((\rho \circ \lambda)(X), (\rho\circ\lambda)(Y)\bigr)
\\
&\hspace{1cm}
	+ \nabla_{(\rho \circ \lambda)(X)}\bigl( \lambda(Y) \bigr)
	- \nabla_{(\rho \circ \lambda)(Y)}\bigl( \lambda(X) \bigr)
	- \mleft[ \lambda(X), \lambda(Y) \mright]_E
\\
&\hspace{1cm}
	- \nabla_{(\rho \circ \lambda)(X)} \mleft( \mleft(\Lambda^{-1} \circ \lambda\mright)(Y) \mright)
	+ \nabla_{(\rho \circ \lambda)(Y)} \mleft( \mleft(\Lambda^{-1} \circ \lambda\mright)(X) \mright)
	+ \Lambda^{-1} \mleft( \mleft[ \lambda(X), \lambda(Y) \mright]_E \mright)
\\
&=
\zeta(X, Y)
	+ \mathrm{d}^{\nabla} \mleft( \Lambda^{-1}\circ \lambda\mright) \bigl((\rho \circ \lambda)(X), (\rho\circ\lambda)(Y)\bigr)
\\
&\hspace{1cm}
	- \nabla_{(\rho \circ \lambda)(X)} \mleft( \mleft(\Lambda^{-1} \circ \lambda \circ \rho \circ \lambda\mright)(Y) \mright)
	+ \nabla_{(\rho \circ \lambda)(Y)} \mleft( \mleft(\Lambda^{-1} \circ \lambda \circ \rho \circ \lambda\mright)(X) \mright)
\\
&\hspace{1cm}
	+ \underbrace{\mleft(\Lambda^{-1} \circ \lambda \circ \rho \mright) \mleft( \mleft[ \lambda(X), \lambda(Y) \mright]_E \mright)}
	_{\mathclap{ = \mleft(\Lambda^{-1} \circ \lambda\mright) \mleft( \mleft[ (\rho\circ\lambda)(X), (\rho\circ\lambda)(Y) \mright]_E \mright) }}
\\
&=
\zeta(X, Y)
	+ \mathrm{d}^{\nabla} \mleft( \Lambda^{-1}\circ \lambda\mright) \bigl((\rho \circ \lambda)(X), (\rho\circ\lambda)(Y)\bigr)
\\
&\hspace{1cm}
	- \mathrm{d}^{\nabla} \mleft( \Lambda^{-1}\circ \lambda\mright) \bigl((\rho \circ \lambda)(X), (\rho\circ\lambda)(Y)\bigr)
\\
&=
\zeta(X, Y).
\eas
\end{proof}

The field redefinition, Def.~\ref{def:FieldRedefinition} and \ref{def:FieldRedefinitionOfThePrimitive}, is also transitive.

\begin{lemmata}{Transitivity of the field redefinition}{TransFieldRedef}
Let $M, N$ be smooth manifolds, $E \to N$ a Lie algebroid, $\nabla$ a connection on $E$, $\zeta \in \Omega^2(N;E)$, $\kappa$ and $g$ fibre metrics on $E$ and $\mathrm{T}N$, respectively. Moreover, let $\lambda, \lambda^\prime \in \Omega^1(N;E)$ such that $\Lambda = \mathds{1}_E - \lambda \circ \rho, \Lambda^\prime \coloneqq \mathds{1}_E - \lambda^\prime \circ \rho \in \sAut(E)$.

Then the field redefinition with $\lambda^\prime$ composed with the field redefinition of $\lambda$ is equivalent to a field redefinition with $\lambda + \lambda^\prime - \lambda^\prime \circ \rho \circ \lambda$.
\end{lemmata}

\begin{remark}
\leavevmode\newline
With this one can also quickly show Lemma \ref{lem:FieldRedefinitionIsInvertible} and \ref{lem:InverseOfZetaLambda} by defining $\lambda^\prime \coloneqq - \Lambda^{-1} \circ \lambda$ such that
\bas
\lambda + \lambda^\prime - \lambda^\prime \circ \rho \circ \lambda
&=
\lambda \underbrace{- \Lambda^{-1} \circ \lambda + \Lambda^{-1} \circ \lambda \circ \rho \circ \lambda}
_{= - \Lambda^{-1} \circ \Lambda \circ \lambda}
=
0,
\eas
which gives trivial transformations.
\end{remark}

\begin{proof}[Proof of Lemma \ref{lem:TransFieldRedef}]
\leavevmode\newline
First observe that
\bas
\Lambda^\prime \circ \Lambda
&=
\mleft( \mathds{1}_E - \lambda^\prime \circ \rho \mright)
\circ \mleft( \mathds{1}_E - \lambda \circ \rho \mright)
\\
&=
\mathds{1}_E 
	- \lambda \circ \rho 
	- \lambda^\prime \circ \rho 
	+ \lambda^\prime \circ \rho \circ \lambda \circ \rho
\\
&=
\mathds{1}_E 
	- \mleft( \lambda 
		+ \lambda^\prime 
		- \lambda^\prime \circ \rho \circ \lambda \mright) \circ \rho
\\
&\eqqcolon
\mathfrak{\Lambda}
\eas
so, $\lambda + \lambda^\prime - \lambda^\prime \circ \rho \circ \lambda$ is a valid element of $\Omega^1(N;E)$ with which one can apply the field redefinition due to the fact that $\Lambda^\prime \circ \Lambda \in \sAut(E)$, thence, also $\mathfrak{\Lambda}\in \sAut(E)$; we also define and calculate similarly
\bas
\widehat{\mathfrak{\Lambda}} 
&\coloneqq 
\widehat{\Lambda}^\prime \circ \widehat{\Lambda}
= 
\mathds{1}_{\mathrm{T}N} 
	- \rho \circ (\lambda + \lambda^\prime - \lambda^\prime \circ \rho \circ \lambda)
\eas
which is an element of $\sAut(\mathrm{T}N)$ (similarly to why $\widehat{\Lambda}$ is), where we denote $\widehat{\Lambda}^\prime \coloneqq \mathds{1}_{\mathrm{T}N} - \rho \circ \lambda^\prime$.
By Remark \ref{rem:FieldRedefOfFullCYMHGT} we only need to check the basic field redefinition of Def.~\ref{def:FieldRedefinition} and \ref{def:FieldRedefinitionOfThePrimitive}, so,
\bas
\widetilde{\widetilde{\varpi_2}^\lambda}^{\lambda^\prime}
&=
\mleft( {}^* \Lambda^\prime \mright)\Bigl(\mleft( {}^* \Lambda \mright) (\varpi_2)+ \underbrace{{}^! \lambda}_{\mathclap{ \stackrel{\eqref{EqPullBackFormelFuerVerschiedeneDefinitionen}}{=} ~ ({}^*\lambda)(\mathrm{D}) }}\Bigr)
	+ {}^!\lambda^\prime
\\
&=
\mleft( {}^*\Lambda^\prime \circ {}^* \Lambda \mright) (\varpi_2)
	+ \underbrace{\bigl({}^*\mleft(\Lambda^\prime \circ \lambda\mright)\bigr)(\mathrm{D})}
	_{= {}^!\mleft(\Lambda \circ \lambda\mright)}
	+ {}^!\lambda^\prime
\\
&=
({}^*\mathfrak{\Lambda})(\varpi_2)
	+ {}^!\mleft( \lambda + \lambda^\prime - \lambda^\prime \circ \rho \circ \lambda \mright).
\eas
For the metrics we immediately have
\bas
\widetilde{\widetilde{\kappa}^\lambda}^{\lambda^\prime}
&=
\kappa \circ \mleft( \Lambda^{-1}, \Lambda^{-1} \mright) \circ \mleft( (\Lambda^\prime)^{-1}, (\Lambda^\prime)^{-1} \mright)
=
\kappa \circ \mleft( \mathfrak{\Lambda}^{-1}, \mathfrak{\Lambda}^{-1} \mright),
\eas
similarly for $g$. Recall again Prop.~\ref{prop:PropsOfBigLambdas} and Cor.~\ref{cor:ConjugationOfDifferentialsAreShitty}, then
\bas
\widetilde{\widetilde{\nabla}^\lambda}^{\lambda^\prime}
&=
\widetilde{\nabla}^\lambda
	+ \mleft( \Lambda^\prime \circ \mathrm{d}^{\mleft(\widetilde{\nabla}^\lambda\mright)^{\mathrm{bas}}} \circ (\Lambda^\prime)^{-1} \mright) \lambda^\prime
\\
&=
\nabla
	+ \underbrace{\mleft( \Lambda \circ \mathrm{d}^{\nabla^{\mathrm{bas}}} \circ \Lambda^{-1} \mright) \lambda}
	_{\mathclap{ = \mleft( \Lambda^\prime \mright)^{-1} \circ \mleft(\mleft( \mathfrak{\Lambda} \circ \mathrm{d}^{\nabla^{\mathrm{bas}}} \circ \mathfrak{\Lambda}^{-1} \mright)\mleft( \Lambda^\prime \circ \lambda \mright)\mright) }}
	+ \Lambda^\prime \circ \Lambda \circ \mleft(
		\mleft(\mathrm{d}^{\nabla^{\mathrm{bas}}} \circ \Lambda^{-1} \circ (\Lambda^\prime)^{-1}\mright)
		\mleft(
			\lambda^\prime \circ \widehat{\Lambda}
		\mright)
	\mright) \circ \mleft( \widehat{\Lambda}^{-1}, \mathds{1}_E \mright)
\\
&=
\nabla
	+ \mleft( \mathfrak{\Lambda} \circ \mathrm{d}^{\nabla^{\mathrm{bas}}} \circ \mathfrak{\Lambda}^{-1} \mright)\underbrace{\mleft( \lambda + \lambda^\prime \circ \widehat{\Lambda} \mright)}
	_{= \lambda + \lambda^\prime - \lambda^\prime \circ \rho \circ \lambda}
\\
&\hspace{1cm}
	- \mleft( \mathfrak{\Lambda} \circ \mathrm{d}^{\nabla^{\mathrm{bas}}} \circ \mathfrak{\Lambda}^{-1} \mright)\mleft( \lambda^\prime \circ \rho \circ \lambda \mright)
\\
&\hspace{1cm}
	+ \mleft( \Lambda^\prime \mright)^{-1} \circ \lambda^\prime \circ \rho \circ \mleft(\mleft( \mathfrak{\Lambda} \circ \mathrm{d}^{\nabla^{\mathrm{bas}}} \circ \mathfrak{\Lambda}^{-1} \mright)\mleft( \Lambda^\prime \circ \lambda \mright)\mright)
\\
&\hspace{1cm}
	+ \mleft(\mleft( \mathfrak{\Lambda} \circ \mathrm{d}^{\nabla^{\mathrm{bas}}} \circ \mathfrak{\Lambda}^{-1} \mright)\mleft( \lambda^\prime \circ \widehat{\Lambda} \mright) \mright) \circ \mleft( \widehat{\Lambda}^{-1} \circ \rho \circ \lambda, \mathds{1}_E \mright)
\\
&=
\widetilde{\nabla}^{\lambda + \lambda^\prime - \lambda^\prime \circ \rho \circ \lambda}
\\
&\hspace{1cm}
	- \mleft( \mathfrak{\Lambda} \circ \mathrm{d}^{\nabla^{\mathrm{bas}}} \mright)\mleft( \mathfrak{\Lambda}^{-1} \circ \lambda^\prime \circ \rho \circ \lambda \mright)
\\
&\hspace{1cm}
	+ \lambda^\prime \circ \rho \circ \Lambda \circ \mathrm{d}^{\nabla^{\mathrm{bas}}} \mleft( \Lambda^{-1} \circ \lambda \mright)
\\
&\hspace{1cm}
	+ \mleft(\mleft( \mathfrak{\Lambda} \circ \mathrm{d}^{\nabla^{\mathrm{bas}}} \circ \mathfrak{\Lambda}^{-1} \mright)\mleft( \lambda^\prime \circ \widehat{\Lambda} \mright) \mright) \circ \mleft( \widehat{\Lambda}^{-1} \circ \rho \circ \lambda, \mathds{1}_E \mright)
\\
&=
\widetilde{\nabla}^{\lambda + \lambda^\prime - \lambda^\prime \circ \rho \circ \lambda}
\\
&\hspace{1cm}
	- \mathfrak{\Lambda} \circ \nabla^{\mathrm{bas}} \circ \mathfrak{\Lambda}^{-1} \circ \lambda^\prime \circ \rho \circ \lambda 
	+ \lambda^\prime \circ \rho \circ \lambda \circ \nabla^{\mathrm{bas}}
\\
&\hspace{1cm}
	+ \lambda^\prime \circ \rho \circ \Lambda \circ \nabla^{\mathrm{bas}} \circ \Lambda^{-1} \circ \lambda
	- \lambda^\prime \circ \rho \circ \Lambda \circ \Lambda^{-1} \circ \lambda \circ \nabla^{\mathrm{bas}}
\\
&\hspace{1cm}
	+ \mathfrak{\Lambda} \circ \nabla^{\mathrm{bas}} \circ \mathfrak{\Lambda}^{-1} \circ \lambda^\prime \circ \widehat{\Lambda} \circ \widehat{\Lambda}^{-1} \circ \rho \circ \lambda
	- \mathfrak{\Lambda} \circ \mathfrak{\Lambda}^{-1} \circ \lambda^\prime \circ \underbrace{\widehat{\Lambda} \circ \nabla^{\mathrm{bas}} \circ \widehat{\Lambda}^{-1} \circ \rho}
	_{\mathclap{ \stackrel{\text{Cor.~\ref{cor:ENablaMitRhoVertauschung}}}{=} \rho \circ \Lambda \circ \nabla^{\mathrm{bas}} \circ \Lambda^{-1} }}
	\circ \lambda
\\
&=
\widetilde{\nabla}^{\lambda + \lambda^\prime - \lambda^\prime \circ \rho \circ \lambda},
\eas
rewriting definitions like $\mathrm{d}^{\nabla^{\mathrm{bas}}} \lambda = \nabla^{\mathrm{bas}} \circ \lambda - \lambda \circ \nabla^{\mathrm{bas}}$, where the basic connection in the first summand is the one on $E$ and the one on $\mathrm{T}N$ in the second summand, \textit{i.e.}
\bas
\mleft(\mathrm{d}^{\nabla^{\mathrm{bas}}} \lambda\mright)(Y, \nu)
&= 
\nabla^{\mathrm{bas}}_\nu \bigl(\lambda(Y)\bigr) 
	- \lambda \mleft(\nabla^{\mathrm{bas}}_\nu Y \mright)
\eas
for all $\nu \in \Gamma(E)$ and $Y \in \mathfrak{X}(N)$. Finally let us look at the field redefinitions of $\zeta$, the calculation is very similar to the proof of Lemma \ref{lem:InverseOfZetaLambda}; the calculation is purely straightforward, just compare the definitions of $\widetilde{\widetilde{\zeta}^\lambda}^{\lambda^\prime}$ with $\widetilde{\zeta}^{\lambda + \lambda^\prime - \lambda^\prime \circ \rho \circ \lambda}$. However, it is very tedious and long, hence, we will omit the calculation; we are going to motivate it differently, using the field redefinition of the field strength provided in Lemma \ref{lem:FinallyNiceTrafoOfEverything}. That is,
\bas
\widetilde{G}^{\lambda + \lambda^\prime - \lambda^\prime \circ \rho \circ \lambda}
&=
\widetilde{F}^{\lambda + \lambda^\prime - \lambda^\prime \circ \rho \circ \lambda}
	+ \frac{1}{2} \biggl({}^*\mleft(\widetilde{\zeta}^{\lambda + \lambda^\prime - \lambda^\prime \circ \rho \circ \lambda}\mright)\biggr)\mleft( \widetilde{\mathfrak{D}}^{\lambda + \lambda^\prime - \lambda^\prime \circ \rho \circ \lambda} \stackrel{\wedge}{,} \widetilde{\mathfrak{D}}^{\lambda + \lambda^\prime - \lambda^\prime \circ \rho \circ \lambda} \mright),
\eas
but also Lemma \ref{lem:FinallyNiceTrafoOfEverything}
\bas
\widetilde{G}^{\lambda + \lambda^\prime - \lambda^\prime \circ \rho \circ \lambda}
&=
({}^*\mathfrak{\Lambda})(G)
=
\mleft( \mleft( {}^*\Lambda^\prime \mright) \circ ({}^*\Lambda) \mright) (G)
=
\mleft( {}^*\Lambda^\prime \mright)\mleft( \widetilde{G}^\lambda \mright)
=
\widetilde{\widetilde{G}^\lambda}^{\lambda^\prime}.
\eas
By the previous results we immediately get
\bas
\widetilde{F}^{\lambda+ \lambda^\prime - \lambda^\prime \circ \rho \circ \lambda}
&=
\widetilde{\widetilde{F}^\lambda}^{\lambda^\prime},
\eas
because $F$ is independent of $\zeta$. Similarly as for $G$ we get by Thm.~\ref{thm:FieldRedefofstandardFieldStrengthF}
\bas
\widetilde{\mathfrak{D}}^{\lambda + \lambda^\prime - \lambda^\prime \circ \rho \circ \lambda}
&=
\widetilde{\widetilde{\mathfrak{D}}^\lambda}^{\lambda^\prime}.
\eas
Then simply compare both sides in $\widetilde{G}^{\lambda + \lambda^\prime - \lambda^\prime \circ \rho \circ \lambda} = \widetilde{\widetilde{G}^\lambda}^{\lambda^\prime}$ to get
\bas
\mleft({}^*\mleft(\widetilde{\zeta}^{\lambda + \lambda^\prime - \lambda^\prime \circ \rho \circ \lambda} - \widetilde{\widetilde{\zeta}^\lambda}^{\lambda^\prime}\mright)\mright)\mleft( \widetilde{\mathfrak{D}}^{\lambda + \lambda^\prime - \lambda^\prime \circ \rho \circ \lambda} \stackrel{\wedge}{,} \widetilde{\mathfrak{D}}^{\lambda + \lambda^\prime - \lambda^\prime \circ \rho \circ \lambda} \mright)
&=
0.
\eas
Since $\mathrm{Dev}$ and $\mathfrak{D}$ are in general non-zero, and by $\widetilde{\mathfrak{D}} = \mathrm{D} - ({}^*\rho)(\varpi_2)$ (so, the minimal coupling stays non-zero if it was initially non-zero), one can conclude
\bas
\widetilde{\zeta}^{\lambda + \lambda^\prime - \lambda^\prime \circ \rho \circ \lambda} 
&=
\widetilde{\widetilde{\zeta}^\lambda}^{\lambda^\prime},
\eas
however, there are edge cases where this argument fails: $M$ could be a point for example, but it is clear that the field redefinition of $\zeta$ is independent of the choice of $M$ such that one can quickly circumvent this problem. Another edge case is $N$ as a point, but then $\zeta \equiv 0$ such that everything is trivially concluded.
%\bas
%\widetilde{\widetilde{\zeta}^\lambda}^{\lambda^\prime}
%&=
%\Lambda^\prime \circ \mleft(
	%\Lambda \circ \zeta \circ \mleft( \widehat{\Lambda}^{-1}, \widehat{\Lambda}^{-1} \mright)
	%+ \widehat{\zeta}^\lambda
%\mright) \circ \mleft( \mleft(\widehat{\Lambda}^\prime\mright)^{-1}, \mleft(\widehat{\Lambda}^\prime\mright)^{-1} \mright)
	%+ \widehat{\widetilde{\zeta}^\lambda}^{\lambda^\prime}
%\\
%&=
%\mathfrak{\Lambda} \circ \zeta \circ \mleft( \widehat{\mathfrak{\Lambda}}^{-1}, \widehat{\mathfrak{\Lambda}}^{-1} \mright)
	%+ \Lambda^\prime \circ \widehat{\zeta}^\lambda \circ \mleft( \mleft(\widehat{\Lambda}^\prime\mright)^{-1}, \mleft(\widehat{\Lambda}^\prime\mright)^{-1} \mright)
	%+ \widehat{\widetilde{\zeta}^\lambda}^{\lambda^\prime},
%\eas
%where $\widehat{\mathfrak{\Lambda}} \coloneqq \widehat{\Lambda}^\prime \circ \widehat{\Lambda}$ and
%where, recalling Eq.~\eqref{AndereFormelFuerNablaTrafoBesserFuerDasRechnen} and \eqref{FormulaForZetaTildeWithZetaEqualzero},
%\bas
%&\Lambda^\prime \circ \widehat{\zeta}^\lambda \circ \mleft( \mleft(\widehat{\Lambda}^\prime\mright)^{-1}, \mleft(\widehat{\Lambda}^\prime\mright)^{-1} \mright)
%\\
%&\hspace{1cm}=
%\Lambda^\prime \circ \widehat{\zeta}^\lambda \circ \mleft( \widehat{\Lambda}, \widehat{\Lambda} \mright) \circ \mleft( \widehat{\mathfrak{\Lambda}}^{-1}, \widehat{\mathfrak{\Lambda}}^{-1} \mright)
%\\
%&\hspace{1cm}=
%- \Lambda^\prime \circ \mleft(
		%\mathrm{d}^{\widetilde{\nabla}^\lambda} \lambda
		%- t_{\widetilde{\nabla}^\lambda_\rho} \circ (\lambda, \lambda)
		%\mright)
		%\circ \mleft( \widehat{\mathfrak{\Lambda}}^{-1}, \widehat{\mathfrak{\Lambda}}^{-1} \mright)
%\\
%&\hspace{1cm}\stackrel{\mathclap{ \eqref{eqDifferentialSplit} }}{=}~ 
%- \Lambda^\prime \circ \Bigl(
		%\underbrace{\mathrm{d}^{\Lambda\circ\nabla\circ \Lambda^{-1}} \lambda}
		%_{\mathclap{ = \mleft( \Lambda \circ \mathrm{d}^\nabla \circ \Lambda^{-1} \mright) \lambda }}
		%+	D \wedge \lambda
		%- t_{\widetilde{\nabla}^\lambda_\rho} \circ (\lambda, \lambda)
	%\Bigr)
	%\circ \mleft( \widehat{\mathfrak{\Lambda}}^{-1}, \widehat{\mathfrak{\Lambda}}^{-1} \mright)
%\\
%&\hspace{1cm}=
%\mathfrak{\Lambda} \circ \biggl( - \mathrm{d}^\nabla \mleft( \Lambda^{-1} \circ \lambda \mright) 
	%+ \mathrm{d}^\nabla \mleft( \Lambda^{-1} \circ \lambda \mright) \circ \mleft( \mathds{1}_{\mathrm{T}N}, \rho \circ \lambda \mright)
	%+ \mathrm{d}^\nabla \mleft( \Lambda^{-1} \circ \lambda \mright) \circ \mleft( \rho \circ \lambda, \mathds{1}_{\mathrm{T}N} \mright)
%\\
%&\hphantom{=\mathfrak{\Lambda} \circ \biggl(}\hspace{2cm}
	%- t_{\nabla_\rho} \circ \mleft( \Lambda^{-1} \circ \lambda, \lambda \mright)
	%- t_{\nabla_\rho} \circ \mleft( \lambda, \Lambda^{-1} \circ \lambda \mright)
	%+ \Lambda^{-1} \circ t_{\widetilde{\nabla}^\lambda_\rho} \circ (\lambda, \lambda)
	%\biggr) \circ \mleft( \widehat{\mathfrak{\Lambda}}^{-1}, \widehat{\mathfrak{\Lambda}}^{-1} \mright)
%\eas
%viewing $D \coloneqq - \mleft( \mathrm{d}^{\Lambda\circ\nabla\circ \Lambda^{-1}} \lambda \mright) \circ \mleft( \mathds{1}_{\mathrm{T}N}, \rho \mright) + \Lambda \circ t_{\nabla_\rho} \circ \mleft( \Lambda^{-1} \circ \lambda, \mathds{1}_E \mright)$ as an element of $\Omega^1(N;\mathrm{End}(E))$,
%and, using Prop.~\ref{prop:PropsOfBigLambdas},
%\bas
%&\Bigl(- t_{\nabla_\rho} \circ \mleft( \Lambda^{-1} \circ \lambda, \lambda \mright)
	%- t_{\nabla_\rho} \circ \mleft( \lambda, \Lambda^{-1} \circ \lambda \mright)
	%+ \Lambda^{-1} \circ \underbrace{t_{\widetilde{\nabla}^\lambda_\rho} \circ (\lambda, \lambda)}
	%_{\mathclap{ = - t_{\mleft(\widetilde{\nabla}^\lambda\mright)^{\mathrm{bas}}} \circ (\lambda, \lambda) }}
%\Bigr)(X, Y)
%\\
%&\hspace{1cm}=
%- \nabla_{(\rho \circ \lambda)(X)} \mleft( \mleft(\Lambda^{-1} \circ \lambda\mright)(Y) \mright)
	%+ \nabla_{\mleft(\rho \circ \Lambda^{-1} \circ \lambda\mright)(Y)} \bigl( \lambda(X) \bigr)
	%+ \mleft[ \lambda(X), \mleft(\Lambda^{-1} \circ \lambda\mright)(Y) \mright]_E
%\\
%&\hspace{2cm}
	%+ \nabla_{(\rho \circ \lambda)(Y)} \mleft( \mleft(\Lambda^{-1} \circ \lambda\mright)(X) \mright)
	%- \nabla_{\mleft(\rho \circ \Lambda^{-1} \circ \lambda\mright)(X)} \bigl( \lambda(Y) \bigr)
	%+ \mleft[ \mleft(\Lambda^{-1} \circ \lambda\mright)(X), \lambda(Y) \mright]_E
%\\
%&\hspace{2cm}
	%- \nabla^{\mathrm{bas}}_{\lambda(X)} \mleft( \mleft( \Lambda^{-1} \circ \lambda \mright) (Y) \mright)
	%+ \nabla^{\mathrm{bas}}_{\lambda(Y)} \mleft( \mleft( \Lambda^{-1} \circ \lambda \mright) (Y) \mright)
	%+ \Lambda^{-1} \mleft( \mleft[ \lambda(X), \lambda(Y) \mright]_E \mright)
%\\
%&\hspace{1cm}=
%- \nabla_{(\rho \circ \lambda)(X)} \mleft( \mleft(\Lambda^{-1} \circ \lambda\mright)(Y) \mright)
	%+ \nabla_{\mleft(\rho \circ \Lambda^{-1} \circ \lambda\mright)(Y)} \bigl( \lambda(X) \bigr)
	%+ \mleft[ \lambda(X), \mleft(\Lambda^{-1} \circ \lambda\mright)(Y) \mright]_E
%\\
%&\hspace{2cm}
	%+ \nabla_{(\rho \circ \lambda)(Y)} \mleft( \mleft(\Lambda^{-1} \circ \lambda\mright)(X) \mright)
	%- \nabla_{\mleft(\rho \circ \Lambda^{-1} \circ \lambda\mright)(X)} \bigl( \lambda(Y) \bigr)
	%+ \mleft[ \mleft(\Lambda^{-1} \circ \lambda\mright)(X), \lambda(Y) \mright]_E
%\\
%&\hspace{2cm}
	%- \mleft[ \lambda(X), \mleft( \Lambda^{-1} \circ \lambda \mright) (Y) \mright]_E
	%- \nabla_{\mleft(\rho \circ \Lambda^{-1} \circ \lambda \mright) (Y)} \bigl( \lambda(X) \bigr)
%\\
%&\hspace{2cm}
	%+ \mleft[ \lambda(Y), \mleft( \Lambda^{-1} \circ \lambda \mright) (X) \mright]_E
	%+ \nabla_{\mleft(\rho \circ \Lambda^{-1} \circ \lambda \mright) (X)} \bigl( \lambda(Y) \bigr)
%\\
%&\hspace{2cm}
	%+ \Lambda^{-1} \mleft( \mleft[ \lambda(X), \lambda(Y) \mright]_E \mright)
%\\
%&\hspace{1cm}=
%- \nabla_{(\rho \circ \lambda)(X)} \mleft( \mleft(\Lambda^{-1} \circ \lambda\mright)(Y) \mright)
	%+ \nabla_{(\rho \circ \lambda)(Y)} \mleft( \mleft(\Lambda^{-1} \circ \lambda\mright)(X) \mright)
	%+ \Lambda^{-1} \mleft( \mleft[ \lambda(X), \lambda(Y) \mright]_E \mright)
%\eas
%for all $X, Y \in \mathfrak{X}(N)$, and therefore by Prop.~\ref{prop:PropsOfBigLambdas}
%\bas
%&\mathfrak{\Lambda} \circ \biggl( - \mathrm{d}^\nabla \mleft( \Lambda^{-1} \circ \lambda \mright) 
	%+ \mathrm{d}^\nabla \mleft( \Lambda^{-1} \circ \lambda \mright) \circ \mleft( \mathds{1}_{\mathrm{T}N}, \rho \circ \lambda \mright)
	%+ \mathrm{d}^\nabla \mleft( \Lambda^{-1} \circ \lambda \mright) \circ \mleft( \rho \circ \lambda, \mathds{1}_{\mathrm{T}N} \mright)
%\\
%&\hphantom{\mathfrak{\Lambda} \circ \biggl(}\hspace{0.5cm}
	%- t_{\nabla_\rho} \circ \mleft( \Lambda^{-1} \circ \lambda, \lambda \mright)
	%- t_{\nabla_\rho} \circ \mleft( \lambda, \Lambda^{-1} \circ \lambda \mright)
	%+ \Lambda^{-1} \circ t_{\widetilde{\nabla}^\lambda_\rho} \circ (\lambda, \lambda)
	%\biggr)(X,Y)
%\\
%&\hspace{0.5cm}=
%\mathfrak{\Lambda} \circ \biggl( 
	%- \mathrm{d}^\nabla \mleft( \Lambda^{-1} \circ \lambda \mright) (X, Y)
	%+ \mathrm{d}^\nabla \mleft( \Lambda^{-1} \circ \lambda \mright) \bigl( X, (\rho \circ \lambda)(Y) \bigr)
	%+ \mathrm{d}^\nabla \mleft( \Lambda^{-1} \circ \lambda \mright) \bigl( (\rho \circ \lambda)(X), Y \bigr)
%\\
%&\hphantom{\mathfrak{\Lambda} \circ \biggl(}\hspace{1.5cm}
	%- \nabla_{(\rho \circ \lambda)(X)} \mleft( \mleft(\Lambda^{-1} \circ \lambda\mright)(Y) \mright)
	%+ \nabla_{(\rho \circ \lambda)(Y)} \mleft( \mleft(\Lambda^{-1} \circ \lambda\mright)(X) \mright)
	%+ \Lambda^{-1} \bigl( \mleft[ \lambda(X), \lambda(Y) \mright]_E \bigr)
%\biggr)
%\\
%&\hspace{0.5cm}=
%\mathfrak{\Lambda} \circ \biggl( 
	%\underbrace{- \mathrm{d}^\nabla \mleft( \Lambda^{-1} \circ \lambda \mright) (X, Y)}
		%_{
			%= - \nabla_X \mleft( \mleft(\Lambda^{-1} \circ \lambda \mright)(Y) \mright)
				%+ \nabla_Y \mleft( \mleft(\Lambda^{-1} \circ \lambda \mright)(X) \mright)
				%+ \mleft(\Lambda^{-1} \circ \lambda\mright)\mleft( [X,Y] \mright)
		%}
%\\
%&\hphantom{\mathfrak{\Lambda} \circ \biggl(}\hspace{1.5cm}
	%+ \nabla_X \mleft( \mleft(\Lambda^{-1} \circ \lambda \circ \rho \circ \lambda\mright)(Y) \mright)
	%- \nabla_Y \mleft( \mleft(\Lambda^{-1} \circ \lambda \circ \rho \circ \lambda\mright)(X)\mright)
%\\
%&\hphantom{\mathfrak{\Lambda} \circ \biggl(}\hspace{1.5cm}
	%- \mleft(\Lambda^{-1} \circ \lambda \mright)\bigl( \mleft[ X, (\rho \circ \lambda)(Y) \mright] \bigr)
	%- \mleft(\Lambda^{-1} \circ \lambda \mright)\bigl( \mleft[ (\rho \circ \lambda)(X), Y \mright] \bigr)
	%+ \underbrace{\Lambda^{-1}}_{\mathclap{ = \mathds{1}_E + \Lambda^{-1} \circ \lambda \circ \rho }} \bigl( \mleft[ \lambda(X), \lambda(Y) \mright]_E \bigr)
%\biggr)
%\\
%&\hspace{0.5cm}=
%\mathfrak{\Lambda} \circ \mleft(
%- \nabla_X \bigl( \lambda(Y) \bigr)
	%+ \nabla_Y \bigl( \lambda(X) \bigr)
	%+ \mleft[ \lambda(X), \lambda(Y) \mright]_E
	%+ \mleft( \Lambda^{-1} \circ \lambda \mright)\mleft( \mleft[ \widehat{\Lambda}(X), \widehat{\Lambda}(Y) \mright] \mright)
%\mright)
%\\
%&\hspace{0.5cm}=
%\mathfrak{\Lambda} \circ \mleft(
	%- \mleft(\mathrm{d}^\nabla \lambda\mright)(X,Y)
	%+ \mleft( \Lambda^{-1} \circ \lambda \mright)\mleft( \mleft[ \widehat{\Lambda}(X), \widehat{\Lambda}(Y) \mright] \mright)
%\mright)
%\\
%&\hspace{0.5cm}=
%- \mleft( \mathrm{d}^{\mathfrak{\Lambda} \circ \nabla \circ \mathfrak{\Lambda}^{-1}}\mleft( \mathfrak{\Lambda} \circ \lambda \mright)\mright)(X, Y)
	%+ \mleft( \Lambda^\prime \circ \lambda \mright)\mleft( \mleft[ \widehat{\Lambda}(X), \widehat{\Lambda}(Y) \mright] \mright)
%\eas
%hence,
%\bas
%\Lambda^\prime \circ \widehat{\zeta}^\lambda \circ \mleft( \mleft(\widehat{\Lambda}^\prime\mright)^{-1}, \mleft(\widehat{\Lambda}^\prime\mright)^{-1} \mright)
%&=
%\mleft(
	%- \mathrm{d}^{\mathfrak{\Lambda} \circ \nabla \circ \mathfrak{\Lambda}^{-1}}\mleft( \mathfrak{\Lambda} \circ \lambda \mright)(X, Y)
	%+ \mleft( \Lambda^\prime \circ \lambda \mright)\bigl( \mleft[ \Lambda(X), \Lambda(Y) \mright] \bigr)
%\mright) \circ \mleft( \widehat{\mathfrak{\Lambda}}^{-1}, \widehat{\mathfrak{\Lambda}}^{-1} \mright).
%\eas
%Using additionally the previous results of this proof, we also have
%\bas
%\widehat{\widetilde{\zeta}^\lambda}^{\lambda^\prime}
%&\coloneqq
%- \mleft(
	%\mathrm{d}^{\widetilde{\nabla}^{\lambda+\lambda^\prime - \lambda^\prime \circ \rho \circ \lambda}} \lambda^\prime 
	%- t_{\widetilde{\nabla}^{\lambda+\lambda^\prime - \lambda^\prime \circ \rho \circ \lambda}_\rho} \circ \mleft(\lambda^\prime, \lambda^\prime\mright)
%\mright)
	%\circ \mleft(\mleft(\widehat{\Lambda}^\prime\mright)^{-1}, \mleft(\widehat{\Lambda}^\prime\mright)^{-1}\mright)
%\\
%&=
%\mleft(
	%- \mleft(\mathrm{d}^{\widetilde{\nabla}^{\lambda+\lambda^\prime - \lambda^\prime \circ \rho \circ \lambda}} \lambda^\prime \mright) \circ \mleft( \widehat{\Lambda}, \widehat{\Lambda} \mright)
	%+ t_{\widetilde{\nabla}^{\lambda+\lambda^\prime - \lambda^\prime \circ \rho \circ \lambda}_\rho} \circ \mleft(\lambda^\prime \circ \widehat{\Lambda}, \lambda^\prime \circ \widehat{\Lambda}\mright)
%\mright)
	%\circ \mleft( \widehat{\mathfrak{\Lambda}}^{-1}, \widehat{\mathfrak{\Lambda}}^{-1} \mright).
%%\\
%%&=
%%\mathrm{d}^{\nabla} \mleft( \Lambda^{-1}\circ \lambda\mright) \circ \mleft(\widehat{\Lambda}, \widehat{\Lambda}\mright)
	%%+ t_{\nabla_\rho} \circ \mleft(\lambda, \lambda\mright)
%\eas
%Altogether, using again Eq.~\eqref{eqDifferentialSplit} as previously but with respect to $\widetilde{\nabla}^{\lambda+\lambda^\prime - \lambda^\prime \circ \rho \circ \lambda}$,
%\bas
%&\mleft(
	%\Lambda^\prime \circ \widehat{\zeta}^\lambda \circ \mleft( \mleft(\widehat{\Lambda}^\prime\mright)^{-1}, \mleft(\widehat{\Lambda}^\prime\mright)^{-1} \mright)
	%+ \widehat{\widetilde{\zeta}^\lambda}^{\lambda^\prime}
%\mright)
	%\circ \mleft(\widehat{\mathfrak{\Lambda}}, \widehat{\mathfrak{\Lambda}}\mright)
%\\
%&\hspace{0.5cm}=
%- \mleft(\mathrm{d}^{\widetilde{\nabla}^{\lambda+\lambda^\prime - \lambda^\prime \circ \rho \circ \lambda}} \lambda^\prime \mright)
	%+ t_{\widetilde{\nabla}^{\lambda+\lambda^\prime - \lambda^\prime \circ \rho \circ \lambda}_\rho} \circ \mleft(\lambda^\prime \circ \widehat{\Lambda}, \lambda^\prime \circ \widehat{\Lambda}\mright)
%\\
%&\hspace{1.5cm}
	%+ \mleft(\mathrm{d}^{\widetilde{\nabla}^{\lambda+\lambda^\prime - \lambda^\prime \circ \rho \circ \lambda}} \lambda^\prime \mright) \circ \mleft( \rho \circ \lambda, \widehat{\Lambda} \mright)
	%+ \mleft(\mathrm{d}^{\widetilde{\nabla}^{\lambda+\lambda^\prime - \lambda^\prime \circ \rho \circ \lambda}} \lambda^\prime \mright) \circ \mleft( \mathds{1}_{\mathrm{T}N}, \rho \circ \lambda \mright)
%\\
%&\hspace{1.5cm}
	%- \mathrm{d}^{\mathfrak{\Lambda} \circ \nabla \circ \mathfrak{\Lambda}^{-1}}\mleft( \mathfrak{\Lambda} \circ \lambda \mright)(X, Y)
	%+ \mleft( \Lambda^\prime \circ \lambda \mright)\mleft( \mleft[ \widehat{\Lambda}(X), \widehat{\Lambda}(Y) \mright] \mright)
%%\\
%%&\hspace{0.5cm}=
%%- \mleft(\mathrm{d}^{\widetilde{\nabla}^{\lambda+\lambda^\prime - \lambda^\prime \circ \rho \circ \lambda}} \lambda^\prime \mright)
	%%+ t_{\widetilde{\nabla}^{\lambda+\lambda^\prime - \lambda^\prime \circ \rho \circ \lambda}_\rho} \circ \mleft(\lambda^\prime \circ \widehat{\Lambda}, \lambda^\prime \circ \widehat{\Lambda}\mright)
%%\\
%%&\hspace{1.5cm}
	%%+ \widetilde{\nabla}^{\lambda+\lambda^\prime - \lambda^\prime \circ \rho \circ \lambda}_{(\rho \circ \lambda)(X)} \mleft( \mleft(\lambda^\prime \circ \widehat{\Lambda} \mright)(Y) \mright)
	%%- \widetilde{\nabla}^{\lambda+\lambda^\prime - \lambda^\prime \circ \rho \circ \lambda}_{\widehat{\Lambda}(Y)} \mleft( \mleft(\lambda^\prime \circ \rho \circ \lambda \mright)(X) \mright)
	%%- \lambda^\prime \mleft( \mleft[ (\rho \circ \lambda)(X), \widehat{\Lambda}(Y) \mright] \mright)
%%\\
%%&\hspace{1.5cm}
	%%+ \widetilde{\nabla}^{\lambda+\lambda^\prime - \lambda^\prime \circ \rho \circ \lambda}_{X} \mleft( \mleft(\lambda^\prime \circ \rho \circ \lambda \mright)(Y) \mright)
	%%- \widetilde{\nabla}^{\lambda+\lambda^\prime - \lambda^\prime \circ \rho \circ \lambda}_{(\rho \circ \lambda)(Y)} \mleft( \lambda^\prime (X) \mright)
	%%- \lambda^\prime \mleft( \mleft[ X, (\rho \circ \lambda)(Y) \mright] \mright)
%%\\
%%&\hspace{1.5cm}
%%- \mathfrak{\Lambda} \Bigl(  
	%%\nabla_X \bigl( \lambda(Y) \bigr)
	%%- \nabla_Y \bigl( \lambda(X) \bigr)
	%%- \lambda\bigl( [X,Y] \bigr)
%%\Bigr)
	%%+ \mleft( \Lambda^\prime \circ \lambda \mright)\mleft( \mleft[ \widehat{\Lambda}(X), \widehat{\Lambda}(Y) \mright] \mright)
%\\
%&\hspace{0.5cm}=
%- \mleft(\mathrm{d}^{\widetilde{\nabla}^{\lambda+\lambda^\prime - \lambda^\prime \circ \rho \circ \lambda}} \lambda^\prime \mright)
	%+ t_{\widetilde{\nabla}^{\lambda+\lambda^\prime - \lambda^\prime \circ \rho \circ \lambda}_\rho} \circ \mleft(\lambda^\prime \circ \Lambda, \lambda^\prime \circ \Lambda\mright)
%\\
%&\hspace{1.5cm}
	%+ \mleft(\mathrm{d}^{\mathfrak{\Lambda} \circ \nabla \circ \mathfrak{\Lambda}^{-1}} \lambda^\prime\mright) \circ \mleft( \rho \circ \lambda, \widehat{\Lambda} \mright)
%\\
%&\hspace{1.5cm}
	%- \mleft( \mathrm{d}^{\mathfrak{\Lambda}\circ\nabla\circ \mathfrak{\Lambda}^{-1}} (\lambda + \lambda^\prime - \lambda^\prime \circ \rho \circ \lambda) \mright) \circ \mleft( \rho \circ \lambda, \rho \circ \lambda^\prime \circ \widehat{\Lambda} \mright)
%\\
%&\hspace{1.5cm}
	%- \mleft( \mathrm{d}^{\mathfrak{\Lambda}\circ\nabla\circ \mathfrak{\Lambda}^{-1}} (\lambda + \lambda^\prime - \lambda^\prime \circ \rho \circ \lambda) \mright) \circ \mleft( \rho \circ \lambda^\prime \circ \rho \circ \lambda, \widehat{\Lambda} \mright)
%\\
%&\hspace{1.5cm}
	%+ \mathfrak{\Lambda} \circ t_{\nabla_\rho} \circ \mleft( \mathfrak{\Lambda}^{-1} \circ (\lambda+ \lambda^\prime - \lambda^\prime \circ \rho \circ \lambda) \circ \rho \circ \lambda, \lambda^\prime \circ \widehat{\Lambda} \mright)
%\\
%&\hspace{1.5cm}
	%+ \mathfrak{\Lambda} \circ t_{\nabla_\rho} \circ \mleft( \lambda^\prime \circ \rho \circ \lambda, \mathfrak{\Lambda}^{-1} \circ (\lambda+ \lambda^\prime - \lambda^\prime \circ \rho \circ \lambda) \circ \widehat{\Lambda} \mright)
%\\
%&\hspace{1.5cm}
	%+ \mleft(\mathrm{d}^{\widetilde{\nabla}^{\lambda+\lambda^\prime - \lambda^\prime \circ \rho \circ \lambda}} \lambda^\prime \mright) \circ \mleft( \mathds{1}_E, \lambda \circ \rho \mright)
%\\
%&\hspace{1.5cm}
	%- \mathrm{d}^{\mathfrak{\Lambda} \circ \nabla \circ \mathfrak{\Lambda}^{-1}}\mleft( \mathfrak{\Lambda} \circ \lambda \mright)(X, Y)
	%+ \mleft( \Lambda^\prime \circ \lambda \mright)\bigl( \mleft[ \Lambda(X), \Lambda(Y) \mright] \bigr)
%\eas
%
%$D \coloneqq 
%- \mleft( \mathrm{d}^{\mathfrak{\Lambda}\circ\nabla\circ \mathfrak{\Lambda}^{-1}} (\lambda + \lambda^\prime - \lambda^\prime \circ \rho \circ \lambda) \mright) \circ \mleft( \mathds{1}_{\mathrm{T}N}, \rho \mright) 
	%+ \mathfrak{\Lambda} \circ t_{\nabla_\rho} \circ \mleft( \mathfrak{\Lambda}^{-1} \circ (\lambda+ \lambda^\prime - \lambda^\prime \circ \rho \circ \lambda), \mathds{1}_E \mright)$
	%
%\bas
%&- \mleft(\mathrm{d}^{\widetilde{\nabla}^{\lambda+\lambda^\prime - \lambda^\prime \circ \rho \circ \lambda}} (\lambda - \lambda^\prime \circ \rho \circ \lambda) \mright)
	%+ t_{\widetilde{\nabla}^{\lambda+\lambda^\prime - \lambda^\prime \circ \rho \circ \lambda}_\rho} \circ \mleft(\lambda, \lambda^\prime \circ \widehat{\Lambda} \mright)
%\\
%&\hspace{0.5cm}
	%+ t_{\widetilde{\nabla}^{\lambda+\lambda^\prime - \lambda^\prime \circ \rho \circ \lambda}_\rho} \circ \mleft(\lambda^\prime \circ \widehat{\Lambda}, \lambda\mright)
	%+ t_{\widetilde{\nabla}^{\lambda+\lambda^\prime - \lambda^\prime \circ \rho \circ \lambda}_\rho} \circ \mleft(\lambda, \lambda \mright)
%\\
%&\hspace{0.5cm}=
%- \widetilde{\nabla}^{\lambda+\lambda^\prime - \lambda^\prime \circ \rho \circ \lambda}_X \bigl((\lambda - \lambda^\prime \circ \rho \circ \lambda) (Y) \bigr)
	%+ \widetilde{\nabla}^{\lambda+\lambda^\prime - \lambda^\prime \circ \rho \circ \lambda}_Y \bigl((\lambda - \lambda^\prime \circ \rho \circ \lambda) (X) \bigr)
	%+ (\lambda - \lambda^\prime \circ \rho \circ \lambda)([X,Y])
%\\
%&\hspace{1.5cm}
	%+ \widetilde{\nabla}^{\lambda+\lambda^\prime - \lambda^\prime \circ \rho \circ \lambda}_{(\rho\circ \lambda)(X)}\mleft( \mleft( \lambda^\prime \circ \widehat{\Lambda} \mright)(Y) \mright)
	%- \widetilde{\nabla}^{\lambda+\lambda^\prime - \lambda^\prime \circ \rho \circ \lambda}_{\mleft( \rho \circ\lambda^\prime \circ \widehat{\Lambda} \mright) (Y)} \bigl( \lambda(X) \bigr)
	%- \mleft[ \lambda(X), \mleft(\lambda^\prime \circ \widehat{\Lambda} \mright) (Y) \mright]_E
%\\
%&\hspace{1.5cm}
	%+ \widetilde{\nabla}^{\lambda+\lambda^\prime - \lambda^\prime \circ \rho \circ \lambda}_{(\rho\circ \lambda)(X)}\mleft( \mleft( \lambda^\prime \circ \widehat{\Lambda} \mright)(Y) \mright)
	%- \widetilde{\nabla}^{\lambda+\lambda^\prime - \lambda^\prime \circ \rho \circ \lambda}_{\mleft( \rho \circ\lambda^\prime \circ \widehat{\Lambda} \mright) (Y)} \bigl( \lambda(X) \bigr)
	%- \mleft[ \lambda(X), \mleft(\lambda^\prime \circ \widehat{\Lambda} \mright) (Y) \mright]_E
%\eas
\end{proof}

\begin{remarks}{Field redefinition as equivalence of CYMH GTs}{FieldredefAsEquivalence}
This finally shows that the field redefinition is an equivalence of CYMH GTs (for fixed $M, N$ and $E$). Reflexivity simply follows due to that $\lambda \equiv 0$ is a valid parameter for the field redefinition, symmetry by Lemma \ref{lem:FieldRedefinitionIsInvertible} and \ref{lem:InverseOfZetaLambda}, and transitivity by Lemma \ref{lem:TransFieldRedef}. Furthermore, by Thm.~\ref{thm:InvarianceUnderTheFieldRedefinition}, the physics stay the same after a field redefinition, which is why one may speak of a \emph{physical} equivalence.
\end{remarks}

As we already argued, starting with a non-flat $\nabla$ and/or a non-zero $\zeta$, it is now natural to ask whether or not there is a field redefinition making $\nabla$ flat and/or $\zeta$ zero, equivalently, whether or not there is an equivalence class with pre-classical and/or classical representative, respectively. We will do this in the next chapter, but let us first state some basic properties of a CYMH GT.

\section{Properties of CYMH GT}\label{PropertiesOFNewTOlleGTs}

\begin{theorems}{Curvature closed under basic connections, by Alexei Kotov}{CurvatureClosed}
Let $E \to N$ be a Lie algebroid over a smooth manifold $N$, and $\nabla$ be a connection on $E$ with vanishing basic curvature. Then
\ba
\mathrm{d}^{\nabla^{\mathrm{bas}}} R_\nabla &= 0.
\ea
\end{theorems}

\begin{remark}
\leavevmode\newline
Alexei Kotov has found this identity, too, with a different approach; this was communicated in a private communication but there is a paper planned about that by Alexei Kotov and Thomas Strobl, planned for 2021.
\end{remark}

\begin{proof}[Proof of Thm.~\ref{thm:CurvatureClosed}]
\leavevmode\newline
We know how the connection acts on the Lie bracket of $E$ due to the vanishing of the basic curvature, hence, let us look at how the curvature acts on the Lie bracket, also using the Jacobi identity of $[ \cdot, \cdot]$,
\bas
R_\nabla(Y, Z) \mleft(\mleft[ \mu, \nu\mright]_E\mright)
&=~
\stackrel{\text{Use } R_\nabla^{\mathrm{bas}} = 0}{\dotsc}
\\
&=
\mleft[ \nabla_Y \nabla_Z \mu, \nu \mright]_E
	+ \mleft[ \nabla_Z \mu, \nabla_Y \nu \mright]_E
	+ \nabla_{\nabla^{\mathrm{bas}}_\nu Y} \nabla_Z \mu
	- \nabla_{\nabla^{\mathrm{bas}}_{\nabla_Z \mu} Y} \nu
	+ \mleft[ \nabla_Y \mu, \nabla_Z \nu \mright]_E \\
	&\quad+ \mleft[ \mu, \nabla_Y \nabla_Z \nu \mright]_E
	+ \nabla_{\nabla^{\mathrm{bas}}_{\nabla_Z \nu} Y} \mu
	- \nabla_{\nabla^{\mathrm{bas}}_\mu Y} \nabla_Z \nu
	+ \nabla_Y \nabla_{\nabla^{\mathrm{bas}}_\nu Z} \mu
	- \nabla_Y \nabla_{\nabla^{\mathrm{bas}}_\mu Z} \nu \\
	&\quad- \Big( Y \leftrightarrow Z \text{ of previous two lines} \Big) \\
	&\quad- \mleft[ \nabla_{[Y, Z]} \mu, \nu \mright]_E
	- \mleft[ \mu, \nabla_{[Y, Z]}\nu \mright]_E
	- \nabla_{\nabla^{\mathrm{bas}}_\nu \mleft( [Y, Z] \mright)} \mu
	+ \nabla_{\nabla^{\mathrm{bas}}_\mu \mleft( [Y, Z] \mright)} \nu \\
&=
R_\nabla\mleft( \nabla^{\mathrm{bas}}_\nu Y, Z \mright)\mu
	+ R_\nabla\mleft( Y, \nabla^{\mathrm{bas}}_\nu Z \mright)\mu
	- R_\nabla\mleft( \nabla^{\mathrm{bas}}_\mu Y, Z \mright)\nu
	- R_\nabla\mleft( Y, \nabla^{\mathrm{bas}}_\mu Z \mright)\nu \\
	&\quad+ \underbrace{\nabla_{\mleft[ \nabla^{\mathrm{bas}}_\nu Y, Z \mright]} \mu}_{= \nabla_{\mleft[ \mleft[ \rho(\nu), Y \mright] + \rho \mleft( \nabla_Y \nu \mright), Z \mright]} \mu}
	+ \nabla_{\mleft[ Y, \nabla^{\mathrm{bas}}_\nu Z \mright]} \mu
	- \nabla_{\mleft[ \nabla^{\mathrm{bas}}_\mu Y, Z \mright]} \nu
	- \nabla_{\mleft[ Y, \nabla^{\mathrm{bas}}_\mu Z \mright]} \nu \\
	&\quad+ \nabla_{\mleft[ \rho \mleft( \nabla_Z \nu \mright), Y \mright] + \rho \mleft( \nabla_Y \nabla_Z \nu\mright)} \mu
	- \nabla_{\mleft[ \rho \mleft( \nabla_Z \mu \mright), Y \mright] + \rho \mleft( \nabla_Y \nabla_Z \mu\mright)} \nu
	- \Big( Y \leftrightarrow Z \Big) \\
	&\quad+ \nabla_{\mleft[ \rho(\mu), [Y, Z] \mright] + \rho \mleft(\nabla_{[Y, Z]}\mu\mright)} \nu
	- \nabla_{\mleft[ \rho(\nu), [Y, Z] \mright] + \rho \mleft(\nabla_{[Y, Z]}\nu\mright)} \mu \\
	&\quad+ \mleft[ \mu, R_\nabla(Y, Z) \nu \mright]_E
	- \mleft[ \nu, R_\nabla(Y, Z) \mu \mright]_E \\
&=
R_\nabla\mleft( \nabla^{\mathrm{bas}}_\nu Y, Z \mright)\mu
	+ R_\nabla\mleft( Y, \nabla^{\mathrm{bas}}_\nu Z \mright)\mu
	- R_\nabla\mleft( \nabla^{\mathrm{bas}}_\mu Y, Z \mright)\nu
	- R_\nabla\mleft( Y, \nabla^{\mathrm{bas}}_\mu Z \mright)\nu \\
	&\quad+ \nabla^{\mathrm{bas}}_\mu \mleft( R_\nabla(Y, Z) \nu \mright)
	- \nabla^{\mathrm{bas}}_\nu \mleft( R_\nabla(Y, Z) \mu \mright) \\
&=
\mleft( \mathrm{d}^{\nabla^{\mathrm{bas}}} R_\nabla \mright)(Y, Z, \mu, \nu)
	+ R_\nabla(Y, Z) \mleft(\mleft[ \mu, \nu\mright]_E\mright) \\
\Leftrightarrow\qquad
0
&= 
\mleft( \mathrm{d}^{\nabla^{\mathrm{bas}}} R_\nabla \mright)(Y, Z, \mu, \nu)
\eas
for all $Y, Z \in \mathfrak{X}(N)$ and $\nu, \mu \in \Gamma(E)$
\end{proof}

So, we know that the basic connection is flat when the basic curvature vanishes, recall Prop.~\ref{prop:SnablamitREnabla}, and that the curvature $R_\nabla$ is closed with respect to the differential induced by the basic connection. The compatibility condition \ref{EqMyFormulationOfZetaCondition} then imposes that the curvature even needs to be exact in order to formulate a gauge theory.
%\footnote{Beware that one can locally not use the Poincaré Lemma although the basic connection is flat. On one hand, the basic connection is an $E$-connection and admits therefore in general not a parallel frame locally; on the other hand, the condition about closedness of scalar valued $E$-forms can also contain purely algebraic equations due to a possible non-trivial kernel of the anchor, so, a similar problem as with the discussed existence of parallel frames.}
%Hence, one might argue that the curvature is also exact if $N$ is contractible, using the Poincaré Lemma. That is due to that the basic connection, as an $E$-connection, has in general not a parallel frame locally, as we have discussed earlier. Thus, its differential is in general locally not a exterior covariant derivative related to a canonical flat connection, because a canonical flat connection implies the existence of a parallel frame by definition; and, thus, we cannot apply the Poincaré Lemma in general. Of course, as a sketch, if we have a parallel frame for the basic connection on $E$ and $\mathrm{T}N$, then we also have a local parallel frame $\mleft( e_a \mright)_a$ of $\bigwedge^2\mathrm{T}^*N \otimes E$ with respect to the basic connection (extended to that space; of course the dual connection to the basic connection will be flat then, too, and the canonical dual frame of a parallel frame will be parallel). Then view forms like the curvature as elements of $\Omega^\bullet\mleft(E; \bigwedge^2\mathrm{T}^*N \otimes E\mright)$, and, so, we get
%\bas
%\mleft(\mathrm{d}^{\nabla^{\mathrm{bas}}} \omega \mright) \mleft( \nu_1, \dotsc, \nu_{k+1} \mright)
%&=
%\mleft(\mathrm{d}_E\omega^a\mright)\mleft( \nu_1, \dotsc, \nu_{k+1} \mright) \otimes e_a
%=
%\mleft(\mathrm{d}_E\omega^a\mright)\mleft( \nu_1, \dotsc, \nu_{k+1} \mright) \otimes e_a
%\eas
%for all $\omega \in \Omega^k\mleft(E; \bigwedge^2\mathrm{T}^*N \otimes E\mright)$ ($k \in \mathbb{N}_0$) and $\nu_1, \dotsc, \nu_{k+1} \in \Gamma(E)$, using $\nabla^{\mathrm{bas}} e_a = 0$.
%
%\begin{corollaries}{Locally compatibility condition about exactness always solved}{ZetaKompatibilitaetLokalGeloest}
%Let $E \to N$ be a Lie algebroid over a smooth and contractible manifold $N$, and $\nabla$ a connection on $E$ with vanishing basic curvature.
%
%Then there exists a $\zeta \in \Omega^2(N;E)$ such that
%\ba
%R_\nabla
%&=
%- \nabla^{\mathrm{bas}} \zeta.
%\ea
%\end{corollaries}
%
%\begin{proof}
%\leavevmode\newline
%By Thm.~\ref{thm:CurvatureClosed} we know that $R_\nabla$ is closed
%\end{proof}

We know that curvatures satisfy a Bianchi identity, let us therefore check what this implies about $\zeta$.

\begin{theorems}{Bianchi identity for the primitives of the connection}{BianchiIdentityForZeta}
Let $E \to N$ be a Lie algebroid over a smooth manifold $N$, and $\nabla$ a connection on $E$ with vanishing basic curvature and for whose curvature there is a $\zeta \in \Omega^2(N;E)$ such that $R_\nabla = - \mathrm{d}^{\nabla^{\mathrm{bas}}} \zeta$. Then
\ba
0
&=
\mleft( \nabla^{\mathrm{bas}}_{\nu_0} \mleft( \mathrm{d}^\nabla \zeta \mright) \mright)(Y_0, Y_1, Y_2)
	- \mleft( \nabla^{\mathrm{bas}}_{\nu_0} \bigl( \zeta \circ \mleft( \mathds{1}_{\mathrm{T}N}, \rho \circ \zeta \mright) \bigr) \mright)(Y_0, Y_1, Y_2)
\nonumber \\
&\hspace{1cm}
	- \mleft( \nabla^{\mathrm{bas}}_{\nu_0} \bigl( \zeta \circ \mleft( \mathds{1}_{\mathrm{T}N}, \rho \circ \zeta \mright) \bigr) \mright)(Y_1, Y_2, Y_0)
	- \mleft( \nabla^{\mathrm{bas}}_{\nu_0} \bigl( \zeta \circ \mleft( \mathds{1}_{\mathrm{T}N}, \rho \circ \zeta \mright) \bigr) \mright)(Y_2, Y_0, Y_1)
\ea
for all $Y_0, Y_1, Y_2 \in \mathfrak{X}(N)$ and $\nu_0 \in \Gamma(E)$, where 
\bas
\bigl(\zeta \circ \mleft( \mathds{1}_{\mathrm{T}N}, \rho \circ \zeta \mright) \bigr) (Y_0, Y_1, Y_2) = \zeta\bigl( Y_0, (\rho \circ \zeta)(Y_1, Y_2) \bigr).
\eas
\end{theorems}

\begin{proof}
\leavevmode\newline
$R_\nabla$ satisfies the Bianchi identity, \textit{i.e.}~
\bas
\mathrm{d}^\nabla R_\nabla
&=
0,
\eas
where we view the curvature as an element of $\Omega^2(N; \mathrm{End}(E))$.
Then use Cor.~\ref{cor:commutationS=0} to get
\bas
0
&=
\mleft( -\mathrm{d}^\nabla R_\nabla \mright) \mleft(Y_0, Y_1, Y_2, \nu_0\mright)
\\
&=
\mleft(\mathrm{d}^\nabla \mathrm{d}^{\nabla^{\mathrm{bas}}} \zeta \mright) \mleft(Y_0, Y_1, Y_2, \nu_0\mright)
\\
&=
\mleft( \mathrm{d}^{\nabla^{\mathrm{bas}}} \mathrm{d}^\nabla \zeta \mright) \mleft(Y_0, Y_1, Y_2, \nu_0\mright)
\\
&\hspace{1cm}
	+ R_\nabla \bigl( Y_0, \mleft(\rho \circ \zeta\mright)(Y_1, Y_2) \bigr) \nu_0
	- R_\nabla \bigl( Y_1, \mleft(\rho \circ \zeta\mright)(Y_0, Y_2) \bigr) \nu_0
	+ R_\nabla \bigl( Y_2, \mleft(\rho \circ \zeta\mright)(Y_0, Y_1) \bigr) \nu_0
\\
&\hspace{1cm}
	- \zeta \bigl( \mleft(\rho \circ R_\nabla\mright)(Y_0, Y_1)\nu_0, Y_2 \bigr)
	+ \zeta \bigl( \mleft(\rho \circ R_\nabla\mright)(Y_0, Y_2)\nu_0, Y_1 \bigr)
	- \zeta \bigl( \mleft(\rho \circ R_\nabla\mright)(Y_1, Y_2)\nu_0, Y_0 \bigr)
\\
&=	
\mleft( \mathrm{d}^{\nabla^{\mathrm{bas}}} \mathrm{d}^\nabla \zeta \mright) \mleft(Y_0, Y_1, Y_2, \nu_0\mright)
\\
&\hspace{1cm}
	- \mleft(\nabla^{\mathrm{bas}}_{\nu_0} \zeta\mright) \bigl( Y_0, \mleft(\rho \circ \zeta\mright)(Y_1, Y_2) \bigr)
	+ \mleft(\nabla^{\mathrm{bas}}_{\nu_0} \zeta\mright) \bigl( Y_1, \mleft(\rho \circ \zeta\mright)(Y_0, Y_2) \bigr) 
\\
&\hspace{1cm}
	- \mleft(\nabla^{\mathrm{bas}}_{\nu_0} \zeta\mright) \bigl( Y_2, \mleft(\rho \circ \zeta\mright)(Y_0, Y_1) \bigr) 
\\
&\hspace{1cm}
	+ \zeta \mleft( \mleft(\nabla^{\mathrm{bas}}_{\nu_0} ( \rho \circ \zeta ) \mright)(Y_0, Y_1), Y_2 \mright)
	- \zeta \mleft( \mleft(\nabla^{\mathrm{bas}}_{\nu_0} ( \rho \circ \zeta )\mright)(Y_0, Y_2), Y_1 \mright)
\\
&\hspace{1cm}
	+ \zeta \mleft( \mleft(\nabla^{\mathrm{bas}}_{\nu_0} ( \rho \circ \zeta)\mright)(Y_1, Y_2), Y_0 \mright)
\eas
for all $Y_0, Y_1, Y_2 \in \mathfrak{X}(N)$ and $\nu_0 \in \Gamma(E)$,
using that $\zeta \in \Omega^{2,0}(N, E;E) \cong \Omega^2(N;E)$, $R_\nabla = - \mathrm{d}^{\nabla^{\mathrm{bas}}} \zeta$ and $\rho \circ \nabla^{\mathrm{bas}} = \nabla^{\mathrm{bas}} \circ \rho$ such that
\bas
\mleft( \rho \circ \nabla^{\mathrm{bas}}_{\nu_0} \zeta \mright) (Y_0,Y_1)
&=
\rho\mleft(\mleft(\nabla^{\mathrm{bas}}_{\nu_0} \zeta \mright)(Y_0, Y_1)\mright)
\\
&=
\rho\mleft(
	\nabla^{\mathrm{bas}}_{\nu_0} \bigl(\zeta (Y_0, Y_1) \bigr)
	- \zeta \mleft( \nabla^{\mathrm{bas}}_{\nu_0} Y_0, Y_1 \mright)
	- \zeta \mleft( Y_0, \nabla^{\mathrm{bas}}_{\nu_0} Y_1 \mright)
\mright)
\\
&=
	\nabla^{\mathrm{bas}}_{\nu_0} \bigl( (\rho \circ \zeta) (Y_0, Y_1) \bigr)
	- (\rho \circ \zeta) \mleft( \nabla^{\mathrm{bas}}_{\nu_0} Y_0, Y_1 \mright)
	- (\rho \circ \zeta) \mleft( Y_0, \nabla^{\mathrm{bas}}_{\nu_0} Y_1 \mright)
\\
&=
\mleft(\nabla^{\mathrm{bas}}_{\nu_0} ( \rho \circ \zeta ) \mright)(Y_0, Y_1).
\eas
We can also write
\bas
\mleft(\nabla^{\mathrm{bas}}_{\nu_0} \zeta\mright) \bigl( Y_0, \mleft(\rho \circ \zeta\mright)(Y_1, Y_2) \bigr)
&=
\nabla^{\mathrm{bas}}_{\nu_0} \Bigl(
	\zeta \bigl( Y_0, \mleft(\rho \circ \zeta\mright)(Y_1, Y_2) \bigr)
\Bigr)
\\
&\hspace{1cm}
	- \zeta \mleft( \nabla^{\mathrm{bas}}_{\nu_0} Y_0, \mleft(\rho \circ \zeta\mright)(Y_1, Y_2) \mright)
	- \zeta \mleft( Y_0, \nabla^{\mathrm{bas}}_{\nu_0} \bigl( \mleft(\rho \circ \zeta\mright)(Y_1, Y_2) \bigr) \mright),
\eas
and (again)
\bas
\mleft(\nabla^{\mathrm{bas}}_{\nu_0}  (\rho \circ \zeta)\mright)(Y_0, Y_1)
&=
\nabla^{\mathrm{bas}}_{\nu_0} \bigl(
	(\rho \circ \zeta) \mleft( Y_0, Y_1 \mright)
\bigr)
	- (\rho \circ \zeta) \mleft( \nabla^{\mathrm{bas}}_{\nu_0} Y_0, Y_1 \mright)
	- (\rho \circ \zeta) \mleft( Y_0, \nabla^{\mathrm{bas}}_{\nu_0} Y_1 \mright),
\eas
such that in total
\bas
0
&=
\mleft( \mathrm{d}^{\nabla^{\mathrm{bas}}} \mathrm{d}^\nabla \zeta \mright) \mleft(Y_0, Y_1, Y_2, \nu_0\mright)
\\
&\hspace{1cm}
	- \nabla^{\mathrm{bas}}_{\nu_0} \Bigl(
	\zeta \bigl( Y_0, \mleft(\rho \circ \zeta\mright)(Y_1, Y_2) \bigr)
\Bigr)
	+ \zeta \mleft( \nabla^{\mathrm{bas}}_{\nu_0} Y_0, \mleft(\rho \circ \zeta\mright)(Y_1, Y_2) \mright)
	+ \zeta \mleft( Y_0, \nabla^{\mathrm{bas}}_{\nu_0} \bigl( \mleft(\rho \circ \zeta\mright)(Y_1, Y_2) \bigr) \mright)
\\
&\hspace{1cm}
	+ \nabla^{\mathrm{bas}}_{\nu_0} \Bigl(
	\zeta \bigl( Y_1, \mleft(\rho \circ \zeta\mright)(Y_0, Y_2) \bigr)
\Bigr)
	- \zeta \mleft( \nabla^{\mathrm{bas}}_{\nu_0} Y_1, \mleft(\rho \circ \zeta\mright)(Y_0, Y_2) \mright)
	- \zeta \mleft( Y_1, \nabla^{\mathrm{bas}}_{\nu_0} \bigl( \mleft(\rho \circ \zeta\mright)(Y_0, Y_2) \bigr) \mright)
\\
&\hspace{1cm}
	- \nabla^{\mathrm{bas}}_{\nu_0} \Bigl(
	\zeta \bigl( Y_2, \mleft(\rho \circ \zeta\mright)(Y_0, Y_1) \bigr)
\Bigr)
	+ \zeta \mleft( \nabla^{\mathrm{bas}}_{\nu_0} Y_2, \mleft(\rho \circ \zeta\mright)(Y_0, Y_1) \mright)
	+ \zeta \mleft( Y_2, \nabla^{\mathrm{bas}}_{\nu_0} \bigl( \mleft(\rho \circ \zeta\mright)(Y_0, Y_1) \bigr) \mright)
\\
&\hspace{1cm}
	- \zeta \mleft( Y_2,
		\nabla^{\mathrm{bas}}_{\nu_0} \bigl(
	(\rho \circ \zeta) \mleft( Y_0, Y_1 \mright)
\bigr)
	- (\rho \circ \zeta) \mleft( \nabla^{\mathrm{bas}}_{\nu_0} Y_0, Y_1 \mright)
	- (\rho \circ \zeta) \mleft( Y_0, \nabla^{\mathrm{bas}}_{\nu_0} Y_1 \mright)
	\mright)
\\
&\hspace{1cm}
	+ \zeta \mleft( Y_1,
	\nabla^{\mathrm{bas}}_{\nu_0} \bigl(
	(\rho \circ \zeta) \mleft( Y_0, Y_2 \mright)
\bigr)
	- (\rho \circ \zeta) \mleft( \nabla^{\mathrm{bas}}_{\nu_0} Y_0, Y_2 \mright)
	- (\rho \circ \zeta) \mleft( Y_0, \nabla^{\mathrm{bas}}_{\nu_0} Y_2 \mright)
	\mright)
\\
&\hspace{1cm}
	- \zeta \mleft( Y_0,
	\nabla^{\mathrm{bas}}_{\nu_0} \bigl(
	(\rho \circ \zeta) \mleft( Y_1, Y_2 \mright)
\bigr)
	- (\rho \circ \zeta) \mleft( \nabla^{\mathrm{bas}}_{\nu_0} Y_1, Y_2 \mright)
	- (\rho \circ \zeta) \mleft( Y_1, \nabla^{\mathrm{bas}}_{\nu_0} Y_2 \mright)
	\mright)
\\
%%%%%%%%%%%%%%%%%%%%%%%%%%%%%%%%%%%%%%%%%%%%%%%%%%%%%%%%%%%%%%%%%%%%%%%%%%%%%%%%%
&=
\underbrace{\mleft( \mathrm{d}^{\nabla^{\mathrm{bas}}} \mathrm{d}^\nabla \zeta \mright) \mleft(Y_0, Y_1, Y_2, \nu_0\mright)}
_{\mathclap{ = \mleft( \nabla^{\mathrm{bas}}_{\nu_0} \mleft( \mathrm{d}^\nabla \zeta \mright) \mright)(Y_0, Y_1, Y_2) }}
\\
&\hspace{1cm}
	- \nabla^{\mathrm{bas}}_{\nu_0} \Bigl(
	\zeta \bigl( Y_0, \mleft(\rho \circ \zeta\mright)(Y_1, Y_2) \bigr)
\Bigr)
	+ \zeta \mleft( \nabla^{\mathrm{bas}}_{\nu_0} Y_0, \mleft(\rho \circ \zeta\mright)(Y_1, Y_2) \mright)
\\
&\hspace{1cm}
	+ \zeta \mleft( Y_0, 
	(\rho \circ \zeta) \mleft( \nabla^{\mathrm{bas}}_{\nu_0} Y_1, Y_2 \mright)
	+ (\rho \circ \zeta) \mleft( Y_1, \nabla^{\mathrm{bas}}_{\nu_0} Y_2 \mright)
	\mright)
\\
&\hspace{1cm}
	- \nabla^{\mathrm{bas}}_{\nu_0} \Bigl(
	\zeta \bigl( Y_1, \mleft(\rho \circ \zeta\mright)(Y_2, Y_0) \bigr)
\Bigr)
	+ \zeta \mleft( \nabla^{\mathrm{bas}}_{\nu_0} Y_1, \mleft(\rho \circ \zeta\mright)(Y_2, Y_0) \mright)
\\
&\hspace{1cm}
	+ \zeta \mleft( Y_1,
	(\rho \circ \zeta) \mleft( \nabla^{\mathrm{bas}}_{\nu_0} Y_2, Y_0 \mright)
	+ (\rho \circ \zeta) \mleft( Y_2, \nabla^{\mathrm{bas}}_{\nu_0} Y_0 \mright)
	\mright)
\\
&\hspace{1cm}
	- \nabla^{\mathrm{bas}}_{\nu_0} \Bigl(
	\zeta \bigl( Y_2, \mleft(\rho \circ \zeta\mright)(Y_0, Y_1) \bigr)
\Bigr)
	+ \zeta \mleft( \nabla^{\mathrm{bas}}_{\nu_0} Y_2, \mleft(\rho \circ \zeta\mright)(Y_0, Y_1) \mright)
\\
&\hspace{1cm}
	+ \zeta \mleft( Y_2,
	(\rho \circ \zeta) \mleft( \nabla^{\mathrm{bas}}_{\nu_0} Y_0, Y_1 \mright)
	+ (\rho \circ \zeta) \mleft( Y_0, \nabla^{\mathrm{bas}}_{\nu_0} Y_1 \mright)
	\mright)
%%%%%%%%%%%%%%%%%%%%%%%%%%%%%%%%%%%%%%%%%%%%%%%%%%%%%%%%%%%%%%%%%%%%%%%%%%%%%%%%
\\
&=
\mleft( \nabla^{\mathrm{bas}}_{\nu_0} \mleft( \mathrm{d}^\nabla \zeta \mright) \mright)(Y_0, Y_1, Y_2)
	- \mleft( \nabla^{\mathrm{bas}}_{\nu_0} \bigl( \zeta \circ \mleft( \mathds{1}_{\mathrm{T}N}, \rho \circ \zeta \mright) \bigr) \mright)(Y_0, Y_1, Y_2)
\\
&\hspace{1cm}
	- \mleft( \nabla^{\mathrm{bas}}_{\nu_0} \bigl( \zeta \circ \mleft( \mathds{1}_{\mathrm{T}N}, \rho \circ \zeta \mright) \bigr) \mright)(Y_1, Y_2, Y_0)
	- \mleft( \nabla^{\mathrm{bas}}_{\nu_0} \bigl( \zeta \circ \mleft( \mathds{1}_{\mathrm{T}N}, \rho \circ \zeta \mright) \bigr) \mright)(Y_2, Y_0, Y_1).
\eas
\end{proof}

Recall Thm.~\ref{thm:modBianchithm} for the following statement.

\begin{theorems}{Primitives of the connection along the foliation of the anchor}{BAlongL}
Let $E \to N$ be a Lie algebroid over a smooth manifold $N$, and $\nabla$ a connection on $E$ with vanishing basic curvature. Then all $\zeta \in \Omega^2(N;E)$ satisfying
\ba
\zeta \circ (\rho, \rho)
&= - t_{\nabla^{\mathrm{bas}}} + H,\label{eq:BrhoaufOrbit}
\ea
where $H \in \Omega^2(E;E)$ with $\nabla^{\mathrm{bas}} H = 0$, also satisfy
\ba
R_\nabla \circ (\rho, \rho)
&=
-\mleft(\mathrm{d}^{\nabla^{\mathrm{bas}}} \zeta \mright) \circ (\rho, \rho, \mathds{1}_E),
\ea
that is,
\bas
R_\nabla \bigl( \rho(\mu), \rho(\nu) \bigr) \eta
&=
-\mleft(\mathrm{d}^{\nabla^{\mathrm{bas}}} \zeta \mright) \bigl(\rho(\mu), \rho(\nu), \eta\bigr)
\eas
for all $\mu, \nu, \eta \in \Gamma(E)$.
\end{theorems}
%
%\begin{motivation}
%\leavevmode\newline
%For invertible anchors $\rho$ Eq. \eqref{eq:BaufOrbit} is easy to motivate. For this we want to use Cor. \ref{cor:commutationS=0} on $\omega = \rho^{-1} \in \Omega^{1,0}(M,E;E)$. Observe first (for $X \in \mathfrak{X}(M)$ and $\mu \in \Gamma(E)$)
%\bas
%\left( \mathrm{d}^{\nabla^{\mathrm{bas}}} \rho^{-1} \right)(X, \mu)
%&= 
%\left( \nabla^{\mathrm{bas}}_\mu \rho^{-1} \right)(X)
%= 
%\nabla^{\mathrm{bas}}_\mu\left( \rho^{-1}(X) \right)
%- \rho^{-1}\left( \nabla^{\mathrm{bas}}_\mu X \right)
%\stackrel{\text{Def. } \ref{def:basicconn}}{=} 0
%\eas
%and ($Y \in \mathfrak{X}(M)$)
%\bas
%\left( \mathrm{d}^\nabla \rho^{-1} \right)(X,Y)
%&= 
%\nabla_X \left( \rho^{-1}(Y) \right)
%- \nabla_Y \left( \rho^{-1}(X) \right)
%- \rho^{-1}([X, Y]) \\
%&= \nabla_{\rho\left(\rho^{-1}(X)\right)} \left( \rho^{-1}(Y) \right)
%- \nabla_{\rho\left(\rho^{-1}(Y)\right)} \left( \rho^{-1}(X) \right)
%- [\rho^{-1}(X), \rho^{-1}(Y)] \\
%&=
%t_{\nabla_\rho}\left( \rho^{-1}(X), \rho^{-1}(Y) \right)
%\stackrel{\text{Prop. } \ref{prop:SnablamitREnabla}}{=}
%-t_{\nabla_{\mathrm{bas}}}\left( \rho^{-1}(X), \rho^{-1}(Y) \right).
%\eas
%Then by Cor. \ref{cor:commutationS=0} on $\omega = \rho^{-1}$
%\bas
%&&0
%&=
%- \left( \mathrm{d}^{\nabla^{\mathrm{bas}}} \left( t_{\nabla_{\mathrm{bas}}} \circ \left( \rho^{-1}, \rho^{-1} \right) \right) \right) (X, Y, \mu)
%+ R_\nabla(X, Y) \mu
%- R_\nabla(Y, X) \mu \\
%&&&\quad- R_\nabla(X, Y) \mu \\
%&\Leftrightarrow&
%R_\nabla
%&=
%\mathrm{d}^{\nabla^{\mathrm{bas}}} \left( t_{\nabla_{\mathrm{bas}}} \circ \left( \rho^{-1}, \rho^{-1} \right) \right)
%\stackrel{\text{Lem. } \ref{lem:commutationanchordifferential}}{=}
%\nabla^{\mathrm{bas}} t_{\nabla_{\mathrm{bas}}} \circ \left( \rho^{-1}, \rho^{-1} \right).
%\eas
%Especially, the compatibility condition with $B$ is fully implied by $S_\nabla = 0$.
%\end{motivation}

\begin{proof}[Proof of Thm.~\ref{thm:BAlongL}]
\leavevmode\newline
That is a trivial consequence of Cor.~\ref{cor:LemmaCurvatureOfDualConnections} and Lemma \ref{lem:commutationanchordifferential}, that is,
\bas
R_\nabla \circ (\rho, \rho)
&=
R_{\nabla_\rho}
=
\nabla^{\mathrm{bas}} t_{\nabla^{\mathrm{bas}}}
\stackrel{\nabla^{\mathrm{bas}} H = 0}{=}
- \nabla^{\mathrm{bas}} \bigl( \zeta \circ (\rho, \rho) \bigr).
\stackrel{ \text{Lem.~\ref{lem:commutationanchordifferential}} }{=} 
\left(- \mathrm{d}^{\nabla^{\mathrm{bas}}} \zeta\right) \circ (\rho, \rho).
\eas
\end{proof}

Therefore one can view the negative of the torsion of the basic connection as a canonical choice for $\zeta$ along the foliation of the anchor. 
In case we decide to take $\zeta \in \Omega^2(N;E)$ such that $\zeta \circ (\rho, \rho) = - t_{\nabla^{\mathrm{bas}}}$, we get:

\begin{corollaries}{Certain classical CYMH GTs implying an abelian structure}{ClassicalTheoriesAreAbelianWithCanonicalChoices}
Let us have the same setup and notation as in Thm.~\ref{thm:FinallyTheGaugeInvarianceWeWant}, \textit{i.e.}~let us assume a CYMH GT. Moreover, assume we have $\zeta \circ (\rho, \rho) = - t_{\nabla^{\mathrm{bas}}}$ and that $N$ is simply connected.

If this CYMH GT is classical, then it is isomorphic to an abelian action Lie algebroid such that $\nabla$ is its canonical flat connection.

In case of tangent bundles, $E = \mathrm{T}N$, this statement is an equivalence, that is, this CYMH GT is classical if and only if it is isomorphic to an abelian action Lie algebroid such that $\nabla$ is its canonical flat connection.
\end{corollaries}

\begin{remark}
\leavevmode\newline
In general one could study whether it is possible to have a connection with vanishing basic curvature on a Lie algebroid which is locally never an action Lie algebroid; in that case the connection could not be flat by Thm.~\ref{thm:ActionLieALgebroid}. However, this is a difficult task; this statement may simplify that, one could just look at abelian action Lie algebroids. With that particular choice for $\zeta$ one would have then a non-classical gauge theory, in case one has a Lie algebroid which is not isomorphic to an abelian action Lie algebroid.
\end{remark}

\begin{proof}[Proof of Cor.~\ref{cor:ClassicalTheoriesAreAbelianWithCanonicalChoices}]
\leavevmode\newline
Classical means that $\nabla$ is flat, and, thus, we have a global isomorphism to an action Lie algebroid $N \times \mathfrak{g}$ for a Lie algebra $\mathfrak{g}$, using that $N$ is simply connected and Thm.~\ref{thm:ActionLieALgebroid}; also recall Remark \ref{remSimplyConnectedEqualsGlobal}. $\nabla$ is then its canonical flat connection.

Classical also implies that $\zeta \equiv 0$, hence, the torsion of $\nabla^{\mathrm{bas}}$ vanishes.\footnote{By the metric compatibility with $\kappa$, $\nabla^{\mathrm{bas}}$ is an $E$-Levi-Civita connection, as we also discussed in Rem.~\ref{remELEVICITAOfBasnbala}.} By Cor.~\ref{cor:AbelianIffNablaBasIsLeviCivita}, $\mathfrak{g}$ is abelian.

If we have $E = \mathrm{T}N$, then just use the equivalence in Cor.~\ref{cor:AbelianIffNablaBasIsLeviCivita}, so, assuming that $E$ is isomorphic to an abelian action Lie algebroid and $\nabla$ is its canonical flat connection, implies that the basic connection has no torsion; since the anchor is now bijective we have $\zeta \equiv 0$.
\end{proof}

Along the transversal directions it will be a bit more difficult as we will see in the next chapter. However, as a first approach one can look at the following proposition, which is based on the assumption that one has partially a parallel frame of the basic connection along the foliation, also using Thm.~\ref{thm:BAlongL}; recall Section \ref{DirectProdsOfLieAlgoids}, and also recall that BLA means bundle of Lie algebras. The setup of the following proposition is basically for Lie algebroids restricted on a suitable neighbourhood of regular points.

\begin{propositions}{Local mixed terms of the primitive of the connection}{MixedTermsOfB}
Let $N$ be a parallelizable smooth manifold, $K \to S$ a BLA over a smooth manifold $S$, and $E = \mathrm{T}N \times K \to N \times S$ as direct product of Lie algebroids, equipped with a connection $\nabla$ with a vanishing basic curvature. Furthermore, assume that there is a global trivialisation $\mleft( f_i \mright)_i$ of $\mathrm{T}N$ such that $\nabla^{\mathrm{bas}} f_i = 0$ (on $E$) for all $i$, and assume that we have a $\zeta \in \Omega^2(N;E)$ with $\zeta \circ (\rho, \rho) = - t_{\nabla^{\mathrm{bas}}}$.

If $\zeta$ additionally satisfies $\zeta(Y, f_i) = \nabla_{Y} f_i$ for all $Y \in \mathfrak{X}(S) \subset \mathfrak{X}(N \times S)$, then
\ba\label{MixeDZetaTermEquation}
R_\nabla\bigl( Y, \rho(\mu) \bigr) \nu
&=
- \mleft( \mathrm{d}^{\nabla^{\mathrm{bas}}} \zeta \mright) \bigl(Y, \rho(\mu), \nu \bigr)
\ea
for all $\mu, \nu \in \Gamma(E)$ and $Y \in \mathfrak{X}(S)$.
\end{propositions}

\begin{remark}
\leavevmode\newline
With $\mathfrak{X}(S) \subset \mathfrak{X}(N \times S)$ we emphasize that we view vector fields of a factor of the base, here $S$, as vector fields on $N \times S$ with values in $S$ and constant along $N$, \textit{i.e.}~the canonical embedding. That is important to keep in mind if one sees notations like $\mathfrak{X}(S)$ in this context.

A word on why we wrote "$\nabla^{\mathrm{bas}} f_i = 0$ (on $E$)". One needs to be careful here, with the basic connection we always mean two connections. However, we have for example $\rho(f_i) = f_i$ such that both versions of the basic connection can act on $f_i$, and as long as $K$ has not zero rank we can not expect that both connections give the same, that is, let $\nu \in \Gamma(K)$, then, on $E$,
\bas
\nabla^{\mathrm{bas}}_\nu f_i
&=
\mleft[ \nu, f_i \mright]_E
	+ \nabla_{f_i} \nu,
\eas
and, on $\mathrm{T}N$,
\bas
\nabla^{\mathrm{bas}}_\nu f_i
&=
\rho(\nabla_{f_i} \nu),
\eas
which is clearly different, even if $\mleft[ \nu, f_i \mright]_E = 0$. However, our imposed condition is about that $f_i$ as an element of $\Gamma(E)$ should be parallel to the basic connection, then we use the usual commutation with the anchor to get
\bas
0
&=
\rho\mleft( \nabla^{\mathrm{bas}} f_i \mright)
=
\nabla^{\mathrm{bas}} \bigl( \rho(f_i) \bigr),
\eas
where we did not write $\rho(f_i)$ as $f_i$ to emphasize that $f_i$ is viewed as an element of $\mathfrak{X}(N)$ on the right hand side. Hence, $\nabla^{\mathrm{bas}} f_i = 0$ in sense of $\mathrm{T}N$ is implied here. In the proof we sometimes write $\rho(f_i)$ for similar reasons of accentuation.
\end{remark}

\begin{proof}[Proof of Prop.~\ref{prop:MixedTermsOfB}]
\leavevmode\newline
We prove Eq.~\eqref{MixeDZetaTermEquation} locally using frames due to its tensorial nature. Let $\mleft( f_a \mright)_a$ be a local frame of $E$, which is given by the frame $\mleft(f_i \mright)_i$ of $\mathrm{T}N$ and by a frame $\mleft(f_\alpha \mright)_\alpha$ of $K$, both frames are canonically embedded into $E$; that is, $f_i$ are constant along $S$, and $f_\alpha$ along $N$. 
Other Latin indices still denote the frame of $\mathrm{T}N$, and other Greek ones the part of $K$, and we clearly have $\rho(f_i) = f_i, \rho(f_\alpha) = 0$; especially, $f_i$ also span the image of the anchor. 
Then
\bas
&&
\nabla^{\mathrm{bas}}_{f_i} Y
&=
\underbrace{\mleft[ f_i, Y \mright]}_{=0}
	+ ~\rho\mleft( \nabla_Y f_i \mright)
=
\rho\mleft( \nabla_Y f_i \mright),
\\
&&
\nabla^{\mathrm{bas}}_{f_\alpha} Y
&=
[ \underbrace{\rho(f_\alpha)}_{=0}, Y ]
	+ \rho\mleft( \nabla_Y f_\alpha \mright)
=
\rho\mleft( \nabla_Y f_\alpha \mright),
\\
&\Rightarrow&
\nabla^{\mathrm{bas}}_{f_a} Y
&=
\rho\mleft( \nabla_Y f_a \mright)
\eas
for all $Y \in \mathfrak{X}(S)$.
%\bas
%&&
%\rho\underbrace{\mleft(\nabla^{\mathrm{bas}}_{f_j} f_i \mright)}_{=0}
%&=
%0,
%\\
%&&
%\rho\mleft(\nabla^{\mathrm{bas}}_{f_\alpha} f_i \mright)
%&=
%\rho\bigl( 
	%\underbrace{\mleft[ f_\alpha, f_i \mright]_E}_{=0}
	%+ \nabla_{f_i} f_\alpha
%\bigr)
%=
%\rho\bigl( \nabla_{f_i} f_\alpha \bigr)
%\\
%&\Rightarrow&
%\eas
By the vanishing of the basic curvature we get
\bas
\nabla_Y\mleft( \mleft[ f_a, f_b \mright]_E \mright)
&=
\mleft[ \nabla_Y f_a, f_b \mright]_E
	+ \mleft[ f_a, \nabla_Y f_b \mright]_E
	+ \nabla_{\nabla^{\mathrm{bas}}_{f_b} Y} f_a
	- \nabla_{\nabla^{\mathrm{bas}}_{f_a} Y} f_b
\\
&=
\mleft[ \nabla_Y f_a, f_b \mright]_E
	+ \mleft[ f_a, \nabla_Y f_b \mright]_E
	+ \nabla_{\rho(\nabla_Y f_b)} f_a
	- \nabla_{\rho(\nabla_Y f_a)} f_b,
\eas
such that, additionally using $t_{\nabla^{\mathrm{bas}}} \stackrel{\text{Cor.~\ref{cor:TorsionOfDualTorsions}}}{=} - t_{\nabla_\rho}$ and the assumptions about $\zeta$,
\bas
\mleft(- \nabla^{\mathrm{bas}}_{f_a} \zeta \mright)\bigl(Y, \rho(f_i)\bigr)
&=
- \nabla^{\mathrm{bas}}_{f_a} \underbrace{\mleft(\zeta\bigl(Y, \rho(f_i)\bigr)\mright)}_{\mathclap{= \nabla_Y f_i}}
	+ \underbrace{\zeta\mleft( \nabla^{\mathrm{bas}}_{f_a} Y, \rho(f_i) \mright)}
		_{= \zeta\mleft( \rho(\nabla_Y f_a), \rho(f_i) \mright)}
	+ \zeta\mleft( Y, \vphantom{\nabla^{\mathrm{bas}}_{f_a} \bigl( \rho(f_i) \bigr)} \smash{\underbrace{\nabla^{\mathrm{bas}}_{f_a} \bigl( \rho(f_i) \bigr)}_{\mathclap{= \rho\mleft( \nabla^{\mathrm{bas}}_{f_a} f_i \mright) = 0 }}} \mright) 
\\
&=
- \mleft[ f_a, \nabla_Y f_i \mright]_E
	- \nabla_{\rho\mleft(\nabla_Y f_i \mright)} f_a
\\
&\hspace{1cm}
	+ \nabla_{\rho(\nabla_Y f_a)} f_i
	- \nabla_{f_i} \nabla_Y f_a
	- \mleft[ \nabla_Y f_a, f_i \mright]
\\
&=
\nabla_Y \underbrace{\mleft( \mleft[ f_i, f_a \mright]_E \mright)}
_{\mathclap{ = - \nabla^{\mathrm{bas}}_{f_a} f_i + \nabla_{f_i} f_a }}
	- \nabla_{f_i} \nabla_Y f_a
\\
&=
\nabla_Y \nabla_{f_i} f_a - \nabla_{f_i} \nabla_Y f_a
\\
&\stackrel{\mathclap{ [Y, f_i] = 0 }}{=}\quad 
R_\nabla(Y, f_i) f_a
\\
&=
R_\nabla\bigl(Y, \rho(f_i)\bigr) f_a.
\eas
\end{proof}
%
%\section{Curvature of gauge transformation, part 2}\label{CurvatureOfGaugeTrafoPart2}
%
%Recall Subsection \ref{CurvatureOfGaugePart1}. We now want to extend Cor.~\ref{cor:FlatnessVonEichtrafos}. We cannot really calculate something similar for all possible functionals $\mathcal{F}^\bullet_E$, because we need a more explicit dependency on fields like $\Phi$ and $A$. When we look at the definitions of the physical terms like the Lagrangian $\mathfrak{L}_{CYMH}$ or the field strength $G$, then we see that the physical tensors we are interested into are contractions of $\varpi_2$, $\mathrm{d}^{{}^*\nabla}\varpi_2$, and $\mathrm{D}$, and wedge products of these forms, while the contractions are done via $\mathrm{ev}$-pullbacks of forms on $N$. Equivalently, the arising wedge products are with respect to the forms $\varpi_2$, the field strength $G$ and the minimal coupling $\mathfrak{D}$; that is, one can view $\varpi_2$, $\mathrm{d}^{{}^*\nabla}\varpi_2$, and $\mathrm{D}$ as the generators of those products, and simply by the definitions of the field strength and the minimal coupling, $\varpi_2$, $G$, and $\mathfrak{D}$ generate the same set.
%
%For the following observe that we can canonically embed $C^\infty(M)$ into $\mathcal{F}^0_E(M) = C^\infty(M\times \mathfrak{M}_E(M; N))$ with $C^\infty(M) \ni f \mapsto \widetilde{f}$, where $\widetilde{f}$ is given by $M\times \mathfrak{M}_E(M; N) \ni (p, \Phi, A) \mapsto \widetilde{f}(p,\Phi, A) \coloneqq f(p)$, which we will still denote by $f$ instead of $\widetilde{f}$.
%
%\begin{definitions}{Functionals polynomial in $\varpi_2$, $G$ and $\mathfrak{D}$}{PolynomialFunctionals}
%Let $M, N$ be two smooth manifolds, $E \to N$ a Lie algebroids, $V \to N$ a vector bundle, $\nabla$ a connection on $E$, and $\zeta \in \Omega^2(N;E)$. Then we define the \textbf{space of polynomial functionals $\gls{Hk}(M; {}^*V)$} as a $C^\infty(M)$-submodule of $\mathcal{F}^k_E(M; {}^*V)$ in the following way:
%\begin{enumerate}
	%\item For the purpose of this definition we denote with $\eta^{(r,s)}$ ($r, s \in \mathbb{N}_0$) an arbitrary element of
	%\bas
	%\Gamma\mleft( \mleft(\bigotimes^r E^*\mright) \otimes \mleft(\bigotimes^s \mathrm{T}^*N \mright) \otimes V \mright).
	%\eas
	%\item We denote with 
	%\bas
	%\mleft\langle
		%\varpi_2, G, \mathfrak{D}
	%\mright\rangle_{\eta^{(r,s)}}
	%\eas
	%the $C^\infty(M)$-submodule of 
	%\bas
	%\mathcal{F}^\bullet_E(M; {}^*V)
	%&\coloneqq
	%\bigoplus_{q \in \mathbb{N}_0} \mathcal{F}^q_E(M; {}^*V)
	%\eas
	%generated by elements of the form
	%\bas
	%\mleft({}^*\eta^{(r,s)}\mright)\Bigl( 
		%\underbrace{\varpi_2 \stackrel{\wedge}{,} \dotsc \stackrel{\wedge}{,} \varpi_2}
		%_{l \text{ times}} \stackrel{\wedge}{,}
		%\underbrace{G \stackrel{\wedge}{,} \dotsc \stackrel{\wedge}{,} G}
		%_{m \text{ times}}\stackrel{\wedge}{,}
		%\underbrace{\mathfrak{D} \stackrel{\wedge}{,} \dotsc \stackrel{\wedge}{,} \mathfrak{D}}
		%_{s \text{ times}}
	%\Bigr)
	%\eas
	%for arbitrary $l, m \in \mathbb{N}_0$ with $l + m = r$. 
	%\item We define another $C^\infty(M)$-submodule $\mleft\langle \varpi_2, G, \mathfrak{D} \mright\rangle_{\eta^{(r,s)},k}$ of $\mathcal{F}^k_E(M; {}^*V)$ by
	%\bas
	%\mleft\langle
		%\varpi_2, G, \mathfrak{D}
	%\mright\rangle_{\eta^{(r,s)},k}
	%&\coloneqq
	%\mleft\langle
		%\varpi_2, G, \mathfrak{D}
	%\mright\rangle_{\eta^{(r,s)}}
	%\cap
	%\mathcal{F}^k_E(M; {}^*V).
	%\eas
	%\item $\mathcal{H}^k_E(M; {}^*V)$ is then defined by 
	%\ba
	%\mathcal{H}^k_E(M; {}^*V)
	%&\coloneqq
	%\sum_{r,s \in \mathbb{N}_0} ~ \sum_{\eta^{(r,s)}}
	%\mleft\langle
		%\varpi_2, G, \mathfrak{D}
	%\mright\rangle_{\eta^{(r,s)},k}.
	%%\Biggl\{
	%%L \in \mathcal{F}^k_E(M; {}^*V)
	%%~ \Bigg| ~
	%%\nonumber\\
	%%&\hspace{1cm}\hphantom{\Biggl\{}
	%%\exists l \in \mathbb{N}_0: ~ 
	%%\exists r_1, \dotsc, r_l, s_1, \dotsc, s_l \in \mathbb{N}_0:~
	%%\exists \eta_1^{(r_1, s_1)}, \dotsc, \eta_l^{(r_l,s_l)}:
	%%\nonumber\\
	%%&\hspace{1cm}\hphantom{\Biggl\{}
	%%L \in
	%%\sum_{i=1}^l
	%%\mleft\langle
		%%\varpi_2, F, \mathfrak{D}
	%%\mright\rangle_{\eta_i^{(r_i, s_i)},k}
	%%\Biggr\}.
	%\ea
%\end{enumerate}
%\end{definitions}
%
%\begin{remark}
%\leavevmode\newline
%By definition, polynomial functionals are closed by multiplications with $C^\infty(M)$, and they build a closed algebra with respect to the wedge product. We will use that fact without mentioning it further.
%\end{remark}
%
%Now back to the curvature $R_\delta$.
%
%\begin{theorems}{Vanishing of the total curvature of $\delta$ on polynomial functionals}{EichtrafoIstIWkrlichKrassFlach}
%Let $M, N$ be smooth manifolds, $E \to N$ a Lie algebroid, $\nabla$ a connection on $E$ with $R^{\mathrm{bas}}_\nabla=0$ and such that there is a $\zeta \in \Omega^2(N;E)$ with $R_\nabla = - \mathrm{d}^{\nabla^{\mathrm{bas}}} \zeta$. Also let ${}^E\nabla$ be a flat $E$-connection on a vector bundle $V \to N$. Then
%\ba
%R_{\delta}(\cdot, \cdot) H
%&=
%0
%\ea
%for all $H \in \mathcal{H}_E^k(M; {}^*V)$ ($k \in \mathbb{N}_0$).
%\end{theorems}
%
%\begin{proof}[Proof of Thm.~\ref{thm:EichtrafoIstIWkrlichKrassFlach}]
%\leavevmode\newline
%Very abstractly, for elements of $\mathcal{H}^k_E(M; {}^*V)$ we have in general sums of terms of the following form
%\bas
%\mleft({}^*h\mright)\mleft(\omega \stackrel{\wedge}{,} \eta\mright),
%\eas
%where $\omega \in \mathcal{H}_E^l(M; {}^*W)$ and $\eta \in \mathcal{H}_E^n(M; {}^*P)$ ($l, n \in \mathbb{N}_0$) for two vector bundles $W, P \to N$, and $h \in \Gamma\mleft( W^* \otimes P^* \otimes V \mright)$. Fix $E$-connections on all of these vector bundles in order to define infinitesimal gauge transformations on all terms, while using dual connections on dual bundles in order to apply the Leibniz rule. Then
%\bas
%\delta_\vartheta \Bigl( \delta_\varepsilon \bigl(
	%\mleft({}^*h\mright)\mleft(\omega \stackrel{\wedge}{,} \eta\mright)
%\bigr) \Bigr)
%&=
%\delta_\vartheta \biggl(
	%\bigl( \delta_\varepsilon \mleft( {}^*  h \mright) \bigr)\mleft(\omega \stackrel{\wedge}{,} \eta\mright)
	%+ \mleft({}^*h\mright)\mleft( \delta_\varepsilon \omega \stackrel{\wedge}{,} \eta\mright)
	%+ \mleft({}^*h\mright)\mleft( \omega \stackrel{\wedge}{,} \delta_\varepsilon \eta\mright)
%\biggr)
%\\
%&=
%\mleft( \delta_\vartheta \delta_\varepsilon \mleft( {}^*  h \mright) \mright)\mleft(\omega \stackrel{\wedge}{,} \eta\mright)
	%+ \mleft( \delta_\varepsilon \mleft( {}^*  h \mright) \mright)\mleft(\delta_\vartheta\omega \stackrel{\wedge}{,} \eta\mright)
	%+ \mleft( \delta_\varepsilon \mleft( {}^*  h \mright) \mright)\mleft(\omega \stackrel{\wedge}{,} \delta_\vartheta\eta\mright)
%\\
%&\hspace{1cm}
	%+\mleft( \delta_\vartheta \mleft( {}^*  h \mright) \mright)\mleft( \delta_\varepsilon \omega \stackrel{\wedge}{,} \eta\mright)
	%+ \mleft({}^*h\mright)\mleft( \delta_\vartheta \delta_\varepsilon \omega \stackrel{\wedge}{,} \eta\mright)
	%+ \mleft({}^*h\mright)\mleft( \delta_\varepsilon \omega \stackrel{\wedge}{,} \delta_\vartheta \eta\mright)
%\\
%&\hspace{1cm}
	%+\mleft( \delta_\vartheta \mleft( {}^*  h \mright) \mright)\mleft( \omega \stackrel{\wedge}{,} \delta_\varepsilon \eta\mright)
	%+ \mleft({}^*h\mright)\mleft( \delta_\vartheta \omega \stackrel{\wedge}{,} \delta_\varepsilon \eta\mright)
	%+ \mleft({}^*h\mright)\mleft( \omega \stackrel{\wedge}{,} \delta_\vartheta\delta_\varepsilon \eta\mright)
%\eas
%for all $\varepsilon, \vartheta \in \mathcal{F}^0_E(M; {}^*E)$,
%thus, as usual in such calculations the mixed terms cancel each other in the total, arriving at
%\bas
%R_{\delta}(\vartheta, \varepsilon) \bigl( 
	%\mleft({}^*h\mright)\mleft(\omega \stackrel{\wedge}{,} \eta\mright)
%\bigr)
%&=
%\bigl(R_{\delta}(\vartheta, \varepsilon)\mleft({}^*h\mright)\bigr)\mleft(\omega \stackrel{\wedge}{,} \eta\mright)
	%+ \mleft({}^*h\mright)\bigl( R_{\delta}(\vartheta, \varepsilon) \omega \stackrel{\wedge}{,} \eta\bigr)
	%+ \mleft({}^*h\mright)\bigl(\omega \stackrel{\wedge}{,} R_{\delta}(\vartheta, \varepsilon) \eta\bigr).
%\eas
%$\omega$ and $\eta$ will be simple wedge products of $\varpi_2$, $G$ and $\mathfrak{D}$ by Def.~\ref{def:PolynomialFunctionals}, and hence we know that the curvature acting on these will be zero by Cor.~\ref{cor:FlatnessVonEichtrafos} (using $R_\nabla^{\mathrm{bas}}=0$), Prop.~\ref{prop:InfinitesimalGaugeTrafoOfMinimalCoupleSmiley} and Cor.~\ref{cor:NewGaugeTrafoOfFieldStrengthG}, using the assumptions on $\nabla$. Therefore we only need to study terms like $R_{\delta}(\vartheta, \varepsilon)\mleft({}^*h\mright)$, using a local frame $\mleft( e_a \mright)_a$ of $E$,
%\bas
%\delta_\vartheta \delta_\varepsilon ({}^*h)
%&=
%- \delta_\vartheta \mleft(
	%\varepsilon^a {}^*\mleft({}^E\nabla_{e_a} h\mright)
%\mright)
%=
%- \delta_\vartheta \varepsilon^a ~ {}^*\mleft({}^E\nabla_{e_a} h\mright)
	%+ \varepsilon^a \vartheta^b ~ {}^*\mleft({}^E\nabla_{e_b} {}^E\nabla_{e_a} h\mright),
%\eas
%and
%\bas
%\delta_{\Delta(\vartheta, \varepsilon)} ({}^*h)
%~~~~&\stackrel{\mathclap{\text{Eq.~\eqref{EqDeltaInFrameKoord}}}}{=}~~~~
%- \mleft(
	%\delta_{\varepsilon} \vartheta^a
	%- \delta_{\vartheta} \varepsilon^a
	%+ \vartheta^b ~ \varepsilon^c ~  \mleft({}^*\bigl( 
	%\mleft[ e_b, e_c \mright]_E
	%\bigr)\mright)^a
%\mright) ~ {}^*\mleft({}^E\nabla_{e_a} h\mright)
%\\
%&=
%\delta_\vartheta \varepsilon^a ~ {}^*\mleft({}^E\nabla_{e_a} h\mright)
	%- \delta_\varepsilon \vartheta^a ~ {}^*\mleft({}^E\nabla_{e_a} h\mright)
	%- \varepsilon^a \vartheta^b ~ {}^*\mleft( {}^E\nabla_{\mleft[ e_b, e_a \mright]_E} h \mright),
%\eas
%in total
%\bas
%R_{\delta}(\vartheta, \varepsilon)\mleft({}^*h\mright)
%&=
%\varepsilon^a \vartheta^b ~ {}^* \underbrace{\mleft(
	%{}^E\nabla_{e_b} {}^E\nabla_{e_a} h
	%- {}^E\nabla_{e_a} {}^E\nabla_{e_b} h
	%- {}^E\nabla_{\mleft[ e_a, e_b \mright]_E} h
%\mright)}_{R_{{}^E\nabla}(e_b,e_a)h}
%=
%0,
%\eas
%where we used that ${}^E\nabla$ is flat. Therefore $R_{\delta}(\vartheta, \varepsilon) \bigl( 	\mleft({}^*h\mright)\mleft(\omega \stackrel{\wedge}{,} \eta\mright) \bigr) = 0$, and since $H$ basically consists of sums of such terms, with factors in $C^\infty(M)$ which are not affected by the infinitesimal gauge transformation, we can conclude
%\bas
%R_{\delta}(\cdot, \cdot) H
%&=
%0.
%\eas
%\end{proof}
%
%\begin{remarks}{Polynomial functionals closed under the gauge transformations}{PolynomeSindZumGlueckGeschlossenNachGaugeTrafo}
%Following the proof, and keeping the same notation and setup, we also see that
%\ba
%\delta_\varepsilon H
%&\in \mathcal{H}^k_E(M; {}^*V)
%\ea
%for all $\varepsilon \in \mathcal{H}^0_E(M; {}^*E)$ and $H \in \mathcal{H}^k_E(M; {}^*V)$. That is, we had
%\bas
%\delta_\varepsilon \bigl(
	%\mleft({}^*h\mright)\mleft(\omega \stackrel{\wedge}{,} \eta\mright)
%\bigr)
%&=
%\bigl( \delta_\varepsilon \mleft( {}^*  h \mright) \bigr)\mleft(\omega \stackrel{\wedge}{,} \eta\mright)
	%+ \mleft({}^*h\mright)\mleft( \delta_\varepsilon \omega \stackrel{\wedge}{,} \eta\mright)
	%+ \mleft({}^*h\mright)\mleft( \omega \stackrel{\wedge}{,} \delta_\varepsilon \eta\mright).
%\eas
%Clearly, $\delta_\varepsilon \mleft( {}^*  h \mright)$ will be just a pullback of an $E$-connection acting on bundles where $h$ lives, so, no problem on that side. Again, $\omega$ and $\eta$ will be without loss of generality simple products of the generators $\varpi_2$, $G$ and $\mathfrak{D}$ by Def.~\ref{def:PolynomialFunctionals}, and beside the factor given by $\varpi_2$ all vanish under the action of $\delta_\varepsilon$ by Prop.~\ref{prop:InfinitesimalGaugeTrafoOfMinimalCoupleSmiley} and Cor.~\ref{cor:NewGaugeTrafoOfFieldStrengthG} (by the imposed conditions on $\nabla$). So, it is only left to prove that $\delta_\varepsilon \varpi_2$ is a polynomial functional, using Eq.~\eqref{EichtrafoVonANochmal},
%\bas
%\delta_{\varepsilon} \varpi_2
%&=
%- ({}^*\nabla) \varepsilon
%=
%- \mathrm{d} \varepsilon^a \otimes {}^*e_a
	%- \varepsilon^a \otimes {}^!(\nabla e_a)
%\eas
%where $\mleft( e_a \mright)_a$ is again a local frame of $E$. Using the definition of pullbacks of forms,
%\bas
%{}^!(\nabla e_a)
%&=
%{}^*\mleft( \nabla_{\mathrm{D}} e_a \mright)
%=
%{}^*\mleft( \nabla_{\mathfrak{D} - ({}^*\rho)(\varpi_2)} e_a \mright)
%=
%{}^*\mleft( \nabla_{\mathfrak{D}} e_a \mright)
	%- {}^*\mleft( \nabla_{({}^*\rho)(\varpi_2)} e_a \mright)
%\eas
%which is just the contraction of ${}^*(\nabla e_a)$ with $\mathfrak{D}$ and $({}^*\rho)(\varpi_2)$, thence, 
%\bas
%{}^!(\nabla e_a)
%&\in \mathcal{H}^1_E(M; {}^*E).
%\eas
%Finally, we only need to discuss $\mathrm{d}\varepsilon^a$, for which it is useful to use coordinates for the contractions. We have $\varepsilon^a \in \mathcal{H}^0_E(M)$, and, so, $\varepsilon^a$ is also just given by terms of the form $({}^*h)(\omega \stackrel{\wedge}{,} \eta)$; we kept the same notation for simplicity, and $h$ is scalar-valued, not to be confused with the previous discussion. As discussed we can also use $\varpi_2$, $\mathrm{d}^{{}^*\nabla}\varpi_2$, and $\mathrm{D}$ as generators, or, equivalently again a change of generators, but given in coordinates, we have generators $\varpi_2^a$, $\mathrm{d}\varpi_2^a$, and $\mathrm{D}^i$, with respect to local coordinate vector fields $\mleft( \partial_i \mright)_i$ of $N$ and we used that $\mleft(\mathrm{d}^{{}^*\nabla}\varpi_2\mright)^a$ is generated by $\varpi_2^a$, $\mathrm{d}\varpi_2^a$ and $\mathrm{D}^i$ by definition: 
%\bas
%\mathrm{d}^{{}^*\nabla}\varpi_2
%&=
%\mathrm{d}\varpi_2^a \otimes {}^*e_a
	%- \varpi_2^a \wedge {}^!(\nabla e_a)
%=
%\mathrm{d}\varpi_2^a \otimes {}^*e_a
	%- \varpi_2^a \wedge {}^*(\nabla_{\mathrm{D}} e_a).
%\eas
%These generators are contracted with components of pullbacks of scalar-valued tensors. Observe, we have
%\ba\label{ComposOfTangenFunctional}
%\mleft(\mathrm{D}^i\mright)(\Phi,A)
%&=
%\mleft( \mathrm{D}\Phi \mright)^i
%=
%\mathrm{d}\Phi^i.
%\ea
%So, applying the Leibniz rule to $\mathrm{d}\bigl( ({}^*h)(\omega \stackrel{\wedge}{,} \eta) \bigr)$ in coordinates in order to calculate $\mathrm{d}\varepsilon^a$, we have first of all contractions of the generators with the differential acting on the pullback of the components of $h$, which is clearly just the pullback of the differential of the components because of the fact that the de-Rham differential commutes with pullbacks. Left are additionally again contractions of products of the generators $\varpi_2^a$, $\mathrm{d}\varpi_2^a$, $\mathrm{D}^i$ and their differentials, which are just given by $\mathrm{d}\varpi_2^a$, using Eq.~\eqref{ComposOfTangenFunctional}. Hence, $\mathrm{d}\varepsilon^a \in \mathcal{H}_E^1(M)$.
%\newline\newline
%Thus, $\delta_{\varepsilon} \varpi_2 \in \mathcal{H}^1_E(M; {}^*E)$, which concludes the argument.
%\end{remarks}
%
%With this we can finally conclude that the subspace of vector fields like $\Psi_\varepsilon \in \mathfrak{X}^E(\mathfrak{M}_E(M; N))$, given as in Def.~\ref{def:TotalInfGaugeTrafoYayy}, is a subalgebra of $\mathfrak{X}(\mathfrak{M}_E(M;N))$.